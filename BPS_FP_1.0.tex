\documentclass[preprint,12pt,eqsecnum,nofootinbib,amsmath,amssymb]{revtex4}

% Date file was last changed:
\newcommand{\datechange}{3/24/2020}
\newcommand{\datestart}{3/23/2020}

% version
\newcommand\draftverson{v1.0}
\newcommand{\fname}{addendumBPS\_1.0.tex}
\newcommand{\laurnumber}{\draftverson  ~\today ~\currenttime}
\newcommand{\mydate}{\datechange}

% history
% v1.0
% taken from
% stopping_power/working/04-addendumBPS/addendumBPS_1.4.tex

% Person who last changed file:>
\newcommand{\whochange}{Robert Singleton}
%
% Project Name, path, informal author names, title
\newcommand{\projname}{Stopping Power}
\newcommand{\dirname}{Clog}
\newcommand{\myauthors}{Robert L. Singleton Jr}
\newcommand{\myrunningtitle}{\fname}
\newcommand{\mytitle}{The BPS Fokker Planck Equation}
%
% printing margins
%

\textwidth=6.5in
\textheight=9.0in

% packages
%
\usepackage{graphicx}  % Include figure files
\usepackage{dcolumn}  % Align table columns on decimal point
\usepackage{bm}             % Bold math: $\bm{\alpha}$
\usepackage{latexsym}  % Several additional symbols
\usepackage{fancyhdr}  % Fancy header package
\usepackage{wrapfig}
\usepackage{comment}
\usepackage{dsfont}
\usepackage{mathtools}
\usepackage{datetime}
\usepackage{mathrsfs}
\usepackage{amsbsy}
%\usepackage{showkeys}% Displays equation and fig names
%\usepackage{hyperref}% Hyperlinked references

% local commands
\newcommand{\overoverline}[1]{ {\overline{\overline{#1}}} }
\newcommand{\EMPTYSET}{\varnothing}
\newcommand{\PROOF}{{\tiny PROOF}}
\newcommand{\ALTPROOF}{{\tiny ALTERNATE PROOF}}
\newcommand{\PAR}{$\blacktriangleright$}
\newcommand{\ENDPF}{$\blacksquare$}
\newcommand{\ENDPROOF}{$\blacksquare$}
%\newcommand{\ENDPF}{\square}
%\newcommand{\ENDPROOF}{$\square$}
\newcommand{\AND}{\wedge}
\newcommand{\OR}{\vee}
\newcommand{\NOT}{\neg}
\newcommand{\EQ}{\equiv}
\newcommand{\IFF}{\leftrightarrow}
\newcommand{\IMP}{\rightarrow}
\newcommand{\T}{{\rm T}}
\newcommand{\F}{{\rm F}}
\newcommand{\LOGEQ}{\sim}
\newcommand{\smDash}{{\rule[1mm]{0.1cm}{0.1mm}}}
\newcommand{\dbar}{{d\hskip-0.12cm \rule[2.2mm]{0.15cm}{0.1mm}}}
\newcommand{\smA}{{\scriptscriptstyle \rm A}}
\newcommand{\smB}{{\rm\scriptscriptstyle B}}
\newcommand{\smN}{{\rm\scriptscriptstyle N}}
\newcommand{\smX}{{\rm\scriptscriptstyle X}}
\newcommand{\bvec}[1]{\mathbf{#1}}
\newcommand{\smP}{{\rm\scriptscriptstyle P}}
\newcommand{\smL}{{\rm\scriptscriptstyle L}}
\newcommand{\smT}{{\rm\scriptscriptstyle T}}
\newcommand{\smC}{{\rm\scriptscriptstyle C}}
\newcommand{\smI}{{\rm\scriptscriptstyle I}}
\newcommand{\smR}{{\rm\scriptscriptstyle R}}
\newcommand{\smS}{{\rm\scriptscriptstyle S}}
\newcommand{\smD}{{\rm\scriptscriptstyle D}}
\newcommand{\smW}{{\rm\scriptscriptstyle W}}
\newcommand{\smCT}{{\rm\scriptscriptstyle CT}}
\newcommand{\smE}{{\rm\scriptscriptstyle E}}
\newcommand{\smae}{{\rm\scriptscriptstyle ae}}
\newcommand{\extend}[2]{ {#1}^\smallfrown{\! #2} }
\newcommand{\smTC}{{\rm\scriptstyle TC}}
\newcommand{\calT}{ {\cal T}}
\newcommand{\calA}{{\cal A}}
\newcommand{\mathfrakA}{\mathfrak{A}}
\newcommand{\mathfrakB}{\mathfrak{B}}
\newcommand{\mathfrakS}{\mathfrak{S}}
\newcommand{\smGr}{{\rm\scriptscriptstyle gr}}
\newcommand{\smLT}{{\rm\scriptscriptstyle <}}
\newcommand{\smGT}{{\rm\scriptscriptstyle >}}
\newcommand{\smY}{{\rm\scriptscriptstyle Y}}

% % baselineskip modes
\newcommand{\bodyskip}{\baselineskip 18pt plus 1pt minus 1pt}
\newcommand{\bibskip}{\baselineskip16pt plus 1pt minus 1pt}
\newcommand{\tableofcontentsskip}{\baselineskip 14pt plus 1pt minus 1pt}
\newcommand{\footnoteskip}{\baselineskip 12pt plus 1pt minus 1pt}
\newcommand{\abstractskip}{\baselineskip 13pt plus 1pt minus 1pt}
\newcommand{\titleskip}{\baselineskip 18pt plus 1pt minus 1pt}
\newcommand{\affiliationskip}{\baselineskip 15pt plus 1pt minus 1pt}
\newcommand{\captionskip}{\footnotesize \baselineskip 12pt plus 1pt minus 1pt}
\newcommand{\enumerateskip}{\baselineskip 14pt plus 1pt minus 1pt}
\newcommand{\theoremskip}{\baselineskip 13pt plus 1pt minus 1pt}

% theorem
%
\newtheorem{theorem}{Theorem}
\newtheorem{corollary}[theorem]{Corollary}
\newtheorem{definition}[theorem]{Definition}
\newtheorem{lemma}[theorem]{Lemma}
\newtheorem{proposition}[theorem]{Proposition}
\newtheorem{example}[theorem]{Example}
%\newtheorem{theorem}{Theorem}
%\newtheorem{corollary}{Corollary}
%\newtheorem{definition}{Definition}

\pagestyle{fancy}
\lhead{\laurnumber}
%\lhead{}
\chead{}
\rhead{}
\lfoot{}
\cfoot{\thepage}
\rfoot{}

%%
%% begin: draw box
%%
%%%%%%%%%%%%%%%%%%%%%%%%%%%%%%
%%
%%  This macro draws a box around around text, taken 
%%  from ``TeX by Example'', by Arvind Borde p76.
%%
%%   To use: 
%%
%%   \vskip0.3cm
%%   \frame{.1}{2}{16.2cm}{\noindent
%%   \begin{eqnarray}
%%     a = b
%%   \end{eqnarray}
%%   }
%%   \vskip0.2cm
%%
%%%%%%%%%%%%%%%%%%%%%%%%%%%%%%%
%%
\def\frame#1#2#3#4{\vbox{\hrule height #1pt    % TOP RULE
  \hbox{\vrule width #1pt\kern #2pt                     % RULE/SPACE ON LEFT
  \vbox{\kern #2pt                                               % TOP SPACE
  \vbox{\hsize #3\noindent #4}                            % BOXED MATERIAL
  \kern #2pt}                                                        % BOTTOM SPACE
  \kern #2pt\vrule width #1pt}                              % RULE/SPACE ON RIGHT
  \hrule height0pt depth #1pt}                            % BOTTOM RULE
}
%%
\def\myframe#1{\vbox{\hrule height 0.1pt    % TOP RULE
  \hbox{\vrule width 0.1pt\kern 2pt                     % RULE/SPACE ON LEFT
  \vbox{\kern 2pt                                               % TOP SPACE
  \vbox{\hsize 16.5cm\noindent #1}                            % BOXED MATERIAL
  \kern 2pt}                                                        % BOTTOM SPACE
  \kern 2pt\vrule width 0.1pt}                              % RULE/SPACE ON RIGHT
  \hrule height0pt depth 0.1pt}                            % BOTTOM RULE
}
%%
%% draws two boxes around text (use sparingly)
%%
\def\fitframe #1#2#3{\vbox{\hrule height#1pt  % TOP RULE
  \hbox{\vrule width#1pt\kern #2pt             % RULE/SPACE ON LEFT
  \vbox{\kern #2pt\hbox{#3}\kern #2pt}         % TOP,MATERIAL,BOT
  \kern #2pt\vrule width#1pt}                  % RULE/SPACE ON RIGHT
  \hrule height0pt depth#1pt}                  % BOTTOM RULE
}
%%
%% draws a box with shadow around text
%%
\def\shframe #1#2#3#4{\vbox{\hrule height 0pt % NO TOP SHADOW
 \hbox{\vrule width #1pt\kern 0pt             % LEFT SHADOW
 \vbox{\kern-#1pt\frame{.3}{#2}{#3}{#4}       % START SHADOW
 \kern-.3pt}                                  % MOVE UP RULE
 \kern-#2pt\vrule width 0pt}                  % STOP SHADOW
 \hrule height #1pt}                          % BOTTOM SHADOW
}
%%
%%
%% end: draw box
%%
%%  To install as a package on a local host.
%%   a. Append the header ``\usepackage{myboxes}'' to the above macro. Name 
%%   the macreo file myboxes.sty.  Move myboxes.sty into $HOME/texmf/tex/mypackages/. 
%%   You might need to type texhash.
%%   b. T use the package write \usepackage{myboxes} in the preamble.

%
\begin{document}

%%% notes info page
%%\hfill{\laurnumber}
%%\vskip0.3cm
%\centerline{{ \Large\bf \projname: \fname}}
%\vskip0.25cm 
%\centerline{\bf \mytitle}
%\vskip0.25cm
%\centerline{\myauthors}
%\vskip0.75cm 
%\baselineskip 14pt plus 1pt minus 1pt
%\begin{flushright}
%Research Notes   \\[3pt]
%{\it Project}:          \\
%\projname                      \\
%  {\it Path of TeX Source}:          \\
%\dirname/\fname                      \\[3pt]
%{\it Last Modified By}:            \\
%\whochange                         \\
%\datechange                        \\[3pt]
%{\it Date Started:}                \\
%\datestart                         \\[3pt]
%{\it Date:}                \\
%\draftverson~ \today ~\currenttime \\
%\end{flushright}

\baselineskip 20pt plus 1pt minus 1pt

%% mini abstract
%\abstractskip
%\noindent
%These are notes on Logic from Ref.~\cite{ref_chang}.  
%\bodyskip

%% title page
\vskip2.0cm
%\pagebreak
\preprint{\laurnumber}


% publication title page
\title{\titleskip
  \mytitle
}

\author{Robert L Singleton Jr}

\affiliation{
University of Leeds\\
School of Mathematics\\
LS2 9JT
}

\date{\datechange}

\begin{abstract}
\abstractskip
\vskip0.3cm 
\noindent

These are notes on the BPS Fokker-Planck equation.


\end{abstract}

%%
\maketitle
%%

% to change page settings
%\thispagestyle{empty}
%\pagestyle{empty}
%\setcounter{page}{0}

%\pagebreak
%\tableofcontentsskip
%\tableofcontents
%%\thispagestyle{empty}

%\pagebreak
\newpage
\bodyskip
%\setcounter{page}{1}

\pagebreak
\clearpage
%

\section{Introduction}
\label{sec:intro}


I would like to start these notes by addressing some potential misconceptions
of the Brown-Preston-Singleton (BPS) paper\,\cite{BPS2005}, and to clarify 
some issues with the presentation of the original manuscript. Given 
the way in which the paper is structured, one might be led to think 
that the BPS stopping power omits small angle scattering, and is 
only valid in the Fokker-Planck limit. I should emphasize, however, 
that the BPS method employs the full Boltzmann equation, and 
indeed includes the scattering effects from all angles. The
Boltzmann short-distance physics, joined together with  the 
long-distance physics of  the Lenard-Balescu equation, provides 
an exact result to leading and next-to-leading order in the plasma 
coupling $g$. The misconception that BPS only takes into account
small angle collisions could well arise from our 
premature emphasis  on the Fokker-Planck equation in Section~IV 
of the BPS paper. In fact, the name of Section~IV  is {\em General 
Formalism}, and this is a misleading title. The section 
should more properly be called {\em The BPS Fokker-Planck Equation}, 
and it should have appeared later in the text. The location and title 
of this section gives the misleading impression that the BPS stopping 
power and temperature equilibration are performed only within the 
Fokker-Planck approximation. This misconception is reinforced by 
the way in which we present our results, namely, in terms of 
coefficients called ${\cal A}$ and ${\cal B}$ (or actually ${\cal A}$ 
and the trace ${\sum}_\ell \, C^{\ell \ell}$),  which were  introduced 
within the context of a Fokker-Planck description. 

To be clear,  BPS uses the method of dimensional
continuation to calculate the rate of Coulomb energy exchange 
$dE/dt$ and the rate of momentum exchange \hbox{${\bf v}
\cdot d{\bf P}/dt$} for a beam of particles with incident velocity
${\bf v}$ in a weakly coupled and fully ionized plasma. These 
two independent (scalar) quantities are calculated within the BPS 
framework, using the short-distance physics of the Boltzmann 
equation and the long distance-physics of the Lenard-Balescu 
equation, and the results are exact to leading order ($g^2\ln g$) 
and to next-to-leading order ($g^2$) in the plasma coupling.
After these calculates were completed, we used them to 
{\em define} the coefficient functions ${\cal A}(v)$ and 
${\cal B}(v)$ of a 
Fokker-Planck (FP) equation, which can then be used to study  
phenomena such as straggling and the angular separation of 
the incident beam. The BPS Fokker-Planck equation is an improved
equation in that it gives $dE/dt$ and ${\bf v} \cdot d{\bf P}/dt$ 
{\em exactly} to order $g^2 \ln g + g^2$, including {\em large 
angle} effects from the Boltzmann equation. However, all other 
quantities calculated from this FP equation are (of course) only 
accurate to leading order and do not include large-angle effects. 
The order of our presentation does not make this strategy 
entirely clear. 

To summarize, we do four things in the BPS paper of Ref.~\cite{BPS2005}:
(i) we calculate $dE/dt$ and ${\bf v} \cdot d{\bf P}/dt$ for a projectile 
in a plasma, exactly to leading and next-to-leading order in~$g$, (ii) we 
calculate the electron-ion temperature equilibration rate $dE/dt$ exactly  
to leading and next-to-leading order in~$g$, (iii) we use these calculations 
to define a Fokker-Planck equation, and (iv) we calculate the quantity 
$dE_\perp/dt$  to leading and next-to-leading order in $g$. Note that 
(i) and (ii) use the full Boltzmann equation, and consequently include 
large and small angle collisions. The resulting Fokker-Planck equation in 
(iii) is meant to be used in the small-angle limit in $\nu=3$ dimensions, 
and, in general, is accurate to leading order. However, since the coefficients 
${\cal A}$ and ${\cal B}$ have been defined in terms of $dE/dt$ and 
${\bf v} \cdot d{\bf P}/dt$, 
the BPS FP equation gives these quantities to leading and next-to-leading
order, including the effects from all angles. One can think of this as a 
``corrected'' Fokker-Plank equation in which the large-angle contributions
to the stopping power (and the momentum deposition) have been included.


%\clearpage
\section{A Fokker-Planck Formulation}
\label{sec_fp_form}
 
\subsection{Basic Formalism and Definitions}
 
Let us now consider a background plasma consisting of multiple 
species $b$ with temperature $T_b \equiv 1/\beta_b$, charge 
$e_b$, and mass $m_b$. Having calculated the basic quantities 
$dE/dt$ and \hbox{${\bf v} \cdot d{\bf P}/dt$} using the method 
of dimensional continuation, we now return to $\nu=3$. Consider 
an universalized swarm (or a beam) of test particles within the 
plasma. Suppose the test particles are identical and have mass 
$m$ and distribution function $f$. Then the Fokker-Planck (FP) 
equation for the beam particles takes the form 
%%
\begin{align}
  \left[
  \frac{\partial }{\partial t} + {\bf v} \cdot {\bm\nabla} 
  \right] f({\bf r}, {\bf p}, t)
  &=
  {\sum}_b\, {\sum}_{k\ell} \, 
  \frac{\partial}{\partial p^k} \, C_b^{k \ell}({\bf r}, {\bf p}, t)
  \bigg[ \beta_b v^\ell + \frac{\partial}{\partial p^\ell}
  \bigg] f({\bf r}, {\bf p}, t)
  \label{eq_bps_fp}
  \ ,
\end{align}
%% 
where ${\bf v} = {\bf p}/m$ and ${\bm\nabla} = \partial/\partial {\bf r}$. 
Given a physical quantity $q({\bf p})$ associated with an individual particle, 
we define the corresponding spatial density and the flux vector by
%%
\begin{align}
  {\mathscr Q}({\bf r}, t)
  &=
  \int \frac{d^3 p}{(2\pi\hbar)^3}\, q({\bf p}) \, f({\bf r}, {\bf p}, t)
  \label{eq_q}
  \\[8pt]
  {\mathscr F}^k({\bf r}, t)
  &=
   \int \frac{d^3 p}{(2\pi\hbar)^3}\, q({\bf p}) \, v^k\, f({\bf r}, {\bf p}, t)
   \label{eq_fk}
  \ .
\end{align}
%% 
After integrating the momentum variable by parts, the FP equation 
(\ref{eq_bps_fp}) implies
%%
\begin{align}
  \frac{\partial}{\partial t}\,{\mathscr Q}({\bf r}, t)
  +
  {\bm\nabla} \cdot \pmb{\mathscr{F}}({\bf r}, t)
  &=
  -{\sum}_b \int \frac{d^3 p}{(2\pi\hbar)^3} \,
  \frac{d Q_b}{dt}({\bf r}, {\bf p}, t) \, f({\bf r}, {\bf p}, t)
  \label{eq_q_continuity}
 \end{align}
%% 
where
%%
\begin{align}
  \frac{d Q_b}{dt}({\bf r}, {\bf p},  t) 
  &\equiv
  {\sum}_{k \ell}
  \left[ \beta_b v^\ell - \frac{\partial}{\partial p^\ell}\right]
  C_b^{k \ell }({\bf r}, {\bf p}, t) \,\frac{\partial }{\partial p^k}\,q({\bf p})
  \ .
  \label{eq_q_dQdt}
\end{align}
%% 
Note that $dQ_b/dt$ is the average rate of loss of the quantity
$q({\bf p})$ from the interaction of the beam with plasma component 
$b$. 


%\pagebreak
\subsection{Application to Energy and Momentum Loss}

The energy density and energy flux take the form
%%
\begin{align}
  {\mathscr U}
  &=
  \int \frac{d^3 p}{(2\pi\hbar)^3}\, \frac{p^2}{2m} \, f({\bf r}, {\bf p}, t)
  \label{eq_E}
  \\[8pt]
  {\mathscr S}^k
  &=
   \int \frac{d^3 p}{(2\pi\hbar)^3}\, \frac{p^2}{2m}\, \frac{p^k}{m}\, 
   f({\bf r}, {\bf p}, t)
   \label{eq_Eflux}
  \ ,
\end{align}
%% 
where $E({\bf p}) \equiv   p^2/2m $ is the kinetic energy of 
a single projectile within the beam. The energy continuity 
equation is
%%
\begin{align}
  \frac{\partial {\mathscr U}}{\partial t}
  +
  {\bm\nabla} \cdot \pmb{\mathscr{S}}
  &=
  -{\sum}_b \int \frac{d^3 p}{(2\pi\hbar)^3} \,
  \frac{d E_b}{dt} \, f({\bf r}, {\bf p}, t)
  \ .
  \label{eq_q_continuity}
\end{align}
%% 
The momentum density and momentum flux are
%%
\begin{align}
  {\mathscr P}^k
  &=
  \int \frac{d^3 p}{(2\pi\hbar)^3}\, p^k\, f({\bf r}, {\bf p}, t)
  \label{eq_P}
  \\[8pt]
  {\mathscr T}^{k\ell}
  &=
   \int \frac{d^3 p}{(2\pi\hbar)^3}\, \frac{p^k p^\ell}{m}\, 
   f({\bf r}, {\bf p}, t)
   \label{eq_Pflux}
  \ ,
\end{align}
%% 
and the continuity equation is
%%
\begin{align}
  \frac{\partial {\mathscr P^k}}{\partial t}
  +
  {\nabla}^\ell  {\mathscr{T}^{k\ell}}
  &=
  -{\sum}_b \int \frac{d^3 p}{(2\pi\hbar)^3} \,
  \frac{d P_b}{dt} \, f({\bf r}, {\bf p}, t)
  \ .
  \label{eq_P_continuity}
\end{align}
%%
The rate of change of energy and momentum takes the form
  %%
\begin{align}
  \frac{d E_b}{dt}
  &=
  {\sum}_{k\ell}
  \left[ \beta_b v^\ell - \frac{\partial}{\partial p^\ell}\right]
  C_b^{k \ell }\,v^k
  \ ,
  \label{eq_E_dEdt}
  \\[11pt]
  \frac{d P^k_b}{dt}
  &=
  {\sum}_{\ell}
  \left[ \beta_b v^\ell - \frac{\partial}{\partial p^\ell}\right]
  C_b^{k \ell }
  \ .
  \label{eq_P_dEdt}
\end{align}
%% 
We have used $\partial E/\partial p^k = p^k/m = v^k$ to express  
(\ref{eq_E_dEdt}), and  $\partial p^k/\partial p^r =\delta^{k r}$ in 
deriving (\ref{eq_P_dEdt}). We can also write (\ref{eq_P_dEdt}) in 
the form
%%
\begin{align}
  v^k \, \frac{dP_b^k}{dt} 
  &= 
  \frac{dE_b}{dt} 
  +
  \frac{1}{m} \, {\sum}_\ell \, C_b^{\ell \ell} \,,
\end{align}
%%
which is BPS Eq.~(4.22). 
By calculating
$dE_b/dt$ and $dP^k/dt$ (or actually $v^k \, dP^k/dt$)
using the BPS method, we can define an FP tensor 
$C_b^{k \ell}$. As calculated by the FP equation, the 
energy and momentum 
exchange in the plasma are given by (\ref{eq_E_dEdt}) and 
(\ref{eq_P_dEdt}). However, since these quantities were calculated
using the BPS method, we see that the BPS Fokker-Planck 
equation gives exact (including large angles) results for the 
rates of energy and momentum exchange. For all other quantities,
the BPS Fokker-Planck equation is only accurate to leading 
order in the small angle limit. 


%\pagebreak
\subsection{Decomposition of the FP Tensor}

We can decompose the FP tensor $C_b^{k\ell}({\bf p})$ into 
longitudinal and transverse components along ${\bf p}$ (or
equivalently along ${\bf v}={\bf p}/m$), 
%%
\begin{align}
  C_b^{k \ell}({\bf p})
  &= 
  {\mathcal A}_b(v) \, \frac{\hat v^k \, \hat v^\ell}{\beta_b v}
   + 
 { \mathcal B}_b(v)  \,\frac{1}{2} \big(\delta^{k \ell } - \hat v^k\,  \hat v^\ell \big)
  \ ,
  \label{eq_Cbkl}
\end{align}
%%
where $\hat{\bf v} = {\bf v}/v$ is the unit vector in the ${\bf v}$ direction. 
This  is BPS Eq.~(4.18). The time dependence has been left implicit, as has 
the spatial dependence (we are in fact only interested in a spatially uniform 
and isotropic plasma). It follows from  (\ref{eq_E_dEdt}) that
%%
\begin{align}
  \frac{dE_b}{dt}
  &= 
  \left[ v - {\sum}_\ell
  \frac{1}{\beta_b m}\, \frac{\partial}{\partial v^\ell}\, \hat v_\ell
  \right]
  {\mathcal A}_b(v)
  \\[8pt]
  &=
  \left[ v -  \frac{2}{\beta_b m v} -\frac{1}{\beta_b m} 
  \frac{\partial \,\,}{\partial v} 
  \right] {\mathcal A}_b(v) 
  \,,
\end{align}
%%
which is BPS Eq.~(4.21). Note that the (three dimensional) trace of
(\ref{eq_Cbkl}) is
%%
\begin{align}
  {\sum}_\ell \, C_b^{\ell \ell}
  &= 
  \frac{ {\mathcal A}_b}{\beta_b v}
   + 
 { \mathcal B}_b  
  \ .
  \label{eq_C_trace_AB}
\end{align}
%%

\subsection{A Columnated Beam}

Now consider a columated beam with indicdent velocity ${\bf v}_p$
and particle mass $m_p$. We can define the stopping power and
momentum loss per unit distance as
%%
\begin{align}
  \frac{d E_b}{dx}(v_p)
  &=
  \frac{1}{v_p}\, \frac{dE_b}{dt}(v_p)
  \\[11pt]
  \frac{d {\bf P}_b}{dx}(v_p)
  &=
  \frac{1}{v_p}\, \frac{d{\bf P}_b}{dt}(v_p)
  \ ,
\end{align}
%% 
so that
%%
\begin{align}
  \frac{d E_b}{dx}(v_p)
  &=
  \left[ 1 -  \frac{2}{\beta_b m_p v_p^2} -\frac{1}{\beta_b m_p v_p} 
  \frac{\partial \,\,}{\partial v_p}  \right] {\mathcal A}_b(v_p) 
  \label{eq_E_dEdt_p}
  \\[11pt]
  {\bf v_p} \cdot \frac{d {\bf P}_b^k}{dx}(v_p)
  &= 
  \frac{dE_b}{dx}(v_p)
  + 
  \frac{1}{m_p v_p} \,{\sum}_\ell \,  C_b^{\ell \ell}(v_p) 
  \\[5pt]
  &=
    \left[ 1 -  \frac{1}{\beta_b m_p v_p^2} -\frac{1}{\beta_b m_p v_p} 
  \frac{\partial \,\,}{\partial v_p}  \right] {\mathcal A}_b(v_p) 
  +  
  \frac{1}{m_p v_p} \,
  {\cal B}_b(v_p)
  \ .
 \label{eq_P_dEdt_p}
\end{align}
%% 
Define the {\em transverse energy} by
%%
\begin{align}
   E_\perp({\bf p})
   =
   \frac{1}{2}\, m_p 
   \Big[
   {\bf v}^2 - ({\bf v} \cdot \hat {\bf v}_p)^2
   \Big]
%   =
%   \frac{1}{2}\, m_p {\sum}_{k \ell} v_p^k v_p^\ell \,
%   \Big[ \delta^{k \ell} - \hat v_p^k \, \hat v_p^\ell
%   \Big]
   \ ,
\end{align}
%% 
where ${\bf p} = m_p {\bf v}$, and $\hat {\bf v}_p = {\bf v}_p/v_p$ 
is the unit vector in the direction of the projectile velocity~${\bf v}_p$. 
We can also write the transverse energy as
%%
\begin{align}
   E_\perp({\bf p})
   =
   \frac{1}{2 m_p} \, {\sum}_{k \ell} \,  p^k p^\ell \,
   \Big[ \delta^{k \ell} - \hat v_p^k \, \hat v_p^\ell
   \Big]
   \ .
\end{align}
%% 
The transerve energy density and energy flux take the form
%%
\begin{align}
  {\mathscr U}_\perp
  &=
  \int \frac{d^3 p}{(2\pi\hbar)^3}\,E_\perp({\bf p}) \, f({\bf r}, {\bf p}, t)
  \label{eq_E_perp}
  \\[8pt]
  {\mathscr S}^k_\perp
  &=
   \int \frac{d^3 p}{(2\pi\hbar)^3}\, E_\perp({\bf p})\, \frac{p^k}{m}\, 
   f({\bf r}, {\bf p}, t)
   \label{eq_Eflux_perp}
  \ ,
\end{align}
%% 
and the continuity equation is
%%
\begin{align}
  \frac{\partial {\mathscr U}_\perp}{\partial t}
  +
  {\bm\nabla} \cdot \pmb{\mathscr{S}}_{\!\perp}
  &=
  -{\sum}_b \int \frac{d^3 p}{(2\pi\hbar)^3} \,
  \frac{d E_{b \, \perp}}{dt} \, f({\bf r}, {\bf p}, t)
  \ .
  \label{eq_q_continuity_perp}
\end{align}
%% 

%%%%%%
\newpage

%%%%%%
%
\begin{acknowledgments}
  I am particularly indebted to Jean-Etienne Sauvestre for reading 
  the BPS manuscript so thoroughly, and for helping clarify a number 
  of potential problems with the structure of the text. 
\end{acknowledgments}


%%%%%%%%
\appendix

%\pagebreak
\section{Straggling and Angular Spreading}

\subsection{Exact Definitions and Relations}

There are three independent quantities
that one can calculate: $dE/dt$, ${\bf v}_p \cdot d{\bf P}/dt$, and
$dE_\perp/dt$. Since $dE/dt  = dE_\perp/dt + dE_\parallel/dt$,
we can trade off $dE_\parallel/dt$ for $dE_\perp/dt$. Every time 
$t$ that $\mathbf v_p$ is specified must be considered as an initial 
time, and could write ${\bf v}_p(t)$ to emphasize that the velocity
is given at the initial time $t$. Furthermore, we must specify all 
ensemble averages $\langle X \rangle$ at the initial time. Let us 
form the following. Since the beam is sharply peaked about ${\bf v}_p$, 
we find
%%
\begin{align}
  \langle v^k(t_0) \rangle 
  &= 
  v_p^k 
  \hskip3.5cm
  T^{kl} = \langle v^k(t_0) v^l(t_0)  \rangle = v_p^k v_p^l 
  \\[10pt]
   P^k &= m \langle v^k(t_0) \rangle  = m v_p^k 
   \hskip1.0cm
   E = \frac{1}{2} m \mathbf v_p^2 = \frac{1}{2} m T^{kk} \,.
\end{align}
%% 
%%%
%\begin{align}
%  P^k &= m \langle v^k(t) \rangle  = m v_p^k\,, 
%  \qquad
%  E = \frac{1}{2} m \mathbf v_p^2 = \frac{1}{2} m T^{kk} \,.
%\end{align}
%%%
These are fixed, given by the initial conditions; however, the time derivative
of a quantity is  is determined by the dynamics,\footnote{
\footnoteskip
Look at footnote 18 in BPS. I am not sure Eq.~(4.33) is correct to order $g^2$ 
under the log. We should also look at Refs. [15] and [16] in BPS.
} % footnote
%%
\begin{align}
  \frac{d}{dt}\big\langle v^k(t) \big\rangle &= 
    \Big\langle \frac{d v^k(t)}{dt} \Big\rangle 
\hskip2.0cm
\frac{d}{dt} T^{kl}(t) =
 \Big\langle \Big[\frac{d v^k(t)}{dt} v^l(t) +
    v^k(t)  \frac{d v^l(t)}{dt} \Big] \Big\rangle 
\\[11pt]
v_p^k\, \frac{dP^k(t)}{dt} 
&= v_p^k \, m \, \Big\langle \frac{d v^k(t)}{dt} \Big\rangle 
\hskip1.2cm
\frac{dE}{dt} = \frac{1}{2}\,m \, \frac{dT^{kk}(t)}{dt}
\ .
\end{align}
%%
Using the tensor $T^{kl}(t)$, we can define the quantities 
$dE_\parallel/dt$ and $dE_\perp/dt$ by taking moments
with respect to the direction of  motion,
%%
\begin{align}
  \hat{\mathbf v}_p 
  &= \frac{{\mathbf v}_p}{|{\mathbf v}_p|}
   = \frac{{\mathbf v}_p}{v_p}
\ ,
\end{align}
%%
namely
%%
\begin{align}
  \frac{d E_\parallel}{dt} 
  &= 
  \hat v^k_p \,\hat v^l_p \, \frac{1}{2}\, m \,\frac{d}{dt} T^{kl}(t) 
  =
  \hat v^k_p \, \hat v^l_p\, m \,
  \Big\langle \frac{d v^k(t)}{dt} \, v^l(t)\Big\rangle 
\\[11pt]
  \frac{d E_\perp}{dt} 
  &=
  \Big[ \delta^{kl}-\hat v^k_p \, \hat v^l_p \Big] 
  \frac{1}{2}\, m \, \frac{d}{dt} T^{kl}(t) 
  =
  \Big[ \delta^{kl}-\hat v^k_p \, \hat v^l_p \Big] m\, 
  \Big\langle \frac{d v^k(t)}{dt} v^l(t)\Big\rangle 
  \ .
\end{align}
%%
Note that
%%
\begin{align}
  \frac{d E}{dt} 
  &=\frac{d E_\parallel}{dt} + \frac{d E_\perp}{dt}
   \,,
\end{align}
illustrating that $dE_\parallel/dt$ and $dE_\perp/dt$ are
not independent in our formulation. 
We close this section by examining the velocity fluctuations,
%%
\begin{align}
  \frac{d \,\,}{dt} \widetilde{T}^{kl}(t) 
  &\equiv
  \frac{d \,\,}{dt} \Big\langle \Big[v^k(t) - \langle v^k(t) \rangle \Big] 
  \Big[v^l(t) - \langle v^l(t) \rangle \Big] \Big\rangle 
  \nonumber\\
  &=
  \frac{d \,\,}{dt} T^{kl}(t) 
    - \Big\langle \frac{d v^k(t)}{dt} \Big\rangle v^l_p 
    -   v^k_p \Big\langle \frac{d v^l(t)}{dt} \Big\rangle 
\ ,
\end{align}
%% 
and thus
%%
\begin{align}
\frac{d \widetilde{E}_\parallel}{dt} &= \frac{d E_\parallel}{dt} 
    - v^k_p\, \frac{d P^k}{dt} 
    \ .
\end{align}
%%
And independent calculation gives
%%
\begin{align} 
\frac{d \widetilde{E}_\perp}{dt} &= \frac{d E_\perp}{dt} \,.
\label{perp}
\end{align}
%%
At high velocities, we expect that 
$d \widetilde{E}_\parallel / dt \approx 0$. Then
$d E_\perp / dt \approx dE/dt - \mathbf v_p \cdot d\mathbf P /dt$.


\subsection{Isotropic Plasma}

In an isotropic plasma, $d{\bf P}/dt$ points in the direction ${\bf v}_p$,
which means that
%%
\begin{align}
\frac{d P^k}{dt}  &= \hat v^k_p\, \dot P(t) 
\ ,
\end{align}
%% 
where $P(t)$ is a scalar. BPS calculates the quantity $\dot P = dP/dt$ to
leading to next-to-leading order in $g$. BPS also calculates $dE/dt$ to
the same order. This determines the ${\cal A}$ and ${\cal B}$ coefficients
defined above. Also note the exact relation
%% 
\begin{align}
  \frac{d}{dt} \, \frac{1}{2}\, m T^{kl}(t) 
  &= 
  \hat v_p^k \,\hat v_p^k \, \frac{dE_\parallel}{dt} 
  + 
  \frac{1}{2} \Big[ \delta^{kl} - \hat v_p^k \, \hat v_p^l \Big] \frac{dE_\perp}{dt} 
  \ ,
\end{align}
%%
which gives a physical interpretation to $dE_\parallel/dt$ and 
$dE_\perp/dt$. 

\subsection{Transverse Energy Loss}


There is an ambiguity in the definition of the BPS Fokker-Planck
equation. We have chosen the coefficients ${\cal A}$ and ${\cal B}$
in such a way as to give exact large-angle corrections to $dE/dt$
and ${\bf v}_p \cdot d{\bf P}/dt$. As such, when we evaluate
$dE_\perp/dt$ by this FP equation, the result is only accurate
to order $g^2 \ln g$.
Equation (4.30) of BPS gives the Fokker-Planck evaluation
%%
\begin{align}
\frac{d E_\perp}{dt} &= \frac{1}{m} \mathcal B \,.
\label{goodperp}
\end{align}
%%
In Section X of BPS, we calculate the $g^2$ correction, thereby
providing all three independent quantities exactly to leading and
next-to-leading order.

**
On the other hand, the discussion after \eqref{perp} suggests that
 %%
\begin{align}
\frac{d E_\perp}{dt} &\approx \frac{d E}{dt} - 
    \mathbf v_p \cdot \frac{d \mathbf P}{dt} \,.
\label{badperp}
\end{align}
%%
But Eqs.(4.23) and (4.24) of BPS yield 
 %%
\begin{align}
\frac{d E}{dt} - \mathbf v_p \cdot \frac{d \mathbf P}{dt} 
&= \frac{1}{m} C^{ll}
\nonumber\\[8pt]
 &= \frac{1}{\beta m v_p} \mathcal A + \frac{1}{m} \mathcal B
\label{fast}
\end{align}
%%
which is far away from BPS Eq.~(4.30) which we quoted in our
Eq.~\eqref{goodperp} except for such high projectile velocities
that $(m_p/m_e) (T / E_p) \ll 1$, in which case Eqs.(10.42) and
(10.45) of BPS show that the $\mathcal A$ term in Eq.~\eqref{fast}
can be neglected. 

At any rate, as described in BPS, the correction (11.21) added to the
approximation (4.30) quoted above gives $d E_\perp /dt $ to order
$\ln g^2 + g^2$.   


%
%%%
%\section{\label{sec:one} Section One of the Appendix}
%

%\clearpage
\begin{thebibliography}{99}
\bibskip


\bibitem{BPS2005}
           Lowell S Brown, Dean L Preston, and Robert L Singleton, Jr,
           {\it Charged Particle Motion in a Highly Ionized Plasma},
           Phys. Rep. {\bf 410} (2005) 237-333.




\end{thebibliography}

\end{document}
        
