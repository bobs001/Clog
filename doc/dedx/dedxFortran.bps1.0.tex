\documentclass[preprint,12pt,eqsecnum,nofootinbib,amsmath,amssymb]{revtex4}

% Date file was last changed:
\newcommand{\datechange}{3/4/2020}
\newcommand{\datestart}{3/4/2020}

% version
\newcommand\draftverson{v1.0}
\newcommand{\fname}{dedxFortran.bps1.0.tex}
\newcommand{\laurnumber}{\draftverson  ~\today ~\currenttime}
\newcommand{\mydate}{\datechange}

% Person who last changed file:>
\newcommand{\whochange}{Robert Singleton}
%
% Project Name, path, informal author names, title
\newcommand{\projname}{Clog Doc}
\newcommand{\dirname}{Clog/doc/dedx}
\newcommand{\myauthors}{Robert Singleton}
\newcommand{\myrunningtitle}{\fname}
\newcommand{\mytitle}{Stopping Power in Clog}
%
% printing margins
%

\textwidth=6.5in
\textheight=9.5in

% packages
%
\usepackage{graphicx}  % Include figure files
\usepackage{dcolumn}  % Align table columns on decimal point
\usepackage{bm}             % Bold math: $\bm{\alpha}$
\usepackage{latexsym}  % Several additional symbols
\usepackage{fancyhdr}  % Fancy header package
\usepackage{wrapfig}
\usepackage{comment}
\usepackage{dsfont}
\usepackage{mathtools}
\usepackage{datetime}
%\usepackage{showkeys}% Displays equation and fig names
%\usepackage{hyperref}% Hyperlinked references

% local commands
\newcommand{\overoverline}[1]{ {\overline{\overline{#1}}} }
\newcommand{\EMPTYSET}{\varnothing}
\newcommand{\PROOF}{{\tiny PROOF}}
\newcommand{\ALTPROOF}{{\tiny ALTERNATE PROOF}}
\newcommand{\PAR}{$\blacktriangleright$}
\newcommand{\ENDPF}{$\blacksquare$}
\newcommand{\ENDPROOF}{$\blacksquare$}
%\newcommand{\ENDPF}{\square}
%\newcommand{\ENDPROOF}{$\square$}
\newcommand{\AND}{\wedge}
\newcommand{\OR}{\vee}
\newcommand{\NOT}{\neg}
\newcommand{\EQ}{\equiv}
\newcommand{\IFF}{\leftrightarrow}
\newcommand{\IMP}{\rightarrow}
\newcommand{\T}{{\rm T}}
\newcommand{\F}{{\rm F}}
\newcommand{\LOGEQ}{\sim}
\newcommand{\smDash}{{\rule[1mm]{0.1cm}{0.1mm}}}
\newcommand{\dbar}{{d\hskip-0.12cm \rule[2.2mm]{0.15cm}{0.1mm}}}
\newcommand{\smA}{{\scriptscriptstyle \rm A}}
\newcommand{\smB}{{\rm\scriptscriptstyle B}}
\newcommand{\smN}{{\rm\scriptscriptstyle N}}
\newcommand{\smX}{{\rm\scriptscriptstyle X}}
\newcommand{\bvec}[1]{\mathbf{#1}}
\newcommand{\smP}{{\rm\scriptscriptstyle P}}
\newcommand{\smL}{{\rm\scriptscriptstyle L}}
\newcommand{\smT}{{\rm\scriptscriptstyle T}}
\newcommand{\smC}{{\rm\scriptscriptstyle C}}
\newcommand{\smI}{{\rm\scriptscriptstyle I}}
\newcommand{\smR}{{\rm\scriptscriptstyle R}}
\newcommand{\smS}{{\rm\scriptscriptstyle S}}
\newcommand{\smQ}{{\rm\scriptscriptstyle Q}}
\newcommand{\smD}{{\rm\scriptscriptstyle D}}
\newcommand{\smO}{{\rm\scriptscriptstyle 0}}
\newcommand{\smW}{{\rm\scriptscriptstyle W}}
\newcommand{\smCT}{{\rm\scriptscriptstyle CT}}
\newcommand{\smQM}{{\rm\scriptscriptstyle QM}}
\newcommand{\smE}{{\rm\scriptscriptstyle E}}
\newcommand{\smae}{{\rm\scriptscriptstyle ae}}
\newcommand{\extend}[2]{ {#1}^\smallfrown{\! #2} }
\newcommand{\smTC}{{\rm\scriptstyle TC}}
\newcommand{\calT}{ {\cal T}}
\newcommand{\calA}{{\cal A}}
\newcommand{\mathfrakA}{\mathfrak{A}}
\newcommand{\mathfrakB}{\mathfrak{B}}
\newcommand{\mathfrakS}{\mathfrak{S}}
\newcommand{\smGr}{{\rm\scriptscriptstyle gr}}
\newcommand{\smLT}{{\rm\scriptscriptstyle <}}
\newcommand{\smGT}{{\rm\scriptscriptstyle >}}
\newcommand{\smY}{{\rm\scriptscriptstyle Y}}

% % baselineskip modes
\newcommand{\bodyskip}{\baselineskip 18pt plus 1pt minus 1pt}
\newcommand{\bibskip}{\baselineskip16pt plus 1pt minus 1pt}
\newcommand{\tableofcontentsskip}{\baselineskip 14pt plus 1pt minus 1pt}
\newcommand{\footnoteskip}{\baselineskip 12pt plus 1pt minus 1pt}
\newcommand{\abstractskip}{\baselineskip 13pt plus 1pt minus 1pt}
\newcommand{\titleskip}{\baselineskip 18pt plus 1pt minus 1pt}
\newcommand{\affiliationskip}{\baselineskip 15pt plus 1pt minus 1pt}
\newcommand{\captionskip}{\footnotesize \baselineskip 12pt plus 1pt minus 1pt}
\newcommand{\enumerateskip}{\baselineskip 14pt plus 1pt minus 1pt}
\newcommand{\theoremskip}{\baselineskip 13pt plus 1pt minus 1pt}

% theorem
%
\newtheorem{theorem}{Theorem}
\newtheorem{corollary}[theorem]{Corollary}
\newtheorem{definition}[theorem]{Definition}
\newtheorem{lemma}[theorem]{Lemma}
\newtheorem{proposition}[theorem]{Proposition}
\newtheorem{example}[theorem]{Example}
%\newtheorem{theorem}{Theorem}
%\newtheorem{corollary}{Corollary}
%\newtheorem{definition}{Definition}

\pagestyle{fancy}
\lhead{\laurnumber}
%\lhead{}
\chead{}
\rhead{}
\lfoot{}
\cfoot{\thepage}
\rfoot{}

%%
%% begin: draw box
%%
%%%%%%%%%%%%%%%%%%%%%%%%%%%%%%
%%
%%  This macro draws a box around around text, taken 
%%  from ``TeX by Example'', by Arvind Borde p76.
%%
%%   To use: 
%%
%%   \vskip0.3cm
%%   \frame{.1}{2}{16.2cm}{\noindent
%%   \begin{eqnarray}
%%     a = b
%%   \end{eqnarray}
%%   }
%%   \vskip0.2cm
%%
%%%%%%%%%%%%%%%%%%%%%%%%%%%%%%%
%%
\def\frame#1#2#3#4{\vbox{\hrule height #1pt    % TOP RULE
  \hbox{\vrule width #1pt\kern #2pt                     % RULE/SPACE ON LEFT
  \vbox{\kern #2pt                                               % TOP SPACE
  \vbox{\hsize #3\noindent #4}                            % BOXED MATERIAL
  \kern #2pt}                                                        % BOTTOM SPACE
  \kern #2pt\vrule width #1pt}                              % RULE/SPACE ON RIGHT
  \hrule height0pt depth #1pt}                            % BOTTOM RULE
}
%%
\def\myframe#1{\vbox{\hrule height 0.1pt    % TOP RULE
  \hbox{\vrule width 0.1pt\kern 2pt                     % RULE/SPACE ON LEFT
  \vbox{\kern 2pt                                               % TOP SPACE
  \vbox{\hsize 16.5cm\noindent #1}                            % BOXED MATERIAL
  \kern 2pt}                                                        % BOTTOM SPACE
  \kern 2pt\vrule width 0.1pt}                              % RULE/SPACE ON RIGHT
  \hrule height0pt depth 0.1pt}                            % BOTTOM RULE
}
%%
%% draws two boxes around text (use sparingly)
%%
\def\fitframe #1#2#3{\vbox{\hrule height#1pt  % TOP RULE
  \hbox{\vrule width#1pt\kern #2pt             % RULE/SPACE ON LEFT
  \vbox{\kern #2pt\hbox{#3}\kern #2pt}         % TOP,MATERIAL,BOT
  \kern #2pt\vrule width#1pt}                  % RULE/SPACE ON RIGHT
  \hrule height0pt depth#1pt}                  % BOTTOM RULE
}
%%
%% draws a box with shadow around text
%%
\def\shframe #1#2#3#4{\vbox{\hrule height 0pt % NO TOP SHADOW
 \hbox{\vrule width #1pt\kern 0pt             % LEFT SHADOW
 \vbox{\kern-#1pt\frame{.3}{#2}{#3}{#4}       % START SHADOW
 \kern-.3pt}                                  % MOVE UP RULE
 \kern-#2pt\vrule width 0pt}                  % STOP SHADOW
 \hrule height #1pt}                          % BOTTOM SHADOW
}
%%
%%
%% end: draw box
%%
%%  To install as a package on a local host.
%%   a. Append the header ``\usepackage{myboxes}'' to the above macro. Name 
%%   the macreo file myboxes.sty.  Move myboxes.sty into $HOME/texmf/tex/mypackages/. 
%%   You might need to type texhash.
%%   b. T use the package write \usepackage{myboxes} in the preamble.

%
\begin{document}

%% notes info page
%\hfill{\laurnumber}
%\vskip0.3cm
\centerline{{ \Large\bf \projname: \fname}}
\vskip0.25cm 
\centerline{\bf \mytitle}
\vskip0.25cm
\centerline{\myauthors}
\vskip0.75cm 
\baselineskip 14pt plus 1pt minus 1pt
\begin{flushright}
Research Notes   \\[3pt]
{\it Project}:          \\
\projname                      \\
  {\it Path of TeX Source}:          \\
\dirname/\fname                      \\[3pt]
{\it Last Modified By}:            \\
\whochange                         \\
\datechange                        \\[3pt]
{\it Date Started:}                \\
\datestart                         \\[3pt]
{\it Date:}                \\
\draftverson~ \today ~\currenttime \\
\end{flushright}

\baselineskip 20pt plus 1pt minus 1pt

%% mini abstract
%\abstractskip
%\noindent
%These are notes on Logic from Ref.~\cite{ref_chang}.  
%\bodyskip

%% title page
\vskip2.0cm
%\pagebreak
\preprint{\laurnumber}

% publication title page
\title{\titleskip
  \mytitle
}

\author{Robert L Singleton Jr}

\affiliation{\affiliationskip
   School of Mathematics\\
   University of Leeds\\
   LS2 9JT
}

%\vskip 0.2cm 
%\affiliation{\affiliationskip
%     %$^1$
%     Los Alamos National Laboratory\\
%     Los Alamos, New Mexico 87545, USA
%}

\date{\datechange}

\begin{abstract}
\abstractskip
\vskip0.3cm 
\noindent
  Physics documentation for the BPS stopping power in the code Clog.
\end{abstract}

%%
\maketitle
%%

% to change page settings
%\thispagestyle{empty}
%\pagestyle{empty}
%\setcounter{page}{0}

\pagebreak
\tableofcontentsskip
\tableofcontents
%\thispagestyle{empty}

%\pagebreak
\newpage
\bodyskip
%\setcounter{page}{1}

\pagebreak
\clearpage

\newpage
\section{The Stopping Power Expression of BPS}

\subsection{Description of the Plasma}

In what follows, I will use the stopping power calculations of
Ref.~\cite{bps} (BPS).  We consider a projectile moving through a hot
fully ionized plasma. The velocity of the projectile is $v_p$, its
charge $e_p$, and its mass $m_p$. The plasma will consist of species
$b$ of charge $e_b$ and mass $m_b$, with number density $n_b$ and
inverse temperature $\beta_b=1/T_b$, where the temperature is measured
in energy units (by convention $b=1$ represents the electron species
and $b \ge 2$ the ion species).  Rationalized units are used for the
charge so that, for example, the Coulomb potential energy in three
dimensions reads $e_p^2 / (4\pi r)$. In summary, the properties
characterizing the projectile and the plasma are
%%
\begin{eqnarray}
  {\rm projectile:} && ~~ e_p ~~ m_p ~~ v_p 
\\
  {\rm plasma:}     && ~~ e_b ~~ m_b ~~ n_b ~~ \beta_b \ .
\end{eqnarray}
%%
For species $b$ the corresponding Debye wave number is defined by
%%
\begin{eqnarray}
  \kappa_b^2 = \beta_b\, n_b \,e_b^2 \ .
\end{eqnarray}
%%
and the total Debye wave number $\kappa_\smD$ is defined by the sum
over all the species
%%
\begin{eqnarray}
  \kappa_\smD^2 = {\sum}_b \, \kappa_b^2 \ .
\end{eqnarray}
%%
The relative and total masses of the projectile and plasma particles
are
%%
\begin{eqnarray}
  \frac{1}{m_{pb} } &=& \frac{1}{m_p } + \frac{1}{m_b } 
\\[5pt]
  M_{pb} &=& m_p + m_b \ .
\end{eqnarray}

For a dilute plasma the dielectric function is given
by\cite{Lifs}
%%
\begin{equation}
  \epsilon({\bf k},\omega) = 1 + {\sum}_c \,
  \frac{e_c^2}{k^2} \int
  \frac{d^\nu {\bf p}_c}{( 2\pi\hbar )^\nu }
  \frac{ 1 }{ \omega - {\bf k} \cdot{\bf v}_c + i \eta}\,
  {\bf k} \cdot \frac{\partial}{\partial {\bf p}_c }
    f_c({\bf p}_c) \,,
\label{epsilon}
\end{equation}
%%
where the prescription $ \eta \to 0^+ $ is implicit and defines the
correct retarded response, and $f_c$ is the Maxwell distribution
function
%%
\begin{eqnarray}
  f_c({\bf p}_c) 
  = n_c \left(\frac{2\pi \hbar^2 \beta_b}
  {m_c}\right)^{\nu/2} \exp\left\{- {\beta_c \over 2} \, 
  m_c v_c^2 \right\} \ .
\label{MB}
\end{eqnarray}
%%

Computing the derivative in Eq.~(\ref{epsilon}) and then 
integrating out the momentum components of ${\bf p}_c$ 
perpendicular to ${\bf k}$ gives the structure
%%
\begin{equation}
  k^2 \, \epsilon( k , k v_p \cos\theta ) =
  k^2 + F(v_p \cos\theta)  \,.
\label{struct}
\end{equation}
%%
The $F$ function appears in the form of a 
dispersion relation
%%
\begin{equation}
  F(u)  =  - \int_{-\infty}^{+\infty} dv \, 
{ \rho_{\rm total}(v) \over u
  - v + i \eta } \,,
\label{disp}
\end{equation}
%%
with the spectral weight
%%
\begin{equation}
  \rho_{\rm total}(v)  = {\sum}_c \, 
	\rho_c(v) \,,
\label{efrelb}
\end{equation}
%%
where
%%
\begin{eqnarray}
\label{spectral}
  \rho_c(v) &=& 
  \kappa^2_c \,\sqrt{ \beta_c m_c \over 2\pi } ~v ~
  \exp\left\{
  -{1 \over 2} \beta_c m_c v^2 \right\} \ .
\end{eqnarray}
%%
For future use, we note that $F$ satisfies the relations
%%
\begin{eqnarray}
\label{fmfstar}
  F(-v) &=& F^*(v) \ .
\\[5pt]
\label{ImF}
  {\rm Im} \, F(v) &=& {1 \over 2i} 
  \left[ F(v) - F^*(v) \right] = 
  \pi  \rho_{\rm total}(v) 
\\[5pt]
\label{rhoodd}
  \rho_c(-v) &=& - \rho_c(v) \ .
\end{eqnarray}
%%
The argument $v$ of $F$ and $\rho_c$ has units of velocity.
The real and imaginary parts of $F$ can be written
\begin{eqnarray}
\label{Fru}
  F_\smR(v)
  &=& 
  \sum_b \kappa_b^2 
  \left[1 - 2 \sqrt{\frac{\beta_b m_b}{2}} ~v~
  {\rm daw}\left(\sqrt{\frac{\beta_b m_b}{2}}\,v 
  \right) \right]
\\[10pt]
\label{Fiu}
  F_\smI(v)
  &=&
  \sqrt{\pi} \sum_b \kappa_b^2 
  \sqrt{\frac{\beta_b m_b}{2}}~~
  v\, \exp\left[-\frac{\beta_b m_b}
  {2}\, v^2\right] = \pi  \rho_{\rm total}(v)  \ ,
\end{eqnarray}
%% 
and the Dawson integral is defined by
%%
\begin{eqnarray}
  {\rm daw}(x) \equiv \int_0^x dy\, 
  e^{y^2 - x^2} = \frac{\sqrt{\pi}}{2}\, e^{-x^2}
  {\rm erfi}(x) \ .
\end{eqnarray}
%% 


\subsection{The Representation of BPS in Ref.~[1]}

\subsubsection{The Classical Result}

The complete energy loss to the plasma species $b$  
in the classical case can be written\cite{bps}
%%
\begin{equation}
  \frac{dE^{\smC}_b}{dx} 
 = 
 \frac{dE^\smC_{b,\smS}}{dx} 
  +
  \frac{dE^\smLT_{b,\smR}}{dx} \,,
\label{doneatlast}
\end{equation}
%%



in which the two contribution are given by\footnote{
\baselineskip 15pt
To save writing, we use $e$
to denote the absolute value of the charge of 
a particle. Thus $e_p e_b$ is always positive 
even if projectile ($p$) and plasma ($b$) 
particles have charges of opposite sign.}
:
%%
\begin{eqnarray}
  && {dE^\smC_{b,\smS} \over dx} =
  {e_p^2 \over 4\pi} \, 
	{\kappa^2_b \over m_p \, v_p} \,
  \left( { m_b \over  2\pi \beta_b  } \right)^{1/2} \,
  \int_0^1 du \, u^{1/2} \,
  \exp\left\{ - {1 \over 2} \,
  \beta_b m_b v^2_p \, u \right\}
\nonumber\\[2 pt]
  && \Bigg\{
  \left[ - \ln
  \left(\beta_b\,  { e_p e_b \, K \over 4 \pi}\, { m_b \over
  m_{pb} } \,
  { u \over 1-u} \right)  + 2 - 2\gamma \right]
  \left[  \beta_b \, M_{pb} \, v_p^2
  -  {1\over u} \right] + {2 \over u}  \Bigg\} \,,
\nonumber\\
  &&
\label{wonderclassicc} 
\end{eqnarray}
%%
and
%%
%% 

%%
\begin{eqnarray}
  \frac{d E^\smLT_{b,\smR}}{dx}  
  &=&
  \frac{e_p^2}{4 \pi } \, \frac{ i }{2 \pi }
  \int_{-1}^{+1} d\cos\theta \, \cos\theta \,
  \frac{\rho_b(v_p\cos\theta)}{\rho_{\rm total}(v_p\cos\theta) } \, 
 F(v_p \cos\theta) \ln \left( \frac{ F(v_p \cos\theta)}{K^2}\right)
\nonumber\\[11 pt]
  &-&  {e_p^2 \over 4 \pi } { i \over 2 \pi }
  { 1 \over \beta_b m_p v_p^2 } 
	{\rho_b(v_p) \over 	\rho_{\rm total}(v_p) }\Bigg[ 
	F(v_p ) \ln \left( { F(v_p ) \over K^2 } \right) 
	- 
	F^*(v_p)\ln \left( { F^*(v_p) \over K^2 } \right) \Bigg] \,.
\label{nunn} 
\end{eqnarray}

The total result does not depend upon the arbitrary
wave number $K$, and choosing $K$ to be a suitable 
multiple of the Debye wave number of the plasma often 
simplifies the final results. 


\subsubsection{Quantum Corrections}

In the previous section, our discussion was only for 
those cases in which classical physics applies.  In 
these cases, the quantum parameters 
%%
\begin{equation}
  \eta_{pb} = {e_p e_b \over 4 \pi \hbar\,v_{pb} } \,,
\end{equation}
%%
are large.  In the energy loss problem, these are the only 
independent dimensionless parameters that entail the quantum 
unit, Planck's constant $\hbar$. The parameters are large 
when the average relative velocity $v_{pb}$ is small, and as 
far as an $\eta_{pb}$ parameter is concerned this corresponds 
to the formal limit $\hbar \to 0$. We now treat
the general case where the size of the quantum parameters 
$\eta_{pb}$ has no restriction. The energy loss to the 
plasma species $b$ in the general case appears as
%%
\begin{equation}
  \frac{ dE_b }{dx} 
  = 
  \frac{ dE^\smC_b }{dx} 
  +
  \frac{ dE^\smQ_b }{dx} \,,
\end{equation}
%%
where, the quantum correction is given by\cite{bps}
%%
\begin{eqnarray}
  { d E^\smQ_b \over dx}  &=&
  { e_p^2 \over 4 \pi } \,
  { \kappa^2_b \over 2 \beta_b m_p v_p^2 }
  \left( { \beta_b m_b \over 2\pi } \right)^{1/2}
  \int_0^\infty dv_{pb} ~
  \left\{ 2\, {\rm Re} \, \psi \left( 1 + i \eta_{pb}
  \right) - \ln \eta^2_{pb}  \right\} 
\nonumber\\
  && \qquad
  \Bigg\{ \left[ 1 + {M_{pb} \over m_b} { v_p \over
  v_{pb} } \left( { 1 \over \beta_b m_b v_p v_{pb} } - 1
  \right) \right] \exp\left\{ - {1 
  \over 2} \beta_b m_b \left( v_p - v_{pb} \right)^2 
  \right\}
\nonumber\\
  && \qquad
  - \left[ 1 + {M_{pb} \over m_b} { v_p \over v_{pb} }
  \left( { 1 \over \beta_b m_b v_p v_{pb} } + 1 \right)
  \right]\exp\left\{- {1 \over 2} 
  \beta_b m_b \left( v_p +
  v_{pb} \right)^2\right\} \Bigg\} \ .
\label{quantumm}
\end{eqnarray}
%%
%% 
Here 
%%
\begin{eqnarray}
v_{pb} &=& | {\bf v}_p - {\bf v}_b| 
\\[5pt]
\eta_{pb} &=& { e_p e_b \over 4 \pi \hbar v_{pb} } \,,
\end{eqnarray}
%%
with $\psi(z)$ being the logarithmic derivative 
of the gamma function, and
%%
\begin{equation}
  {\rm Re} \, \psi(1+i\eta)  =
  \sum_{k=1}^\infty { 1 \over k}\, { \eta^2 \over k^2 +
  \eta^2 }  -\gamma \,.
\end{equation}

\subsection{A More Compact Representation of BPS}

For coding purposes I will write the classical
contribution as
\vskip0.5cm 

%
\centerline{
\frame{.1}{10}{15cm}{\noindent
\begin{eqnarray}
\label{Fdedxc}
  {dE^\smC_b \over dx}(v_p) &=&
  \frac{e_p^2\, \kappa^2_b}{4\pi} \, {m_b \over m_p} \,
  \frac{1}{\sqrt{2\pi\, \beta_b m_b v_p^2}} \,
  \int_0^1 du \, u^{1/2} \,e^{ -\beta_b m_b v^2_p \, u/2 }
  \Bigg\{ \frac{2}{u} ~+~
\nonumber\\[2 pt]
  && 
  \left[ - \ln
  \left(\beta_b\,  { e_p e_b \, K \over 4 \pi}\, 
  { m_b \over m_{pb} } \, { u \over 1-u} \right)  
  + 2 - 2\gamma \right]\left[  \beta_b \, M_{pb} \, 
  v_p^2 -  {1\over u} \right] \Bigg\} 
\nonumber\\[15pt]
  && 
  + ~ {e_p^2 \over 4 \pi } { 1 \over 4 \pi }
  \int_{-1}^{+1} du \, u \,H_b(v_p u) 
  ~-~\frac{e_p^2}{4 \pi }\, \frac{1}{2\pi}\,
  \frac{1}{\beta_b m_p v_p^2} \, H_b(v_p) \ .
\end{eqnarray}
}}



\noindent
where the $K$-dependent function is defined by

%%
\begin{eqnarray}
\nonumber
\\[-10pt]
\label{Hbdef}
  H_b(v) \equiv
  i \,\frac{\rho_b(v)}{\rho_{\rm total}(v)} 
  \Bigg[ F(v) \ln \left( { F(v) \over K^2 } \right) - 
  F^*(v)\ln \left( { F^*(v) \over K^2 } \right) \Bigg] \ .
\\[-10pt]
\nonumber
\end{eqnarray}
%%

\noindent
The quantum correction can be expressed as

\vskip0.5cm 
%
\centerline{
\frame{.1}{10}{15cm}{\noindent
\begin{eqnarray} \hskip-0.25cm 
  { d E^\smQ_b \over dx}(v_p)  &=&
  \frac{e_p^2 \kappa_b^2}{4 \pi} \,
  \frac{2}{\sqrt{2\pi\beta_b m_p v_p^2} }\, 
  e^{-\beta_b m_b v_p^2/2} \hskip-0.15cm \int_0^\infty 
  \hskip-0.4cm du e^{-\beta_b m_b v_p^2\, u^2/2}
  \bigg\{ \hskip-0.1cm \ln(\eta_b/u) 
  \hskip-0.05cm - \hskip-0.1cm 
  {\rm Re} \,\psi \hskip-0.05cm \left( \hskip-0.05cm
  1 + i \eta_b/u \right) \hskip-0.1cm  \bigg\}
\nonumber\\
  &&
  \Bigg\{ 
  {M_{pb} \over m_p} \, \frac{1}{ u} \left( 
  \cosh(\beta_b m_b v_p^2\, u) -
  \frac{\sinh(\beta_b m_b v_p^2\, u)}{\beta_b m_b v_p^2\, u}
  \right)  
  -
  \frac{m_b}{m_p}\, \sinh(\beta_b m_b v_p^2 u) \Bigg\} \ .
\nonumber
\\[20pt] &&
\label{Fdedxqm}
  \eta_b = \frac{e_p e_b}{4\pi \hbar v_p}
\end{eqnarray}
%%  
}}
\vskip0.5cm 



\noindent
To express the classical stopping power in the form 
(\ref{Fdedxc}), note that the
first term is same as (\ref{wonderclassicc}), while
the next two terms of (\ref{Fdedxc}) derive from 
(\ref{nunn}) in the following way. The last term of 
(\ref{Fdedxc}) follows trivially from (\ref{Hbdef}),
while the second term follows from the first line of 
(\ref{nunn}) with the variable replacement $u=\cos\theta$
and the reflection property (\ref{fmfstar}) of $F$. In
particular, we use

\vbox{
%%
\begin{eqnarray}
\nonumber
  -\hskip-0.2cm \int_{-1}^1 \hskip-0.15cm
  du \,u\, \frac{\rho_b(v_p u)}
  {\rho_{\rm total}(v_p u)} F^*(v_p u)
  \ln \hskip-0.1cm \left( \frac{ F^*(v_p u)}{K^2} \right) 
&=&
\nonumber
  \int_{-1}^1 \hskip-0.1cm du \,u\, \frac{\rho_b(-v_p u)}
  {\rho_{\rm total}(-v_p u)} F^*(-v_p u)
  \ln \hskip-0.1cm \left( \frac{ F^*(-v_p u)}{K^2} \right) 
\\[10pt]
&=&
  \int_{-1}^1 \hskip-0.1cm du \,u\, \frac{\rho_b(v_p u)}
  {\rho_{\rm total}(v_p u)} F(v_p u)
  \ln \hskip-0.1cm \left( \frac{ F(v_p u)}{K^2} \right) \,,
\end{eqnarray}
%%
}


\noindent
from which it follows that 
%%
\begin{eqnarray}
\nonumber
  \int_{-1}^1 du \, u \, H_b(v_p u) 
  &=&  
  i \int_{-1}^1 du \,u \, {\rho_b(v_p u) \over 
  \rho_{\rm total}(v_p u) } \Bigg[ F(v_p u ) \ln
  \left( { F(v_p u ) \over K^2 } \right) - F^*(v_p u)
  \ln \left( { F^*(v_p u) \over K^2 } \right) \Bigg] 
\\[10pt]
  &=&
  2\, i \int_{-1}^1 du \,u\, \frac{\rho_b(v_p u)}
  {\rho_{\rm total}(v_p u)}~ F(v_p u)
  \ln \left( \frac{ F(v_p u)}{K^2} \right) \ .
\end{eqnarray}
%%
The representation (\ref{Fdedxqm}) of the quantum correction 
(\ref{quantumm}) follows from 
%%
\begin{eqnarray}
  { d E^\smQ_b \over dx}  &=&
  { e_p^2 \over 4 \pi } \,
  { \kappa^2_b \over 2 \beta_b m_p v_p^2 }
  \left( { \beta_b m_b \over 2\pi } \right)^{1/2}
  \int_0^\infty dv_{pb} ~
  \left\{ 2\, {\rm Re} \, \psi \left( 1 + i \eta_{pb}
  \right) - \ln \eta^2_{pb}  \right\} 
\nonumber\\
  && \qquad
  \Bigg\{ \left[ 1 + {M_{pb} \over m_b} { v_p \over
  v_{pb} } \left( { 1 \over \beta_b m_b v_p v_{pb} } - 1
  \right) \right] \exp\left\{ - {1 
  \over 2} \beta_b m_b \left( v_p - v_{pb} \right)^2 
  \right\}
\nonumber\\
  && \qquad
  - \left[ 1 + {M_{pb} \over m_b} { v_p \over v_{pb} }
  \left( { 1 \over \beta_b m_b v_p v_{pb} } + 1 \right)
  \right]\exp\left\{- {1 \over 2} 
  \beta_b m_b \left( v_p +
  v_{pb} \right)^2\right\} \Bigg\} 
\\[15pt]
\nonumber
  &=&
  { e_p^2 \over 4 \pi } \,
  { \kappa^2_b \over \beta_b m_p v_p }
  \left( { \beta_b m_b \over 2\pi } \right)^{1/2}
  \int_0^\infty \hskip-0.2cm du ~
  \left\{ {\rm Re} \, \psi \left( 1 + i \eta_b/u
  \right) - \ln(\eta_b/u)  \right\} 
  \hskip0.5cm : v_{pb} = v_p u
\nonumber\\
  && \qquad
  \Bigg\{ \left[ 1 + {M_{pb} \over m_b} { 1 \over u } 
  \left( { 1 \over \beta_b m_b v_p^2\, u } - 1
  \right) \right] \exp\left\{ - {1 
  \over 2} \beta_b m_b v_p^2 \left( u-1 \right)^2 
  \right\}
\nonumber\\
  && \qquad
  - \left[ 1 + {M_{pb} \over m_b} { 1\over u }
  \left( { 1 \over \beta_b m_b v_p^2\,u } + 1 \right)
  \right]\exp\left\{- {1 \over 2} 
  \beta_b m_b v_p^2 \left(u + 1 \right)^2\right\} \Bigg\} 
\\[15pt]
  &=&
  { e_p^2 \over 4 \pi } \,
  \kappa^2_b  \frac{e^{-\beta_b m_b v_p^2/2} }
  {\sqrt{2\pi\beta_b m_p v_p^2} }\,\int_0^\infty 
  \hskip-0.2cm du ~e^{-\beta_b m_b v_p^2\, u^2/2}
  \left\{ {\rm Re} \, \psi \left( 1 + i \eta_b/u
  \right) - \ln(\eta_b/u)  \right\} 
\nonumber\\ \nonumber
  && \hskip-1cm 
  \Bigg\{ \left[ \frac{m_b}{m_p} + {M_{pb} \over m_p u}
  \left( { 1 \over \beta_b m_b v_p^2\, u } - 1
  \right) \right] e^{\beta_b m_b v_p^2\, u}
  - \left[ \frac{m_b}{m_p} + {M_{pb} \over m_p u} 
  \left( { 1 \over \beta_b m_b v_p^2\,u } + 1 \right)
  \right]e^{- \beta_b m_b v_p^2\, u } \Bigg\} 
\\[15pt]
  &=&
  { e_p^2 \over 4 \pi } \,
  \kappa^2_b  \frac{2\,e^{-\beta_b m_b v_p^2/2} }
  {\sqrt{2\pi\beta_b m_p v_p^2} }\,\int_0^\infty 
  \hskip-0.2cm du ~e^{-\beta_b m_b v_p^2\, u^2/2}
  \bigg\{ {\rm Re} \, \psi \left( 1 + i \eta_b/u
  \right) - \ln(\eta_b/u)  \bigg\}
\nonumber\\ \nonumber
  && 
  \Bigg\{ \frac{m_b}{m_p}\, \sinh(\beta_b m_b v_p^2 u) 
  + {M_{pb} \over m_p} \, \frac{1}{ u} \left( 
  { \sinh(\beta_b m_b v_p^2\, u) \over \beta_b m_b v_p^2\, u } 
  - \cosh(\beta_b m_b v_p^2\, u) \right) \Bigg\} \ .
\end{eqnarray}
%%


\clearpage
\section{Units}

\subsection{Thermal Velocity} 

It will be convenient to express the projectile velocity $v_p$ 
in units of the thermal velocity of a reference plasma species. 
For particles of mass $m$, we define the thermal velocity 
$v_{\rm th}$ from the equipartition theorem, 
%%
\begin{eqnarray}
  \frac{1}{2}\, m \, v_{\rm th}^2 =
  \frac{3}{2}\, T \ ,
\end{eqnarray}
%%
and therefore $v_{\rm th}=\sqrt{3/\beta m}$. 
Since there are variations on this definition
in the literature, we will define the the 
{\em dimensionless} thermal velocity $\bar v_p$ 
of the projectile as
%%
\begin{eqnarray}
\label{vpvth}
  v_p 
  &=& \bar v_p \cdot v_{\rm th} 
\\[5pt]
  v_{\rm th} 
  &\equiv& \sqrt{\frac{r}{\beta_m\, m}} \ ,
\end{eqnarray}
%%
where $m$ is the mass of a reference plasma species, $\beta_m$ 
is the corresponding temperature, and and $r$ is an arbitrary 
real number. We will usually take $m$ to be the electron mass, 
and $r=3$ (but some authors take $r=2$ or $r=1$, and for this 
reason we keep $r$ arbitrary for now). From here on we will 
also express mass in units of $m$, {\em i.e.} in terms of 
the {\em dimensionless} mass ratio $m_b^\smO = m_b/m$, or 
$m_{pb}^\smO=m_{pb}/m$, or $M_{pb}^\smO=M_{pb}/m$. Dimensionless 
combinations such as  $\beta_b m_b \, v_p^2$ will always appear 
together, and using (\ref{vpvth}) we can express these as 
$\beta_b m_b \, v_p^2 = r_b\, m_b^\smO \,\bar v_p^2$ where 
we have defined the quantity 
%%
\begin{eqnarray}
\label{rbdef}   
  r_b \equiv r \, \frac{\beta_b}{\beta_m} \ .
\end{eqnarray}
%%
The classical contribution takes the form
\begin{eqnarray}
  {dE^\smC_b \over dx}(\bar v_p) &=&
  \frac{e_p^2 \, \kappa^2_b}{4\pi} \, 
  \frac{m_b^\smO}{m_p^\smO} \,
  \frac{1}{\sqrt{2\pi\, r_b m_b^\smO \bar v_p^2}} \,
  \int_0^1 du \, u^{1/2} \,e^{ -r_b\, m_b^\smO \,\bar 
  v_p^2 \, u/2 }\Bigg\{ \frac{2}{u} ~+~
\nonumber\\[2 pt]
  && 
  \left[ - \ln
  \left(\beta_b\,  { e_p e_b \, K \over 4 \pi}\, 
  { m_b^\smO \over m_{pb}^\smO } \, { u \over 1-u} \right)  
  + 2 - 2\gamma \right]\left[  r_b \, M_{pb}^\smO \, 
  \bar v_p^2 -  {1\over u} \right] \Bigg\} 
\nonumber\\[15pt]
  && 
  + ~ {e_p^2 \over 4 \pi } { 1 \over 4 \pi }
  \int_{-1}^{+1} du \, u \,H_b(v_p u) 
  ~-~\frac{e_p^2}{4 \pi }\, \frac{1}{2\pi}\,
  \frac{1}{r_b m_p^\smO v_p^2} \, H_b(v_p) \ ,
\end{eqnarray}
%%
while the quantum correction is
%%
\begin{eqnarray}
  \hskip-0.2cm 
  { d E^\smQ_b \over dx}(\bar v_p)  &=& 
  \frac{e_p^2\,\kappa_b^2}{4 \pi} \,
  \frac{2}{\sqrt{2\pi\, r_b m_p^\smO \bar v_p^2} }\, 
  e^{-r_b m_b^\smO \bar v_p^2/2} \hskip-0.15cm 
  \int_0^\infty 
  \hskip-0.4cm du e^{-r_b m_b^\smO \bar v_p^2\, u^2/2}
  \bigg\{ \hskip-0.1cm \ln(\eta_b/u) 
  \hskip-0.05cm - \hskip-0.05cm 
  {\rm Re} \, \psi \hskip-0.05cm 
  \left( 1 + i \eta_b/u \right) 
  \hskip-0.1cm  \bigg\}
\nonumber\\
  && 
  \Bigg\{ 
  {M_{pb}^\smO \over m_p^\smO} \, \frac{1}{ u} \hskip-0.1cm
  \left( 
  \cosh(r_b m_b^\smO \bar v_p^2\, u)  - 
  \frac{\sinh(r_b m_b^\smO \bar v_p^2\, u)}{r_b m_b^\smO \bar v_p^2\, u } 
  \right)  
  -
  \frac{m_b^\smO}{m_p^\smO}\, \sinh(r_b m_b^\smO \bar 
  v_p^2\, u) \Bigg\} \ .
\nonumber
\end{eqnarray}
%%
The quantum correction is complete as it stands, but
the classical contribution requires more work to
re-express $H_b(v_p)$ as a function of the dimensionless
velocity $\bar v_p$. 

We continue by expressing $\rho_c(v)$ of (\ref{spectral}) 
in terms of the dimensionless variable $\bar v$ defined 
by $v = v_{\rm th}\, \bar v$: 
%%
\begin{eqnarray}
\label{spectralbar}
  \rho_c(v) 
  &=& 
  \kappa^2_c \,\sqrt{ \beta_c m_c v_{\rm th}^2
  \over 2\pi } ~\bar v~ e^{-\beta_c m_c v_{\rm th}^2 
  \, \bar v^2/2} 
  =
  \kappa^2_c \,\sqrt{ r_c m^\smO \over 2\pi } 
  ~\bar v~ e^{-r_c m^\smO \, \bar v^2/2} \ ,
\end{eqnarray}
%%
which allows us to define the spectral weight as

\vskip0.5cm 
\frame{.1}{1}{15cm}{\noindent
%%
\begin{eqnarray}
  \bar \rho_c(x) &=& 
  \frac{\kappa^2_c}{\sqrt{\pi}}
  ~x~ e^{-x^2}  
\\[5pt]
  \bar \rho_{\rm total}(\bar v)  &=& 
  {\sum}_b \, \bar \rho_b(x_b) 
  ~~~~~~
  x_b = \bar v \,\sqrt{r_b m^\smO/2}  \ .
\label{efrelbbar}
\end{eqnarray}
%%
}
\vskip0.4cm


\noindent
Note that $\bar \rho$ itself still has dimensions of 
$\kappa^2$. We then define 
%%
\begin{equation}
  \bar F(\bar v)  =  - \int_{-\infty}^{+\infty} 
  d\bar u \, { \bar \rho_{\rm total}(\bar u) \over 
  \bar v - \bar u + i \eta } \,,
\label{dispbarbar}
\end{equation}
where now $\bar v$, $\bar u$, and $\eta$ are dimensionless, 
and the real and imaginary parts become
%%

\vskip0.5cm 
\frame{.1}{1}{15cm}{\noindent
%%
\begin{mathletters}
\begin{eqnarray}
\label{Frubar}
  \bar F_\smR(\bar v)
  &=& 
  \sum_b \kappa_b^2 \bigg[1 - 2 x_b\,
  {\rm daw}\left(x_b \right) \bigg]
  ~~~~~~
  x_b = \bar v \,\sqrt{r_b m^\smO/2} 
\\[10pt]
\label{Fiubar}
  \bar F_\smI(\bar v)
  &=&
  \sqrt{\pi} \sum_b \kappa_b^2\, x_b\, e^{-x_b^2} 
  = \pi  \bar \rho_{\rm total}(\bar v)  \ .
\end{eqnarray}
\end{mathletters}
%% 
}
\vskip0.3cm

\noindent
Note that $F(v) = F(v_{\rm th} \bar v) =\bar F (\bar v)$ 
and that $\rho_c(v) = \rho_c(v_{\rm th} \bar v) =\bar
\rho_c(\bar v)$. This prompts us to define 

\vskip0.5cm 
\frame{.1}{1}{15cm}{\noindent
%%
\begin{eqnarray}
\nonumber
\\[-10pt]
  \bar H_b(\bar v) \equiv
  i \,{\bar \rho_b(\bar v) \over \bar \rho_{\rm total}(\bar v) } 
  \Bigg[ \bar F(\bar v) \ln\left( {\bar F(\bar v) \over K^2 } 
  \right) - \bar F^*(\bar v)\ln \left( {\bar F^*(\bar v) 
  \over K^2 } \right) \Bigg] \ ,
\\[-10pt]
\nonumber
\end{eqnarray}
%%
}
\vskip0.3cm

\noindent
so that $H_b(v) = H_b(v_{\rm th} \bar v) =\bar H_b (\bar v)$.
In particular the last two term of the classical piece contain 
%%
\begin{eqnarray}
  H_b(v_p u) = H_b(v_{\rm th} \bar v_p\, u) =
  \bar H_b\hskip-0.1cm \left(\bar v_p u\right) \ , 
\end{eqnarray}
%%
which allows us to write the expression completely
in terms of $\bar H_b(\bar v)$. 
%%
\begin{eqnarray}
  {dE^\smC_b \over dx}(\bar v_p) &=&
  \frac{e_p^2 \, \kappa^2_b}{4\pi} \, 
  \frac{m_b^\smO}{m_p^\smO} \,
  \frac{1}{\sqrt{2\pi\, r_b m_b^\smO \bar v_p^2}} \,
  \int_0^1 du \, u^{1/2} \,e^{ -r_b\, m_b^\smO \,\bar 
  v_p^2 \, u/2 }\Bigg\{ \frac{2}{u} ~+~
\nonumber\\[2 pt]
  && 
  \left[ - \ln
  \left(\beta_b\,  { e_p e_b \, K \over 4 \pi}\, 
  { m_b^\smO \over m_{pb}^\smO } \, { u \over 1-u} \right)  
  + 2 - 2\gamma \right]\left[  r_b \, M_{pb}^\smO \, 
  \bar v_p^2 -  {1\over u} \right] \Bigg\} 
\nonumber\\[15pt]
  && \hskip-0.5cm 
  + ~ {e_p^2 \over 4 \pi } { 1 \over 4 \pi }\int_{-1}^{+1} 
  du \, u \,\bar H_b\left(\bar v_p u\right) ~-~
  \frac{e_p^2}{4 \pi }\, \frac{1}{2\pi}\,
  \frac{1}{r_b m_p^\smO \bar v_p^2} \, 
  \bar H_b\left(\bar v_p\right) \ .
\end{eqnarray}
%%
In summary, the classical contribution and the
quantum correction are

%
\frame{.1}{15}{16cm}{\noindent
%%
\begin{eqnarray}
\hskip-1cm 
  {dE^\smC_b \over dx}(\bar v_p) &=&
  \frac{e_p^2 \, \kappa^2_b}{4\pi} \, 
  \frac{m_b^\smO}{m_p^\smO} \,
  \frac{1}{\sqrt{2\pi\, r_b m_b^\smO \bar v_p^2}} \,
  \int_0^1 du \, u^{1/2} \,e^{ -r_b\, m_b^\smO \,\bar 
  v_p^2 \, u/2 }\Bigg\{ \frac{2}{u} ~+~
\nonumber\\[2 pt]
  && 
  \left[ - \ln
  \left(\beta_b\,  { e_p e_b \, K \over 4 \pi}\, 
  { m_b^\smO \over m_{pb}^\smO } \, { u \over 1-u} \right)  
  + 2 - 2\gamma \right]\left[  r_b \, M_{pb}^\smO \, 
  \bar v_p^2 -  {1\over u} \right] \Bigg\} 
\nonumber\\[15pt]
  && \hskip-0.7cm 
  + ~ {e_p^2 \over 4 \pi } { 1 \over 4 \pi }
  \int_{-1}^{+1} du \, u \,\bar H_b\left(\bar v_p u\right)
  ~-~\frac{e_p^2}{4 \pi }\, \frac{1}{2\pi}\,
  \frac{1}{r_b m_p^\smO \bar v_p^2} \, 
  \bar H_b\left(\bar v_p\right)
\\[30pt]
  { d E^\smQ_b \over dx}(\bar v_p)  &=& 
  \frac{e_p^2\,\kappa_b^2}{4 \pi} \,
  \frac{2}{\sqrt{2\pi\, r_b m_p^\smO \bar v_p^2} }\, 
  e^{-r_b m_b^\smO \bar v_p^2/2} \hskip-0.15cm 
  \int_0^\infty 
  \hskip-0.4cm du e^{-r_b m_b^\smO \bar v_p^2\, u^2/2}
  \Bigg\{ \hskip-0.1cm \ln\left(\frac{\bar\eta_b}{\bar 
  v_p u}\right) 
  \hskip-0.05cm - \hskip-0.05cm 
  {\rm Re} \, \psi \hskip-0.05cm 
  \left( 1 + i \frac{\bar\eta_b}{\bar v_p u} \right) 
  \hskip-0.1cm  \Bigg\}
\nonumber\\
  && 
  \Bigg\{ 
  {M_{pb}^\smO \over m_p^\smO} \, \frac{1}{ u} \hskip-0.1cm
  \left( 
  \cosh(r_b m_b^\smO \bar v_p^2\, u)  -
  \frac{\sinh(r_b m_b^\smO \bar v_p^2\, u)}{r_b m_b^\smO \bar v_p^2\,u} 
  \right)  
  -
  \frac{m_b^\smO}{m_p^\smO}\, \sinh(r_b m_b^\smO \bar 
  v_p^2\, u) \Bigg\} \ .
\nonumber
\\[10pt] &&
  \bar \eta_b = \frac{e_p e_b}{4\pi \hbar\, v_{\rm th}}
\end{eqnarray}
}
\vskip0.5cm 

\noindent
It is convenient to change $u$-integration variables 
in the quantum term to $\bar u = \bar v_p u$:

%%
\begin{eqnarray}
  { d E^\smQ_b \over dx}(\bar v_p)  
  &=& 
  \frac{e_p^2\,\kappa_b^2}{4 \pi} \,
  \frac{2}{\sqrt{2\pi\, r_b m_p^\smO \bar v_p^2} }\, 
  e^{-r_b m_b^\smO \bar v_p^2/2} \hskip-0.15cm 
  \int_0^\infty 
  \hskip-0.4cm du e^{-r_b m_b^\smO \bar v_p^2\, u^2/2}
  \Bigg\{ \hskip-0.1cm \ln\left(\frac{\bar\eta_b}{\bar 
  v_p u}\right) 
  \hskip-0.05cm - \hskip-0.05cm 
  {\rm Re} \, \psi \hskip-0.05cm 
  \left( 1 + i \frac{\bar\eta_b}{\bar v_p u} \right) 
  \hskip-0.1cm  \Bigg\}
\nonumber\\\nonumber
  && 
  \Bigg\{ 
  {M_{pb}^\smO \over m_p^\smO} \, \frac{1}{ u} \hskip-0.1cm
  \left( 
  \cosh(r_b m_b^\smO \bar v_p^2\, u)  -
  { \sinh(r_b m_b^\smO \bar v_p^2\, u) \over r_b m_b^\smO 
  \bar v_p^2\, u } 
  \right)  
  -
  \frac{m_b^\smO}{m_p^\smO}\, \sinh(r_b m_b^\smO \bar 
  v_p^2\, u) \Bigg\} 
\\[30pt]
  &=&
  \frac{e_p^2\,\kappa_b^2}{4 \pi} \,
  \frac{2}{\sqrt{2\pi\, r_b m_p^\smO \bar v_p^2} }\, 
  e^{-r_b m_b^\smO \bar v_p^2/2} \hskip-0.15cm 
  \int_0^\infty 
  \hskip-0.1cm \frac{d\bar u}{\bar v_p}\, e^{-r_b m_b^\smO \bar u^2/2}
  \Bigg\{ \hskip-0.1cm \ln\left(\frac{\bar\eta_b}{\bar u}\right) 
  \hskip-0.05cm - \hskip-0.05cm {\rm Re} \, \psi \hskip-0.05cm 
  \left( 1 + i \frac{\bar\eta_b}{\bar u} \right) \hskip-0.1cm  \Bigg\}
\nonumber\\\nonumber
  && 
  \Bigg\{ 
  {M_{pb}^\smO \over m_p^\smO} \, \frac{\bar v_p}{\bar u} \hskip-0.1cm
  \left( 
  \cosh(r_b m_b^\smO \bar v_p\, \bar u)  -
  { \sinh(r_b m_b^\smO \bar v_p\, \bar u) \over r_b m_b^\smO 
  \bar v_p\, \bar u } 
  \right)  
  -
  \frac{m_b^\smO}{m_p^\smO}\, \sinh(r_b m_b^\smO \bar 
  v_p\,\bar u) \Bigg\} 
\\[30pt]
  &=&
  \frac{e_p^2\,\kappa_b^2}{4 \pi} \,
  \frac{2}{\sqrt{2\pi\, r_b m_p^\smO \bar v_p^2} }\, 
  e^{-r_b m_b^\smO \bar v_p^2/2} \hskip-0.15cm 
  \int_0^\infty 
  \hskip-0.1cm d u \, e^{-r_b m_b^\smO u^2/2}
  \Bigg\{ \hskip-0.1cm \ln\left(\frac{\bar\eta_b}{u}\right) 
  \hskip-0.05cm - \hskip-0.05cm {\rm Re} \, \psi \hskip-0.05cm 
  \left( 1 + i \frac{\bar\eta_b}{u} \right) \hskip-0.1cm  \Bigg\}
\nonumber\\\nonumber
  && 
  \Bigg\{ 
  {M_{pb}^\smO \over m_p^\smO} \, \frac{1}{u} \hskip-0.1cm
  \left( 
  \cosh(r_b m_b^\smO \bar v_p\, u)  - 
  { \sinh(r_b m_b^\smO \bar v_p\, u) \over r_b m_b^\smO 
  \bar v_p\, u } 
  \right)  
  -
  \frac{m_b^\smO}{m_p^\smO}\, \frac{\sinh(r_b m_b^\smO \bar v_p\,u)}
  {\bar v_p} \Bigg\}  \ .
\end{eqnarray}


\subsection{Atomic Units}

Distances will be measured in units of the Bohr Radius,
%%
\begin{eqnarray}
\label{bohrradiusdef}
  a_0 = \frac{4\pi \hbar^2}{m_e e^2}
  = 5.3 \times 10^{-9}\,{\rm cm} \ ,
\end{eqnarray}
%%
and temperature in units of ${\rm eV}$:
%%
\begin{mathletters}
\label{azeroquantities}  
\begin{eqnarray}
  K &=& K^\smO\, a_o^{-1}
\\
  n_b &=& n_b^\smO\, a_0^{-3} 
\\
  \beta_b &=& \beta_b^\smO\, {\rm eV}^{-1}
\\
  m c^2 &=& m^\smO \, {\rm eV}
\\
  m_b^\smO &=& m_b/m \ , ~~ m_p^\smO = m_p/m \ ,
  ~~{\rm and} ~~ M_{pb}^\smO = M_{pb}/m
\\
  e_b &=& Z_b\, e ~~~{\rm and}~~
  e_p  = Z_p\, e \ .
\end{eqnarray}
\end{mathletters}
%%
Any quantity with a superscript zero is
dimensionless. The binding energy of the 
electron in a Hydrogen atom is $B_e \,
{\rm eV} = 13.6\,{\rm eV}$, and upon 
using the expression
%%
\begin{eqnarray}
\label{eazeroone}
  \frac{e^2}{4\pi a_0} = 2B_e\,{\rm eV} 
  = 27\,{\rm eV} \ ,
\end{eqnarray}
%%
or
%%
\begin{eqnarray}
\label{eazerotwo} 
  e^2 = 8\pi B_e \cdot {\rm eV} a_0 \ .
\end{eqnarray}
%%
The inverse-squared Debye length takes the form
%%
\begin{eqnarray}
\nonumber
  \kappa_b^2 
  &=& \beta_b n_b\, e_b^2
  = 
  \bigg(\beta_b^\smO\,{\rm eV}^{-1}\bigg)
  \bigg(n_b^\smO\, a_0^{-3}\bigg)
  \bigg(Z_b^2\, 8 \pi B_e\cdot {\rm eV} a_0 \bigg)
\\[5pt]
  &=&
  8 \pi B_e \,
  \beta_b^\smO Z_b^2\,n_b^\smO \,\cdot a_0^{-2} \ ,
\end{eqnarray}
%%
and although we will not use it as such, the combination 
$e^2 \kappa_b^2$ can be written
%%
\begin{eqnarray}
  e_p^2\, \kappa_b^2 
  = 
  64 \pi^2 B_e^2 \,
  \beta_b^\smO Z_p^2 Z_b^2\,n_b^\smO \,\cdot {\rm eV}/a_0 \ .
\end{eqnarray}
%%
The dimensionless, quantity under the log takes the form
%%
\begin{eqnarray}
  \beta_b e_p e_b K/4\pi
  &=& (\beta_b^\smO \, {\rm eV}^{-1}) \cdot (Z_p Z_b  
  8\pi B_e \,{\rm eV} a_0 ) \cdot (K^\smO a_0^{-1})/4\pi
\\
  &=&
  2 B_e \,\beta_b^\smO Z_p Z_b \, K^\smO \ .
\end{eqnarray}
%%

We define the {\em dimensionless} thermal velocity $\bar 
v_{\rm th}$ as the thermal velocity in units of the speed 
of light, {\em i.e.} by  $v_{\rm th}=c\, \bar v_{\rm th}$. 
Since we have defined $m c^2 =m^\smO \, {\rm eV}$ and 
$\beta_m = \beta_m^\smO\, {\rm eV}^{-1}$, the thermal velocity 
can be expressed as $v_{\rm th}= \,c~\sqrt{r/\beta_m^\smO\, 
m^\smO}$:
%%

\vskip-0.25cm 
\vbox{
\begin{eqnarray}
  v_{\rm th} &=& c \,  \bar v_{\rm th} 
  ~~~~{\rm where}~~ c=2.998 \hskip-0.05cm \times
   \hskip-0.05cm 10^{10}\,{\rm cm/s} \ , ~~~{\rm and}
\\[10pt]
  \bar v_{\rm th} 
  &=& \,\sqrt{\frac{r}{\beta_m^\smO\, m^\smO}} \ .
\end{eqnarray}
}
%%

\vskip-0.4cm 
\noindent
It will be convenient to define the quantum parameter
\hbox{$\bar\eta_b \equiv e_p e_b/ 4\pi \hbar v_{\rm th}$} 
set by the thermal speed, so that 
%%

\vskip-0.5cm 
\vbox{
\begin{eqnarray}
  \bar\eta_b  = \left( \frac{Z_p Z_b}
  {4\pi\hbar c \, \bar v_{\rm th}}
  \right) \bigg(8\pi\, B_e \cdot {\rm eV}
  a_0\bigg) = 
  \frac{2 B_e\, Z_p Z_b}{\bar v_{\rm th}}~ 
  \frac{{\rm eV} a_0}{\hbar c} ~~~ ;
\end{eqnarray}
%%
}

\vskip-0.4cm 
\noindent
however, $\hbar c = 197.3\, {\rm eV}{\rm
nm} = 3.723 \times 10^3\,{\rm eV}a_0$
and therefore

%%
\vskip0.2cm 
\frame{.1}{10}{17cm}{\noindent
\begin{eqnarray}
  \bar\eta_b  = \frac{e_p e_b}{4\pi \hbar v_{\rm 
  th}} 
  = \frac{2 B_e\, Z_p Z_b}{\bar v_{\rm th}}~
  {\scriptscriptstyle \times}\,
  2.686 \times 10^{-4} \ .
\end{eqnarray}
}
%%

\noindent
With the following dimensionless parameters, 
%%

\vskip0.5cm 
\frame{.1}{1}{17cm}{\noindent
\begin{mathletters}
\label{aefbcdefs}
\begin{eqnarray}
  A_b
  &\equiv&
  \sqrt{\frac{r_b m_b^\smO}{2}}
  ~~~~~~
  E_b
  \equiv  
  \frac{m_b^\smO}{m_p^\smO} ~
  \frac{1}{\sqrt{2\pi r_b m_b^\smO}}
  ~~~~~~
  F_b \equiv 
  \frac{2}{\sqrt{2\pi r_b m_b^\smO}}
  ~~~~~~
  B_b
  \equiv
  r_b M_{pb}^\smO
\\[5pt]
  C_b
  &\equiv& 
  2 - 2\gamma - \ln\hskip-0.1cm \bigg(
  2 B_e\, \beta_b^\smO Z_p Z_b K^\smO 
  m_b^\smO/m_{pb}^\smO\bigg)
  ~~~~ {\rm with~~} r_b \equiv 
  r\, \beta_b^\smO/\beta_m^\smO \ ,
\end{eqnarray}
\end{mathletters}
}
%%

\noindent
the stopping power takes the form

%%
\vskip0.2cm 
\frame{.1}{1}{17cm}{\noindent
\begin{eqnarray}
\hskip-1cm 
  {dE^\smC_b \over dx} &=&
  \frac{e_p^2 \, \kappa^2_b}{4\pi} \, 
   \frac{E_b}{\bar v_p} 
  \int_0^1 du \, u^{1/2} \,e^{ -(A_b \,\bar v_p)^2 \,u }
  \Bigg\{ \frac{2}{u} +
  \left[ C_b - \ln \left(  { u \over 1-u} \right)  
  \right]\left[  B_b \, \bar v_p^2 -  
  {1\over u} \right] \Bigg\} 
\nonumber\\[15pt]
  && 
  + ~ {e_p^2 \over 16 \pi^2 }
  \int_{-1}^{+1} du \, u \,\bar H_b\left(\bar v_p \, u\right)
  ~-~\frac{e_p^2}{8 \pi^2 m_p^\smO }\, 
  \frac{1}{r_b \bar v_p^2} \, \bar H_b\left(\bar v_p \right)
\\[30pt]
  { d E^\smQ_b \over dx}  &=& 
  \frac{e_p^2\,\kappa_b^2}{4 \pi} \,\frac{F_b}{\bar v_p} \,
  e^{-(A_b \, \bar v_p)^2} \hskip-0.15cm \int_0^\infty 
  \hskip-0.4cm du e^{-A_b^2 \, u^2}
  \Bigg\{ \hskip-0.1cm \ln\left(\frac{\bar\eta_b}{u}\right) 
  \hskip-0.05cm - \hskip-0.05cm 
  {\rm Re} \, \psi \hskip-0.05cm 
  \left( 1 + i \frac{\bar\eta_b}{u} \right) 
  \hskip-0.1cm  \Bigg\}
\nonumber\\
  && 
  \Bigg\{ {M_{pb}^\smO \over m_p^\smO} \, \frac{1}{ u} 
  \hskip-0.1cm
  \left( 
  \cosh(2 A_b^2 \bar v_p u) -
  { \sinh(2 A_b^2 \bar v_p u) \over 2 A_b^2 \bar v_p u }
  \right)  -
  \frac{m_b^\smO}{m_p^\smO}\, \frac{\sinh(2 A_b^2 \, 
  \bar v_p u)}{\bar v_p}\,\Bigg\} \ .
\end{eqnarray}
%%
}



\subsection{Break into Smaller Pieces for Coding}

We will break the stopping power into small component
functions and handle each separately in our coding.

\vskip0.5cm 
\frame{.1}{1}{17cm}{\noindent
%%
\begin{eqnarray}
  \frac{dE^\smC_b}{dx}
  &=&
  \frac{c_1}{\bar v_p}\,\kappa^2_b 
  E_b ~ I_1(A_b \bar v_p,\, B_b \bar v_p^2, \, C_b) +
  c_2\, I_2(\bar v_p)
  - \frac{c_3}{r_b \bar v_p^2} \, \bar H_b(\bar v_p)
\\[10pt]
  \frac{ d E^\smQ_b}{dx} 
  &=&
  \frac{c_1}{\bar v_p}\,
  \kappa_b^2 \, F_b \, e^{-(A_b \, \bar v_p)^2} ~  
  I_\smQM(A_b^2, \, \bar\eta_b, \, \bar v_p)
\end{eqnarray}
%%
}



\vskip0.3cm
\noindent
where we define the functions of dimensionless arguments

\vskip0.5cm 
\frame{.1}{1}{17cm}{\noindent
%%
\begin{eqnarray}
\hskip-1cm 
  I_1(a,b,c) &=&
  \int_0^1 du \,e^{ -a^2 u }
  \Bigg\{ \frac{2}{\sqrt{u}} + \left[ c - \ln \left(  
  { u \over 1-u} \right)  \right]\left[  b\,\sqrt{u} 
  -  \frac{1}{\sqrt{u}} \right] \Bigg\} 
\nonumber\\[10pt]
  I_2(a) &=&
  \int_{-1}^{+1} du \, u \,\bar H_b(a \,  u) 
\\[10pt]
  I_\smQM(a,e,v)    &=& 
  \int_0^\infty du e^{-a u^2}\Bigg\{\ln\left(
  \frac{e}{u}\right) - {\rm Re} \, \psi \hskip-0.05cm 
  \left( 1 + i \frac{e}{u} \right) \Bigg\}
\nonumber\\[5pt]
  && 
  \Bigg\{ 
  {M_{pb}^\smO \over m_p^\smO} \, \frac{1}{u} 
  \left(\cosh(2a v\,u) - 
  {\sinh(2 a v\, u) \over 2 a v\, u} \right)  -
  \frac{m_b^\smO}{m_p^\smO}\, \frac{\sinh(2 a v\, u)}{v} 
  \Bigg\} \ .
\end{eqnarray}
%%
}

\vskip0.3cm
\noindent
and the constants
%%
\begin{eqnarray}
\label{cdefs}
  c_1  &=& \frac{e_p^2}{4\pi} =
  2 B_e Z_p^2 \cdot {\rm eV} a_0
\\[5pt]
  c_2  &=& \frac{e_p^2}{16\pi^2} =
  \frac{B_e Z_p^2}{2\pi} \cdot 
  {\rm eV}\,a_0
\\[5pt]
  c_3  &=& \frac{e_p^2}{8\pi^2}\,
  \frac{1}{m_p^\smO} =
  \frac{B_e Z_p^2}{\pi m_p^\smO} 
  \cdot {\rm eV}\,a_0 \ .
\end{eqnarray}
%%


\newpage
\section{Asymptotic Limits}

The only limit employed in the code so far is
the small velocity limit for the classical term.
See dedxAsymptotic1.x.tex for the other limits.

\subsection{Classical}

\subsubsection{Small Velocity Limit}

In this section we seek the small velocity limit 
$v_p \to 0$ of the classical contribution\footnote{
\tighten
Incidentally, large velocities are inconsistent  with 
the purely classical limit, so we can consider $v_p \to 
\infty$ only when we take quantum corrections into account.
That is to say, while we may formally take the large 
velocity limit of the function $dE_b^\smC/dx$, it alone
has no physical validity unless we also add to this term 
to the large velocity limit of the quantum piece $dE_b^\smQ/dx$.
}
$dE_b^\smC/dx$. 
As illustrated in Eq.~(\ref{doneatlast}), the classical 
contribution $dE_b^\smC/dx$ is given by the sum of two
terms, Eqs.~(\ref{wonderclassicc}) and (\ref{nunn}). 
The term (\ref{wonderclassicc}) becomes
%%
\begin{eqnarray}
  v_p &\to& 0 ~~{\rm or}~~ \beta_b m_b v_p^2/2 \ll 1\, :
\nonumber\\[3pt]
{d E^\smC_{b,\smS} \over dx}   
&=&
  {e_p^2 \over 4\pi} \, 
  { \kappa_b^2 \over m_p \, v_p} \, 
  \left( { m_b \over  2\pi \beta_b } \right)^{1/2} 
  \Bigg\{
  \left[  \ln\left(\beta_b  { e_p e_b \over 16 \pi} K
  { m_b \over m_{pb} } \right)  +  2\gamma \right]
  \left[2-  \beta_b v_p^2 \left( {2\over3} m_p + m_b  \right) \,
  \right]
\nonumber\\[5pt]
  && \hskip5cm 
  - \frac{2}{3}\,\beta_b m_b v_p^2 \Bigg\} ~+~
  {\cal O}(v_p^2) \ ,
\label{wonderclassiclim}
\end{eqnarray}
%%
where we have used the following elementary $u$ integrals
$$
  \int_0^1 du \, u^{1/2} = {2\over3} \,, \qquad
  \int_0^1 du \, u^{1/2} \ln\left( { 1 -u \over u} \right) = {4\over3} 
  \left( \ln 2 - 1 \right) \,,
$$
$$
  \int_0^1 du \, u^{-1/2} = 2 \,, \qquad\qquad
  \int_0^1 du \, u^{-1/2} \ln\left( { 1 -u \over u} \right) = 4 \ln 2  \,.
$$



To obtain the small velocity behavior of 
Eq.~(\ref{nunn}), we first add and
subtract $v_p\cos\theta/v$ in the numerator of 
the integrand of (\ref{disp}) to get
%%
\begin{equation}
  F(v_p  \cos\theta)  = \kappa^2_\smD  - 
  {\sum}_c \, \kappa^2_c \, 
  \int_{-\infty}^{+\infty} dv \, { v_p 
  \cos\theta  \over v_p \cos\theta + i 
  \eta - v}\sqrt{ \beta_c m_c \over 2\pi } 
  \exp\left\{ - { 1 \over 2} \beta_c m_c v^2
  \right\} \,,
\label{structt}
\end{equation}
%%
where
%%
\begin{equation}
  \kappa^2_\smD = {\sum}_c \, \kappa_c^2
\end{equation}
%%
is the total squared Debye wave number of 
the plasma. We now make use of the relation
%%
\begin{equation}
  {1 \over v_p \cos\theta -v + i \eta} =
  - i \pi \delta(v_p \cos\theta - v ) +
  {\cal P} { 1 \over v_p \cos\theta - v } \,,
\end{equation}
%%
in which ${\cal P}$ denotes the principal part
prescription. Since ${\cal P} (1/x)$ defines 
an odd function, the translation $ u = v - 
v_p \cos\theta$ of the integration variable
gives
%%
\begin{eqnarray}
  F(v_p  \cos\theta)  &=& \kappa^2_\smD  + 
  \pi \, i \, {\sum}_c \, \rho_c(v_p \cos\theta) 
\nonumber\\
  &&
  - 2 \, v_p \cos\theta \, {\sum}_c \, 
  \kappa^2_c\, \sqrt{ \beta_c m_c \over 2\pi } 
  \exp\left\{ - {1 \over 2} \beta_c m_c \, (v_p 
  \cos\theta)^2\right\}
\nonumber\\
  && \quad \times
  \int_0^{\infty}  { du \over u} \, \sinh 
  \left( \beta_c m_c \, u \, v_p \cos\theta 
  \right) \, \exp\left\{-{1 \over 2} \beta_c 
  m_c \, u^2 \right\} \,.
\label{structtt}
\end{eqnarray}
%%
In this form the small $v_p$ limit is reduced 
to the evaluation of elementary Gaussian integrals 
and we have
%%
\begin{eqnarray}
  v_p \to 0 \,: &&
\nonumber\\
  F(v_p  \cos\theta)  &=& \kappa^2_\smD -
  {\sum}_c \, \kappa^2_c \, \beta_c \, m_c \,
   v_p^2 \cos^2\theta + O(v_p^4)  
+ \pi i \, \rho_{\rm total}(v_p\cos\theta) \,,
\nonumber\\
\label{smstruct}
\end{eqnarray}
%%
where we note that $\rho_{\rm total}(u)$ starts 
out at order $u$. 
Placing this result in Eq.~(\ref{nunn})  produces
%%
\begin{eqnarray}
  v_p &\to& 0 \,:
\nonumber\\
  {d E^\smLT_{b,\smR} \over dx}  &=&
 - {e_p^2 \over 4 \pi } \, \kappa_b^2
   \left( {\beta_b m_b \over 2 \pi  } \right)^{1/2} 
\, v_p \, {2\over3} \, \left[ 
\ln\left({\kappa_\smD \over K} \right) + {1 \over 2}
  \right]
\nonumber\\
  &&
    +  {e_p^2 \over 4 \pi } \,  
   {\kappa_b^2 \over m_p v_p}
 \left( { m_b \over 2 \pi \beta_b} \right)^{1/2} 
\Bigg\{ \left[ 1 - {1\over2} \beta_b m_b v_p^2 \right]
\left[2\ln \left({\kappa_\smD \over K}\right) + 1 \right]
\nonumber\\
&& \qquad
   - {\sum}_c  \, { \kappa_c^2 \over
  \kappa_\smD^2 } \, \beta_c \,  m_c \, v_p^2  
 + { \pi \over 12} \, v_p^2 \, \left[ {\sum}_c \, 
	{\kappa^2_c \over \kappa_\smD^2} 
\left(\beta_c  m_c \right)^{1/2} \right]^2 \Bigg\} \,.
\label{lessreglim}
\end{eqnarray}
%%
The result (\ref{wonderclassiclim}) added to the small 
velocity limit (\ref{lessreglim})
produces


\vskip0.5cm 
\centerline{
\frame{.1}{10}{18cm}{\noindent
%%
\begin{eqnarray}
  v_p &\to& 0 ~~{\rm or}~~ \beta_b m_b v_p^2/2 \ll 1\, :
\nonumber\\[3pt]
  {d E^\smC_b \over dx}  &=& 
 {e_p^2 \over 4\pi} \, \frac{\kappa_b^2}{m_p v_p}\,
  \left({m_b \over  2\pi \beta_b } \right)^{1/2}
  \Bigg\{
  \left[  \ln\left(\beta_b  { e_p e_b \over 16 \pi} \,
  \kappa_\smD { m_b \over m_{pb} } \right) + {1\over2} + 
  2\gamma \right] \left[2 - \beta_b v_p^2\left(
  \frac{2}{3}\, m_p + m_b\right)
  \right]
\nonumber\\[5pt]
  && - {2\over3} \, \beta_b m_b v_p^2 -
     \, {\sum}_c  \, { \kappa_c^2 \over
  \kappa_\smD^2 } \, \beta_c m_c \, v_p^2    
 + { \pi \over 12} \, v_p^2 \, \left[ {\sum}_c \, 
	{\kappa^2_c \over \kappa_\smD^2} 
 \left(\beta_c  m_c \right)^{1/2} \right]^2 \Bigg\} 
 ~+~ {\cal O}(v_p^2) \,.
\label{lessreglimm}
\end{eqnarray}
%%
}}

\vskip0.5cm
\noindent
Let us now express this result in code variables.
%%
\begin{eqnarray}
  v_p &\to& 0 ~~{\rm or}~~ \beta_b m_b v_p^2/2 \ll 1\,
  ~~{\rm or}~~ r_b m_b^\smO \bar v_p^2/2 \ll 1 :
\nonumber\\[3pt]
  {d E^\smC_b \over dx}  
  &=& 
  {e_p^2 \kappa_b^2 \over 4\pi} \, \frac{m_b}{m_p}\,
  {1 \over  \sqrt{2\pi \beta_b m_b v_p^2} }
  \Bigg\{
  \left[  \ln\left(\beta_b  { e_p e_b \over 16 \pi} \,
  \kappa_\smD { m_b \over m_{pb} } \right) + {1\over2} + 
  2\gamma \right] \left[2 - \beta_b m_b v_p^2\left(
  1 + \frac{2}{3}\, \frac{m_p}{m_b}\right)
  \right]
\nonumber\\[5pt]
  && - {2\over3} \, \beta_b m_b v_p^2 -
  \, {\sum}_c  \, { \kappa_c^2 \over
  \kappa_\smD^2 } \, \beta_c m_c \, v_p^2    
  + { \pi \over 12} \, \left[ {\sum}_c \, 
  {\kappa^2_c \over \kappa_\smD^2} 
 \left(\beta_c  m_c v_p^2\right)^{1/2} 
 \right]^2 \Bigg\} 
\\[15pt]
\label{clsmallvpone}
  &=&
  {e_p^2 \kappa_b^2 \over 4\pi} \, 
  \frac{m_b^\smO}{m_p^\smO}\,
  {1 \over  \sqrt{2\pi r_b m_b^\smO  \bar v_p^2} }
  \Bigg\{
  \left[  \ln\left(\beta_b  { e_p e_b \over 16 \pi} \,
  \kappa_\smD \frac{m_b}{m_{pb}} \right) + 
  \frac{1}{2} + 
  2\gamma \right] \left[2 - r_b m_b^\smO  \bar 
  v_p^2\left(
  1 + \frac{2}{3}\, \frac{m_p^\smO}{m_b^\smO}\right)
  \right]
\nonumber\\[5pt]
  && - {2\over3} \, r_b m_b^\smO \bar v_p^2 -
  \, {\sum}_c  \, { \kappa_c^2 \over
  \kappa_\smD^2 } \, r_c m_c^\smO \, \bar v_p^2    
  + { \pi \over 12} \, \left[ {\sum}_c \, 
  {\kappa^2_c \over \kappa_\smD^2} 
 \left(r _c  m_c^\smO \bar v_p^2\right)^{1/2} 
 \right]^2 \Bigg\} \ ,
\end{eqnarray}
%%
where we have written the expression (i) in dimensionless
units $\bar v_p$ of the thermal velocity $v_{\rm th}$
defined in (\ref{vpvth}), (ii) in dimensionless inverse
temperature units defined in (\ref{rbdef}), and (iii) in
terms of dimensionless masses $m_b^\smO$ defined 
in (\ref{azeroquantities}). We can write this expression
in terms of the variables $A_b$ and $E_b$ defined in 
(\ref{aefbcdefs}), which we repeat here for convenience, 
and a new variable $G_b$:
%%
\begin{eqnarray}
  A_b
  &\equiv&
  \sqrt{\frac{r_b m_b^\smO}{2}}
  ~~~~~~
  E_b
  \equiv  
  \frac{m_b^\smO}{m_p^\smO} ~
  \frac{1}{\sqrt{2\pi r_b m_b^\smO}}
\\[5pt]
  G_b
  &\equiv& 
  \frac{1}{2} + 2\gamma + \ln\hskip-0.1cm \bigg(
  2 B_e\, \beta_b^\smO Z_p Z_b \kappa_\smD^\smO 
  m_b^\smO/m_{pb}^\smO\bigg)
  =
  \frac{1}{2} + 2\gamma + \ln\hskip-0.1cm \bigg(
  2 B_e\, \beta_b^\smO Z_p Z_b \kappa_\smD^\smO 
  m_b^\smO/m_{pb}^\smO\bigg) \ .
\end{eqnarray}
%%
In expressing the final form of $G_b$, we have used 
Eqs.~(\ref{bohrradiusdef}), (\ref{azeroquantities}),
(\ref{eazeroone}), and (\ref{eazerotwo}). We also
define $c_1 = e_p^2/4\pi$, as in (\ref{cdefs}), and
write (\ref{clsmallvpone}) as 

\vskip0.5cm 
\centerline{
\frame{.1}{10}{15cm}{\noindent
%%
\begin{eqnarray}
  v_p &\to& 0 ~~{\rm or}~~ (A_b \bar v_p)^2 \ll 1\, :
\nonumber\\[3pt]
  {d E^\smC_b \over dx}  
  &=&
  \frac{c_1}{\bar v_p}\, \kappa_b^2 \, 
  E_b 
  \Bigg\{ 2 G_b \left[1 - A_b^2  \bar v_p^2\left(
  1 + \frac{2}{3}\, \frac{m_p^\smO}{m_b^\smO}\right)
  \right]
\nonumber\\[5pt]
  && - {4\over3} \, A_b^2 \bar v_p^2 -
  \, 2 \sum_c  \, { \kappa_c^2 \over
  \kappa_\smD^2 } \, A_c^2 \, \bar v_p^2    
  + { \pi \over 6} \, \left[ {\sum}_c \, 
  {\kappa^2_c \over \kappa_\smD^2} 
  \, A_c  \bar v_p \right]^2 \Bigg\} + 
  {\cal O}\bigg((A_b \bar v_p)^2\bigg) \ .
\end{eqnarray}
%%
}}



\newpage
\section{Fitting}

\subsection{The General Method}


Fits to functions $f(x)$ will be performed with rational 
polynomials over the positive real numbers. In particular, 
we will find rational functions $R(x)$ and $Q(x)$ such that
%%
\begin{eqnarray}
  f(x) \approx R(x) \, Q(x) \ ;
\end{eqnarray}
%%
we require that $Q(x)$ asymptotes to one for $x \to 0$ 
and $x\to\infty$, while $R(x)$ is designed to capture 
the asymptotic behavior of $f(x)$. For example, suppose 
$f(x) \sim x$ for $x \sim 0$, and $f(x) \sim 1/x$  for 
large $x$, then we may take  $R=x/(x^2 +1)$. We will take 
$Q$ to be of the form
%%
\begin{eqnarray}
  Q_n(x) = \frac{\sum_{\ell=0}^n b_\ell x^\ell}
  {\sum_{\ell=0}^n a_\ell x^\ell} \ ,
\end{eqnarray}
%%
with $b_n=a_n=1$ and $a_0=b_0$, i.e.
%%
\begin{eqnarray}
\\ \nonumber
  Q_n(x) = \frac{x^n ~+~ b_{n-1} x^{n-1} ~+~ b_{n-2} 
  x^{n-2} ~+~ \cdots ~+~ b_2 x^2 ~+~ b_1 x ~+~ b_0}
  {x^n ~+~ a_{n-1} x^{n-1} ~+~ a_{n-2} 
  x^{n-2} ~+~ \cdots ~+~ a_2 x^2 ~+~ a_1 x ~+~ b_0} \ .
\\ \nonumber
\end{eqnarray}
%%
Note that $Q_n(x) \to 1$ for $x\to\infty$ and $x\to 0$. 
We will determine the $2n-1$ variables $b_\ell$ (with
$\ell=0, 1, \cdots, n-1$) and $a_\ell$ (with $\ell=1, 2, 
\cdots, n-1$) by requiring that the spline-fit agrees
at points $x_m$ with $m=0, 2, \cdots 2n-2$, i.e.
%%
\begin{eqnarray}
  R_m Q_n(x_m) = f_m ~~~~~~~ m=0, 2, \cdots, 2n-2 \ .
\\ \nonumber
\end{eqnarray}
%%
where $f_m=f(x_m)$ and $R_m = R(x_m)$. Writing $F_m=f_m/
R_m$, we find $2n-1$ linear equations
%%
\begin{eqnarray}
  \sum_{\ell=1}^{n-1}\left(x_m^\ell \cdot b_\ell - x_m^\ell
  F_m \cdot a_\ell \right)~+~ (1-F_m) \cdot b_0 = (F_m-1) 
  x_m^n ~~~~~ m=0,1,\cdots, 2n-2 \ ,
\\ \nonumber
\end{eqnarray}
%%
which may be solved for $b_\ell$ and $a_\ell$. 



\subsection{An Example: the Dawson Integral}

We need to fit the Dawson integral
%%
\begin{eqnarray}
  {\rm daw}(x) = \int_0^x dy\, e^{y^2-x^2} \ ,
\end{eqnarray}
%% 
and to evaluate the this function analytically at a few 
points we use 
%%
\begin{eqnarray}
  {\rm daw}(x) =  \frac{\sqrt{\pi}}{2}\, 
  e^{-x^2}\,{\rm erfi}(x) \ .
\end{eqnarray}
%% 
For a detailed evaluation of the integrals and limits
that follow, see daw.nb. The asymptotic forms for large 
and small $x$ are
%%
\begin{eqnarray}
\nonumber
  x \ll 1: \hskip1cm \\  
  {\rm daw}(x) &=& x - \frac{2x^3}{3} +
  \frac{4 x^5}{15} + {\cal O}(x^7) 
  \hskip1.55cm 0.1\% {\rm ~error~for~} x < x_{\rm min}=0.5
\\[10pt]
\nonumber
  x \gg 1: \hskip1cm \\  
  {\rm daw}(x) &=& \frac{1}{2x} + \frac{1}{4 x^3} +
  \frac{3}{8 x^5} + {\cal O}(x^{-7}) 
  \hskip1cm 0.1\% {\rm ~error~for~} x > x_{\rm max}=5\ .
\end{eqnarray}
%%
We can use these analytic expressions for $x<x_{\rm min}$
and $x>x_{\rm max}$. For values in between we will fit
to rational functions with $n=6$, and approximate 
%%
\begin{eqnarray}
  {\rm daw}(x) = R(x) \, Q_6(x) \ .
\end{eqnarray}
%%
The rational function $Q_6(x)$ goes to unity for large and 
small $x$, so we must choose $R(x)$ to yield the correct 
asymptotic behavior. We have seen that ${\rm daw}(x) \sim x$
for $x << 1$ and ${\rm daw}(x) \sim 1/2x$ for $x \gg 1$,
so we take 
%%
\begin{eqnarray}
  R(x)=\frac{x}{2 x^2 +1} \ .
\end{eqnarray}
%%
For $n=6$, we need $2n-1=11$ data points: we will take
$m$ to start at zero and end at ten, and we choose 
%%
\begin{eqnarray}
\begin{array}{lll}
  m~~~~ & x_m~~~ & {\rm daw}(x_m)   \\ \hline
  0    & 0.92413   &   0.541044     \\[-8pt]
  1    & 0.2       &   0.194751     \\[-8pt]
  2    & 0.5       &   0.424436     \\[-8pt]
  3    & 0.7       &   0.510504     \\[-8pt]
  4    & 1.2       &   0.507273     \\[-8pt]
  5    & 1.4       &   0.456507     \\[-8pt]
  6    & 1.6       &   0.399940     \\[-8pt]
  7    & 2.0       &   0.301340     \\[-8pt]
  8    & 3.0       &   0.178271     \\[-8pt]
  9    & 4.0       &   0.129348     \\[-8pt]
 10    & 8.0       &   0.0630002 ~~~~~~,
\end{array}
\end{eqnarray}
%%
and solving the linear equations gives
%%
\begin{eqnarray}
  \begin{array}{lrr}
  \ell~~~~ & \hfill b_\ell \hfill & \hskip3cm a_\ell \hskip1.5cm  \\ \hline
  0    &    5.735938802443  &   5.7359388024432  \\[-8pt]
  1    &   -6.736660071378  &  -6.8237204895090  \\[-8pt]
  2    &   19.979442278715  &  13.3804115903096  \\[-8pt]
  3    &  -18.550635026076  & -14.2130723670491  \\[-8pt]
  4    &   12.265136090570  &  11.1714434417979  \\[-8pt]
  5    &   -4.672858126848  &  -4.6630338746894 
  \end{array}   
\end{eqnarray}
%%

\vskip0.1cm 
In the source code, the parameters are set in {\rm 
globalvars.f90} and the the polynomial fits take place 
in the subroutine {\rm daw()} of {\rm dedx.f90}. 
\vskip1cm 

{
\noindent
globalvars.f90:
\baselineskip12pt
\begin{verbatim}
MODULE globalvars
!
! mathematical constants
!
  REAL,    PARAMETER :: PI   =3.141592654   ! pi
  REAL,    PARAMETER :: SQPI =1.772453851   ! sqrt(pi)
  REAL,    PARAMETER :: GAMMA=0.577215665   ! Euler Gamma
  REAL,    PARAMETER :: LOG2 =0.6931471806  ! ln(2)
  REAL,    PARAMETER :: LOG4 =1.386294361   ! ln(4)
  REAL,    PARAMETER :: LOG8 =2.079441542   ! ln(8)
  REAL,    PARAMETER :: LOG16=2.772588722   ! ln(16)
  REAL,    PARAMETER :: ZETA3=1.202056903   ! zeta(3)
  REAL,    PARAMETER :: EXP2E=3.172218958   ! exp(2*GAMMA)

....
....

! daw() approximates Dawson's integral by rational
! functions with coefficients:
!
  INTEGER, PARAMETER          :: NNDAW=3, NMDAW=2*NNDAW-1 ! NMDAW=5
  REAL,    DIMENSION(0:NMDAW) :: DWB, DWA
  PARAMETER (            &
  DWB=(/                 &
    5.73593880244318E0,  & !b0
   -6.73666007137766E0,  & !b1
    1.99794422787154E1,  & !b2
   -1.85506350260761E1,  & !b3
    1.22651360905700E1,  & !b4
   -4.67285812684807E0/),& !b5
  DWA=(/                 &
    DWB(0),              & !a0
   -6.82372048950896E0,  & !a1
    1.33804115903096E1,  & !a2
   -1.42130723670491E1,  & !a3
    1.11714434417979E1,  & !a4
   -4.66303387468937E0/) ) !a5

....
....

END MODULE globalvars
\end{verbatim}
}

\vskip1cm 

{
\noindent
dedx.f90:
\baselineskip12pt
\begin{verbatim}
!
! See daw.nb for details.
!
 FUNCTION daw(x)
   USE globalvars
   IMPLICIT NONE
   REAL, INTENT(IN) :: x
   REAL  daw

   REAL,    PARAMETER :: XMIN=0.4D0, XMAX=5.D0
   REAL    :: x3, x5, xx, ra, rc
   INTEGER :: n
   IF (x .LE. XMIN) THEN
      x3=x*x*x
      x5=x3*x*x
      daw=x - 2.D0*x3/3.D0 + 4.D0*x5/15.D0
   ELSEIF (x .GE. XMAX) THEN
      x3=x*x*x
      x5=x3*x*x
      daw=1.D0/(2.D0*x)+1.D0/(4.D0*x3)+3.D0/(8.D0*x5)
   ELSE
      ra=0.E0
      rc=0.E0
      xx=1.E0
      DO n=0,NMDAW
         ra=ra+DWA(n)*xx
         rc=rc+DWB(n)*xx
         xx=x*xx
      ENDDO
      ra=ra+xx
      rc=rc+xx
      daw=x/(1.E0+2.E0*x*x)
      daw=daw*rc/ra
   ENDIF
 END FUNCTION daw
\end{verbatim}
}


\newpage
\section{The Main Subroutine}

The main driving subroutine that returns the stopping power
is called dedx\_bps and takes the form
\vskip0.75cm

{
\noindent
dedx.f90:
\baselineskip12pt
\begin{verbatim}
 SUBROUTINE dedx_bps(nni, vp, zp, mp, betab, zb, mb, nb, &
     dedxtot, dedxsumi, dedxctot, dedxcsumi, dedxqtot, dedxqsumi)

   A. define useful global variables
   B. print diagnostics, such as plasma coupling (if desired)
   C. call classical and quantum stopping power

 END SUBROUTINE dedx_bps
\end{verbatim}
}

\noindent
where the input and output variables are described by the
following table.

\vskip0.5cm
{\center
\vbox{
\begin{tabular}{|l|l|l|}\hline
~variable name~~  &~input or output~~ &~description~~              \\\hline
~nni              &~input             &~number of ion species      \\[-5pt]
~vp               &~input             &~projectile velocity $\bar v_p$
in thermal units $v_{\rm th}$~                                     \\[-5pt]
~zp               &~input             &~charge of projectile       \\[-5pt]
~mp               &~input             &~mass of projectile in eV   \\[-5pt]
~betab            &~input             &~inverse temperature array of
plasma in ${\rm eV}^{-1}$                                          \\[-5pt]
~zb               &~input             &~charge array of plasma     \\[-5pt]
~mb               &~input             &~mass array of plasma in eV \\[-5pt]
~nb               &~input             &~number density array of
plasma in ${\rm cm}^{-3}$                                     \\\hline
~dedxtot          &~output            &~total stopping power in
${\rm MeV}/\mu{\rm m}$                                              \\[-5pt]
~dedxsumi         &~output            &~ionic stopping power in 
${\rm MeV}/\mu{\rm m}$                                              \\[-3pt]
~dedxctot         &~output            &~classical contribution to
the total stopping power~                                           \\[-5pt]
~dedxcsumi        &~output            &~classical contribution to
the ionic stopping power~                                           \\[-5pt]
~dedxqtot         &~output            &~quantum contribution to
the total stopping power~                                           \\[-5pt]
~dedxqsumi        &~output            &~quantum contribution to
the ionic stopping power~                                           \\\hline
\end{tabular}
}
}

\vskip0.5cm
\noindent
The subroutine (a) defines some useful global variables, 
(b) prints diagnostics, and (b) evaluates the stopping
power. Global variables are declared as such, but not
necessarily set, in the module globalvars.f90. This module 
also defines some parameters, such as $\pi$ and the Euler 
constant $\gamma$. 

\pagebreak
{
\noindent
globalvars.f90:
\baselineskip12pt
\begin{verbatim}
MODULE globalvars
!
! mathematical constants
!
  REAL,    PARAMETER :: PI   =3.141592654   ! pi
  REAL,    PARAMETER :: SQPI =1.772453851   ! sqrt(pi)
  REAL,    PARAMETER :: GAMMA=0.577215665   ! Euler Gamma
  REAL,    PARAMETER :: LOG2 =0.6931471806  ! ln(2)
  REAL,    PARAMETER :: LOG4 =1.386294361   ! ln(4)
  REAL,    PARAMETER :: LOG8 =2.079441542   ! ln(8)
  REAL,    PARAMETER :: LOG16=2.772588722   ! ln(16)
  REAL,    PARAMETER :: ZETA3=1.202056903   ! zeta(3)
  REAL,    PARAMETER :: EXP2E=3.172218958   ! exp(2*GAMMA)
!
! physical parameters and conversion factors
!
  REAL,    PARAMETER :: BE=13.6             ! binding energy of Hydrogen
  REAL,    PARAMETER :: CC=2.998E10         ! speed of light
  REAL,    PARAMETER :: MPEV =0.938271998E9 ! proton mass in eV
  REAL,    PARAMETER :: MEEV =0.510998902E6 ! electron mass in eV
  REAL,    PARAMETER :: AMUEV=0.931494012E9 ! AMU in eV
  REAL,    PARAMETER :: KTOEV=8.61772E-5    ! conversion factor
  REAL,    PARAMETER :: CMTOA0=1.8867925E8  ! conversion factor
  REAL,    PARAMETER :: MTR=1.E-6           ! length unit
  REAL,    PARAMETER :: EV=1.E6             ! energy unit
  REAL,    PARAMETER :: CONVFACT=CMTOA0*(MTR*100.)/EV
!
! misc parameters
!
  INTEGER, PARAMETER :: R=3        ! thermal velocity parameter
  INTEGER, PARAMETER :: I=1        ! plasma species index
  REAL               :: K          ! arbitrary wave number units a0^-1 

! plasma parameters: values set in dedx_bps
!
! REAL,    DIMENSION(1:NNB)          :: kb2, ab, bb, cb, eb, fb, rb, gb
! REAL,    DIMENSION(1:NNB)          :: ab2, etb, rmb0, rrb0, mb0
  REAL,    DIMENSION(:), ALLOCATABLE :: kb2, ab, bb, cb, eb, fb, rb, gb
  REAL,    DIMENSION(:), ALLOCATABLE :: ab2, etb, rmb0, rrb0, mb0
  LOGICAL, DIMENSION(:), ALLOCATABLE :: lzb
  REAL    :: cp1, cp2, cp3, vth, vthc, mp0, kd
  INTEGER :: NNB  ! number of plasma species = ni+1
....
....
END MODULE globalvars
\end{verbatim}
}

\vskip0.5cm
\noindent
The last part of globalvars.f90 not shown here involves 
parameters
for the functions ${\rm daw}(x)$, $J_1(x)$, $J_2(x)$,
$J_3(x)$, and $J_4(x)$, which are used to calculate the
classical contribution to the stopping power and will be 
described later.

\subsection{Defining Parameters and the Main Call}

~~
\vskip-1cm 
Some general parameters and mass parameters:
%%
\begin{mathletters}
\begin{eqnarray}
  {\rm mb0} &=& m_b^\smO = m_b/m
\\[-2pt]
  {\rm rb} &=& r_b = 
  r \beta_b/\beta_m 
\\[-2pt]
  {\rm rmb0} &=& M_{pb}^\smO/m_p^\smO
\\[-2pt]
  {\rm rrb0} &=& m_b^\smO/m_p^\smO
\\[-2pt]
  {\rm kb2} &=& \kappa_b^2 = 8 \pi B_e \,
  \beta_b^\smO Z_b^2\,n_b^\smO ~~~~
  {\rm kb} = \sqrt{{\rm kb2}}
\\[-2pt]
  {\rm cp1} &=& c_1 = e_p^2/4\pi =
  2 B_e Z_p^2 
\\[-2pt]
  {\rm cp2} &=& c_2  = e_p^2/16\pi^2 =
  B_e Z_p^2/2\pi
\\[-2pt]
  {\rm cp3} &=& c_3  = e_p^2/8\pi^2 m_p^\smO =
  B_e Z_p^2/\pi m_p^\smO 
\end{eqnarray}
%%
\vskip-0.5cm 
Classical parameters:
%%
\begin{eqnarray}
  {\rm ab} &=& A_b = 
  \sqrt{r_b m_b^\smO/2}
  ~~~~~~{\rm ab2} \equiv {\rm ab}
  \cdot {\rm ab}
\\[-2pt]
  {\rm bb} &=& B_b =
  r_b M_{pb}^\smO = r_b (m_p^\smO + m_b^\smO)
\\[-2pt]
  {\rm cb} &=& C_b = 
  2 - 2\gamma - \ln\hskip-0.1cm \bigg(
  2 B_e\, \beta_b^\smO Z_p Z_b K^\smO 
  m_b^\smO/m_{pb}^\smO\bigg)
\\[-2pt]
  {\rm gb} &=&   G_b =
  \frac{1}{2} + 2\gamma + \ln\hskip-0.1cm \bigg(
  2 B_e\, \beta_b^\smO Z_p Z_b \kappa_\smD^\smO 
  m_b^\smO/m_{pb}^\smO\bigg) 
  ~~{\rm : for~the~small~} \bar v_p {\rm ~limit}
\\[-2pt]
  {\rm eb} &=& E_b = 
  \frac{m_b^\smO}{m_p^\smO} ~
  \frac{1}{\sqrt{2\pi r_b m_b^\smO}}
\end{eqnarray}
%%
\vskip-0.2cm 
Quantum parameters:
%%
\begin{eqnarray}
  {\rm fb} &=& F_b = 
  \frac{2}{\sqrt{2\pi r_b m_b^\smO}}
\\
  {\rm etb} &=& \bar\eta_b = 
  \frac{e_p e_b}{4\pi\hbar\, v_{\rm th}}
  = \frac{2 B_e\, Z_p Z_b}{\bar v_{\rm th}}~
  {\scriptscriptstyle \times}\,
  2.686 \times 10^{-4} 
\end{eqnarray}
\end{mathletters}
%%

\vskip-0.1cm 
{
\noindent
dedx.f90:
\baselineskip12pt
\begin{verbatim}
 SUBROUTINE dedx_bps(nni, vp, zp, mp, betab, zb, mb, nb, &
     dedxtot, dedxsumi, dedxctot, dedxcsumi, dedxqtot, dedxqsumi)
   USE globalvars

   IMPLICIT NONE
   INTEGER,                     INTENT(IN)  :: nni    ! number of ions
   REAL,    DIMENSION(1:nni+1), INTENT(IN)  :: betab  ! plasma temp array
   REAL,    DIMENSION(1:nni+1), INTENT(IN)  :: mb, nb ! mass and density 
   INTEGER, DIMENSION(1:nni+1), INTENT(IN)  :: zb     ! charge array
   REAL,                        INTENT(IN)  :: vp     ! projectile velocity
   REAL,                        INTENT(IN)  :: zp     ! projectile charge
   REAL,                        INTENT(IN)  :: mp     ! projectile mass
   REAL,                        INTENT(OUT) :: dedxtot,  dedxsumi
   REAL,                        INTENT(OUT) :: dedxctot, dedxcsumi
   REAL,                        INTENT(OUT) :: dedxqtot, dedxqsumi
   REAL, DIMENSION(1:nni+1) :: mpb0, rpb0
   REAL :: betam, kd, mm
   REAL                     :: e, gd
   REAL, DIMENSION(1:nni+1) :: gpb
   REAL, DIMENSION(1:nni+1) :: ub2, mpb, loglamb

   NNB=nni+1
   ALLOCATE(kb2(1:NNB),ab(1:NNB),bb(1:NNB),cb(1:NNB))
   ALLOCATE(eb(1:NNB),fb(1:NNB),rb(1:NNB),gb(1:NNB))
   ALLOCATE(ab2(1:NNB),etb(1:NNB),rmb0(1:NNB),rrb0(1:NNB))
   ALLOCATE(mb0(1:NNB))
!
! plasma parameters
!
   betam=betab(I)                ! inv temp of index plasma species
   rb=R*betab/betam              ! r_b array
   kb2=8*PI*BE*betab*zb*zb*nb    ! inv Debye length squared
   kd=SUM(ABS(kb2))              ! total inv Debye length
   kd=SQRT(kd)                   ! units aa0^-1
   K =kd                         ! set K to Debye 

   mm=mb(I)                      ! mass of index plasma species  
   mp0=mp/mm                     ! rescaled proj mass
   cp1=2*BE*zp**2                ! units of eV-a0 
   cp2=(BE*zp**2)/(2*PI)         ! units of eV-a0 
   cp3=(BE*zp**2)/(PI*mp0)       ! dimensionless parameter   
   vthc=SQRT(R/(betam*mm))       ! thermal velocity of mm: units of c
   vth =CC*vthc                  ! thermal velocity of mm: units cm/s 

   mb0 =mb/mm                    ! rescaled plasma masses
   mpb0=mp0 + mb0                ! Mpb0
   rpb0=mp0*mb0/mpb0             ! mpb0
   rmb0=mpb0/mp0                 ! rm0=rMb0=(mp0+mb0)/mp0
   rrb0=mb0/mp0                  ! rr0=rmb0=mb0/mp0

   ab  =SQRT(rb*mb0/2)            
   ab2 =ab*ab                    
   bb  =rb*mpb0                   
   eb  =(mb0/mp0)/SQRT(2*PI*rb*mb0)
   etb =2*BE*ABS(zp*zb)*(2.686E-4)/vthc
   fb  =2/SQRT(2*PI*rb*mb0)
   WHERE ( zb /= 0 )             ! do not take log of zero
      lzb=.TRUE.                 ! flag for future use
      cb  =2 - 2*GAMMA - LOG(ABS((2*BE)*betab*zp*zb*K*mb0/rpb0))
      gb  =0.5 + 2*GAMMA + Log(ABS(0.5*BE*&    ! for small vp limit
           betab*zp*zb*kd*mb0/rpb0))           !
   ELSEWHERE
      cb=0
      gb=0
      lzb=.FALSE.
   ENDWHERE
...
...
\end{verbatim}
}

\subsection{Diagnostics}

We now display several diagnostic parameters (commented 
out for a production run): (i) check for charge neutrality, 
(ii) display the arbitrary wavenumber $K$ in comparison to 
the Debye wave number $\kappa_\smD$, (iii) calculation the 
plasma coupling constants $g_{pb}$, (iv) display the thermal 
velocity in ${\rm cm/s}$, (v) and print the Coulomb logarithm
as defined by Li and Petrasso.

\vbox{
\baselineskip12pt
\begin{verbatim}
...
...
! check for charge neutrality
!
!   e=SUM(zb*nb)
!   PRINT *, 'charge  = ', e

! print K and kd
! 
!   PRINT *, 'K(a0^-1)= ', K
!   PRINT *, 'kd      = ', kd

! g-factors
!
!   gpb=2*BE*betam*SQRT(kb2)       ! g_pb 
!   gd =2*BE*betam*kd              ! g_d
!   PRINT *,'gD      = ', gd

! thermal velocity (cm/s)
!
!   PRINT *, 'vth     = ', vth
!

! Coulomb log
!
!   ub2=(vp*vthc)**2 + 2/(betab*mb)**2     ! velocity (units of c)
!   mpb=rpb0*mm                            ! reduced mass array (eV)
!   loglamb=(8*PI*zb*zb)**2/(mpb*ub2)**2   ! a0 = 5.29*10^-11 m 
!   loglamb=loglamb + (2.69E-4)**2/(2*mpb*mpb*ub2) ! = 2.69*10^-4 eV
!   loglamb=-0.5*LOG(loglamb)
!   PRINT *, 'log(Lam)=',loglamb
...
...
\end{verbatim}
}


\pagebreak
\subsection{Call Classical and Quantum Stopping Power}

The subroutine dedxc calculates the classical contribution
while dedxq finds the quantum correction:

\vbox{
\baselineskip12pt
\begin{verbatim}
...
...

   CALL dedxc(vp,dedxctot,dedxcsumi)   ! returned in MeV/mu-m
   CALL dedxq(vp,dedxqtot,dedxqsumi)   !
   dedxtot =dedxctot  + dedxqtot
   dedxsumi=dedxcsumi + dedxqsumi

   ! PRINT *,"       MeV/mu-m    MeV/mu-m"
   ! PRINT *,"tot:",dedxtot,  dedxsumi
   ! PRINT *,"cl :",dedxctot, dedxcsumi
   ! PRINT *,"qm :",dedxqtot, dedxqsumi

   DEALLOCATE(kb2,ab,bb,cb,eb,fb,rb,gb)
   DEALLOCATE(ab2,etb,rmb0,rrb0,mb0,lzb)
 END SUBROUTINE dedx_bps
\end{verbatim}
}

\newpage
\section{The Quantum Correction}

Since the quantum correction is easier to calculate
than the classical contribution, we start here
first. The subroutine dedxq evaluates 
%%
\begin{eqnarray}
\label{dedxqmdef}
  { d E^\smQ_b \over dx} &=&
  \frac{c_1}{\bar v_p} \hskip0.3cm \cdot \hskip0.1cm 
  \kappa_b^2 \, F_b  \hskip0.3cm \cdot \hskip0.1cm 
  I_\smQM(A_b^2, \, \bar\eta_b, \, \bar v_p)
\\[-13pt]\nonumber  &&
  \hskip-0.2cm\underbrace{\hskip0.3cm}_{{\rm cp1/vp}}
  \hskip0.3cm \underbrace{\hskip0.5cm}_{{\rm kb2*fb}}
  \hskip0.7cm \underbrace{\hskip2.5cm}_{{\rm dedxqi}}
\end{eqnarray}
%%
where we define
%%
\begin{eqnarray}
\label{IQMdef}
  I_\smQM(a,e,v)    &=& e^{-a v^2}
  \int_0^\infty du e^{-a u^2}\Bigg\{\ln\left(
  \frac{e}{u}\right) - {\rm Re} \, \psi \hskip-0.05cm 
  \left( 1 + i \frac{e}{u} \right) \Bigg\}
\nonumber\\[5pt]
  && 
  \Bigg\{ 
  {M_{pb}^\smO \over m_p^\smO} \, \frac{1}{u} 
  \left(\cosh(2a v\,u) - {\sinh(2 a v\, u) \over 2 a v\, u} 
  \right)  -
  \frac{m_b^\smO}{m_p^\smO}\, \frac{\sinh(2 a v\, u)}{v} 
  \Bigg\} \ .
\end{eqnarray}
%%
The first two terms of (\ref{dedxqmdef}) are multiplicative 
factors, and all the work is performed by a call to a subroutine
dedxqi that evaluates the integral (\ref{IQMdef}). The structure 
of the quantum driving routine is of the form

{
\baselineskip12pt
\begin{verbatim}
 SUBROUTINE dedxq(vp, dedxqtot, dedxqsumi)
  ...
  ...
  A. loop over species ib=1,NNB (ib=1 is the electron)
  B. calculate IQM for each ib, i.e. call dedxqi
  C. multiply by factors kb2(ib)*fb(ib) and cp1/vp
  ...
  ...
 END SUBROUTINE dedxq
\end{verbatim}
}

\noindent
The full subroutine is listed here:

{
\baselineskip12pt
\begin{verbatim}
 SUBROUTINE dedxq(vp, dedxqtot, dedxqsumi)
   USE globalvars
   
   IMPLICIT NONE
   REAL, INTENT(IN) :: vp
   REAL, INTENT(OUT):: dedxqtot, dedxqsumi

   REAL, DIMENSION(1:NNB) :: qmb
   REAL                   :: dedxqi, a1, a2, e, rm, rr
   INTEGER                :: ib

   qmb=0
   DO ib=1,NNB            ! sum over plasma species
      IF ( lzb(ib) ) THEN ! computle only if zb(ib) /= 0 
         a1=ab(ib)
         a2=a1*a1
         e=etb(ib)
         rm=rmb0(ib)      ! rmb0=rMb0=(mp0+mb0)/mp0
         rr=rrb0(ib)      ! rrb0=rmb0=mb0/mp0
         qmb(ib)=qmb(ib)+dedxqi(vp,a2,e,rm,rr)
         qmb(ib)=qmb(ib)*kb2(ib)*fb(ib)
      ELSE
         qmb(ib)=0        ! don't compute if zb(ib) = 0
      ENDIF
   ENDDO
   dedxqsumi=SUM(qmb(2:NNB))
   dedxqtot =qmb(1)
   dedxqtot =dedxqtot+dedxqsumi

   dedxqtot=CONVFACT*(cp1/vp)*dedxqtot
   dedxqsumi=CONVFACT*(cp1/vp)*dedxqsumi
 END SUBROUTINE dedxq
\end{verbatim}
}

\noindent
The subroutine dedxq is basically a wrapper for
the function dedxqi (to be described momentarily)
that sums 
the quantum components over all species $b = {\rm ib}$
from 1 to NNB (the total number of plasma species; 
charged zero species are not counted). Our
convention is that $b=1$ corresponds to electrons, so
that the ionic contribution dedxqsumi does not include 
the $b=1$ term.  During the loop
over species: (i) the function dedxqi is called, (ii)
the result is multiplied by $\kappa_b^2 F_b = {\rm kb2(ib)
* fb(ib)}$, and (iii) the final result is multiplies by
$c_1/\bar v_p = {\rm cp1/vp}$. The factor ${\rm CONVFACT}$
converts from ${\rm eV/a_0}$ to ${\rm MeV/ \mu m}$. The
heart of the subroutine is the function dedxqi, which 
performs a Gaussian quadrature over the integrand d\_dedxq,
or returns its value based on analytic limits if they
apply (the large mass limit for ions and the large thermal 
velocity limit for electrons). 

During the loop over species ${\rm ib}$, the arguments
of the function dedxqi(vp,a,e,rm,rr) have the following
meaning:
\vskip0.5cm

{\center
\vbox{
\begin{tabular}{|l|l|}\hline
~variable name~~  &~description~~              \\\hline
~vp               &~projectile velocity $\bar v_p$ in thermal 
units $v_{\rm th}$~                                         \\[-5pt]
~a                &~$A_b^2={\rm ab(ib)}*{\rm ab(ib)}$       \\[-5pt]
~e                &~$\bar\eta_b={\rm etb(ib})$              \\[-5pt]
~rm               &~$M_{pb}^\smO/m_p^\smO={\rm rmb0(ib)}$   \\[-5pt]
~rr               &~$m_b^\smO/m_p^\smO={\rm rrb0(ib)}$      \\\hline
\end{tabular}
}
}

\vskip0.5cm
\noindent
while the function itself is defined to be

\vbox{
%%
\begin{eqnarray}
   {\rm dedxqi}({\rm a},{\rm e},{\rm v},{\rm rm},{\rm rr})    
  &=& e^{- {\rm a} {\rm v}^2}
  \int_0^\infty du\, e^{-{\rm a} u^2}\Bigg\{\ln\left({\rm e}/u\right) - 
  {\rm repsi}\left({\rm e}/u\right) \Bigg\}
\nonumber\\[5pt]
  && 
  \Bigg\{ 
  {\rm rm} \, \frac{1}{u} 
  \left(\cosh(2{\rm a} {\rm v}\,u) -
  {\sinh(2 {\rm a} {\rm v\, u}) \over 2 {\rm a} {\rm v}\, u} 
  \right)  -
  {\rm rr}\, \frac{\sinh(2 {\rm a} {\rm v}\, u)}{{\rm v}} 
  \Bigg\} \ ,
\end{eqnarray}
%%
}

\noindent
with ${\rm repsi(x)} \equiv {\rm Re}\,\psi(1 + i\, x)$. 


\subsection{The Quantum Integral by Gaussian Quadrature} 

We now look at dedxqi in more detail. For the moment we
will revert to the notation of (\ref{IQMdef}), and write 
dedxqi as
%%
\begin{eqnarray}
  I_\smQM(a,e,v)    &=& 
  \frac{M_{pb}^\smO}{m_p^\smO}\,Q_2(a,e,v) -
  \frac{m_b^\smO}{m_p^\smO}\, Q_1(a,e,v) \ ,
\end{eqnarray}
%%
where 
%%
\begin{eqnarray}
  Q_1(a,e,v) &=&\frac{1}{v}\,
  e^{- a v^2}
  \int_0^\infty du \, e^{-a u^2} G(e/u)\,
  \sinh(2 a v\, u)
\\[5pt]
  Q_2(a,e,v) &=&
  e^{- a v^2}
  \int_0^\infty du \, e^{-a u^2}
  G(e/u)\, \frac{1}{u} 
  \left(\cosh(2a v\,u) - {\sinh(2 a v\, u) \over 2 a v\, u} 
  \right)  \ ,
\end{eqnarray}
%%
with
%%
\begin{eqnarray}
  G(y) &=& 
  \ln\left(y \right) - {\rm Re} \, \psi \hskip-0.05cm 
  \left( 1 + i y \right) \ .
\end{eqnarray}
%%
For Gaussian quadrature, and for deriving a useful analytic 
result based on a saddle point approximation (to be described
more fully in a moment), it is convenient 
to change variables to $x=u/v$. We can then can write the 
integrals in terms of two parameters,
%%
\begin{eqnarray}
  r &=& e/v
\\
  s &=& a v^2 \ ,
\end{eqnarray}
%%
as follows:
%%
\begin{mathletters}
\label{qonetwodef}
\begin{eqnarray}
  Q_1(r,s) &=& e^{-s}
  \int_0^\infty dx \, e^{-s x^2} G(r/x)\, \sinh(2 s x) \
\\[5pt]
  Q_2(r,s) &=&
  e^{- s}\int_0^\infty dx \, e^{-s x^2}
  G(r/x)\, \frac{1}{x} 
  \left(\cosh(2 s x) - {\sinh(2 s x) \over 2 s x} \right)  \ .
\end{eqnarray}
\end{mathletters}
%%
We can also expand the hyperbolic functions and write 
%%
\begin{eqnarray}
  Q_1(r,s) &=&\frac{1}{2}
  \int_0^\infty dx \,G(r/x)\, \left[\,e^{-s(x-1)^2} -
  e^{-s(x+1)^2} \right] 
\\[5pt]
  Q_2(r,s) &=&
  \frac{1}{2}\int_0^\infty dx \, 
  G(r/x)\, \frac{1}{x} 
  \left[\,e^{-s(x-1)^2} + e^{-s(x+1)^2} - 
  \frac{e^{-s(x-1)^2} - e^{-s(x+1)^2}}{2 s\, x} \right]   \ ,
\end{eqnarray}
%%
which illustrates that the integrand peaks as $x=1$, 
or at $u=v$ in terms of the old variable, with a width
of order $1/\sqrt{s}$. 

For large and small values of $s$ we can perform analytic
approximations; however, in the most general case we must 
resort to a numerical integration scheme. As previously 
noted, the integrand d\_dedxq(r,s,rm,rr,x) is peaked  about 
$x=1$ with a width of order $1/\sqrt{{\rm s}}$, and we can
therefore restrict our numerical integration between the
limits $1 \pm N/\sqrt{{\rm s}}$, taking a large value such
as $N=30$ for safety. If the lower integration limit becomes 
negative with this prescription we set it to zero. Omitting
most of the variable declarations, the Gaussian quadrature 
routine is:


\vskip0.5cm 
{
\baselineskip12pt
\begin{verbatim}
 FUNCTION dedxqi(v, a, e, rm, rr)
!
! This function performs the integration numerically by
! Gaussian Quadrature. The polynomial P3(x)=(5*x^3-3*x)/2 
! is employed, and I have defined the appropriate weights 
! W13, W2 and relative position UPM in parameter statements. 
!
   IMPLICIT NONE     
  
   REAL, INTENT(IN) :: v, a, e, rm, rr
   REAL :: dedxqi
  
   REAL,    PARAMETER :: UPM=0.7745966692E0
   REAL,    PARAMETER :: W13=0.5555555556E0, W2=0.8888888889E0
   INTEGER, PARAMETER :: NG=10000  ! must be even
   REAL,    PARAMETER :: NN=30.E0
...
   Special analytic limits will be placed here    
...
   r=e/v
   s=a*v*v

   x0=1.E0 - NN/SQRT(s)
   x0=MAX(0.,x0)
   x1=1.E0 + NN/SQRT(s)
   dx=(x1-x0)/NG
   dedxqi=0.E0
   x=x0-dx
   DO ix=1,NG,2
!     
      x=x+2.E0*dx
      dedxqi=dedxqi+W2*d_dedxq(r,s,rm,rr,x)
!
      xm=x-dx*UPM
      dedxqi=dedxqi+W13*d_dedxq(r,s,rm,rr,xm)
!
      xm=x+dx*UPM
      dedxqi=dedxqi+W13*d_dedxq(r,s,rm,rr,xm)
   ENDDO
   dedxqi=dedxqi*dx
 END FUNCTION dedxqi
\end{verbatim}
}


\noindent
The integrand d\_dedxq is

{
\baselineskip12pt
\begin{verbatim}
 FUNCTION d_dedxq(r, s, rm, rr, x)
   IMPLICIT NONE
   REAL, INTENT(IN) :: r, s, rm, rr, x
   REAL, PARAMETER  :: SXMAX=0.05
   REAL :: d_dedxq
   REAL :: repsi, rx, sx, sh, ch
   REAL :: ep, em, xm1, xp1
   rx=r/x
   sx=2*s*x
   xm1=x-1
   xp1=x+1
   ep=EXP(-s*xp1*xp1)
   em=EXP(-s*xm1*xm1)
   sh=0.5E0*(em-ep)         ! sh and ch are 
   IF (sx .GT. SXMAX) THEN  ! not sinh or cosh
      ch=0.5E0*(em+ep)      ! 
      ch=(ch - sh/sx)/x
   ELSE
      ch=2.E0*sx/3 + (1.E0/15.E0 - 1.E0/(6.E0*s))*sx*sx*sx
      ch=s*ch*EXP(-s)
   ENDIF
   d_dedxq=LOG(ABS(rx)) - repsi(rx)
   d_dedxq=d_dedxq*(rm*ch - rr*sh)
 END FUNCTION d_dedxq
\end{verbatim}
}

\noindent
The function ${\rm repsi}(x)$ will be coded in a
moment. 
Note that if the argument ${\rm sx} \equiv 2 {\rm s} x$
is smaller than ${\rm SXMAX}=0.05$ then we expand 
%%
\begin{eqnarray} 
  \frac{1}{2 x} 
  \left[\,e^{-s(x-1)^2} + e^{-s(x+1)^2} - 
  \frac{e^{-s(x-1)^2} - e^{-s(x+1)^2}}{2 s\, x} 
  \right]  = s\, e^{-s} \,
  \left[\frac{2}{3}~ {\rm sx} + \left(
  \frac{1}{15} - \frac{1}{6 s}\right){\rm sx}^3
  \right] \ .
\end{eqnarray}

\subsection{The Large-s Limit}

We can approximate the $s \gg 1$ limit analytically by
a saddle-point evaluation of the $x$-integral. For the 
value of the parameter $a$ passed to the subroutine dedxqi, 
this limit corresponds to
%%
\begin{eqnarray}
  s = \frac{1}{2}\, \beta_b m_b v_p^2 =
  \frac{r}{2}\, \frac{\beta_b}{\beta_m} \,
  \frac{m_b}{m}\, {\bar v}_p^2 \, \gg 1 \ .
\end{eqnarray}
%%
The dimensionless projectile velocity ${\bar v}_p$
is in units of the thermal velocity $v_{\rm th}$
defined in (\ref{vpvth}), for some reference species 
of mass $m$ and inverse temperature $\beta_m$. We will 
always take the reference species to be the electron. 
The value $r=3$ is the default, in which case $v_{\rm 
th}$ correspond to the thermal root-mean-squared 
velocity of the electron. Each plasma component will 
usually have a different value for $s$, with the
difference being especially acute between electrons
and ions because of the mass ratio $m_b/m$ (which is
one for electrons and several thousand for ions). We 
see that the large-$s$ limit encompasses several physical 
regimes, and is typically realized when: (i) the 
projectile velocity $v_p$ is much greater than the 
thermal velocity of the electron $v_{\rm th}$, so 
that ${\bar v}_p = v_p/v_{\rm th}$ is large, (ii) the 
temperature $T_b$ of species $b$ is much less than 
the electron temperature $T_m$, so that $\beta_b/
\beta_m$ is large, and (iii) when the species $b$ is
an ion, so that $m_b/m$ is large. For the case of ions,
the value of $s$ can be of order $10^3$ or more. However,
as the ion velocity decreases, the value of $s$ decreases 
as the square of the velocity, so $s$ need not always 
be large for ions. 


Let us now look at the saddle-point approximation to the 
integrals $Q_1$ and $Q_2$  in the case of large $s$. We 
will start with $Q_1$ and evaluate the integral for $s 
\gg 1$:
%%
\begin{eqnarray}
  Q_1(r,s) 
  &=&
  \frac{1}{2}\int_0^\infty dx \,G(r/x)\, 
  \left[\,e^{-s(x-1)^2} - e^{-s(x+1)^2} \right] 
\\[5pt]
  &=&
  \frac{1}{2}\int_{-1}^\infty dx \,G\!\left(\frac{r}{x+1}
  \right)\, \left[\,e^{-s x^2} - e^{-s(x+2)^2} \right] 
  \hskip1cm :~x \to x+1
\\[5pt]
  &\approx&
  \frac{1}{2}\int_{-1}^\infty dx \,G\!\left(\frac{r}{x+1}
  \right)\, e^{-s x^2} 
  \hskip1cm :{\rm dropping~the~second~exponential}
\\[5pt]
  &=&
  \frac{1}{2}\int_{-1}^\infty dx \left[G_0(r) + G_1(r) \, x +
  \frac{1}{2}G_2(r) \, x^2 + \cdots \right]e^{-s x^2} 
  \hskip0.1cm :{\rm expanding~} G {\rm ~in~~} x
\\[5pt]
\nonumber
  &\approx&
  \frac{1}{2}\int_{-\infty}^\infty dx
  \left[G_0(r) + G_1(r) \, x + \frac{1}{2}G_2(r) \, 
   x^2 + \cdots \right] e^{-s x^2} 
  \hskip0.1cm :{\rm extending~lower~int~limit}.
\end{eqnarray}
%%
The expansion coefficients are $x$-independent and 
given by 
%%
\begin{eqnarray}
  G_n(r) &=& \frac{d^n}{dx^n}\,G\!\left(\frac{r}{x+1}
  \right) \Bigg\vert_{x=0} ~~~\Rightarrow
\\[10pt]
  G_0(r) &=& G(r)
\\
  G_1(r) &=& - r G^\prime(r)
\\
  G_2(r) &=& 2 r G^\prime(r) + r^2 G^{\prime\prime}(r) \ ,
\end{eqnarray}
%%
with 


%%
\vbox{
\begin{eqnarray}
  G(y) &=& \ln(y) - {\rm Re}\,\psi(1 + i y) \equiv
  \ln(y) - {\rm repsi}(y)
\\[5pt]
  G^\prime(r) &=& \frac{1}{y} + {\rm Im}\,\psi^\prime(1+ i y)
  \equiv \frac{1}{y} - {\rm repsi1}(y)
\\
  G^{\prime\prime}(r) &=& -\frac{1}{y^2} + 
  {\rm Re}\,\psi^{\prime\prime}(1+ i y) 
  \equiv -\frac{1}{y^2} - {\rm repsi2}(y)  \ .
\end{eqnarray}
}%%

\noindent
For numerical use we will construct fits to the functions
$G$, $G^\prime$ and $G^{\prime\prime}$, or actually to 
${\rm repsi}$, ${\rm repsi1}$ and ${\rm repsi2}$, but first 
let's continue with the integral, which can now be performed 
exactly using, 


\vbox{
%%
\begin{eqnarray}
  && \int_{-\infty}^\infty dx e^{-s x^2} = \sqrt{\frac{\pi}{s}}
\hskip1cm   
  \int_{-\infty}^\infty dx\, x \, e^{-s x^2} = 0
\hskip1cm 
  \int_{-\infty}^\infty dx\, x^2 \, e^{-s x^2} = 
  \frac{\sqrt{\pi}}{2 s^{3/2}} \ ,
\end{eqnarray}
%%
}

\noindent
thereby giving  

%%
\vskip0.5cm 
\frame{.1}{10}{16.2cm}{\noindent
\begin{eqnarray}
  Q_1(r,s) 
  &\approx&
  \frac{\sqrt{\pi}}{2}\left[\, 
  \frac{G_0(r)}{s^{1/2}} + \frac{G_2(r)}{4 s^{3/2}}\,
  \right] \ .
\end{eqnarray}
}
%%


We can evaluate the integral $Q_2$ in a similar fashion, 
%%
\begin{eqnarray}
  Q_2(r,s) 
  &\approx&
  \frac{1}{2}\int_{-\infty}^\infty dx \, 
  G\left(\frac{r}{x+1}\right)\, 
  \left[\,\frac{1}{x+1}   - \frac{1}{2 s\, (x+1)^2} 
  \right]  \, e^{-sx^2}
\\[5pt]
  &=&
  \frac{1}{2}\int_{-\infty}^\infty dx
  \left[H_0(r,s) + H_1(r,s) \, x + \frac{1}{2}\,H_2(r,s) \,
   x^2 + \cdots \right] e^{-s x^2} \ ,
\end{eqnarray}
%%
where the expansion coefficients are 
%%
\begin{eqnarray}
  H_n(r,s) &=& \frac{d^n}{dx^n}\,G\!\left(\frac{r}{x+1}
  \right) \left[\,\frac{1}{x+1}   - \frac{1}{2 s\, (x+1)^2} 
  \right] \Bigg\vert_{x=0} ~~~\Rightarrow
\\[10pt]
  H_0(r,s) &=& G_0(r)\left( 1 - \frac{1}{2s} \right)
\\[5pt]
  H_1(r,s) &=& G_1(r)\left( 1 - \frac{1}{2s} \right)
  -G_0(r) \left( 1 - \frac{1}{s} \right)
\\[5pt]
  H_2(r,s) &=& G_2(r)
  \left( 1 - \frac{1}{2s} \right) 
  - 2\,G_1(r)\left( 1 - \frac{1}{s} \right) + 
  2\,G_0(r)\left( 1 - \frac{3}{2s} \right)\ ,
\end{eqnarray}
%%
and the integral therefore becomes

%%
\vskip0.5cm 
\frame{.1}{10}{16.2cm}{\noindent
\begin{eqnarray}
  Q_2(r,s) 
  &\approx&
  \frac{\sqrt{\pi}}{2}\left[\, 
  \frac{H_0(r,s)}{s^{1/2}} + \frac{H_2(r,s)}{4 s^{3/2}}\,
  \right] \ .
\end{eqnarray}
}
%%
In summary, for large $s$ the quantum integral can
be written as 
%%
\begin{eqnarray}
   && \hskip-5.4cm s \gg 1 : ~
\\
\nonumber
  I_\smQM \equiv 
  {\rm dedxqi}({\rm a},{\rm e},{\rm v},{\rm rm},{\rm rr})    
  &=& {\rm rm}\, Q_2({\rm r},{\rm s}) - {\rm rr}\,
  Q_1({\rm r},{\rm s})
\\
\nonumber &&
  {\rm r} = {\rm e}/{\rm v} {\rm ~and~} {\rm s}= {\rm a} 
  {\rm v}^2 \ ,
\end{eqnarray}
%%
with an accuracy of about $0.2\%$ for $s > s_{\rm max}=10$.
The corresponding subroutine is listed below, with most
of the variable declarations omitted for brevity. 


{
\baselineskip12pt
\begin{verbatim}
 FUNCTION dedxqi(v, a, e, rm, rr)
   IMPLICIT NONE     
   REAL, INTENT(IN) :: v, a, e, rm, rr
   REAL :: dedxqi
   ...  
   REAL,    PARAMETER :: SQPI =1.772453851    ! sqrt(pi)
   REAL,    PARAMETER :: SMAX=10.             ! cut on s
   ...

   r=e/v   
   s=a*v*v
   IF ( s .GT. SMAX) THEN          ! large s can be performed analytically
      g0 =LOG(ABS(r)) - repsi(r)   ! this case is usually realized for ions
      dg =1/r - repsi1(r)          ! 
      ddg=-1/(r*r) - repsi2(r)     ! 
      g1=-r*dg                     ! 
      g2=2*r*dg + r*r*ddg          !
      s1=1/s                       ! s1=1/s
      s2=s1/2                      ! s2=1/2*s
      s3=3*s2                      ! s3=3/2*s
      s05=SQRT(s)                  ! s05=s^(1/2)
      s15=s*s05                    ! s15=s^(3/2)
      h0=g0*(1-s2)
      h2=g2*(1-s2) - 2*g1*(1-s1) + 2*g0*(1-s3)
      q1=SQPI/2
      q2=SQPI/2
      q1=q1*(g0/s05 + 0.25E0*g2/s15)
      q2=q2*(h0/s05 + 0.25E0*h2/s15)
      dedxqi=rm*q2 - rr*q1
   ELSE                            ! otherwise do integral numerically
      ...
      Previously listed Gaussian quadrature method here
      ...
   ENDIF
 END FUNCTION dedxqi
\end{verbatim}
}

\noindent
This subroutine calls the three special functions ${\rm repsi}(x) 
= {\rm Re}\,\psi(1 + i x)$, ${\rm repsi1}(x) = -{\rm Im}\,
\psi^\prime(1 + i x)$, and ${\rm repsi2}(x) = -{\rm Re}\,
\psi^{\prime\prime}(1 + i x)$. The polygamma functions
are usually not part of the repertoire of standard mathematical
functions accompanying Fortran, so we will construct these
functions by finding analytic expressions for large and
small $x$ and by using fits to polynomials and other functions 
at intermediate values. Rather than using canned functions from
a mathematical library, constructing the functions we need ourselves 
will make the code more portable. 

\subsubsection{Fits to the Digamma Function and its Derivatives}

We collect here a number of useful results for finding expansions 
of the polygamma functions in the limit of large and small 
arguments. These results are taken from Ref.~\cite{abst}, Chapter~6 
on pp~259,260. In finding the small-$x$ expansion of $\psi(1 + i x)$ 
and its derivatives with respect to $x$ we use:
%%
\begin{eqnarray}
  \psi^{(n)}(1)&=&(-1)^{n+1}\, n!\,\zeta(n+1)
  ~~ n=1,2,3, \cdots   ~~~:6.4.2  ~ ,
\end{eqnarray}
%%
and $\psi(1)=-\gamma$, where $\gamma=0.57721 \cdots\,$ is
the Euler constant. For a general complex argument $z$
we can Taylor expand 
%%
\begin{eqnarray}
  \psi^{(n)}(1 + z) &=& \sum_{m=0}^\infty 
  \frac{1}{m!}\, \psi^{(n+m)}(1) \,  z^m =
  \sum_{m=0}^\infty 
  \frac{1}{m!}\, (-1)^{n+m+1}\, (n+m)!\,\zeta(n+m+1) 
  \,  z^m \ ,
\end{eqnarray}
%%
so that 
%%
\begin{eqnarray}
  \psi^{(n)}(1 + i x) &=& \sum_{m=0}^\infty 
  \frac{1}{m!}\, (-1)^{n+m+1}\, (n+m)!\,
  \zeta(n+m+1) \,  (i x)^m \ .
\end{eqnarray}
%%
For small $x$, the series can be truncated at the 
appropriate order to give the desired accuracy.


To find the large-argument expansion we first use
%%
\begin{eqnarray}
  \psi^{(n)}(1 + z) &=& \  \psi^{(n)}(z) + (-1)^n \, 
  n!\, z^{-(1+n)} ~~n=0,1,2,\cdots   ~~~: 6.4.6 ~ ,
\end{eqnarray}
%%
and then employ the asymptotically large limits
%%
\begin{eqnarray}
\nonumber
  && \hskip-1.5cm 
  \vert z \vert \to \infty ~{\rm with}~ {\rm arg}(z)<\pi :
\\
  \psi(z) &=& \ln z - \frac{1}{2z} - \frac{1}{12 z^2} +
  \frac{1}{120 z^4} - \frac{1}{252 z^6} + {\cal O}(z^{-8})
  \hskip2.4cm : 6.3.18
\\[10pt]
  \psi^{(1)}(z) &=& \frac{1}{z} + \frac{1}{2z^2} +
  \frac{1}{6 z^3} - \frac{1}{30 z^5} + \frac{1}{42 z^7}
  - \frac{1}{30 z^9} + {\cal O}(z^{-11})
  \hskip1.65cm :6.4.12
\\[10pt]
  \psi^{(2)}(z) &=& -\frac{1}{z^2} - \frac{1}{z^3} -
  \frac{1}{2 z^4} + \frac{1}{6 z^6} - \frac{1}{6 z^8}
  + \frac{3}{10 z^{10}} - \frac{5}{6 z^{12}} + {\cal O}(z^{-14})
  ~~ :6.4.13 ~ .
\end{eqnarray}
%%
For the case of interest, when the argument becomes 
$z=i x$, this gives
%%
\begin{eqnarray}
\nonumber
  \psi(1 + i x) &=& \psi(ix) + (ix)^{-1}
\\
\label{psitosix}
  &=&
   - \frac{i}{2 x} + \frac{i \pi}{2} + \ln x + 
  \frac{1}{12 x^2} + \frac{1}{120 x^4} + \frac{1}{252 x^6} 
  + {\cal O}(x^{-8})
\\[10pt]
\nonumber
  \psi^\prime(1 + i x) &=& \psi^\prime(i x) - (i x)^{-2}
\\
  &=&
  \frac{1}{2 x^2} -\frac{i}{x}  + \frac{i}{6 x^3} 
  + \frac{i}{30 x^5} + \frac{i}{42 x^7}
  + \frac{i}{30 x^9} + {\cal O}(x^{-11})
\\[10pt]
\nonumber
  \psi^{\prime\prime}(1 + i x) &=& \psi^{\prime\prime}(i x) 
  + 2(i x)^{-3}
\\
  &=&
  \frac{i}{x^3} + \frac{1}{x^2}  - \frac{1}{2 x^4} 
  - \frac{1}{6 x^6} - \frac{1}{6 x^8}
  - \frac{3}{10 x^{10}} - \frac{5}{6 x^{12}} + 
  {\cal O}(x^{-14}) \ .
\end{eqnarray}
%%
It is sufficient to keep only the first three terms
in each expansion, and so for large $x$ we have
%%
\begin{eqnarray}
  {\rm repsi}(x) = \phantom{-}
  {\rm Re}\,\psi(1 + i x) &=&
  \ln x + \frac{1}{12 x^2} + \frac{1}{120 x^4} + 
  {\cal O}(x^{-6})
\\[10pt]
  {\rm repsi1}(x) = 
  -{\rm Im}\,\psi^\prime(1 + i x) &=& 
  \frac{1}{x}  - \frac{1}{6 x^3} 
  - \frac{1}{30 x^5} - {\cal O}(x^{-7})
\\[10pt]
  {\rm repsi2}(x) = 
  -{\rm Re}\,\psi^{\prime\prime}(1 + i x) &=& 
  -\frac{1}{x^2}  + \frac{1}{2 x^4} 
  + \frac{1}{6 x^6} + {\cal O}(x^{-8}) \ .
\end{eqnarray}
%%

\pagebreak
\subsubsection{The Digamma Function: repsi}

The integrand calls the function ${\rm repsi}(x) \equiv 
{\rm Re}\,\psi(1 + i x)$. The asymptotic limits of 
${\rm repsi}(x)$ for large and small $x$ are
%%
\begin{eqnarray}
\nonumber
  x < x_{\rm min}=0.16: \hskip1cm \\  
\label{repsixmin}
  {\rm repsi}(x) &=& -\gamma + \zeta(3)x^2 + {\cal O}(x^4) 
  \hskip2.8cm {\rm max~error~} 0.1\% 
\\[10pt]
\nonumber
  x > x_{\rm max}=1.5: \hskip1cm \\  
\label{repsixmax}
  {\rm repsi}(x) &=& \ln x +\frac{1}{12 x^2} +
  \frac{1}{120\, x^4} + {\cal O}(x^{-6}) 
  \hskip1cm {\rm max~error~}0.1\% \ .
\end{eqnarray}
%%
For intermediate values between $x_{\rm min}$ and 
$x_{\rm max}$ we use the fit
%%
\begin{eqnarray}
  {\rm repsi}(x) = -\gamma + \,
  \frac{1}{2}\ln\left(1 + \frac{e^{2\gamma}\,x^4 +
  2 \zeta(3) x^2 }{(1+x^2)} \right)
  \left[1 - \frac{1}{10}\exp\left(- \frac{4x}{3} - 
  \frac{9}{8 x} \right)\right]^{-1} \ .
\end{eqnarray}
%%
which has an maximum error of 0.1\% around $x \sim 0.5$ 
and $x \sim 2$. The numerical values of the constants
are $\gamma=0.5772156649$, $\zeta(3)=1.202056903$, and 
$e^{2\gamma}=3.172218958$. The function ${\rm repsi}(x)$
is now listed:

{
\baselineskip12pt
\begin{verbatim}
 FUNCTION repsi(x)  
   USE globalvars
   IMPLICIT NONE
   REAL, INTENT(IN) :: x
   REAL, PARAMETER :: XMIN=0.16E0, XMAX=1.5E0
   REAL, PARAMETER :: TZETA3=2.404113806E0 ! 2*ZETA(3)
   REAL, PARAMETER :: A=0.1E0, B=1.33333E0, C=1.125E0
   REAL :: repsi
   REAL :: x2, x4
   IF (x .LE. XMIN) THEN
      x2=x**2
      repsi=-GAMMA + ZETA3*x2
   ELSEIF (x .GE. XMAX) THEN
      x2=x**2
      x4=x2*x2
      repsi=LOG(x)+1.D0/(12.D0*x2)+1.D0/(120.D0*x4)
   ELSE
      x2=x*x
      repsi=0.5E0*LOG(1 + (EXP2E*x2*x2 + TZETA3*x2)/(1+x2))
      repsi=repsi/(1 - A*EXP(-B*x - C/x)) - GAMMA
   ENDIF
 END FUNCTION repsi
\end{verbatim}
}
\vskip1cm

\subsubsection{The Derivative: repsi1}

The integrand also calls the function ${\rm repsi1}(x)=
\frac{d}{dx}\, {\rm Re}\, \psi(1 + i x)=- {\rm Im}\, 
\psi^\prime(1 + i x)$, which has the asymptotic limits
%%
\begin{eqnarray}
\nonumber
  x < x_{\rm min}=0.14: \hskip1cm \\  
  {\rm repsi1}(x) &=& 2 \zeta(3)\,x
  - 4 \zeta(5)\,x^3
  \hskip2.6cm {\rm max~error~} 0.1\% 
\\ \nonumber
  2 \zeta(3) &=& 2.404113806319188
  \hskip0.7cm 
  4\zeta(5) = 4.14771102057348
\\[10pt]
\nonumber
  x > x_{\rm max}=1.9 : \hskip1cm \\  
  {\rm repsi1}(x) &=& \frac{1}{x} - \frac{1}{6 x^3} -
  \frac{1}{30 x^5} + {\cal O}(x^{-7})
  \hskip1cm {\rm max~error~} 0.08\%   \ .
\end{eqnarray}
%%
In contrast to the previous case, the asymptotic limits
are polynomials, and at intermediate values of $x$ between 
$x_{\rm min}$ and $x_{\rm max}$  we can therefore perform 
polynomial fits. The function 
${\rm repsi1}(x)$ is always positive except at at $x=0$
where it vanishes; it then increases to a maximum of about 
${\rm repsi1}\sim 0.9$ around $x \sim 0.7$; and then decreases 
asymptotically to zero as $x$ increases further. We could 
easily fit the function between the entire region $x_{\rm min} 
< x < x_{\rm max}$ with a single polynomial; however, I found 
it more convenient to fit between $x_{\rm min}<x<x_1$ and $x_1 
< x < x_{\rm max}$, with $x_1=0.7$, using two separate polynomials. 
In the first region $x_{\rm min}<x<x_1$ the function increases 
monotonically, and in the second region $x_1 < x < x_{\rm max}$ 
it decreases monotonically, and therefore upon using separate 
polynomials in each region we can obtain more accuracy with 
a lower order. Least-square fits to $5^{\rm th}$ order give
%%
\begin{eqnarray}
  x_{\rm min} < x < x_1=0.7 : ~~~
  {\rm repsi1}(x) &=& \sum_{\ell=0}^5 a_\ell\, x^\ell
\\
 0.7= x_1 < x < x_{\rm max}: ~~~
  {\rm repsi1}(x) &=& \sum_{\ell=0}^5 b_\ell\, x^\ell \ ,
\end{eqnarray}
%%
with the coefficients given by 
%%
\begin{eqnarray}
  \begin{array}{lrr}
  \ell~~~~ & \hfill a_\ell \hfill & \hskip3cm b_\ell \hskip1.5cm  \\ \hline
  0    &  0.004211521868683916   &  -0.25386287337370820 \\[-8pt]
  1    &  2.314767988469241000   &   4.60092985583543200 \\[-8pt]
  2    &  0.761843932767193200   &  -6.76154044407838200 \\[-8pt]
  3    & -7.498711815965575000   &   4.46723854889984100 \\[-8pt]
  4    &  7.940030433629257000   &  -1.44439009761387350 \\[-8pt]
  5    &  2.749533936429732000   &   0.18595402917922707
  \end{array}  
\end{eqnarray}
%%

\noindent
See the Mathematica notebook IQM.nb for details.
The full subroutine for ${\rm repsi1}(x)$ becomes

{
\baselineskip12pt
\begin{verbatim}
!              d
! repsi1(x) = --- Re[ Psi(1 + I*x) = -Im Psi'(1 + I*x). 
!             dx
 FUNCTION repsi1(x)  
   IMPLICIT NONE
   REAL, INTENT(IN) :: x
   REAL :: repsi1

   REAL, PARAMETER :: XMIN=0.14E0, X1=0.7E0 ,XMAX=1.9E0
   REAL, PARAMETER :: ZETA32=2.404113806319188 ! 2*ZETA(3)
   REAL, PARAMETER :: ZETA54=4.147711020573480 ! 4*ZETA(5)
   REAL, PARAMETER :: a0= 0.004211521868683916
   REAL, PARAMETER :: a1= 2.314767988469241000
   REAL, PARAMETER :: a2= 0.761843932767193200
   REAL, PARAMETER :: a3=-7.498711815965575000
   REAL, PARAMETER :: a4= 7.940030433629257000
   REAL, PARAMETER :: a5=-2.749533936429732000
   REAL, PARAMETER :: b0=-0.253862873373708200
   REAL, PARAMETER :: b1= 4.600929855835432000
   REAL, PARAMETER :: b2=-6.761540444078382000
   REAL, PARAMETER :: b3= 4.467238548899841000
   REAL, PARAMETER :: b4=-1.444390097613873500
   REAL, PARAMETER :: b5= 0.185954029179227070
   REAL :: xi
   IF ( x .LE. XMIN) THEN             ! x < xmin=0.14 
      repsi1=ZETA32*x - ZETA54*x*x*x  ! accurate to 0.1%
   ELSEIF (x .LE. x1) THEN
      repsi1=a5                       ! xmin < x < x1=0.7
      repsi1=a4 + repsi1*x            ! accurate to 0.002%
      repsi1=a3 + repsi1*x            ! a0 + a1*x + a2*x^2 +
      repsi1=a2 + repsi1*x            ! a3* x^3 + a4*x^4 + a5*x^5
      repsi1=a1 + repsi1*x
      repsi1=a0 + repsi1*x
   ELSEIF (x .LE. xmax) THEN          ! x1 < x < xmax=1.9
      repsi1=b5                       ! accurate to 0.1%
      repsi1=b4+repsi1*x              ! b0 + b1*x + b2*x^2 + 
      repsi1=b3+repsi1*x              ! b3*x^3 + b4*x^4 +
      repsi1=b2+repsi1*x              ! b5*x^5
      repsi1=b1+repsi1*x              
      repsi1=b0+repsi1*x            
   ELSE
      xi=1/x                          ! x > xmax=1.9
      repsi1=-1.E0/30.E0              ! accurate to 0.08%
      repsi1=repsi1*xi                ! 1/x - 1/6x^3 - 1/30x^5
      repsi1=-1.E0/6.D0 + repsi1*xi   ! 
      repsi1=repsi1*xi                !
      repsi1=1.E0 + repsi1*xi         !
      repsi1=repsi1*xi                !
   ENDIF
 END FUNCTION repsi1
\end{verbatim}
}

Note the manner in which the polynomials have been 
calculated. One might be tempted to define the
values of the $a_\ell$ in an array ${\rm a}(0\!:\!5)$
and to write 

{
\baselineskip12pt
\begin{verbatim}
  repsi1=0
  DO i=0,5
    repsi1 = repsi1 + a(i)*(x**i)
  ENDDO                            ;
\end{verbatim}
}

\noindent
however, it takes far fewer operations to 
construct the polynomial backward in the
fashion 

{
\baselineskip12pt
\begin{verbatim}
  repsi1=0
  DO i=5,0,-1
    repsi1 = a(i) + repsi1*x
  ENDDO                            ;
\end{verbatim}
}

\noindent
or, as I have chosen to implement this procedure,

{
\baselineskip12pt
\begin{verbatim}
  repsi1=b5
  repsi1=b4 + repsi*x
  repsi1=b3 + repsi*x
  repsi1=b2 + repsi*x
  repsi1=b1 + repsi*x
  repsi1=b0 + repsi*x   .
\end{verbatim}
}

\noindent
In the latter there are only ten operations, five additions
and five multiplications. For an $N^{\rm th}$ order polynomial
this method would only require $2N$ operations. In the former 
there are $\ell+2$ operations for each loop $\ell=0, \cdots, 5$ 
(there are $\ell$ multiplications for $x^\ell$, another multiplication 
for $a_\ell \cdot x^\ell$, and a single addition), giving
a total of 27 operations. Again, for an $N^{\rm th}$ order
polynomial the former method would require $\sum_{\ell=0}^N
(\ell + 2) = (N+1)(N+4)/2$ operations. The more obvious former 
method is quadratic while the latter method is only linear.  

\subsubsection{The Second Derivative: repsi2}

The final polygamma function is ${\rm repsi2}(x)=\frac{d^2}{dx^2}\, 
{\rm Re}\, \psi(1 + i x)=- {\rm Re}\, \psi^{\prime\prime}(1 + i x)$,
which has the asymptotic forms
%%
\begin{eqnarray}
  x < x_{\rm min}=0.18: \hskip1cm \\  
\nonumber
  {\rm repsi2}(x) &=& 2 \zeta(3)  - 12 \zeta(5) \, x^2
 + 30 \zeta(7)\, x^4
  \hskip1cm {\rm max~error~} 0.1\% 
\\ \nonumber
  2 \zeta(3) &=& 2.404113806319188
  \hskip1.2cm 
  12 \zeta(5)  = 12.44313306172044
\\[-8pt] \nonumber
  30 \zeta(7) &=& 30.250478321457685
\\[10pt]
\nonumber
  x > x_{\rm max}=2.5: \hskip1cm \\  
  {\rm repsi2}(x) &=& -\frac{1}{x^2} + \frac{1}{2 x^4} +
  \frac{1}{6 x^6} + {\cal O}(x^{-8})
  \hskip1.2cm {\rm max~error~} 0.1\% \ .
\end{eqnarray}
%%
As before, it is convenient to divide the interval 
$x_{\rm min} < x < x_{\rm max}$ into the two regions
\hbox{$x_{\rm min} < x < x_1$} and $x_1< x < x_{\rm max}$,
this time with $x_1=1.2$. In the first interval the
function ${\rm repsi2}(x)$ drops monotonically from a 
maximum of ${\rm repsi2} \sim 2.4$ at $x=0$ to a minimum 
of ${\rm repsi2} \sim -0.4$ at $x \sim 1.2$ (the function
crosses zero at $x=0.678157$). In the second interval 
beyond $x \sim 1.2$, the function remains negative but
begins to increase and asymptotically approaches the 
$x$-axis from below as $x$ gets large. This is the
motivation for dividing the interval around $x_1=1.2$
and performing separate least-square fits in the two 
subintervals. Working to $7^{\rm th}$ and $6^{\rm th}$ 
orders for the first and second intervals gives 
%%
\begin{eqnarray}
  x_{\rm min} < x < x_1=1.2 : ~~~
  {\rm repsi1}(x) &=& \sum_{\ell=0}^7 a_\ell\, x^\ell
\\
 1.2= x_1 < x < x_{\rm max}: ~~~
  {\rm repsi1}(x) &=& \sum_{\ell=0}^6 b_\ell\, x^\ell \ ,
\end{eqnarray}
%%
with the coefficients
%%
\begin{eqnarray}
  \begin{array}{lrr}
  \ell~~~~ & \hfill a_\ell \hfill & \hskip3cm b_\ell \hskip1.5cm  \\ \hline
  0    &   2.42013533575662130  &   4.98436272402513600  \\[-8pt]
  1    &  -0.41115258967949336  & -16.65464665310595300  \\[-8pt]
  2    &  -8.09116694062588400  &  20.69413003620411000  \\[-8pt]
  3    & -24.93648245588276400  & -13.37268378509369200  \\[-8pt]
  4    & 114.81090561524718000  &   4.83094787289278800  \\[-8pt]
  5    &-170.85454523278196000  &  -0.92976482601030100  \\[-8pt]
  6    & 128.84024667658247000  &   0.07456475055097825  \\[-8pt]
  7    & -50.24590900103020600  &  
  \end{array}  
\end{eqnarray}
%%


\noindent
The full subroutine for ${\rm repsi2}(x)$ is listed below. 

{
\baselineskip12pt
\begin{verbatim}
!              d^2
! repsi2(x) = ---- Re[ Psi(1 + I*x) = -Re Psi''(1 + I*x). 
!             dx^2
 FUNCTION repsi2(x)  ! needs to be completed
   IMPLICIT NONE
   REAL, INTENT(IN) :: x
   REAL :: repsi2

   REAL, PARAMETER :: XMIN=0.18E0, X1=1.2E0 ,XMAX=2.5E0
   REAL, PARAMETER :: ZETA32=2.4041138063191880 ! 2*ZETA(3)
   REAL, PARAMETER :: ZETA512=12.44313306172044 ! 12*ZETA(5)
   REAL, PARAMETER :: ZETA730=30.250478321457685! 30*ZETA(7)
   REAL, PARAMETER :: a0= 2.42013533575662130
   REAL, PARAMETER :: a1=-0.41115258967949336
   REAL, PARAMETER :: a2=-8.09116694062588400
   REAL, PARAMETER :: a3=-24.9364824558827640
   REAL, PARAMETER :: a4=114.8109056152471800
   REAL, PARAMETER :: a5=-170.854545232781960
   REAL, PARAMETER :: a6=128.8402466765824700
   REAL, PARAMETER :: a7=-50.2459090010302060 
   REAL, PARAMETER :: a8= 8.09941032385266400
   REAL, PARAMETER :: b0= 4.98436272402513600
   REAL, PARAMETER :: b1=-16.6546466531059530
   REAL, PARAMETER :: b2= 20.6941300362041100
   REAL, PARAMETER :: b3=-13.3726837850936920 
   REAL, PARAMETER :: b4= 4.83094787289278800 
   REAL, PARAMETER :: b5=-0.92976482601030100
   REAL, PARAMETER :: b6= 0.07456475055097825
   REAL :: xi, xx
   IF ( x .LE. XMIN) THEN
      xx=x*x
      repsi2=ZETA32 - ZETA512*xx + ZETA730*xx*xx ! x < xmin=0.18
   ELSEIF (x .LE. x1) THEN                       ! accurate to 0.1%
      repsi2=a8                                  !
      repsi2=a7 + repsi2*x                       ! xmin < x < x1=1.2
      repsi2=a6 + repsi2*x                       ! accurate to 0.01% 
      repsi2=a5 + repsi2*x                       ! a0 + a1*x + a2*x^2 +
      repsi2=a4 + repsi2*x                       ! a3*x^3 + a4*x^4 +
      repsi2=a3 + repsi2*x                       ! a5*x^5 + a6*x^6 +
      repsi2=a2 + repsi2*x                       ! a7*x&7 + a8*x^8 
      repsi2=a1 + repsi2*x
      repsi2=a0 + repsi2*x 
   ELSEIF (x .LE. xmax) THEN                     ! x1 < x < xmax=2.5
      repsi2=b6                                  ! accurate to 0.2%
      repsi2=b5+repsi2*x                         ! b0 + b1*x + b2*x^2 +
      repsi2=b4+repsi2*x                         ! b3*x^3 + b4*x^4 +
      repsi2=b3+repsi2*x                         ! b5*x^5 + b6*x^6
      repsi2=b2+repsi2*x          
      repsi2=b1+repsi2*x          
      repsi2=b0+repsi2*x          
   ELSE
      xi=1/x                                     ! x > xmax=2.5
      xi=xi*xi                                   ! accurate to 0.07%
      repsi2= 1.E0/6.E0                          ! -1/x^2 + 1/2x^4 + 
      repsi2= 0.5E0 + repsi2*xi                  ! 1/6x^6
      repsi2=-1. + repsi2*xi                     !
      repsi2=repsi2*xi
   ENDIF
 END FUNCTION repsi2
\end{verbatim}
}
\vskip1cm


\pagebreak
\subsection{The Small-s Limit}

We can also analytically approximate the integrals $Q_1$ and 
$Q_2$ defined in (\ref{qonetwodef}) in the $s \ll 1$ limit. 
In this case it is convenient to change the integration variable 
to $y=\sqrt{a}\,u$, and to define the parameters
%%
\begin{eqnarray}
  r &=& \sqrt{a}\,e
\\
  s &=& a\, v^2 ~~~{\rm (as~before)} \ ,
\end{eqnarray}
%%
which allows us to express the integrals as
%%
\begin{mathletters}
\label{qonetwoagain}
\begin{eqnarray}
  Q_1(r,s) &=& 
  \frac{1}{\sqrt{s}}\, e^{-s}
  \int_0^\infty dy \, e^{-y^2} G(r/y)\, \sinh\left(2 \sqrt{s}\,
   y \right) 
\\[5pt]
  Q_2(r,s) &=&
  e^{-s}\int_0^\infty dy \, e^{-y^2}
  G(r/y)\, \frac{1}{y} 
  \left(\cosh\left(2 \sqrt{s}\, y \right) - 
  {\sinh\left(2 \sqrt{s}\, y\right) 
  \over 2 \sqrt{s}\, y} \right)  \ .
\end{eqnarray}
\end{mathletters}
%%
We will keep the parameter $r$ general, and expand the
hyperbolic functions for small arguments,
%%
\begin{eqnarray}
  \sinh x &=& x + \frac{x^3}{6} + \cdots 
\\[5pt]
  \frac{1}{x} 
  \left(\cosh x - \frac{\sinh x}{x} 
  \right) &=& \frac{x}{3} + \frac{x^3}{30} +
  \cdots \ .
\end{eqnarray}
%%
This gives
%%
\begin{eqnarray}
  Q_1(r,s) &=& 
  \frac{1}{\sqrt{s}}\, e^{-s}
  \int_0^\infty dy \, e^{-y^2} G(r/y)\, \left[
  (2 \sqrt{s}\, y) + \frac{1}{6}\,\left(2 \sqrt{s}\, y
  \right)^3  \right]
\\[5pt]
  Q_2(r,s) &=&
  2\sqrt{s}\,e^{-s}\int_0^\infty dy \, e^{-y^2}
  G(r/y)\, \left[
  \frac{1}{3}\,(2 \sqrt{s}\, y) + 
  \frac{1}{30}\,\left(2 \sqrt{s}\, y
  \right)^3  \right] \ ,
\end{eqnarray}
%%
or more succinctly 
%%
\begin{eqnarray}
  Q_1(r,s) &=& e^{-s}\left[
  2 K_1(r) + \frac{4\,s}{3}\, K_3(r)  \right]
\\[5pt]
  Q_2(r,s) &=& e^{-s}\left[ 
  \frac{4\, s}{3}\,  K_1(r) + \frac{8\,s^2}{15}\,  K_3(r)
  \right]\ ,
\end{eqnarray}
%%
with 
%%
\begin{eqnarray}
  K_1(x) &=& \int_0^\infty dy \, y\, e^{-y^2} G(x/y)
\\[5pt]
  K_3(x) &=& \int_0^\infty dy \, y^3\, e^{-y^2} G(x/y) \ .
\end{eqnarray}
%%
In this way we have reduced the 2-parameter integrals 
$Q_1(r,s)$ and $Q_2(r,s)$
to 1-parameter integrals $K_1(x)$ and $K_3(x)$, which 
we can now fit numerically (and find their analytic 
asymptotic limits for large and small $x$). The quantum 
integral for small-$s$ becomes, 
%%
\begin{eqnarray}
   && \hskip-5.4cm   s \ll 1 : ~
\\[5pt]
\nonumber
  I_\smQM \equiv 
  {\rm dedxqi}({\rm a},{\rm e},{\rm v},{\rm rm},{\rm rr})    
  &=& 
  e^{-s}\left[
  \left( \frac{4\,{\rm s}}{3}\,{\rm rm} - 2\,{\rm rr}
  \right) K_1({\rm r}) +
  \left( \frac{8\,{\rm s}^2}{15}\,{\rm rm} - \frac{4\,
  {\rm s}}{3}\,{\rm rr}\right) K_3({\rm r})
  \right] 
\\[5pt]
\nonumber &&
  {\rm r} = {\rm \sqrt{a}}\,{\rm v} {\rm ~and~} {\rm s}= {\rm a} 
  {\rm v}^2 \ ,
\end{eqnarray}
%%
with an accuracy between 0.2\% and 0.5\% for $s < s_{\rm min}=
0.05$ (as $r$ varies from 0.1 to 10 or so the accuracy 
decreases slightly). The relevant source code is listed 
below:


{
\baselineskip12pt
\begin{verbatim}
 FUNCTION dedxqi(v, a, e, rm, rr)
   IMPLICIT NONE     
  
   REAL, INTENT(IN) :: v, a, e, rm, rr
   REAL :: dedxqi
   ...  
   REAL,    PARAMETER :: SMAX=10., SMIN=0.05  ! cuts on s
   ...
   s=a*v*v
   IF ( s .GT. SMAX) THEN          
     ...
      Previously listed large-s method here
     ...
   ELSEIF ( s .LT. SMIN) THEN      ! small s can be performed analytically
      r=SQRT(a)*e                  ! this case is usually realized for elec.
      i1=kqm1(r)                   ! accuracy < 0.5%
      i2=kqm3(r)
      dedxqi=(4.*rm*s/3. - 2*rr)*i1
      dedxqi=dedxqi+ 4.*(4*rm*s*s/15.- rr*s/3.)*i2   
      dedxqi=EXP(-s)*dedxqi
   ELSE                            
      ...
      Previously listed Gaussian quadrature method here
      ...
   ENDIF
 END FUNCTION dedxqi
\end{verbatim}
}

\noindent
The functions $K_1(x)$ and $K_3(x)$ have been called 
${\rm kqm1}$ and ${\rm kqm3}$ in the above subroutine.

\pagebreak
\subsubsection{Fitting the Functions $K_1$ and $K_3$}

We now code the functions $K_1(x)$ and $K_2(x)$ by analytically
calculating the large and small $x$-limits, {\em i.e.} for $x 
< x_{\rm min}$ and $x > x_{\rm max}$ where the values $x_{\rm 
min}$ and $x_{\rm max}$ will chosen to give an accuracy of
order a tenth of a percent or so, and then by interpolating 
between these limits with least-squares fits. 

We will first concentrate on the $x > x_{\rm max}$ regime,
as this turns out to be the easier case, and we consider 
the functions
%%
\begin{eqnarray}
\label{kndef}
  K_n(x) &=& \int_0^\infty \!\!dy\, y^n \, e^{-y^2} G(x/y) 
\\[5pt]
\nonumber
  {\rm with}~G(x)&=&\ln x - {\rm Re}\,\psi(1 + i x)
\end{eqnarray}
%%
for positive integral values of $n$. In the end we are only 
interested in the cases \hbox{$n=1$ and $3$}; however, for
clarity and to simplify the intermediate expressions  we will 
work with a general $K_n$ for now. The exponent in (\ref{kndef}) 
ensures that the integrand is negligible unless $y \lesssim 1$, 
and so the argument of $G(x/y)$ remains large when $x$ is large. 
We can expand $G(x/y)$ using (\ref{psitosix}), which for $x > 
x_{\rm max}$ (where $x_{\rm xmax}$ is to be determined) allows 
us to write
%%
\begin{eqnarray}
  G(x/y) = -\frac{y^2}{12 x^2} -
  \frac{y^4}{120\, x^4} - \frac{y^6}{252\, x^6} 
  + {\cal O}(x^{-8}) \ ,
\end{eqnarray}
%%
and therefore 
%%
\begin{eqnarray}
  K_n(x) &=& -\int_0^\infty \!\!dy \, e^{-y^2} 
  \left[ \frac{y^{2+n}}{12 x^2} +
  \frac{y^{4+n}}{120\, x^4}  +   
  \frac{y^{6+n}}{252\, x^4}  + 
  \right] + {\cal O}(x^{-8}) \ .
\end{eqnarray}
%%
For $n=1$ and $3$ we need to evaluate the integrals,
%%
\begin{eqnarray}
  \int_0^\infty \!\!dy \, y^3 \, e^{-y^2} = \frac{1}{2}
  \hskip0.6cm 
  \int_0^\infty \!\!dy \, y^5 \, e^{-y^2} = 1
  \hskip0.6cm 
  \int_0^\infty \!\!dy \, y^7 \, e^{-y^2} = 3
  \hskip0.6cm 
  \int_0^\infty \!\!dy \, y^9 \, e^{-y^2} = 12 \ ,
\end{eqnarray}
%%
which gives 
%%
\begin{eqnarray}
\nonumber
  && \hskip-2cm x > x_{\rm max}=3.2 :
\\[3pt]
  K_1(x) &=& -\frac{1}{24\, x^2} -\frac{1}{120\, x^4}
  - \frac{1}{84\, x^6}
 \hskip1cm {\rm max~error}~ 0.1\%
\\[15pt]
\nonumber
  && \hskip-2cm x > x_{\rm max}=2.5 :
\\[3pt]
  K_3(x) &=& -\frac{1}{12\, x^2} -\frac{1}{40\, x^4}
  - \frac{3}{64\, x^6}
 \hskip1.2cm {\rm max~error}~ 0.1\% ~~ .
\end{eqnarray}
%%
By working to such a high order we are able to obtain
an accuracy of less than 0.1\% for reasonably small
values of $x_{\rm max}$ of order one. 

Let us now find an analytic approximation to $K_n(x)$ for
$x < x_{\rm min}$, where we will choose $x_{\rm min}$ to 
give an accuracy of about 0.1\%. This limit is more involved 
than the previous case. One might be tempted to use the 
small-$x$ limit (\ref{repsixmin}) for ${\rm repsi}(x)$
and to write 
%%
\begin{eqnarray}
  K_n(x) &=& \int_0^\infty dy\, y^n \, e^{-y^2} \bigg[\,
  \ln(x/y) + \gamma - \zeta(3) (x/y)^2 + \cdots \,\bigg] \ .
\end{eqnarray}
%%
Indeed, for $n \ge 2$ this is a reasonably good approximation,
and for the case of interest we find 
%%
\begin{eqnarray}
  K_3(x) &=& 
  \int_0^\infty dy\, y^3 \, e^{-y^2} \bigg[\,
  \ln(x/y) + \gamma - \zeta(3) (x/y)^2 + \cdots \,\bigg]
\\[5pt]
  &=&
  (\,\ln \!x + \gamma) \, 
  \underbrace{\int_0^\infty \!\!dy\, y^3 \, e^{-y^2}}_{1/2} - 
  \underbrace{\int_0^\infty \!\!dy\, y^3 \, \ln y \, 
  e^{-y^2}}_{(1-\gamma)/4} -
  \zeta(3) x^2 \underbrace{\int_0^\infty \!\!dy\, y \, 
  e^{-y^2}}_{1/2}  + \cdots 
\\[8pt]
  &=&
  \frac{1}{2}\,\ln\!x + \frac{3\gamma}{4} - \frac{1}{4} -
  \frac{1}{2}\,\zeta(3) x^2 + \cdots \ .
\end{eqnarray}
%%
This expansion is accurate to 0.1\% for $x \!<\! x_{\rm min}=
0.15$. The problem arises when one uses (\ref{repsixmin})
to approximate $K_1$:
%%
\begin{eqnarray}
  K_1(x) &=& 
  \int_0^\infty dy\, y \, e^{-y^2} \bigg[\,
  \ln(x/y) + \gamma  - \zeta(3)(x/y)^2 \cdots \,\bigg]
\\[5pt]
  &=&
  (\,\ln \!x + \gamma) \, 
  \underbrace{\int_0^\infty \!\!dy\, y \, e^{-y^2}}_{1/2} - 
  \underbrace{\int_0^\infty \!\!dy\, y \ln y \, 
  e^{-y^2}}_{-\gamma/4} - \zeta(3) x^2 \hskip-0.6cm
  \underbrace{\int_0^\infty \!\!\frac{dy}{y} \, 
  e^{-y^2}}_{{\rm log~divergence~at~}y=0} + \cdots 
\\[8pt]
\label{K1first}
  &=&
  \frac{1}{2}\,\ln\!x + \frac{3\gamma}{4} +   \cdots \ .
\end{eqnarray}
%%
Ignoring the divergence and using only the first two terms
to approximate $K_1$ gives an accuracy of 3.5\% for $x \sim
0.1$, and the accuracy improves to only 2\% for $x \sim 0.01$.
For many applications, this might be good enough; however,
we have consistently achieved an accuracy of 0.1\%, so we
must achieve that figure of merit here as well. The problem,
of course, is that the quadratic term containing $\zeta(3) x^2$
is proportional to $y^{-2}$. Combined with the single 
factor of $y$ in $K_1$, this term leads to the logarithmic 
singularity at $y=0$. A singularity arises only for the $n=1$
case, as the integral of $y^{n-2}$ converges at $y=0$ for $n 
\ge 2$. To find $K_1$, we  must therefore be more careful in 
expanding $G(x/y)$. The full expansion of the digamma function
gives 
%%
\begin{eqnarray}
  {\rm Re}\,\psi(1 + i z) = -\gamma + \sum_{k=1}^\infty
  \frac{1}{k}\, \frac{z^2}{k^2 + z^2} \ .
\end{eqnarray}
%%
We have already calculated the contribution from the
first term. We found that trouble arose when, for
small arguments $z$, we approximated  the sum by 
%%
\begin{eqnarray}
\label{ksumapprox}
  \sum_{k=1}^\infty \frac{1}{k}\, 
  \frac{z^2}{k^2 + z^2}  \approx 
  \sum_{k=1}^\infty \frac{z^2}{k^3} = 
  \zeta(3)\, z^2 \ .
\end{eqnarray}
%%
Inside the integral we must use this approximation
with extreme care, since $z=x/y$ becomes large as 
$y \to 0$ regardless of how small we take $x$. We 
can make the sum look similar to (\ref{ksumapprox}) 
by writing 
%%
\begin{eqnarray}
  \sum_{k=1}^\infty
  \frac{1}{k}\, \frac{z^2}{k^2 + z^2} =
  \sum_{k=1}^\infty
  \frac{x^2}{k^3}\,\frac{d}{d y^2} \ln\left(
  k^2 y^2/x^2 + 1 \right) \ .
\end{eqnarray}
%%
The contribution from the sum,  which we must
add to (\ref{K1first}), can therefore be written
%%
\begin{eqnarray}
  \bar K_1(x) 
  &\equiv& 
  -\int_0^\infty \!\!dy\, y \, e^{-y^2}
  \sum_{k=1}^\infty
  \frac{1}{k}\, \frac{x^2}{k^2 + x^2}
  = -\sum_{k=1}^\infty \frac{x^2}{k^3}\,
  \int_0^\infty \!\!dy\,y\,  e^{-y^2} \frac{d}{d y^2} 
  \ln\left(k^2 y^2/x^2 + 1 \right)
\\[8pt]
  &=&
  -\frac{1}{2}\sum_{k=1}^\infty \frac{x^2}{k^3}\,
  \int_0^\infty \!\!du\,  e^{-u} \frac{d}{d u} 
  \ln\left(k^2 u/x^2 + 1 \right)
\\[8pt]
  &=&
  -\frac{1}{2}\sum_{k=1}^\infty \frac{x^2}{k^3}\,
  \int_0^\infty \!\!du\,  e^{-u} 
  \ln\left(k^2 u/x^2 + 1 \right) \ .
\end{eqnarray}
%%
In the last equality we have integrated by parts, 
%%
\begin{eqnarray}
\nonumber
  \int_0^\infty \!\!du\,  e^{-u} \frac{d}{d u} 
  \ln\left(k^2 u/x^2 + 1 \right)
  &=&
  e^{-u} \ln\left(k^2 u/x^2 + 1 \right) 
  \Bigg\vert_{u=0}^\infty +
  \int_0^\infty \!\!du\,  e^{-u} 
  \ln\left(k^2 u/x^2 + 1 \right)
\\[8pt]
  &=&
  \int_0^\infty \!\!du\,  e^{-u} 
  \ln\left(k^2 u/x^2 + 1 \right) \ .
\end{eqnarray}
%%
The previous log-divergence at $u=0$ (or $y=0$) is
now regulated by the $1$ inside the logarithm, which
consequently vanishes at $u=0$. Completing the
integral is now straightforward:
%%
\begin{eqnarray}
  \bar K_1(x) 
  &=&
  -\frac{1}{2}\sum_{k=1}^\infty \frac{x^2}{k^3}\,
  \int_0^\infty \!\!du\,  e^{-u} \Bigg[
  \ln\left(\frac{u}{x^2}\right) +
  \underbrace{\ln\left( k^2 + \frac{x^2}{u} 
  \right)}_{k~{\rm dep.~isolated~here }} \, \Bigg]
%\\[5pt]
%  &=&
%  \frac{x^2}{2}\, \zeta(3) \int_0^\infty \!\!du\, 
%  e^{-u} \ln\!\left(\frac{x^2}{u}\right) -
%  \frac{x^2}{2}\sum_{k=1}^\infty \frac{1}{k^3}\,
%  \int_0^\infty \!\!du\,  e^{-u} 
%  \ln\!\left( k^2 + \frac{x^2}{u} \right)
\\[15pt]
  &=&
  x^2 \ln\! x\, \underbrace{\sum_{k=1}^\infty
  \frac{1}{k^3}}_{\zeta(3)} ~ \underbrace{
  \int_0^\infty \!\!du\,\, e^{-u}}_{1} ~-~
  \frac{x^2}{2}\, \underbrace{\sum_{k=1}^\infty
  \frac{1}{k^3}}_{\zeta(3)} ~  \underbrace{
  \int_0^\infty \!\!du\,\ln\! u\,e^{-u}}_{
  -\gamma}
\\[5pt] && \hskip1cm 
\nonumber
  -x^2\underbrace{\sum_{k=1}^\infty \frac{\ln k}
  {k^3}}_{-\zeta^\prime(3)} ~ \underbrace{
  \int_0^\infty \!\!du\,  e^{-u}}_{1} ~-~
  \frac{x^2}{2}\underbrace{\sum_{k=1}^\infty 
  \frac{1}{k^3}\,\int_0^\infty \!\!du\,  e^{-u} 
  \ln\!\left( 1 + \frac{x^2}{u k^2} \right)}_{
  {\rm remainder}}    
\\[5pt]
  &=&
  \zeta(3)\, x^2 \ln x \,+\, 
  \bigg[\, \frac{1}{2} \,\zeta(3)\, \gamma +
  \zeta^\prime(3) \,\bigg]\, x^2 \,+\, {\rm 
  remainder} \ .
\end{eqnarray}
%%
We have now found both the leading-log term ($x^2\ln x$) 
and the sub-leading contribution (the $x^2$ piece). In 
other words, we have found the complete contribution of 
the form $A x^2 \ln x + B x^2 =A\, x^2 \ln(C x)$, which 
amounts to finding the constant in front of the log ($A=
\zeta(3)$ in this case) {\em and} the constant $C$ inside 
the log (or $B=A \ln C = \zeta^\prime(3) + \zeta(3) 
\gamma/2$). For small enough $x$, which will turn out to 
be of order a tenth, the remainder term can be neglected 
and still provide an accuracy of about 0.1\%. Finally, one 
comment on $\zeta^\prime(3)=-0.198126$ is in order. This 
is the derivative of the zeta-function $\zeta(s)$ evaluated
at $s=3$. Normally one thinks of the zeta-function as being 
defined at integer values of $s$, but one can analytically 
continue to the complex $s$-plane since the series
%%
\begin{eqnarray}
  \zeta(s) = \sum_{k=1}^\infty \frac{1}{k^s} 
\end{eqnarray}
%%
converges for ${\rm Re}\, s > 1$. We can then take the 
derivative of the zeta-function:
%%
\begin{eqnarray}
  \zeta^\prime(s) = \sum_{k=1}^\infty \frac{d}{ds}\,
  k^{-s}
  = - \sum_{k=1}^\infty \frac{\ln k}{k^s} \ .
\end{eqnarray}
%%
In summary, we have derived the small-$x$ limits
%%
\begin{eqnarray}
\nonumber
  && \hskip-1.5cm x < x_{\rm min}=0.15 :
\\[3pt]
  K_1(x) &=& \frac{1}{2}\,\ln\!x + \frac{3\gamma}{4} +
  \zeta(3)\, x^2 \ln x \,+\, 
  \bigg[\, \zeta^\prime(3) + \frac{1}{2} \,\zeta(3)
  \gamma \, \bigg]\, x^2
 \hskip1cm {\rm max~error}~ 0.2\%
\\[15pt]
\nonumber
  && \hskip-1.5cm x < x_{\rm min}=0.15 :
\\[3pt]
  K_3(x) &=& 
  \frac{1}{2}\,\ln\!x + \frac{3\gamma}{4} - 
  \frac{1}{4} - \frac{1}{2}\,\zeta(3)\, x^2 
 \hskip4.9cm {\rm max~error}~ 0.1\% ~~ .
\end{eqnarray}
%%
The numerical value of the quadratic term in $K_1$
is $\zeta^\prime(3) + \zeta(3)\gamma/2=0.148797$.

At intermediate values we perform least-squares fits
to the following functional forms, 
%%
\begin{eqnarray}
\nonumber
  && \hskip-1.55cm 0.15=x_{\rm min} \!<\! x \!<\! x_{\rm max}=3.2 :
\\
  K_1(x) &=& 
  \sum_{\ell=0}^4 c_\ell\, x^\ell  +
  \sum_{\ell=1}^4 \frac{d_\ell}{x^\ell} + d_0\ln\!x
 \hskip1cm {\rm max~error}~ 0.2\%
\\[15pt]
\nonumber
  &&\hskip-1.55cm 0.15=x_{\rm min} \!<\! x \!<\! x_{\rm max}=2.5 :
\\
  K_3(x) &=& 
  \sum_{\ell=0}^4 c_\ell\, x^\ell  +
  \sum_{\ell=1}^4 \frac{d_\ell}{x^\ell} + d_0\ln\!x
 \hskip1cm {\rm max~error}~ 0.04\% ~~ ,
\end{eqnarray}
%%
giving the coefficients
%%
\begin{eqnarray}
  \begin{array}{l|lrr}
 K_1  &\ell~~~~& \hfill c_\ell \hfill & \hskip3cm d_\ell \hskip1.5cm  \\ \hline
      &  ~0 &  0.25109815055856566000  & -0.18373957827052560000    \\[-8pt]
      &  ~1 & -0.02989283169766254700  & -0.33121125339783110000    \\[-8pt]
      &  ~2 &  0.03339880139150325000  &  0.04022076263527408400    \\[-8pt]
      &  ~3 & -0.00799128953390392700  & -0.00331897950305779480    \\[-8pt]
      &  ~4 &  0.00070251863606810650  &  0.00012313574797356784    \\[-8pt]
 \end{array}
\end{eqnarray}
%%
and
%%
\begin{eqnarray}
  \begin{array}{l|lrr}
 K_1  &   \ell~~~~ & \hfill c_\ell \hfill   & \hskip3cm d_\ell \hskip1.5cm \\ \hline
      &  ~0 &  0.691191700599840900000 &  0.835543536679762600000   \\[-8pt]
      &  ~1 & -1.094849572900974000000 &  0.047821976622976340000   \\[-8pt]
      &  ~2 &  0.318264617154601400000 &  0.000053594881446931025   \\[-8pt]
      &  ~3 & -0.060275957444801354000 & -0.000268040997573199600   \\[-8pt]
      &  ~4 &  0.005112428730167831000 &  0.000015765134162582942   \\[-8pt]
 \end{array}
\end{eqnarray}
%%

\vskip0.3cm 
\noindent
The source code for these functions, called kqm1 and kqm3 in Fortran,
is now listed:

{
\baselineskip12pt
\begin{verbatim}
!
!             /Infinity
!             |
! kqm1(x)  =  | dy   y exp(-y^2) [ln(x/y) - repsi(x/y)]
!             |
!            /0
!
!
 FUNCTION kqm1(x)  
   IMPLICIT NONE
   REAL, INTENT(IN) :: x
   REAL :: kqm1

   REAL, PARAMETER :: XMIN=0.15E0, XMAX=3.2E0 

   REAL, PARAMETER :: a0= 0.4329117486761496454549449429 ! 3*GAMMA/4
   REAL, PARAMETER :: a1= 1.2020569031595942854250431561 ! ZETA(3)
   REAL, PARAMETER :: a2= 0.1487967944177345026410993331 ! ZETA'(3)+GAMMA*
   REAL, PARAMETER :: b2=-0.0416666666666666666666666667 !-1/24  ZETA(3)/2
   REAL, PARAMETER :: b4=-0.0083333333333333333333333333 !-1/120
   REAL, PARAMETER :: b6=-0.0119047619047619047619047619 !-1/84
   REAL, PARAMETER :: c0= 0.25109815055856566000
   REAL, PARAMETER :: c1=-0.02989283169766254700
   REAL, PARAMETER :: c2= 0.03339880139150325000
   REAL, PARAMETER :: c3=-0.00799128953390392700
   REAL, PARAMETER :: c4= 0.00070251863606810650
   REAL, PARAMETER :: d0=-0.18373957827052560000
   REAL, PARAMETER :: d1=-0.33121125339783110000
   REAL, PARAMETER :: d2= 0.04022076263527408400
   REAL, PARAMETER :: d3=-0.00331897950305779480
   REAL, PARAMETER :: d4= 0.00012313574797356784
   REAL :: x2, lx, xi
   IF ( x .LE. XMIN) THEN                        ! x < xmin=0.15: to 0.06%
      x2=x*x                                     ! ln(x)/2 + 3*GAMMA/4 +
      lx=LOG(x)                                  ! ZETA(3)*X^2*ln(x) +
      kqm1=0.5E0*lx + a0 + a1*x2*lx + a2*x2      ! [ZETA'(3) + GAMMA*
                                                 ! ZETA(3)/2]*x^2
                                                 !
   ELSEIF ( x .GE. XMAX ) then                   ! x > xmax=3.2: to 0.12%
      xi=1/x                                     ! -1/24*x^2 - 1/120*x^4 -
      x2=xi*xi                                   ! 1/84*x^6
      kqm1=b6                                    !
      kqm1=b4 + kqm1*x2                          !
      kqm1=b2 + kqm1*x2                          !
      kqm1=kqm1*x2                               !
   ELSE
      xi=1/x                                     ! xmin < x < xmax
      lx=LOG(x)                                  ! fit accurate to 0.2%
      kqm1=c4                                    ! c0 + c1*x + c2*x^2 +
      kqm1=c3+kqm1*x                             ! c3*x^3 + c4*x^4 +
      kqm1=c2+kqm1*x                             ! d0*ln(x) +
      kqm1=c1+kqm1*x                             ! d1/x + d2/x^2 +
      kqm1=c0+kqm1*x + d0*lx                     ! d3/x^3 + d4/x^4
      lx=d4                                      !
      lx=d3+lx*xi                                !
      lx=d2+lx*xi                                !
      lx=d1+lx*xi                                !
      lx=lx*xi                                   !
      kqm1=kqm1+lx
   ENDIF
 END FUNCTION kqm1

!
!             /Infinity
!             |
! kqm3(x)  =  | dy   y^3 exp(-y^2) [ln(x/y) - repsi(x/y)]
!             |
!            /0
!
!
 FUNCTION kqm3(x)  
   IMPLICIT NONE
   REAL, INTENT(IN) :: x
   REAL :: kqm3

   REAL, PARAMETER :: XMIN=0.15E0, XMAX=2.5E0 
   REAL, PARAMETER :: a0= 0.1829117486761496454549449429 ! 3*GAMMA/4 - 1/4
   REAL, PARAMETER :: a2=-0.6010284515797971427073328102 !-ZETA(3)/2
   REAL, PARAMETER :: b2=-0.0833333333333333333333333333 !-1/12
   REAL, PARAMETER :: b4=-0.025                          !-1/40
   REAL, PARAMETER :: b6=-0.046875                       !-3/64
   REAL, PARAMETER :: c0= 0.691191700599840900000
   REAL, PARAMETER :: c1=-1.094849572900974000000
   REAL, PARAMETER :: c2= 0.318264617154601400000
   REAL, PARAMETER :: c3=-0.060275957444801354000
   REAL, PARAMETER :: c4= 0.005112428730167831000
   REAL, PARAMETER :: d0= 0.835543536679762600000
   REAL, PARAMETER :: d1= 0.047821976622976340000
   REAL, PARAMETER :: d2= 0.000053594881446931025
   REAL, PARAMETER :: d3=-0.000268040997573199600
   REAL, PARAMETER :: d4= 0.000015765134162582942
   REAL :: x2, lx, xi
   IF ( x .LE. XMIN) THEN                        ! x < xmin=0.15: to 0.1%
      x2=x*x                                     ! ln(x)/2 + 3*GAMMA/4 -1/4
      lx=LOG(x)                                  ! -[ZETA(3)/2]*x^2
      kqm3=0.5E0*lx + a0 + a2*x2                 ! 
                                                 ! 
                                                 !
   ELSEIF ( x .GE. XMAX ) then                   ! x > xmax=2.5: to 0.25%
      xi=1/x                                     ! -1/12*x^2 - 1/40*x^4 -
      x2=xi*xi                                   ! 3/64*x^6
      kqm3=b6                                    !
      kqm3=b4 + kqm3*x2                          !
      kqm3=b2 + kqm3*x2                          !
      kqm3=kqm3*x2                               !
   ELSE
      xi=1/x                                     ! xmin < x , xmax
      lx=LOG(x)                                  ! fit accurate to 0.04%
      kqm3=c4                                    ! c0 + c1*x + c2*x^2 +
      kqm3=c3+kqm3*x                             ! c3*x^3 + c4*x^4 +
      kqm3=c2+kqm3*x                             ! d0*ln(x) +
      kqm3=c1+kqm3*x                             ! d1/x + d2/x^2 +
      kqm3=c0+kqm3*x + d0*lx                     ! d3/x^3 + d4/x^4
      lx=d4                                      !
      lx=d3+lx*xi                                !
      lx=d2+lx*xi                                !
      lx=d1+lx*xi                                !
      lx=lx*xi                                   !
      kqm3=kqm3+lx
   ENDIF
 END FUNCTION kqm3
\end{verbatim}
}

\newpage
\section{The Classical Contribution}

We now turn to the classical contribution
%%
\begin{eqnarray}
  {dE^\smC_b \over dx}&=&
  \frac{c_1}{\bar v_p}\hskip0.2cm\cdot\hskip0.2cm
  \kappa^2_b E_b \hskip0.2cm\cdot\hskip0.2cm
   I_1(A_b \bar v_p,\, B_b \bar v_p^2, \, C_b) 
  ~~+~~
  c_2\hskip0.1cm\cdot\hskip0.1cm I_2(\bar v_p)
  ~~-~~ 
  \frac{c_3}{r_b \bar v_p^2} \hskip0.2cm\cdot\hskip0.2cm
   \bar H_b(\bar v_p) \ ,
\\[-12pt]\nonumber  &&
  \hskip-0.2cm\underbrace{\hskip0.3cm}_{{\rm cp1/vp}}
  \hskip0.3cm \underbrace{\hskip0.5cm}_{{\rm kb2*eb}}
  \hskip0.7cm \underbrace{\hskip3cm}_{{\rm intone}}
  \hskip1.0cm\underbrace{\hskip0.3cm}_{{\rm cp2}}
  \hskip0.2cm \underbrace{\hskip1.4cm}_{{\rm inttwo}}
  \hskip0.3cm\underbrace{\hskip0.3cm}_{{\rm cp3/vp2*rb}}
  \hskip0.2cm \underbrace{\hskip1.2cm}_{{\rm hi}}
\end{eqnarray}
%%
in which 
%%
\begin{eqnarray}
  I_1(a,b,c) &=&
  \int_0^1 du \,e^{ -a^2 u }
  \Bigg\{ \frac{2}{\sqrt{u}} + \left[ c - \ln \left(  
  { u \over 1-u} \right)  \right]\left[  b\,\sqrt{u} 
  -  \frac{1}{\sqrt{u}} \right] \Bigg\} 
\nonumber\\[10pt]
  I_2(a) &=&
  \int_{-1}^{+1} du \, u \,\bar H_b(a \,u)  \ .
\end{eqnarray}
%%

\noindent
The general structure of the classical routine dedxc is 

\vskip0.3cm 
{
\noindent
dedx.f90:
\baselineskip12pt
\begin{verbatim}
 SUBROUTINE dedxc(vp, dedxctot, dedxcsumi)
   USE globalvars
   ...
   ...
   A. loop over plasma species ib=1,NNB (ib=1 is electron)
   B. compute classical stopping power
      1. small velocity limit
      2. or numerical evalulation
         a. compute intone (I1)
         b. compute inttwo (I2)
         c. compute hi
   ...
   ...
 END SUBROUTINE dedxc
\end{verbatim}
}
\vskip0.2cm

\noindent
or in complete form

\vskip0.5cm 
{
\noindent
dedx.f90:
\baselineskip12pt
\begin{verbatim}
 SUBROUTINE dedxc(vp, dedxctot, dedxcsumi)
   USE globalvars
   
   IMPLICIT NONE
   REAL, INTENT(IN) :: vp
   REAL, INTENT(OUT):: dedxctot, dedxcsumi

   REAL, PARAMETER        :: ABV20=0.001
   REAL, DIMENSION(1:NNB) :: abv, abv2, clb
   REAL                   :: abv2ib, ss, vp2, kd2, kd4
   REAL                   :: hi, inttwo, intone, a, b, c, ke
   INTEGER                :: ib
!
! define input variables
!
   vp2=vp*vp
   abv=ab*vp      ! for intone, inttwo, hi
   clb=0          ! initialize classical to zero
   DO ib=1,NNB    ! loop over plasma components
      IF ( lzb(ib) ) THEN ! computle only if zb(ib) /= 0 
         abv2=abv*abv                 !
         abv2ib=abv2(ib)              ! cut on each component
         IF (abv2ib .LT. ABV20) THEN  ! small velocity limit is analytic
            kd2 =kd*kd
            kd4 =kd2*kd2
            clb(ib)=clb(ib)+2*gb(ib)*(1-abv2ib*(1 + &
                 (2./3.)*(mp0/mb0(ib))))
            clb(ib)=clb(ib)-(4./3.)*abv2ib -2*SUM(kb2*abv2)/kd2
            ss=(SUM(kb2*abv))
            ss=ss*ss
            clb(ib)=clb(ib)+(PI/6.)*ss/kd4
            clb(ib)=clb(ib)*cp1*kb2(ib)*eb(ib)/vp 
         ELSE                  ! general velocities are numerical
!
! int1: dedxc=(cp1/vp)*intone(abv,bbv,kb2,eb,cb)
!
            a=abv(ib)
            b=bb(ib)*vp2
            c=cb(ib)
            ke=kb2(ib)*eb(ib)
            clb(ib)=clb(ib)+(cp1/vp)*ke*intone(a,b,c)
!       
! int2: dedxc=dedxc + cp2*inttwo(abv,kb2) 
!
            clb(ib)=clb(ib)+cp2*inttwo(ib,abv)
!
! H: dedxc=dedxc - (cp3/vp2)*h(abv,kb2)
!
            clb(ib)=clb(ib)-(cp3/vp2)*hi(ib,abv)/rb(ib)
         ENDIF
      ELSE
         clb(ib)=0  ! don't compute when zb(ib) = 0
      ENDIF
   ENDDO
   dedxcsumi=SUM(clb(2:NNB))
   dedxctot =clb(1)
   dedxctot =dedxctot+dedxcsumi

   dedxctot =CONVFACT*dedxctot   ! convert to MeV/mu-m
   dedxcsumi=CONVFACT*dedxcsumi  !
 END SUBROUTINE dedxc
\end{verbatim}
}

\noindent
We will look at each of these processes in reverse
order, starting with the function hi, then inttwo,
and finally intone.



\subsection{Contribution $\bm{H_b}$}


The function ${\rm hi(ib,abv)}$ calculates 
$\bar H_b(\bar v_p)$ with $b={\rm ib}$, where we have
defined 
%%
\begin{eqnarray}
\label{Hbx}
  \bar H_b(\bar v_p) = 
  i \,{\bar \rho_b(\bar v_p) \over \bar \rho_{\rm total}(\bar v_p) } 
  \Bigg[ \bar F(\bar v_p) \ln\left( {\bar F(\bar v_p) \over K^2 } 
  \right) - \bar F^*(\bar v_p)\ln \left( {\bar F^*(\bar v_p) 
  \over K^2 } \right) \Bigg] \ ,
\end{eqnarray}
%%
with the real and imaginary parts of $F$ being
%%
\begin{mathletters}
\begin{eqnarray}
  \bar F_\smR(\bar v_p)
  &=& 
  \sum_b \kappa_b^2 \bigg[1 - 2 x_b\,
  {\rm daw}\left(x_b \right) \bigg]
  ~~~~~ x_b = A_b \bar v_p
\\[10pt]
  \bar F_\smI(\bar v_p)
  &=&
  \sqrt{\pi} \sum_b \kappa_b^2\, x_b \, e^{-x_b^2} 
  = \pi  \bar \rho_{\rm total}(\bar v_p)  \ .
\end{eqnarray}
\end{mathletters}
%% 
Let us first look at the subroutine ${\rm fri(xb, fr, fi, 
fabs, farg)}$ that returns the necessary parts of $\bar F$. 
The arguments are described by 




\vskip0.5cm
{\center
\vbox{
\begin{tabular}{|l|l|}\hline
~variable name~~  &~description~~              \\\hline
~xb               &~$x_b = A_b \bar v_p$ array              \\[-5pt]
~fr               &~$F_r$ array                             \\[-5pt]
~fi               &~$F_i$ array                             \\[-5pt]
~fabs             &~$|F|=\sqrt{F_r^2 + F_i^2}$ array        \\[-5pt]
~farg             &~array of angles between real and imag   \\\hline
\end{tabular}
} 
}

\vskip0.5cm
\noindent
and the subroutine itself is
\vskip0.5cm 

{
\noindent
dedx.f90:
\baselineskip12pt
\begin{verbatim}
 SUBROUTINE fri(xb, fr, fi, fabs, farg)
   USE globalvars
   IMPLICIT NONE
   REAL, DIMENSION(1:NNB), INTENT(IN)  :: xb
   REAL,                   INTENT(OUT) :: fr,  fi, fabs, farg
   REAL    :: x, daw, d
   INTEGER :: ib
   fr=0
   fi=0
   DO ib=1,NNB
      x=xb(ib)
      d=daw(x)
      fr=fr+(kb2(ib)*(1-2*x*d))
      fi=fi+kb2(ib)*x*EXP(-x*x)
   ENDDO
   fi=fi*SQPI
   fabs=SQRT(fr*fr + fi*fi)
   farg=ATAN2(fi,fr)
 END SUBROUTINE fri 
\end{verbatim}
}

To calculate the function ${\rm hi}({\rm ib},{\rm abv})$,
we write (\ref{Hbx}) as
%%
\begin{eqnarray}
  \bar H(\bar v_p) &\equiv&  i \,\bigg[ \bar F \ln(\bar F/K^2) - 
  \bar F^*\ln(\bar F^*/K^2) \bigg]
\\[5pt]
\label{hbsimp}
  \bar H_b(\bar v_p) &=& \pi \bar H(\bar v_p) \, 
  \frac{\rho_b(x_b)}{\bar F_\smI(\bar v_p)}
 = \pi \bar H(\bar v_p) \, \frac{\kappa_b^2 x_b e^{-x_b^2}/
  \sqrt{\pi}}{\sqrt{\pi} \sum_c \kappa_c^2\, x_c\, 
  e^{-x_c^2} }
  ~~~ x_b = A_b \bar v_p \ .
\end{eqnarray}
%%
The logarithm can be written
%%
\begin{eqnarray}
\nonumber
  \ln \bar F &=& \ln( \bar F_\smR + i \bar F_\smI) = 
  \ln|\bar F|e^{i\,{\rm arg} \bar F}
\\
  &=& \ln|\bar F| + i \, {\rm arg} \bar F \ .
\end{eqnarray}
%%
For the reflection properties to hold we must choose 
$-\pi < {\rm arg}\bar F \le \pi$. Upon expanding $\bar H$ 
we find
%%
\begin{eqnarray}
\nonumber
  \bar H 
  &=&  
  -2\, {\rm Im}\, \{ \bar F \ln(\bar F/K^2) \}
  =  -2\, (\bar F_\smR\, {\rm Im}\ln(\bar F/K^2)
  + \bar F_\smI\, {\rm Re} \ln(\bar F/K^2) )
\\
  &=&
  -2\, [\, \bar F_\smR\, {\rm arg}\bar F
  + \bar F_\smI\,\ln(|\bar F|/K^2) \, ]
\nonumber
\end{eqnarray}
%%
The subroutine hi now takes the form

\vskip0.5cm 
{
\noindent
dedx.f90:
\baselineskip12pt
\begin{verbatim}
 FUNCTION hi(ib, abv)
   USE globalvars
   IMPLICIT NONE
   REAL,    DIMENSION(1:NNB), INTENT(IN) :: abv
   INTEGER,                   INTENT(IN) :: ib
   REAL                                  :: hi
   REAL, PARAMETER :: EPS=1.E-6, XB2MAX=10.D0
   REAL    :: fr, fi, fabs, farg, h, ebc
   REAL    :: xb, xb2, xc, xc2, xbc
   INTEGER :: ic
   CALL fri(abv, fr, fi, fabs, farg)
   h=-2*(fi*LOG(fabs/K**2) + fr*farg)
   xb =abv(ib)
   xb2=xb*xb
!
! Express Fi in terms of its sum and write 
!
!   hi = kb^2*xb*Exp[-xb^2]/Sum_c kc^2*xc*Exp[-xc^2]
!
! then calculate hi^-1 first, excluding kb^2*xb since
! some components of kb^2*xb might be zero.
   hi=0                                       
   DO ic=1,NNB                               
      xc =abv(ic)                             
      xc2=xc*xc                              
      xbc=xb2-xc2                            
      IF (xbc .LT. XB2MAX) THEN                
         ebc=EXP(xb2-xc2)                    
      ELSE                                   
         hi=0    ! ebc is large and dominates the sum, 
         RETURN  ! and its inverse is almost zero
      ENDIF
      hi=hi + kb2(ic)*xc*ebc                 
   ENDDO
   hi=kb2(ib)*xb/hi  ! invert and include kb^2*xb
   hi=h*hi
 END FUNCTION hi
\end{verbatim}
}

\vskip0.2cm
\noindent
In calculating (\ref{hbsimp}) we must be careful about
numerically multiplying powers of large and small 
exponents together. Numerically, we will find that
%%
\begin{eqnarray}
 e^{-({\rm very~large})} \times e^{+\, {\rm large}} =
 e^{-({\rm very~large})} \times {\rm number} = 0
 ~~~({\rm to~machine~precision}) \ ,
\end{eqnarray}
%%
when in fact the product of the two exponentials should
be a very small but nonzero number. To avoid this and
obtain more accuracy, we shall first multiply the
exponentials by hand,
%%
\begin{eqnarray}
 e^{-({\rm very~large})} \times e^{+\, {\rm large}}=
 e^{-({\rm very~large})\,+{\rm large}} \ ,
\end{eqnarray}
%%
since the combined exponent will not be as large 
(in absolute value) as either one individually.
In fact, we essentially calculate reciprocal
%%
\begin{eqnarray}
  \bar H_b^{-1} &=& 
  \sum_c \kappa_c^2\, x_c\,e^{x_b^2-x_c^2}~
  \frac{1}{\kappa_b^2 x_b\, \bar H} \ ,
\end{eqnarray}
and then take the reciprocal again. 


\subsection{Contribution $\bm{I_2}$}

%%
\begin{eqnarray}
  I_2(a) &=&
  \int_{-1}^{+1} du \, u \,\bar H_b(a \,u) =
  2 \int_0^1 du \, u \,\bar H_b(a \,u)  \ .
\end{eqnarray}
%%


{
\noindent
dedx.f90:
\baselineskip12pt
\begin{verbatim}
 FUNCTION inttwo(ib, abv)
   USE globalvars
    
   IMPLICIT NONE     
   REAL, DIMENSION(1:NNB), INTENT(IN) :: abv
   INTEGER,                INTENT(IN) :: ib
   REAL                               :: inttwo
   
   REAL, PARAMETER :: UPM=0.7745966692E0
   REAL, PARAMETER :: W13=0.5555555556E0, W2=0.8888888889E0
   
   INTEGER, PARAMETER :: NG=2000  ! NG must be even
   REAL,    PARAMETER :: U0=0.E0, U1=1.E0, DU=1.E0/NG
   
   REAL    :: u, um, hi, uu(NNB)
   INTEGER :: iu, ibb
   
   inttwo=0
   u=U0-DU
   DO iu=1,NG,2
!
      u=u+2.E0*DU
      DO ibb=1,NNB
         uu(ibb)=u*abv(ibb)
      ENDDO
      inttwo=inttwo+W2*u*hi(ib,uu)
!
      um=u-DU*UPM
      DO ibb=1,NNB
         uu(ibb)=um*abv(ibb)
      ENDDO
      inttwo=inttwo+W13*um*hi(ib,uu)
!
      um=u+DU*UPM
      DO ibb=1,NNB
         uu(ibb)=um*abv(ibb)
      ENDDO
      inttwo=inttwo+W13*um*hi(ib,uu)
   ENDDO
   inttwo=2*inttwo*DU
 END FUNCTION inttwo
\end{verbatim}
}


\subsection{Contribution $\bm{I_1}$}


Recall that the first classical contribution is
%%
\begin{mathletters}
\begin{eqnarray}
  I_1(a,b,c)
  &=&
  \int_0^1 \hskip-0.1cm du \, e^{-a^2 u} \Bigg\{ 
  \frac{2}{\sqrt{u}} + \left[ c - \ln\hskip-0.1cm 
  \left(\frac{u}{1-u} \right) \right] \Bigg[b \sqrt{u} 
  - \frac{1}{\sqrt{u}} \Bigg] \, \Bigg\} 
\\[10pt]\nonumber
  &=& 
   \sqrt{\pi}\, {\rm erf}(a)\hskip-0.1cm
  \left[\frac{2-c}{a} + \frac{bc}{2 a^3} \right] 
  - \frac{bc}{a^2}\,e^{-a^2} \hskip-0.1cm 
  + J_3(a^2) - J_1(a^2) + b\bigg[J_2(a^2) - J_4(a^2) 
  \bigg] \ ,
\\
\end{eqnarray}
\end{mathletters}
%%
where the functions $J_1 \cdots J_2$ are defined

\pagebreak
%%
\begin{mathletters}
\begin{eqnarray}
  J_1(x) &\equiv&
  \int_0^1 du\,e^{-x u} \, \frac{\ln(1-u)}
  {\sqrt{u}} 
\\[5pt]
  J_2(x) &\equiv&
  \int_0^1 du\,e^{-x u} \, \sqrt{u}\, 
  \ln(1-u)
\\[5pt]
  J_3(x) &\equiv& 
  \int_0^1 du\,e^{-x u} \, \frac{\ln u}
  {\sqrt{u}} 
\\[5pt]
  J_4(x) &\equiv &
  \int_0^1 du\,e^{-x u} \, \sqrt{u}\, 
  \ln \hskip-0.05cm u  \ .
\end{eqnarray}
\end{mathletters}
%%

\noindent
The source code is
\hskip0.2cm 

{
\noindent
dedx.f90:
\baselineskip12pt
\begin{verbatim}
 FUNCTION intone(a, b, c) 
   USE globalvars

   IMPLICIT NONE
   REAL, INTENT(IN) :: a, b, c
   REAL             :: intone

   REAL    :: bc, a2, a3, erfa, expa, ferf
   REAL    :: j1, j2, j3, j4
   bc=b*c
   a2=a*a
   a3=a2*a
   erfa=SQPI*ferf(a)  ! see ferf.f
   expa=EXP(-a2)
   intone=erfa*((2-c)/a + bc/(2*a3))-bc*expa/a2
   intone=intone + j3(a2) - j1(a2) + b*(j2(a2) -j4(a2))
 END FUNCTION intone
\end{verbatim}
}

\noindent
For a detailed evaluation of the integrals and limits
that follow, see intJ1234.nb. For the moment note that
we have performed a few of the integrals exactly, namely
%%
\begin{mathletters}
\begin{eqnarray}
  \int_0^1 du\,e^{-a^2 u}\,\frac{1}{\sqrt{u}} 
  &=& \frac{\sqrt{\pi}}{a}\,{\rm erf}(a)
\\[5pt]
  \int_0^1 du\,e^{-a^2 u} \,\sqrt{u} \, 
  &=&
  \frac{{\sqrt \pi}}{2 a^3}\,{\rm erf}(a) -
  \frac{e^{-a^2}}{a^2} \ ,
\end{eqnarray}
\end{mathletters}
%%

Let us now look at the integrals $J_1 \, \cdots \, J_4$,
which can be expressed in terms of hypergeometric functions
%%
\begin{eqnarray}
  \phantom{.}_p F_q({\bf a}, {\bf p},z)
  \equiv
  \sum_{k=0}^\infty \frac{(a_1)_k \cdots 
  (a_p)_k} {(b_1)_k \cdots (b_q)_k}\,
  \frac{z^k}{k!} \ ,
\end{eqnarray}
%%
where ${\bf a}$ and ${\bf b}$ are $p$ and $q$ dimensional 
vectors, respectively, and the Pochhammer symbol is
%%
\begin{mathletters}
\begin{eqnarray}
 (s)_0 &\equiv& 1
\\
 (s)_k &\equiv& s(s+1) \cdots (s+k-1) \ .
\end{eqnarray}
\end{mathletters}
%%
The integrals $J_3$ and $J_4$ are easily expressed
as
%%
\begin{mathletters}
\begin{eqnarray}
  J_3(a^2) &=&
  \int_0^1 du\,e^{-a^2 u} \, \frac{\ln u} {\sqrt{u}} 
  =
  -4 \phantom{.}_2 F_2({\bf 1}/2,\,{\bf 3}/2,- a^2)
  \hskip0.5cm  
\\[5pt]
  J_4(a^2) &=&
  \int_0^1 du\,e^{-a^2 u} \, \sqrt{u}\, 
  \ln \hskip-0.05cm u 
  =
  -\frac{4}{9} \phantom{.}_2 F_2({\bf 3}/2,\,{\bf 5}/2,- a^2) 
\\[5pt] \nonumber
&& \hskip2.5cm 
   {\bf 1}=(1,1) ~~~ {\bf 3} = 3 \cdot {\bf 1} ~~~
   {\bf 5} = 5 \cdot {\bf 1} \ .
\end{eqnarray}
\end{mathletters}
%%
The integrals $J_1$ and $J_2$ are somewhat more complicated,
%%
\begin{mathletters}
\begin{eqnarray}
  J_1(a^2) &\equiv&
  \int_0^1 du\,e^{-a^2 u} \, \frac{\ln(1-u)}
  {\sqrt{u}} 
  =
  \sqrt{\pi}\, \bigg[
  -\,\gamma \, \frac{{\rm erf}(a)}{a} ~+~
  \,_1\bar F_1(1/2,\,3/2,\,-a^2)
  \bigg]
\\[5pt]
  J_2(a^2) &\equiv&
  \int_0^1 du\,e^{-a^2 u} \, \sqrt{u}\, 
  \ln(1-u)
  = \gamma \, \frac{e^{-a^2}}{a^2} +
  \frac{\sqrt{\pi}}{2a^2}\,
  \bigg[\hskip-0.1cm 
  -\gamma \frac{{\rm erf}(a)}{a} +
  \,_1\bar F_1(1/2,\,3/2,-a^2)
\\\nonumber && \hskip9.65cm 
  +~ \,_1\bar F_1(3/2,\,3/2,\,-a^2)
  \bigg] \ ,
\end{eqnarray}
\end{mathletters}
%%
where $\gamma = 0.577216 \cdots$ is the Euler gamma constant, 
and where the functions $\,_1\bar F_1(x,y,z)$ are defined to 
be the regularized hypergeometric functions $\,_1 F_1(x,y,z)/
\Gamma(y)$ with a derivative of the second argument:
%%
\begin{eqnarray}
  \,_1\bar F_1(x,y,z) \equiv \frac{\partial}{\partial y}\,
  \frac{\,_1 F_1(x,y,z)}{\Gamma(y)} \ .
\end{eqnarray}
%%
However, these forms for $J_1 \cdots J_4$ are not very
useful numerically. 


\subsubsection{$J_1(x)$}

We now consider the integral
%%
\begin{eqnarray}
  J_1(x) &\equiv&
  \int_0^1 du\,e^{-x\, u} \, \frac{\ln(1-u)}
  {\sqrt{u}} \ ,
\end{eqnarray}
%%
where $x \ge 0$. We can obtain an analytic solutions for small and large 
values of the argument,


\pagebreak
%%
\begin{eqnarray}
\nonumber
  x \ll 1: \hskip1cm && 
\\[-5pt]
  J_1(x) &=& m x + b  
  \hskip2.7cm  
   m= \frac{4}{9}\,\left( 4 - \ln8 \right) 
  ~~~ b=-(4-\ln16)  
\\\nonumber &&
  \hskip-1.3cm {\rm error~} 0.3\% {\rm ~for~} x< x_{\rm min}=0.1
\\[10pt]
\nonumber
  x \gg 1: \hskip1cm &&
\\
  J_1(x) &=& a\, x^{-3/2} + g \, x^{-5/2}
  \hskip1cm  a=-\frac{\sqrt{\pi}}{2}
  ~~~   g=-\frac{3\sqrt{\pi}}{8} \ .
\\\nonumber &&
  \hskip-1.3cm {\rm error~} 0.3\% {\rm ~for~} x> x_{\rm max}=20
\end{eqnarray}
%%
In the former case, we expand the exponent to linear order 
in $x$, while in the latter case the support lies near 
$u \sim 1$ and we replace the upper limit by infinity and 
expand $\ln(1-u)=-u - u^2/2 + {\cal O}(u^3)$. 

For intermediate points between $x_{\rm min}$ and $x_{\rm max}$ 
we will approximate the function by the product of rational functions 
%%
\begin{eqnarray}
  J_1(x) = R(x) \, Q_6(x) \ .
\end{eqnarray}
%% 
The function $R(x)$ is designed to capture the
asymptotic behavior of $J_1$, and we take
%%
\begin{eqnarray}
\\\nonumber
  R(x)= 
  \underbrace{~\frac{m x + b}{c\, x^{7/2} + 1}~~}_{x\ll 1:~mx+b}  + 
  \underbrace{~~\frac{e\, a\, x^2}{e\, x^{7/2} +1}~~}_{x\gg 1:~ 
  a\, x^{-3/2}}\ ,
\end{eqnarray}
%%
where $x^2$ in the numerator of the second term is chosen
so that this term dominates over the first term at large $x$.
In summary, we take
%%
\begin{eqnarray}
\begin{array}{lrr}
  {\rm coeff} ~~~ & {\rm numerical} & \hskip2cm {\rm exact}    \\ \hline
  m               &   0.843565            & 4(4-\ln8)/9        \\[-8pt]
  b               &  -1.22741             & -(4-\ln16)         \\[-8pt]
  c               &  0.1                  & {\rm arbitrary}    \\[-8pt]
  e               &  0.2                  & {\rm arbitrary}    \\[-8pt]
  a               & -0.886227             & -\sqrt{\pi}/2 
\end{array}
\end{eqnarray}
%
% J1MM= 0.843565E0
% J1BB=-1.22741E0
% J1CC= 0.1E0
% J1EE= 0.2E0
% J1AA=-0.886227E0
\noindent
For $n=6$ we need $2n-1=11$ data points: we will take
$m$ to start at zero and end at ten, and we choose 
%%
\begin{eqnarray}
\begin{array}{lll}
  m~~~~ & x_m~~~ & J_1(x_m)   \\ \hline
  0    & 0.1       &   -1.14532     \\[-8pt]
  1    & 0.2       &   -1.06942     \\[-8pt]
  2    & 0.5       &   -0.874098    \\[-8pt]
  3    & 1.0       &   -0.633428    \\[-8pt]
  4    & 2.0       &   -0.35207     \\[-8pt]
  5    & 3.0       &   -0.211855    \\[-8pt]
  6    & 4.0       &   -0.137756    \\[-8pt]
  7    & 5.0       &   -0.0960144   \\[-8pt]
  8    & 6.0       &   -0.0709186   \\[-8pt]
  9    & 7.0       &   -0.0548658   \\[-8pt]
 10    & 8.0       &   -0.04401 
\end{array}
\end{eqnarray}
%%
and solving the linear equations gives
%%
\begin{eqnarray}
  \begin{array}{lrr}
  \ell~~~~ & \hfill b_\ell \hfill & \hskip1cm a_\ell \hskip1.5cm  \\ \hline
  0    &    -921.277      &   -921.277    \\[-8pt]
  1    &     781.631      &    774.26     \\[-8pt]
  2    &    -327.675      &   -211.002    \\[-8pt]
  3    &     39.2781      &   -1.1106     \\[-8pt]
  4    &    -0.278985     &   33.5015     \\[-8pt]
  5    &   -1.68781       &  -11.749      \\[-8pt]
  \end{array}
\end{eqnarray}
%%


{
\noindent
globalvars.f90:
\baselineskip12pt
\begin{verbatim}
! j1() approximates the j1 integral by rations
! functions with coefficients:
!
  INTEGER, PARAMETER         :: NNJ1=3, NMJ1=2*NNJ1-1 ! NMJ1=5
  REAL,    DIMENSION(0:NMJ1) :: J1B, J1A
  PARAMETER (           &
  J1B=(/                &
   -926.65E0  ,         & !b0
   787.016E0  ,         & !b1
   -329.764E0 ,         & !b2
   39.7406E0  ,         & !b3
   -0.173896E0,         & !b4
   -1.66913E0 /),       & !b5
  J1A=(/                &
    J1B(0),             & !a0
    787.165E0 ,         & !a1
    -213.584E0,         & !a2
    -1.04219E0,         & !a3
    33.594E0  ,         & !a4
   -11.8391E0/)         ) !a5
  REAL, PARAMETER :: J1MM= (4./9.)*(4.-LOG8)    ! 0.8535815 
  REAL, PARAMETER :: J1BB=-(4.-LOG16)           !-1.2274113
  REAL, PARAMETER :: J1AA=-SQPI/2.              !-0.8862270
  REAL, PARAMETER :: J1CC= 0.1E0
  REAL, PARAMETER :: J1EE= 0.2E0
  REAL, PARAMETER :: J1GG=-3.*SQPI/8.
\end{verbatim}
}

{
\noindent
dedx.f90:
\baselineskip12pt
\begin{verbatim}
 FUNCTION j1(x)
   USE globalvars   
   IMPLICIT NONE
   REAL, INTENT(IN) :: x
   REAL :: j1

   REAL,    PARAMETER   :: XMIN=0.1D0, XMAX=20.D0
   REAL    :: x2, x4
   REAL    :: xx, ra, rc
   REAL    :: y, y3, y5
   INTEGER :: n
!
! analytic asymptotic forms
!
  IF (x .LE. XMIN) THEN
     j1=J1MM*x+J1BB
  ELSEIF (x .GT. XMAX) THEN
     y=SQRT(x)           ! x^1/2
     y3=x*y              ! x^3/2
     y5=y3*x             ! x^5/2
     j1=J1AA/y3 +J1GG/y5
  ELSE
!
! numerical asymptotic form
!
     x2=x*x
     x4=x**3.5
     j1=(J1MM*x+J1BB)/(J1CC*x4+1) + J1EE*J1AA*x2/(J1EE*x4+1)
!
! spline correction
! 
     ra=0.E0
     rc=0.E0
     xx=1.E0 
     DO n=0,NMJ1
        ra=ra+J1A(n)*xx
        rc=rc+J1B(n)*xx
        xx=x*xx
     ENDDO
     ra=ra+xx
     rc=rc+xx
     j1=j1*rc/ra
  ENDIF
 END FUNCTION j1
\end{verbatim}
}


\subsubsection{$J_2(a)$}

We will now consider the integral
%%
\begin{eqnarray}
  J_2(x) &\equiv&
  \int_0^1 du\,e^{-x u} \, \sqrt{u}\, 
  \ln(1-u)
\end{eqnarray}
%%
where $x \ge 0$. The analytic solutions at asymptotic
values of $x$ are

%%
\vbox{
\begin{eqnarray}
\nonumber
  x \ll 1: \hskip1cm &&
\\[-5pt]
  J_2(x) &=& m x + b 
  \hskip2.5cm
  m= \frac{4}{75}\,(23 - 15\ln2)
  ~~~ b=-\frac{4}{9}\,( 4 - \ln8) 
\\\nonumber &&
  \hskip-1.3cm {\rm error~} 0.3\% {\rm ~for~} x< x_{\rm min}=0.1
\\[10pt]
\nonumber
  x \gg 1: \hskip1cm &&
\\
  J_2(x) &=& a\,x^{-5/2} + g\,x^{-7/2} 
  \hskip0.8cm a = - \frac{3\sqrt{\pi}}{4}
  ~~~ g =  - \frac{15\sqrt{\pi}}{16} \ .
\\\nonumber &&
  \hskip-1.3cm {\rm error~} 0.3\% {\rm ~for~} x> x_{\rm min}=30
\end{eqnarray}
%%
}

\noindent
As before, in the former case we expand the exponent to linear 
order in $x$, and in the latter case the support lies near 
$u \sim 1$ and we replace the upper limit by infinity and 
expand $\ln(1-u)=-u - u^2/2 + {\cal O}(u^2)$. 

For intermediate values of $x$ between $x_{\rm min}$ and
$x_{\rm max}$ we approximate the integral by
%%
\begin{eqnarray}
  J_2(x) = R(x) \, Q_6(x) \ ,
\end{eqnarray}
%%
where we take
%%
\begin{eqnarray}
\\\nonumber
  R(x)= 
  \underbrace{~\frac{m x + b}{c\, x^{9/2} + 1}~~}_{x\ll 1:~mx+b}  + 
  \underbrace{~~\frac{e\, a\, x^2}{e\, x^{9/2} +1}~~}_{x\gg 1:~ 
  a\, x^{-5/2}}\ ,
\end{eqnarray}
%%
with $x^2$ in the numerator of the second term being chosen
so that this term dominates over the first term at large $x$.
In summary, we take
%%
\begin{eqnarray}
\begin{array}{lrr}
  {\rm coeff} ~~~ & {\rm numerical} & \hskip3cm {\rm exact}    \\ \hline
  m               &   0.663779            & 4(23 - 15\ln2)/75  \\[-8pt]
  b               &  -0.853582            & -4( 4 - \ln8)/9    \\[-8pt]
  c               &   0.5                 & {\rm arbitrary}    \\[-8pt]
  e               &   0.2                 & {\rm arbitrary}    \\[-8pt]
  a               &  -1.32934             & -3\sqrt{\pi}/4 
\end{array}
\end{eqnarray}
%! J2MM= 0.663779E0
%! J2BB=-0.853582E0
%! J2CC= 0.5E0
%! J2EE= 0.2E0
%! J2AA=-1.32934E0

\noindent
For $n=6$ we need $2n-1=11$ data points: we will take
$m$ to start at zero and end at ten, and we choose 
\vskip-0.8cm

%%
\vbox{
\begin{eqnarray}
\begin{array}{lll}
  m~~~~ & x_m~~~   & J_2(x_m)       \\ \hline
  0    & 0.1       &  -0.789096     \\[-8pt]
  1    & 0.2       &  -0.729766     \\[-8pt]
  2    & 0.3       &  -0.675165     \\[-8pt]
  3    & 0.5       &  -0.578622     \\[-8pt]
  4    & 1.0       &  -0.396407     \\[-8pt]
  5    & 2.0       &  -0.192906     \\[-8pt]
  6    & 3.0       &  -0.0992046    \\[-8pt]
  7    & 4.0       &  -0.054271     \\[-8pt]
  8    & 5.0       &  -0.0316815    \\[-8pt]
  9    & 6.0       &  -0.0197116    \\[-8pt]
 10    & 7.0       &  -0.0130054 
\end{array}
\end{eqnarray}
}
%

\noindent
and solving the linear equations gives
\vskip-0.2cm

%%
\vbox{
\begin{eqnarray}
  \begin{array}{lrr}
  \ell~~~~ & \hfill b_\ell \hfill & \hskip1cm a_\ell \hskip1.5cm  \\ \hline
  0    &    82.3208      &    82.3208    \\[-8pt]
  1    &   -263.406      &   -262.701    \\[-8pt]
  2    &    316.627      &    315.347    \\[-8pt]
  3    &     -176.8      &   -178.477    \\[-8pt]
  4    &    55.9024      &    58.7802    \\[-8pt]
  5    &   -8.50148      &   -9.99801    \\[-8pt]
  \end{array}
\end{eqnarray}
%%
}

{
\noindent
globalvars.f90:
\baselineskip12pt
\begin{verbatim}
! j2() approximates the j2 integral by rations
! functions with coefficients:
!
  INTEGER, PARAMETER         :: NNJ2=3, NMJ2=2*NNJ2-1 ! NMJ2=5
  REAL,    DIMENSION(0:NMJ2) :: J2B, J2A
  PARAMETER (           &
  J2B=(/                &
    87.1714E0 ,         & !b0
    -277.584E0,         & !b1
    329.082E0 ,         & !b2
    -180.982E0,         & !b3
    56.7202E0 ,         & !b4
    -8.60238E0/),       & !b5
   J2A=(/               & 
    J2B(0),             & !a0
    -277.693E0,         & !a1
    329.801E0 ,         & !a2
    -184.219E0,         & !a3
    59.9325E0 ,         & !a4
    -10.1138E0/)        ) !a5
  REAL, PARAMETER ::  J2MM= 4.*(23.-15.*LOG2)/75. ! 0.6721489
  REAL, PARAMETER ::  J2BB=-4.*(4.-LOG8)/9.       !-0.8535815
  REAL, PARAMETER ::  J2AA=-3.*SQPI/4.            !-1.3293405 
  REAL, PARAMETER ::  J2CC= 0.5E0
  REAL, PARAMETER ::  J2EE= 0.2E0
  REAL, PARAMETER ::  J2GG=-15.*SQPI/16.
\end{verbatim}
}


{
\noindent
dedx.f90:
\baselineskip12pt
\begin{verbatim}
 FUNCTION j2(x)
   USE globalvars
   IMPLICIT NONE
   REAL, INTENT(IN) :: x
   REAL :: j2

   REAL,    PARAMETER   :: XMIN=0.1, XMAX=30.D0
   REAL    :: x2, x4
   REAL    :: y, y5, y7
   REAL    :: xx, ra, rc
   INTEGER :: n
!
! analytic asymptotic forms
!
  IF (x .LE. XMIN) THEN
     j2=J2MM*x+J2BB
  ELSEIF (x .GT. XMAX) THEN
     y=SQRT(x)          ! x^1/2
     y5=x*x*y           ! x^5/2
     y7=y5*x            ! x^7/2
     j2=J2AA/y5 +J2GG/y7
  ELSE
!
! numerical asymptotic form
!
     x2=x*x
     x4=x**4.5
     j2=(J2MM*x+J2BB)/(J2CC*x4+1) + &
        J2EE*J2AA*x2/(J2EE*x4+1)
!
! spline correction
! 
     ra=0.E0
     rc=0.E0
     xx=1.E0
     DO n=0,NMJ2
        ra=ra+J2A(n)*xx
        rc=rc+J2B(n)*xx
        xx=x*xx
     ENDDO
     ra=ra+xx
     rc=rc+xx
     j2=j2*rc/ra
  ENDIF
 END FUNCTION j2
\end{verbatim}
}


\subsubsection{$J_3(x)$}

We will now consider the integral
%%
\begin{eqnarray}
  J_3(x) &\equiv& 
  \int_0^1 du\,e^{-x u} \, \frac{\ln u}
  {\sqrt{u}} 
\end{eqnarray}
%%
where $x \ge 0$. We can obtain an analytic solutions for small and large 
values of the argument: $x \ll 1$ and $x \gg 1$. In the former case, we 
expand the exponent to linear order and find
%%
\begin{eqnarray}
\nonumber
  x \ll 1: \hskip1cm &&
\\[-5pt]
  J_3(x) &=& m x + b 
  \hskip4cm m= 4/9 ~~~ b=-4  
\\\nonumber &&
  \hskip-1.3cm {\rm error~} 0.3\% {\rm ~for~} x< x_{\rm min}=0.4
\\[10pt]
\nonumber
  x \gg 1: \hskip1cm &&
\\
\label{j3largex}
  J_3(x) &=& (a_1 \ln x + a_2) x^{-1/2} 
  \hskip1.8cm a_1 = -\sqrt{\pi} ~~~ 
  a_2 =  -\sqrt{\pi}\,(\gamma + \ln4)
\\\nonumber &&
  \hskip-1.3cm {\rm error~} 0.3\% {\rm ~for~} x> x_{\rm max}=2.3
\end{eqnarray}
%%
The former case is obtained, as before, by expanding the
exponent to linear order in $x$. In the latter case the support 
lies near $u \sim 1$ and we replace the upper limit by infinity 
and write
%%
\begin{eqnarray}
\nonumber
  J_3(x) 
  &\approx&
  \int_0^\infty  du\,e^{-x u} \, \frac{\ln u}{\sqrt{u}} 
  =
  \int_0^\infty  \frac{du}{x} ~ e^{-u} \, \frac{\ln (u/x)}
  {\sqrt{(u/x)}} ~~~ : u \to u/x
\\[5pt]
  &=&
  x^{-1/2}\int_0^\infty  du ~ e^{-u} \, 
  \frac{\ln u  - \ln x}{\sqrt{u}} \ ,
\end{eqnarray}
%%
which gives (\ref{j3largex}) after doing the integrals.

For intermediate values of $x$ between $x_{\rm min}$ and
 $x_{\rm max}$, we approximate
%%
\begin{eqnarray}
  J_3(x) = R(x) \, Q_6(x) \ ,
\end{eqnarray}
%%
where we take
%%
\begin{eqnarray}
\\\nonumber
  R(x)= 
  \underbrace{~\frac{m x + b}{c\, x^{7/2} + 1}~~}_{x\ll 1:~mx+b}  + 
  \underbrace{~~\frac{e\,[a_1\ln(x+1)+a_2]x^2}{e\, x^{5/2} +1}~~}_{x\gg 1:~ 
  (a_1\ln x + a_2)\, x^{-1/2}}\ ,
\end{eqnarray}
%%
with $x^2$ in the numerator of the second term chosen so that this 
term dominates over the first term at large $x$. In summary, we take
%%
\begin{eqnarray}
\begin{array}{lrr}
  {\rm coeff} ~~~ & {\rm numerical} & \hskip3cm {\rm exact}    \\ \hline
  m               &  0.444444          &  4/9                  \\[-8pt]
  b               &  -4                &  -4                   \\[-8pt]
  c               &  0.1               &  {\rm arbitrary}      \\[-8pt]
  e               &  0.2               &  {\rm arbitrary}      \\[-8pt]
  a_1             & -1.77245           &  -\sqrt{\pi}          \\[-8pt]
  a_2             & -3.48023           &  -\sqrt{\pi}\,(\gamma+\ln4)
\end{array}
\end{eqnarray}
%! J3MM = 4./9.
%! J3BB =-4.0E0
%! J3CC = 0.1E0
%! J3EE = 0.2E0
%! J3AA1=-1.77245E0
%! J3AA2=-3.48023E0

\noindent
For $n=6$ we need $2n-1=11$ data points: we will take
$m$ to start at zero and end at ten, and we choose 
%%
\begin{eqnarray}
\begin{array}{lll}
  m~~~~ & x_m~~~ & J_3(x_m)        \\ \hline
  0    & 0.1       &   -3.95634    \\[-8pt]
  1    & 0.2       &   -3.91421    \\[-8pt]
  2    & 0.5       &   -3.7962     \\[-8pt]
  3    & 0.7       &   -3.72387    \\[-8pt]
  4    & 1.2       &   -3.56202    \\[-8pt]
  5    & 1.4       &   -3.50388    \\[-8pt]
  6    & 1.6       &   -3.44903    \\[-8pt]
  7    & 2.0       &   -3.3481     \\[-8pt]
  8    & 3.0       &   -3.13707    \\[-8pt]
  9    & 4.0       &   -2.96948    \\[-8pt]
 10    & 8.0       &   -2.53355
\end{array}
\end{eqnarray}
%
and solving the linear equations gives
%%
\begin{eqnarray}
  \begin{array}{lrr}
  \ell~~~~ & \hfill b_\ell \hfill & \hskip1cm a_\ell \hskip1.5cm  \\ \hline
  0    &    24.9508   &    24.9508      \\[-8pt]
  1    &  -0.925829   &  -0.913013      \\[-8pt]
  2    &    9.63281   &    13.5336      \\[-8pt]
  3    &   -2.94514   &    -1.2945      \\[-8pt]
  4    &    6.44874   &    4.60507      \\[-8pt]
  5    &   -2.10201   &   -1.88678      \\[-8pt]
  \end{array}
\end{eqnarray}
%%


{
\noindent
globalvars.f90:
\baselineskip12pt
\begin{verbatim}
! j3() approximates the j3 integral by rations
! functions with coefficients:
!
  INTEGER, PARAMETER         :: NNJ3=3, NMJ3=2*NNJ3-1 ! NMJ3=5
  REAL,    DIMENSION(0:NMJ3) :: J3B, J3A
  PARAMETER (           &
  J3B=(/                &
   24.9719E0  ,         & !b0
   -0.923982E0,         & !b1
   9.62659E0  ,         & !b2
   -2.93352E0 ,         & !b3
   6.44425E0  ,         & !b4
   -2.10031/) ,         & !b5
   J3A=(/               &
    J3B(0),             & !a0
    -0.926079E0,        & !a1
    13.5296E0  ,        & !a2
    -1.28659E0 ,        & !a3
    4.59814E0  ,        & !a4
    -1.88505E0/)        ) !a5
  REAL, PARAMETER ::  J3MM = 4./9.                ! 0.444444 
  REAL, PARAMETER ::  J3BB =-4.0E0                !-4.0       
  REAL, PARAMETER ::  J3AA1=-SQPI                 !-1.7724539 
  REAL, PARAMETER ::  J3AA2=-SQPI*(GAMMA+LOG4)    !-3.4802318 
  REAL, PARAMETER ::  J3CC = 0.1E0 
  REAL, PARAMETER ::  J3EE = 0.2E0 
\end{verbatim}
}


{
\noindent
dedx.f90:
\baselineskip12pt
\begin{verbatim}
 FUNCTION j3(x)
   USE globalvars
   IMPLICIT NONE
   REAL, INTENT(IN) :: x
   REAL :: j3

   REAL,    PARAMETER   :: XMIN=0.4, XMAX=2.3D0
   REAL    :: x2, x3, x4
   REAL    :: xx, ra, rc
   REAL    :: y
   INTEGER :: n
!
! analytic asymptotic forms
!
  IF (x .LE. XMIN) THEN
     j3=J3MM*x+J3BB
  ELSEIF (x .GE. XMAX) THEN
     y=SQRT(x)
     j3=(J3AA1*LOG(x) + J3AA2)/y
  ELSE
!
! numerical asymptotic form 
!
     x2 =x*x
     x3 =x**2.5
     x4 =x**3.5
     j3=(J3MM*x+J3BB)/(J3CC*x4+1) + &
        J3EE*(J3AA1*LOG(1+x)+J3AA2)*x2/(J3EE*x3+1)
!
! spline correction
! 
     ra=0.E0
     rc=0.E0
     xx=1.E0
     DO n=0,NMJ3
        ra=ra+J3A(n)*xx
        rc=rc+J3B(n)*xx
        xx=x*xx
     ENDDO
     ra=ra+xx
     rc=rc+xx
     j3=j3*rc/ra
  ENDIF
END FUNCTION j3
\end{verbatim}
}


\subsubsection{$J_4(x)$}


We will now consider the integral
%%
\begin{eqnarray}
  J_4(x) &\equiv &
  \int_0^1 du\,e^{-x u} \, \sqrt{u}\, 
  \ln \hskip-0.05cm u  \ .
\end{eqnarray}
%%
where $x \ge 0$. We can obtain an analytic solutions for small and large 
values of the argument: $x \ll 1$ and $x \gg 1$. In the former case, we 
expand the exponent to linear order and find
%%
\begin{eqnarray} 
\nonumber
  x \ll 1: \hskip1cm &&
\\[-5pt]
  J_4(x) &=& b + m x + f x^2
  \hskip3.5cm   b= -4/9  ~~~ m= 4/25 ~~~ f= -2/49
\\\nonumber &&
  \hskip-1.3cm {\rm error~} 0.01\% {\rm ~for~} x< x_{\rm min}=0.18
\\[10pt]
\nonumber
  x \gg 1: \hskip1cm &&
\\
  J_4(x) &=& (a_1 \ln x + a_2) x^{-3/2}
  \hskip1.5cm a_1 = -\frac{\sqrt{\pi}}{2}~~~
  a_2 = \frac{\sqrt{\pi}}{2}\,(2 - \gamma - \ln4)
\\\nonumber &&
  \hskip-1.3cm {\rm error~} 0.3\% {\rm ~for~} x> x_{\rm max}=4.7
\end{eqnarray}
%%
The former case is evaluated as before, while in the latter case 
the support lies near $u \sim 1$ and we replace the upper limit by 
infinity and write
%%
\begin{eqnarray}
  J_4(x) 
  &\approx&
  \int_0^\infty  du\,e^{-x u} \, \sqrt{u} \, \ln u
  =
  \int_0^\infty  \frac{du}{x} ~ e^{-u} \, 
  \sqrt{(u/x)} \, \ln (u/x)
  ~~~ : u \to u/x
\\[5pt]
  &=&
  x^{-3/2}\int_0^\infty  du ~ e^{-u} \, 
  \frac{\ln u  - \ln x}{\sqrt{u}} \ .
\end{eqnarray}
%%

At intermediate values of $x$ between $x_{\rm min}$ and
$x_{\rm max}$ we approximate
%%
\begin{eqnarray}
  J_4(x) = R(x) \, Q_6(x) \ .
\end{eqnarray}
%%
We take the rational approximation $R$ to be 
%%
\begin{eqnarray}
\\\nonumber
  R(x)= 
  \underbrace{~\frac{m x + b}{c\, x^{7/2} + 1}~~}_{x\ll 1:~mx+b}  + 
  \underbrace{~~\frac{e\,[a_1\ln(x+1)+a_2]x^2}{e\, x^{7/2} 
  +1}~~}_{x\gg 1:~ (a_1\ln x + a_2)\, x^{-3/2}}\ ,
\end{eqnarray}
%%
where $x^2$ in the numerator of the second term is chosen
so that this term dominates over the first term at large $x$.
In summary, we take
%%
\begin{eqnarray}
\begin{array}{lrr}
  {\rm coeff} ~~~ & {\rm numerical} & \hskip3cm {\rm exact}       \\ \hline
  m               &  0.16              &   4/25                   \\[-8pt]
  b               & -0.444444          &  -4/9                    \\[-8pt]
  c               &  0.1               &  {\rm arbitrary}         \\[-8pt]
  e               &  0.1               &  {\rm arbitrary}         \\[-8pt]
  a_1             & -0.886227          &  -\sqrt{\pi}/2           \\[-8pt]
  a_2             &  0.0323384         &  \sqrt{\pi}\,(2-\gamma -\ln4)/2
\end{array}
\end{eqnarray}
%! J4MM = 0.158781E0
%! J4BB =-0.444444E0
%! J4CC = 0.1E0
%! J4EE = 0.1E0
%! J4AA1=-0.886227E0
%! J4AA2= 0.0323384E0

\noindent
For $n=6$ we need $2n-1=11$ data points: we will take
$m$ to start at zero and end at ten, and we choose 
%%
\begin{eqnarray}
\begin{array}{lll}
  m~~~~& x_m~~~   & J_4(x_m)        \\ \hline
  0    & 0.1      &    -0.428845    \\[-8pt]
  1    & 1.0      &    -0.318233    \\[-8pt]
  2    & 10       &    -0.0635077   \\[-8pt]
  3    & 20       &    -0.0293211   \\[-8pt]
  4    & 30       &    -0.0181472   \\[-8pt]
  5    & 40       &    -0.0127948   \\[-8pt]
  6    & 50       &    -0.00971452  \\[-8pt]
  7    & 60       &    -0.00773775  \\[-8pt]
  8    & 70       &    -0.00637363  \\[-8pt]
  9    & 80       &    -0.00538212  \\[-8pt]
 10    & 90       &    -0.00463275
\end{array}
\end{eqnarray}
%
and solving the linear equations gives
%%
\begin{eqnarray}
  \begin{array}{lrr}
  \ell~~~~~~~ & \hfill b_\ell \hfill & \hskip3cm a_\ell \hskip1.5cm  \\ \hline
  0    &   1.745934124305194E7    &  1.745934124305194E7     \\[-8pt]
  1    &  -2.227978010029586E7    & -2.238319621183343E7     \\[-8pt]
  2    &   1.295264548871798E6    &  1.477987858851782E6     \\[-8pt]
  3    &               -785340    &              -788364     \\[-8pt]
  4    &               6120.99    &               6041.5     \\[-8pt]
  5    &                533.886   &              533.792     \\[-8pt]
  \end{array}
\end{eqnarray}
%%


{\noindent
globalvars.f90:
\baselineskip12pt
\begin{verbatim}
! j4() approximates the j4 integral by rations
! functions with coefficients:
!
  INTEGER, PARAMETER         :: NNJ4=3, NMJ4=2*NNJ4-1 ! NMJ4=5
  REAL,    DIMENSION(0:NMJ4) :: J4B, J4A
  PARAMETER (             &
  J4B=(/                  &
   1.368871985536256E7,   & !b0
   -1.51863372154072E7,   & !b1
   920692.E0  ,           & !b2
   -590484.E0 ,           & !b3
   1763.2E0   ,           & !b4
   415.931E0/),           & !b5
   J4A=(/                 &
    J4B(0),               & !a0
    -1.5303334226695618E7,& !a1
    1.064250585959179E6  ,& !a2
    -592292.E0    ,       & !a3
    1697.91E0     ,       & !a4
    415.831E0/)           ) !a5
\end{verbatim}
}


{
\noindent
dedx.f90:
\baselineskip12pt
\begin{verbatim}
 FUNCTION j4(x)
   USE globalvars
   IMPLICIT NONE
   REAL, INTENT(IN) :: x
   REAL :: j4

   REAL,    PARAMETER   :: XMIN=0.18D0, XMAX=4.7D0
   REAL    :: x2, x4
   REAL    :: xx, ra, rc
   REAL    :: y, y3
   INTEGER :: n
!
! analytic asymptotic forms
!
  IF (x .LE. XMIN) THEN
     j4=J4MM*x+J4BB
  ELSEIF (x .GE. XMAX) THEN
     y=SQRT(x)                   ! x^1/2
     y3=x*y                      ! x^3/2
     j4=(J4AA1*LOG(x) + J4AA2)/y3
  ELSE 
!
! numerical asymptotic form 
!
     x2 =x*x
     x4 =x**3.5
     j4=(J4MM*x+J4BB)/(J4CC*x4+1) + &
         J4EE*(J4AA1*LOG(1+x)+J4AA2)*x2/(J4EE*x4+1)
!
! spline correction
! 
     ra=0.E0
     rc=0.E0
     xx=1.E0
     DO n=0,NMJ4
        ra=ra+J4A(n)*xx
        rc=rc+J4B(n)*xx
        xx=x*xx
     ENDDO
     ra=ra+xx
     rc=rc+xx
     j4=j4*rc/ra
  ENDIF
 END FUNCTION j4
\end{verbatim}
}

\pagebreak
\centerline{\bf Appendix A}
\vskip0.3cm 

We now list the complete source code sequentially. 

\vskip0.5cm
{
\noindent
dedx\_main.f90:
\baselineskip12pt
\begin{verbatim}
PROGRAM dedxmain
!
! This is the driver for the BPS dedx subroutine,
! version 3.12
!
  USE globalvars

  IMPLICIT NONE
  INTEGER :: nni ! number of ions excluding electrons
  REAL,    DIMENSION(:), ALLOCATABLE :: beta, mb, nb, zb
  REAL    :: vp, zp, mp, ee
  REAL    :: dedxtot, dedxsumi
  REAL    :: ne, te, ti, betae, betai
  REAL    :: vpe, betam, vv
  REAL    :: dedxctot, dedxqtot, dedxcsumi, dedxqsumi

! projectile - DT plasma (NNB=3 or nni=2)
!
  nni=2         
  ALLOCATE(mb(1:nni+1),nb(1:nni+1),zb(1:nni+1))

  ee=3.54E0    ! projectile energy (MeV)
  ee=0.01E0    ! projectile energy (MeV)
  zp=2         ! projectile charge
  mp=4*MPEV    ! projectile mass
  te=60.       ! temperature of electrons (keV)
  ti=te        ! temperature of ions      (keV)
  ne=1.4448E26 ! electron numr density (cm^-3)
  zb(1)=-1.    ! species charges
  zb(2)=+1.    !
  zb(3)=+1.    !
  mb(1)=MEEV   ! species masses
  mb(2)=2*MPEV !
  mb(3)=3*MPEV !
  nb(1)=1.     ! number density with charge neutrality
  nb(2:nni+1)=1./(zb(2:nni+1)*nni) 

  ALLOCATE(beta(1:nni+1))            ! inverse temperature
  betae=1.E-3/te                     ! array (eV^-1)
  betai=1.E-3/ti                      
  beta(1)=betae                      
  IF ( nni .GE. 1) THEN
     beta(2:nni+1)=betai  
  ENDIF              
  nb=nb*ne/CMTOA0**3                 ! number density in
                                     ! atomic units (a0^-3)
! convert E to vp 
!
  betam=beta(I)
  vv=SQRT(R/(betam*mb(I))) ! vthc: thermal velocity units of c
  vp=vpe(ee,mp,vv) 
  PRINT *, "E :", ee ! MeV
  PRINT *, "vp:", vp ! units of vth for species I
  CALL dedx_bps(nni,vp,zp,mp,beta,zb,mb,nb,dedxtot,dedxsumi,&
       dedxctot,dedxcsumi,dedxqtot,dedxqsumi)
  PRINT *, ee, dedxtot
  PRINT *, ee, dedxsumi

  DEALLOCATE(beta,mb,nb,zb)
END PROGRAM dedxmain
  
FUNCTION vpe(ee, mp, vv) 
!
! 
!
! vpe= projectile velocity in units of vth
! vv = thermal velocity in units of c (vthc)
!
  USE globalvars
  IMPLICIT NONE
  REAL ee, mp, vv, vpe
  vpe=SQRT(2*ee/(mp/EV))/vv
END FUNCTION vpe
\end{verbatim}
}

\vskip1cm
{
\noindent
globalvars.f90:
\baselineskip12pt
\begin{verbatim}
MODULE globalvars
!
! mathematical constants
!
  REAL,    PARAMETER :: PI   =3.141592654   ! pi
  REAL,    PARAMETER :: SQPI =1.772453851   ! sqrt(pi)
  REAL,    PARAMETER :: GAMMA=0.577215665   ! Euler Gamma
  REAL,    PARAMETER :: LOG2 =0.6931471806  ! ln(2)
  REAL,    PARAMETER :: LOG4 =1.386294361   ! ln(4)
  REAL,    PARAMETER :: LOG8 =2.079441542   ! ln(8)
  REAL,    PARAMETER :: LOG16=2.772588722   ! ln(16)
  REAL,    PARAMETER :: ZETA3=1.202056903   ! zeta(3)
  REAL,    PARAMETER :: EXP2E=3.172218958   ! exp(2*GAMMA)
!
! physical parameters and conversion factors
!
  REAL,    PARAMETER :: BE=13.6             ! binding energy of Hydrogen
  REAL,    PARAMETER :: CC=2.998E10         ! speed of light
  REAL,    PARAMETER :: MPEV =0.938271998E9 ! proton mass in eV
  REAL,    PARAMETER :: MEEV =0.510998902E6 ! electron mass in eV
  REAL,    PARAMETER :: AMUEV=0.931494012E9 ! AMU in eV
  REAL,    PARAMETER :: KTOEV=8.61772E-5    ! conversion factor
  REAL,    PARAMETER :: CMTOA0=1.8867925E8  ! conversion factor
  REAL,    PARAMETER :: MTR=1.E-6           ! length unit
  REAL,    PARAMETER :: EV=1.E6             ! energy unit
  REAL,    PARAMETER :: CONVFACT=CMTOA0*(MTR*100.)/EV
!
! misc parameters
!
  INTEGER, PARAMETER :: R=3        ! thermal velocity parameter
  INTEGER, PARAMETER :: I=1        ! plasma species index
  REAL               :: K          ! arbitrary wave number units a0^-1 

! plasma parameters: values set in dedx_bps
!
! REAL,    DIMENSION(1:NNB)          :: kb2, ab, bb, cb, eb, fb, rb, gb
! REAL,    DIMENSION(1:NNB)          :: ab2, etb, rmb0, rrb0, mb0
  REAL,    DIMENSION(:), ALLOCATABLE :: kb2, ab, bb, cb, eb, fb, rb, gb
  REAL,    DIMENSION(:), ALLOCATABLE :: ab2, etb, rmb0, rrb0, mb0
  LOGICAL, DIMENSION(:), ALLOCATABLE :: lzb
  REAL    :: cp1, cp2, cp3, vth, vthc, mp0, kd
  INTEGER :: NNB  ! number of plasma species = ni+1

! daw() approximates Dawson's integral by rational
! functions with coefficients:
!
  INTEGER, PARAMETER          :: NNDAW=3, NMDAW=2*NNDAW-1 ! NMDAW=5
  REAL,    DIMENSION(0:NMDAW) :: DWB, DWA
  PARAMETER (            &
  DWB=(/                 &
    5.73593880244318E0,  & !b0
   -6.73666007137766E0,  & !b1
    1.99794422787154E1,  & !b2
   -1.85506350260761E1,  & !b3
    1.22651360905700E1,  & !b4
   -4.67285812684807E0/),& !b5
  DWA=(/                 &
    DWB(0),              & !a0
   -6.82372048950896E0,  & !a1
    1.33804115903096E1,  & !a2
   -1.42130723670491E1,  & !a3
    1.11714434417979E1,  & !a4
   -4.66303387468937E0/) ) !a5

! j1() approximates the j1 integral by rations
! functions with coefficients:
!
  INTEGER, PARAMETER         :: NNJ1=3, NMJ1=2*NNJ1-1 ! NMJ1=5
  REAL,    DIMENSION(0:NMJ1) :: J1B, J1A
  PARAMETER (           &
  J1B=(/                &
   -926.65E0  ,         & !b0
   787.016E0  ,         & !b1
   -329.764E0 ,         & !b2
   39.7406E0  ,         & !b3
   -0.173896E0,         & !b4
   -1.66913E0 /),       & !b5
  J1A=(/                &
    J1B(0),             & !a0
    787.165E0 ,         & !a1
    -213.584E0,         & !a2
    -1.04219E0,         & !a3
    33.594E0  ,         & !a4
   -11.8391E0/)         ) !a5
  REAL, PARAMETER :: J1MM= (4./9.)*(4.-LOG8)    ! 0.8535815 
  REAL, PARAMETER :: J1BB=-(4.-LOG16)           !-1.2274113
  REAL, PARAMETER :: J1AA=-SQPI/2.              !-0.8862270
  REAL, PARAMETER :: J1CC= 0.1E0
  REAL, PARAMETER :: J1EE= 0.2E0
  REAL, PARAMETER :: J1GG=-3.*SQPI/8.

! j2() approximates the j2 integral by rations
! functions with coefficients:
!
  INTEGER, PARAMETER         :: NNJ2=3, NMJ2=2*NNJ2-1 ! NMJ2=5
  REAL,    DIMENSION(0:NMJ2) :: J2B, J2A
  PARAMETER (           &
  J2B=(/                &
    87.1714E0 ,         & !b0
    -277.584E0,         & !b1
    329.082E0 ,         & !b2
    -180.982E0,         & !b3
    56.7202E0 ,         & !b4
    -8.60238E0/),       & !b5
   J2A=(/               & 
    J2B(0),             & !a0
    -277.693E0,         & !a1
    329.801E0 ,         & !a2
    -184.219E0,         & !a3
    59.9325E0 ,         & !a4
    -10.1138E0/)        ) !a5
  REAL, PARAMETER ::  J2MM= 4.*(23.-15.*LOG2)/75. ! 0.6721489
  REAL, PARAMETER ::  J2BB=-4.*(4.-LOG8)/9.       !-0.8535815
  REAL, PARAMETER ::  J2AA=-3.*SQPI/4.            !-1.3293405 
  REAL, PARAMETER ::  J2CC= 0.5E0
  REAL, PARAMETER ::  J2EE= 0.2E0
  REAL, PARAMETER ::  J2GG=-15.*SQPI/16.

! j3() approximates the j3 integral by rations
! functions with coefficients:
!
  INTEGER, PARAMETER         :: NNJ3=3, NMJ3=2*NNJ3-1 ! NMJ3=5
  REAL,    DIMENSION(0:NMJ3) :: J3B, J3A
  PARAMETER (           &
  J3B=(/                &
   24.9719E0  ,         & !b0
   -0.923982E0,         & !b1
   9.62659E0  ,         & !b2
   -2.93352E0 ,         & !b3
   6.44425E0  ,         & !b4
   -2.10031/) ,         & !b5
   J3A=(/               &
    J3B(0),             & !a0
    -0.926079E0,        & !a1
    13.5296E0  ,        & !a2
    -1.28659E0 ,        & !a3
    4.59814E0  ,        & !a4
    -1.88505E0/)        ) !a5
  REAL, PARAMETER ::  J3MM = 4./9.                ! 0.444444 
  REAL, PARAMETER ::  J3BB =-4.0E0                !-4.0       
  REAL, PARAMETER ::  J3AA1=-SQPI                 !-1.7724539 
  REAL, PARAMETER ::  J3AA2=-SQPI*(GAMMA+LOG4)    !-3.4802318 
  REAL, PARAMETER ::  J3CC = 0.1E0 
  REAL, PARAMETER ::  J3EE = 0.2E0 

! j4() approximates the j4 integral by rations
! functions with coefficients:
!
  INTEGER, PARAMETER         :: NNJ4=3, NMJ4=2*NNJ4-1 ! NMJ4=5
  REAL,    DIMENSION(0:NMJ4) :: J4B, J4A
  PARAMETER (             &
  J4B=(/                  &
   1.368871985536256E7,   & !b0
   -1.51863372154072E7,   & !b1
   920692.E0  ,           & !b2
   -590484.E0 ,           & !b3
   1763.2E0   ,           & !b4
   415.931E0/),           & !b5
   J4A=(/                 &
    J4B(0),               & !a0
    -1.5303334226695618E7,& !a1
    1.064250585959179E6  ,& !a2
    -592292.E0    ,       & !a3
    1697.91E0     ,       & !a4
    415.831E0/)           ) !a5
  REAL, PARAMETER :: J4MM = 4./25.                    ! 0.16 
  REAL, PARAMETER :: J4BB =-4./9.                     !-0.4444444E0
  REAL, PARAMETER :: J4AA1=-SQPI/2.                   !-0.8862269E0  
  REAL, PARAMETER :: J4AA2=SQPI*(2.-GAMMA-LOG4)/2.    ! 3.23383974E-02 
  REAL, PARAMETER :: J4CC = 0.1E0
  REAL, PARAMETER :: J4EE = 0.1E0 

END MODULE globalvars

!
! D-T plasma:
!
!      PARAMETER (ZB(1)=-1)        ! charge of 1-st component
!      PARAMETER (ZB(2)=+1)        ! charge of 2-nd component
!      PARAMETER (ZB(3)=+1)        ! charge of 3-rd component
!      PARAMETER (MB(1)=MEEV)      ! mass of   1-st component
!      PARAMETER (MB(2)=2*MPEV)    ! mass of   2-nd component
!      PARAMETER (MB(3)=3*MPEV)    ! mass of   3-rd component
!      PARAMETER (NB(1)=NE)        ! density of 1-st comp
!      PARAMETER (NB(2)=0.5*NE)    ! density of 2-nd comp
!      PARAMETER (NB(3)=0.5*NE)    ! density of 3-rd comp
\end{verbatim}
}

\vskip1cm
{
\noindent
dedx.f90:
\baselineskip12pt
\begin{verbatim}
! These subroutines implements the BPS stopping power: ``Charged 
! Particle Motion in a Highly Ionized Plasma'', L. Brown, D. Preston, 
! and R. Singleton Jr., Physics Reports, Vol. 410, No. 4, 237-333 
! (2005); or arXiv:physics/0501084.
! 
! See BPSx.xx/doc/BPS_phys_rep.pdf for the full Phys. Rep. paper,
! and see BPSx.xx/doc/doc.pdf for code documentation.
!
! Robert Singleton, LANL, X-7
!
! v3.13: Jan-06, 4rd production version
! v3.12: May-05, 3rd production version
! v3.09: May-04, 2nd production version
! v3.04: Jan-03, 1st production version
!
 SUBROUTINE dedx_bps(nni, vp, zp, mp, betab, zb, mb, nb, &
     dedxtot, dedxsumi, dedxctot, dedxcsumi, dedxqtot, dedxqsumi)
   USE globalvars

   IMPLICIT NONE
   INTEGER,                     INTENT(IN)  :: nni    ! number of ions
   REAL,    DIMENSION(1:nni+1), INTENT(IN)  :: betab  ! plasma temp array
   REAL,    DIMENSION(1:nni+1), INTENT(IN)  :: mb, nb ! mass and density
   REAL,    DIMENSION(1:nni+1), INTENT(IN)  :: zb     ! charge array
   REAL,                        INTENT(IN)  :: vp     ! projectile velocity
   REAL,                        INTENT(IN)  :: zp     ! projectile charge
   REAL,                        INTENT(IN)  :: mp     ! projectile mass
   REAL,                        INTENT(OUT) :: dedxtot,  dedxsumi
   REAL,                        INTENT(OUT) :: dedxctot, dedxcsumi
   REAL,                        INTENT(OUT) :: dedxqtot, dedxqsumi

   REAL, DIMENSION(1:nni+1) :: mpb0, rpb0
   REAL :: betam, mm

   REAL                     :: e, gd
   REAL, DIMENSION(1:nni+1) :: gpb

   REAL, DIMENSION(1:nni+1) :: ub2, mpb, loglamb

   NNB=nni+1
   ALLOCATE(kb2(1:NNB),ab(1:NNB),bb(1:NNB),cb(1:NNB))
   ALLOCATE(eb(1:NNB),fb(1:NNB),rb(1:NNB),gb(1:NNB))
   ALLOCATE(ab2(1:NNB),etb(1:NNB),rmb0(1:NNB),rrb0(1:NNB))
   ALLOCATE(mb0(1:NNB),lzb(1:NNB))
!
! plasma parameters
!
   betam=betab(I)                ! inv temp of index plasma species
   rb=R*betab/betam              ! r_b array
   kb2=8*PI*BE*betab*zb*zb*nb    ! inv Debye length squared
   kd=SUM(ABS(kb2))              ! total inv Debye length
   kd=SQRT(kd)                   ! units a0^-1
   K =kd                         ! set K to Debye 

   mm=mb(I)                      ! mass of index plasma species  
   mp0=mp/mm                     ! rescaled proj mass
   cp1=2*BE*zp**2                ! units of eV-a0 
   cp2=(BE*zp**2)/(2*PI)         ! units of eV-a0 
   cp3=(BE*zp**2)/(PI*mp0)       ! dimensionless parameter   
   vthc=SQRT(R/(betam*mm))       ! thermal velocity of mm: units of c
   vth =CC*vthc                  ! thermal velocity of mm: units cm/s 

   mb0 =mb/mm                    ! rescaled plasma masses
   mpb0=mp0 + mb0                ! Mpb0
   rpb0=mp0*mb0/mpb0             ! mpb0
   rmb0=mpb0/mp0                 ! rm0=rMb0=(mp0+mb0)/mp0
   rrb0=mb0/mp0                  ! rr0=rmb0=mb0/mp0
   ab  =SQRT(rb*mb0/2)            
   ab2 =ab*ab                    
   bb  =rb*mpb0                   
   eb  =(mb0/mp0)/SQRT(2*PI*rb*mb0)
   etb =2*BE*ABS(zp*zb)*(2.686E-4)/vthc
   fb  =2/SQRT(2*PI*rb*mb0)
   WHERE ( zb /= 0 )             ! do not take log of zero
      lzb=.TRUE.                 ! flag for future use
      cb  =2 - 2*GAMMA - LOG(ABS((2*BE)*betab*zp*zb*K*mb0/rpb0))
      gb  =0.5 + 2*GAMMA + Log(ABS(0.5*BE*&    ! for small vp limit
           betab*zp*zb*kd*mb0/rpb0))           !
   ELSEWHERE
      cb=0
      gb=0
      lzb=.FALSE.
   ENDWHERE

! check for charge neutrality
!
   e=SUM(zb*nb)
   PRINT *, 'charge  = ', e

! print K and kd
! 
   PRINT *, 'K(a0^-1)= ', K
   PRINT *, 'kd      = ', kd

! g-factors
!
   gpb=2*BE*betam*SQRT(kb2)       ! g_pb 
   gd =2*BE*betam*kd              ! g_d
   PRINT *,'gD      = ', gd

! thermal velocity (cm/s)
!
   PRINT *, 'vth     = ', vth
!
! Coulomb log (I think I left out a factor of k_D from the 
!              argument of the log)
!
   ub2=(vp*vthc)**2 + 2/(betab*mb)**2     ! velocity (units of c)
   mpb=rpb0*mm                            ! reduced mass array (eV)
   loglamb=(8*PI*zb*zb)**2/(mpb*ub2)**2   ! a0 = 5.29*10^-11 m 
   loglamb=loglamb + (2.69E-4)**2/(2*mpb*mpb*ub2) ! = 2.69*10^-4 eV
   loglamb=-0.5*LOG(loglamb)
   PRINT *, 'log(Lam)=',loglamb

   CALL dedxc(vp,dedxctot,dedxcsumi)      ! returned in MeV/mu-m
   CALL dedxq(vp,dedxqtot,dedxqsumi)   !
   dedxtot =dedxctot  + dedxqtot
   dedxsumi=dedxcsumi + dedxqsumi
!!
   PRINT *,"       MeV/mu-m    MeV/mu-m"
   PRINT *,"tot:",dedxtot,  dedxsumi
   PRINT *,"cl :",dedxctot, dedxcsumi
   PRINT *,"qm :",dedxqtot, dedxqsumi
!!
   DEALLOCATE(kb2,ab,bb,cb,eb,fb,rb,gb)
   DEALLOCATE(ab2,etb,rmb0,rrb0,mb0,lzb)
 END SUBROUTINE dedx_bps


!
! quantum correction:
!
! This subroutine calculates the quantum correction dedxq.
! It returns dedxqtot and dedxqsumi. The following functions 
! and subroutines are defined here:
!
! Subroutine: dedxq
!
! Function  : dedxqi
!
! Function  : d_dedxq 
!
! Function  : repsi
!
 SUBROUTINE dedxq(vp, dedxqtot, dedxqsumi)
   USE globalvars
   
   IMPLICIT NONE
   REAL, INTENT(IN) :: vp
   REAL, INTENT(OUT):: dedxqtot, dedxqsumi

   REAL, DIMENSION(1:NNB) :: qmb
   REAL                   :: dedxqi, a1, a2, e, rm, rr
   INTEGER                :: ib


   qmb=0
   DO ib=1,NNB            ! sum over plasma species
      IF ( lzb(ib) ) THEN ! computle only if zb(ib) /= 0 
         a1=ab(ib)
         a2=a1*a1
         e=etb(ib)
         rm=rmb0(ib)      ! rmb0=rMb0=(mp0+mb0)/mp0
         rr=rrb0(ib)      ! rrb0=rmb0=mb0/mp0
         qmb(ib)=qmb(ib)+dedxqi(vp,a2,e,rm,rr)
         qmb(ib)=qmb(ib)*kb2(ib)*fb(ib)
      ELSE
         qmb(ib)=0        ! don't compute if zb(ib) = 0
      ENDIF
   ENDDO
   dedxqsumi=SUM(qmb(2:NNB))
   dedxqtot =qmb(1)
   dedxqtot =dedxqtot+dedxqsumi

   dedxqtot=CONVFACT*(cp1/vp)*dedxqtot
   dedxqsumi=CONVFACT*(cp1/vp)*dedxqsumi
 END SUBROUTINE dedxq


!
! a = ab(ib)*ab(ib)
! e = etb(ib)
!
!
!           /Infinity
!           |
! dedxq  =  | du  d_dedxq(u)
!           |
!           /0
!
!
 FUNCTION dedxqi(v, a, e, rm, rr)
!
! This function performs the integration numerically by
! Gaussian Quadrature. The number of intervals NG that [u0,u1]
! is broken into is hardwired in a PARAMETER statement, but 
! it be changed by the user (must be even).
!
! NOTE on Gaussian Quadrature:
!
! The polynomial P3(x)=(5*x^3-3*x)/2 is employed, and I have
! defined the appropriate weights W13, W2 and relative position
! UPM in parameter statements. 
!
   IMPLICIT NONE     
  
   REAL, INTENT(IN) :: v, a, e, rm, rr
   REAL :: dedxqi
  
   REAL,    PARAMETER :: UPM=0.7745966692E0
   REAL,    PARAMETER :: W13=0.5555555556E0, W2=0.8888888889E0
   INTEGER, PARAMETER :: NG=10000  ! must be even
   REAL,    PARAMETER :: NN=30.E0

   REAL,    PARAMETER :: SQPI =1.772453851    ! sqrt(pi)
   REAL,    PARAMETER :: SMAX=10., SMIN=0.05  ! cuts on s
   REAL     :: q1, q2, dg, ddg, g0, g1, g2, h0, h2
   REAL     :: s1, s2, s3, s05, s15 
   REAL     :: repsi, repsi1, repsi2

   REAL     :: i1, i2, kqm1, kqm3

   REAL     :: x0, x1, dx
   REAL     :: x, xm, d_dedxq
   INTEGER  :: ix
   REAL     :: r, s

   s=a*v*v
   IF ( s .GT. SMAX) THEN          ! large s can be performed analytically
      r=e/v                        ! this case is usually realized for ions
      g0 =LOG(ABS(r)) - repsi(r)   ! 
      dg =1/r - repsi1(r)          ! 
      ddg=-1/(r*r) - repsi2(r)     ! 
      g1=-r*dg                     ! 
      g2=2*r*dg + r*r*ddg          !
      s1=1/s                       ! s1=1/s
      s2=s1/2                      ! s2=1/2*s
      s3=3*s2                      ! s3=3/2*s
      s05=SQRT(s)                  ! s05=s^(1/2)
      s15=s*s05                    ! s15=s^(3/2)
      h0=g0*(1-s2)
      h2=g2*(1-s2) - 2*g1*(1-s1) + 2*g0*(1-s3)
      q1=SQPI/2
      q2=SQPI/2
      q1=q1*(g0/s05 + 0.25E0*g2/s15)
      q2=q2*(h0/s05 + 0.25E0*h2/s15)
      dedxqi=rm*q2 - rr*q1
   ELSEIF ( s .LT. SMIN) THEN      ! small s can be performed analytically
      r=SQRT(a)*e                  ! this case is usually realized for elec.
      i1=kqm1(r)                   ! accuracy < 0.5%
      i2=kqm3(r)
      dedxqi=(4.*rm*s/3. - 2*rr)*i1
      dedxqi=dedxqi+ 4.*(4*rm*s*s/15.- rr*s/3.)*i2   
      dedxqi=EXP(-s)*dedxqi
   ELSE                            ! otherwise do integral numerically
      r=e/v
      x0=1.E0 - NN/SQRT(s)         ! by Gaussian quadrature
      x0=MAX(0.,x0)
      x1=1.E0 + NN/SQRT(s)
      dx=(x1-x0)/NG
      dedxqi=0.E0
      x=x0-dx
      DO ix=1,NG,2
!     
         x=x+2.E0*dx
         dedxqi=dedxqi+W2*d_dedxq(r,s,rm,rr,x)
!
         xm=x-dx*UPM
         dedxqi=dedxqi+W13*d_dedxq(r,s,rm,rr,xm)
!
         xm=x+dx*UPM
         dedxqi=dedxqi+W13*d_dedxq(r,s,rm,rr,xm)
      ENDDO
      dedxqi=dedxqi*dx
   ENDIF
 END FUNCTION dedxqi


 FUNCTION d_dedxq(r, s, rm, rr, x)
   IMPLICIT NONE
   REAL, INTENT(IN) :: r, s, rm, rr, x
   REAL, PARAMETER  :: SXMAX=0.05
   REAL :: d_dedxq
   REAL :: repsi, rx, sx, sh, ch
   REAL :: ep, em, xm1, xp1
   rx=r/x
   sx=2*s*x
   xm1=x-1
   xp1=x+1
   ep=EXP(-s*xp1*xp1)
   em=EXP(-s*xm1*xm1)
   sh=0.5E0*(em-ep)         ! sh and ch are 
   IF (sx .GT. SXMAX) THEN  ! not sinh or cosh
      ch=0.5E0*(em+ep)      ! 
      ch=(ch - sh/sx)/x
   ELSE
      ch=2.E0*sx/3 + (1.E0/15.E0 - 1.E0/(6.E0*s))*sx*sx*sx
      ch=s*ch*EXP(-s)
   ENDIF
   d_dedxq=LOG(ABS(rx)) - repsi(rx)
   d_dedxq=d_dedxq*(rm*ch - rr*sh)
 END FUNCTION d_dedxq


!
! This is a fit to repsi(x) = Re Psi(1 + I*x), where
! Psi is the PolyGamma function. The accuracy is 0.1%.
!
 FUNCTION repsi(x)  
   USE globalvars
   IMPLICIT NONE
   REAL, INTENT(IN) :: x
   REAL, PARAMETER :: XMIN=0.16E0, XMAX=1.5E0
   REAL, PARAMETER :: TZETA3=2.404113806319188 ! 2*ZETA(3)
   REAL, PARAMETER :: A=0.1E0, B=1.33333E0, C=1.125E0
   REAL :: repsi
   REAL :: x2, x4
   IF (x .LE. XMIN) THEN
      x2=x**2
      repsi=-GAMMA + ZETA3*x2
   ELSEIF (x .GE. XMAX) THEN
      x2=x**2
      x4=x2*x2
      repsi=LOG(x)+1.D0/(12.D0*x2)+1.D0/(120.D0*x4)
   ELSE
      x2=x*x
      repsi=0.5E0*LOG(1 + (EXP2E*x2*x2 + TZETA3*x2)/(1+x2))
      repsi=repsi/(1 - A*EXP(-B*x - C/x)) - GAMMA
   ENDIF
 END FUNCTION repsi

!              d
! repsi1(x) = --- Re[ Psi(1 + I*x) = -Im Psi'(1 + I*x). 
!             dx
 FUNCTION repsi1(x)  
   IMPLICIT NONE
   REAL, INTENT(IN) :: x
   REAL :: repsi1

   REAL, PARAMETER :: XMIN=0.14E0, X1=0.7E0 ,XMAX=1.9E0
   REAL, PARAMETER :: ZETA32=2.404113806319188 ! 2*ZETA(3)
   REAL, PARAMETER :: ZETA54=4.147711020573480 ! 4*ZETA(5)
   REAL, PARAMETER :: a0= 0.004211521868683916
   REAL, PARAMETER :: a1= 2.314767988469241000
   REAL, PARAMETER :: a2= 0.761843932767193200
   REAL, PARAMETER :: a3=-7.498711815965575000
   REAL, PARAMETER :: a4= 7.940030433629257000
   REAL, PARAMETER :: a5=-2.749533936429732000
   REAL, PARAMETER :: b0=-0.253862873373708200
   REAL, PARAMETER :: b1= 4.600929855835432000
   REAL, PARAMETER :: b2=-6.761540444078382000
   REAL, PARAMETER :: b3= 4.467238548899841000
   REAL, PARAMETER :: b4=-1.444390097613873500
   REAL, PARAMETER :: b5= 0.185954029179227070
   REAL :: xi
   IF ( x .LE. XMIN) THEN             ! x < xmin=0.14 
      repsi1=ZETA32*x - ZETA54*x*x*x  ! accurate to 0.1%
   ELSEIF (x .LE. x1) THEN
      repsi1=a5                       ! xmin < x < x1=0.7
      repsi1=a4 + repsi1*x            ! accurate to 0.002%
      repsi1=a3 + repsi1*x            ! a0 + a1*x + a2*x^2 +
      repsi1=a2 + repsi1*x            ! a3* x^3 + a4*x^4 + a5*x^5
      repsi1=a1 + repsi1*x
      repsi1=a0 + repsi1*x
   ELSEIF (x .LE. xmax) THEN          ! x1 < x < xmax=1.9
      repsi1=b5                       ! accurate to 0.1%
      repsi1=b4+repsi1*x              ! b0 + b1*x + b2*x^2 + 
      repsi1=b3+repsi1*x              ! b3*x^3 + b4*x^4 +
      repsi1=b2+repsi1*x              ! b5*x^5
      repsi1=b1+repsi1*x              
      repsi1=b0+repsi1*x            
   ELSE
      xi=1/x                          ! x > xmax=1.9
      repsi1=-1.E0/30.E0              ! accurate to 0.08%
      repsi1=repsi1*xi                ! 1/x - 1/6x^3 - 1/30x^5
      repsi1=-1.E0/6.D0 + repsi1*xi   ! 
      repsi1=repsi1*xi                !
      repsi1=1.E0 + repsi1*xi         !
      repsi1=repsi1*xi                !
   ENDIF
 END FUNCTION repsi1


!              d^2
! repsi2(x) = ---- Re[ Psi(1 + I*x) = -Re Psi''(1 + I*x). 
!             dx^2
 FUNCTION repsi2(x) 
   IMPLICIT NONE
   REAL, INTENT(IN) :: x
   REAL :: repsi2

   REAL, PARAMETER :: XMIN=0.18E0, X1=1.2E0 ,XMAX=2.5E0
   REAL, PARAMETER :: ZETA32=2.4041138063191880 ! 2*ZETA(3)
   REAL, PARAMETER :: ZETA512=12.44313306172044 ! 12*ZETA(5)
   REAL, PARAMETER :: ZETA730=30.250478321457685! 30*ZETA(7)
   REAL, PARAMETER :: a0= 2.42013533575662130
   REAL, PARAMETER :: a1=-0.41115258967949336
   REAL, PARAMETER :: a2=-8.09116694062588400
   REAL, PARAMETER :: a3=-24.9364824558827640
   REAL, PARAMETER :: a4=114.8109056152471800
   REAL, PARAMETER :: a5=-170.854545232781960
   REAL, PARAMETER :: a6=128.8402466765824700
   REAL, PARAMETER :: a7=-50.2459090010302060 
   REAL, PARAMETER :: a8= 8.09941032385266400
   REAL, PARAMETER :: b0= 4.98436272402513600
   REAL, PARAMETER :: b1=-16.6546466531059530
   REAL, PARAMETER :: b2= 20.6941300362041100
   REAL, PARAMETER :: b3=-13.3726837850936920 
   REAL, PARAMETER :: b4= 4.83094787289278800 
   REAL, PARAMETER :: b5=-0.92976482601030100
   REAL, PARAMETER :: b6= 0.07456475055097825
   REAL :: xi, xx
   IF ( x .LE. XMIN) THEN
      xx=x*x
      repsi2=ZETA32 - ZETA512*xx + ZETA730*xx*xx ! x < xmin=0.18
   ELSEIF (x .LE. x1) THEN                       ! accurate to 0.1%
      repsi2=a8                                  !
      repsi2=a7 + repsi2*x                       ! xmin < x < x1=1.2
      repsi2=a6 + repsi2*x                       ! accurate to 0.01% 
      repsi2=a5 + repsi2*x                       ! a0 + a1*x + a2*x^2 +
      repsi2=a4 + repsi2*x                       ! a3*x^3 + a4*x^4 +
      repsi2=a3 + repsi2*x                       ! a5*x^5 + a6*x^6 +
      repsi2=a2 + repsi2*x                       ! a7*x&7 + a8*x^8 
      repsi2=a1 + repsi2*x
      repsi2=a0 + repsi2*x 
   ELSEIF (x .LE. xmax) THEN                     ! x1 < x < xmax=2.5
      repsi2=b6                                  ! accurate to 0.2%
      repsi2=b5+repsi2*x                         ! b0 + b1*x + b2*x^2 +
      repsi2=b4+repsi2*x                         ! b3*x^3 + b4*x^4 +
      repsi2=b3+repsi2*x                         ! b5*x^5 + b6*x^6
      repsi2=b2+repsi2*x          
      repsi2=b1+repsi2*x          
      repsi2=b0+repsi2*x          
   ELSE
      xi=1/x                                     ! x > xmax=2.5
      xi=xi*xi                                   ! accurate to 0.07%
      repsi2= 1.E0/6.E0                          ! -1/x^2 + 1/2x^4 + 
      repsi2= 0.5E0 + repsi2*xi                  ! 1/6x^6
      repsi2=-1. + repsi2*xi                     !
      repsi2=repsi2*xi
   ENDIF
 END FUNCTION repsi2

!
!             /Infinity
!             |
! kqm1(x)  =  | dy   y exp(-y^2) [ln(x/y) - repsi(x/y)]
!             |
!            /0
!
!
 FUNCTION kqm1(x)  
   IMPLICIT NONE
   REAL, INTENT(IN) :: x
   REAL :: kqm1

   REAL, PARAMETER :: XMIN=0.15E0, XMAX=3.2E0 

   REAL, PARAMETER :: a0= 0.4329117486761496454549449429 ! 3*GAMMA/4
   REAL, PARAMETER :: a1= 1.2020569031595942854250431561 ! ZETA(3)
   REAL, PARAMETER :: a2= 0.1487967944177345026410993331 ! ZETA'(3)+GAMMA*
   REAL, PARAMETER :: b2=-0.0416666666666666666666666667 !-1/24  ZETA(3)/2
   REAL, PARAMETER :: b4=-0.0083333333333333333333333333 !-1/120
   REAL, PARAMETER :: b6=-0.0119047619047619047619047619 !-1/84
   REAL, PARAMETER :: c0= 0.25109815055856566000
   REAL, PARAMETER :: c1=-0.02989283169766254700
   REAL, PARAMETER :: c2= 0.03339880139150325000
   REAL, PARAMETER :: c3=-0.00799128953390392700
   REAL, PARAMETER :: c4= 0.00070251863606810650
   REAL, PARAMETER :: d0=-0.18373957827052560000
   REAL, PARAMETER :: d1=-0.33121125339783110000
   REAL, PARAMETER :: d2= 0.04022076263527408400
   REAL, PARAMETER :: d3=-0.00331897950305779480
   REAL, PARAMETER :: d4= 0.00012313574797356784
   REAL :: x2, lx, xi
   IF ( x .LE. XMIN) THEN                        ! x < xmin=0.15: to 0.06%
      x2=x*x                                     ! ln(x)/2 + 3*GAMMA/4 +
      lx=LOG(x)                                  ! ZETA(3)*X^2*ln(x) +
      kqm1=0.5E0*lx + a0 + a1*x2*lx + a2*x2      ! [ZETA'(3) + GAMMA*
                                                 ! ZETA(3)/2]*x^2
                                                 !
   ELSEIF ( x .GE. XMAX ) then                   ! x > xmax=3.2: to 0.12%
      xi=1/x                                     ! -1/24*x^2 - 1/120*x^4 -
      x2=xi*xi                                   ! 1/84*x^6
      kqm1=b6                                    !
      kqm1=b4 + kqm1*x2                          !
      kqm1=b2 + kqm1*x2                          !
      kqm1=kqm1*x2                               !
   ELSE
      xi=1/x                                     ! xmin < x , xmax
      lx=LOG(x)                                  ! fit accurate to 0.2%
      kqm1=c4                                    ! c0 + c1*x + c2*x^2 +
      kqm1=c3+kqm1*x                             ! c3*x^3 + c4*x^4 +
      kqm1=c2+kqm1*x                             ! d0*ln(x) +
      kqm1=c1+kqm1*x                             ! d1/x + d2/x^2 +
      kqm1=c0+kqm1*x + d0*lx                     ! d3/x^3 + d4/x^4
      lx=d4                                      !
      lx=d3+lx*xi                                !
      lx=d2+lx*xi                                !
      lx=d1+lx*xi                                !
      lx=lx*xi                                   !
      kqm1=kqm1+lx
   ENDIF
 END FUNCTION kqm1

!
!             /Infinity
!             |
! kqm3(x)  =  | dy   y^3 exp(-y^2) [ln(x/y) - repsi(x/y)]
!             |
!            /0
!
!
 FUNCTION kqm3(x)  
   IMPLICIT NONE
   REAL, INTENT(IN) :: x
   REAL :: kqm3

   REAL, PARAMETER :: XMIN=0.15E0, XMAX=2.5E0 
   REAL, PARAMETER :: a0= 0.1829117486761496454549449429 ! 3*GAMMA/4 - 1/4
   REAL, PARAMETER :: a2=-0.6010284515797971427073328102 !-ZETA(3)/2
   REAL, PARAMETER :: b2=-0.0833333333333333333333333333 !-1/12
   REAL, PARAMETER :: b4=-0.025                          !-1/40
   REAL, PARAMETER :: b6=-0.046875                       !-3/64
   REAL, PARAMETER :: c0= 0.691191700599840900000
   REAL, PARAMETER :: c1=-1.094849572900974000000
   REAL, PARAMETER :: c2= 0.318264617154601400000
   REAL, PARAMETER :: c3=-0.060275957444801354000
   REAL, PARAMETER :: c4= 0.005112428730167831000
   REAL, PARAMETER :: d0= 0.835543536679762600000
   REAL, PARAMETER :: d1= 0.047821976622976340000
   REAL, PARAMETER :: d2= 0.000053594881446931025
   REAL, PARAMETER :: d3=-0.000268040997573199600
   REAL, PARAMETER :: d4= 0.000015765134162582942
   REAL :: x2, lx, xi
   IF ( x .LE. XMIN) THEN                        ! x < xmin=0.15: to 0.1%
      x2=x*x                                     ! ln(x)/2 + 3*GAMMA/4 -1/4
      lx=LOG(x)                                  ! -[ZETA(3)/2]*x^2
      kqm3=0.5E0*lx + a0 + a2*x2                 ! 
                                                 ! 
                                                 !
   ELSEIF ( x .GE. XMAX ) then                   ! x > xmax=2.5: to 0.25%
      xi=1/x                                     ! -1/12*x^2 - 1/40*x^4 -
      x2=xi*xi                                   ! 3/64*x^6
      kqm3=b6                                    !
      kqm3=b4 + kqm3*x2                          !
      kqm3=b2 + kqm3*x2                          !
      kqm3=kqm3*x2                               !
   ELSE
      xi=1/x                                     ! xmin < x < xmax
      lx=LOG(x)                                  ! fit accurate to 0.04%
      kqm3=c4                                    ! c0 + c1*x + c2*x^2 +
      kqm3=c3+kqm3*x                             ! c3*x^3 + c4*x^4 +
      kqm3=c2+kqm3*x                             ! d0*ln(x) +
      kqm3=c1+kqm3*x                             ! d1/x + d2/x^2 +
      kqm3=c0+kqm3*x + d0*lx                     ! d3/x^3 + d4/x^4
      lx=d4                                      !
      lx=d3+lx*xi                                !
      lx=d2+lx*xi                                !
      lx=d1+lx*xi                                !
      lx=lx*xi                                   !
      kqm3=kqm3+lx
   ENDIF
 END FUNCTION kqm3

!
! classical contribution:
!
!
! This source code calculates the classical dedxc for each 
! plasma component. It returns dedxctot and dedxcsumi. The
! following functions and subroutines are defined here:
!
! Subroutine: dedxc
!
! Function  : intone
!
! Function  : inttwo
!
! Function  :  hi
!
! Function  : j1, j2, j3 ,j4
!
! Subroutine: fri
!
! Function  : daw
!
 SUBROUTINE dedxc(vp, dedxctot, dedxcsumi)
   USE globalvars
   
   IMPLICIT NONE
   REAL, INTENT(IN) :: vp
   REAL, INTENT(OUT):: dedxctot, dedxcsumi

   REAL, PARAMETER        :: ABV20=1.E-4
   REAL, DIMENSION(1:NNB) :: abv, abv2, clb
   REAL                   :: abv2ib, ss, vp2, kd2, kd4
   REAL                   :: hi, inttwo, intone, a, b, c, ke
   INTEGER                :: ib
!
! define input variables
!
   vp2=vp*vp
   abv=ab*vp      ! for intone, inttwo, hi
   clb=0          ! initialize classical to zero
   DO ib=1,NNB    ! loop over plasma components
      IF ( lzb(ib) ) THEN ! computle only if zb(ib) /= 0 
         abv2=abv*abv                 !
         abv2ib=abv2(ib)              ! cut on each component
         IF (abv2ib .LT. ABV20) THEN  ! small velocity limit is analytic
            kd2 =kd*kd
            kd4 =kd2*kd2
            clb(ib)=clb(ib)+2*gb(ib)*(1-abv2ib*(1 + &
                 (2./3.)*(mp0/mb0(ib))))
            clb(ib)=clb(ib)-(4./3.)*abv2ib -2*SUM(kb2*abv2)/kd2
            ss=(SUM(kb2*abv))
            ss=ss*ss
            clb(ib)=clb(ib)+(PI/6.)*ss/kd4
            clb(ib)=clb(ib)*cp1*kb2(ib)*eb(ib)/vp 
         ELSE                  ! general velocities are numerical
!
! int1: dedxc=(cp1/vp)*intone(abv,bbv,kb2,eb,cb)
!
            a=abv(ib)
            b=bb(ib)*vp2
            c=cb(ib)
            ke=kb2(ib)*eb(ib)
            clb(ib)=clb(ib)+(cp1/vp)*ke*intone(a,b,c)
!       
! int2: dedxc=dedxc + cp2*inttwo(abv,kb2) 
!
            clb(ib)=clb(ib)+cp2*inttwo(ib,abv)
!
! H: dedxc=dedxc - (cp3/vp2)*h(abv,kb2)
!
            clb(ib)=clb(ib)-(cp3/vp2)*hi(ib,abv)/rb(ib)
         ENDIF
      ELSE
         clb(ib)=0  ! don't compute when zb(ib) = 0
      ENDIF
   ENDDO
   dedxcsumi=SUM(clb(2:NNB))
   dedxctot =clb(1)
   dedxctot =dedxctot+dedxcsumi

   dedxctot =CONVFACT*dedxctot   ! convert to MeV/mu-m
   dedxcsumi=CONVFACT*dedxcsumi  !
 END SUBROUTINE dedxc

!
! Much of this integration was performed analytically,
! and the rest can be expressed in terms of Hypergeometric
! functions j1 ... j4 (which we in turn fit)
!
 FUNCTION intone(a, b, c) 
   USE globalvars

   IMPLICIT NONE
   REAL, INTENT(IN) :: a, b, c
   REAL             :: intone

   REAL    :: bc, a2, a3, erfa, expa, ferf
   REAL    :: j1, j2, j3, j4
   bc=b*c
   a2=a*a
   a3=a2*a
   erfa=SQPI*ferf(a)  ! see ferf.f
   expa=EXP(-a2)
   intone=erfa*((2-c)/a + bc/(2*a3))-bc*expa/a2
   intone=intone + j3(a2) - j1(a2) + b*(j2(a2) -j4(a2))
 END FUNCTION intone

!
!              /1
!              |
! inttwo = 2 * | du u*H(u*abv)
!              |
!              /0
!
! This subroutine performs the integration numerically by
! Gaussian Quadrature. The number of intervals NG that [0,1]
! is broken into is hardwired in PARAMETER statement, but 
! it can be changed by the user (must be even).
!
! NOTE on Gaussian Quadrature:
!
! The polynomial P3(x)=(5*x^3-3*x)/2 is employed, and I have
! defined the appropriate weights W13, W2 and relative position
! UPM in parameter statements. 
!
 FUNCTION inttwo(ib, abv)
   USE globalvars
    
   IMPLICIT NONE     
   REAL, DIMENSION(1:NNB), INTENT(IN) :: abv
   INTEGER,                INTENT(IN) :: ib
   REAL                               :: inttwo
   
   REAL, PARAMETER :: UPM=0.7745966692E0
   REAL, PARAMETER :: W13=0.5555555556E0, W2=0.8888888889E0
   
   INTEGER, PARAMETER :: NG=2000  ! NG must be even
   REAL,    PARAMETER :: U0=0.E0, U1=1.E0, DU=1.E0/NG
   
   REAL    :: u, um, hi, uu(NNB)
   INTEGER :: iu, ibb
   
   inttwo=0
   u=U0-DU
   DO iu=1,NG,2
!
      u=u+2.E0*DU
      DO ibb=1,NNB
         uu(ibb)=u*abv(ibb)
      ENDDO
      inttwo=inttwo+W2*u*hi(ib,uu)
!
      um=u-DU*UPM
      DO ibb=1,NNB
         uu(ibb)=um*abv(ibb)
      ENDDO
      inttwo=inttwo+W13*um*hi(ib,uu)
!
      um=u+DU*UPM
      DO ibb=1,NNB
         uu(ibb)=um*abv(ibb)
      ENDDO
      inttwo=inttwo+W13*um*hi(ib,uu)
   ENDDO
   inttwo=2*inttwo*DU
 END FUNCTION inttwo

!
! The ib-th component of Hb is hi=Sqrt[Pi]*kb^2*xb*Exp[-xb^2]*H/Fi 
! with Fi=Sqrt[Pi]Sum_c kc^2*xc*Exp[-xc^2] where xb=abv(ib) and 
! kb^2=kkb2(ib)
!                                     
 FUNCTION hi(ib, abv)
   USE globalvars
   IMPLICIT NONE
   REAL,    DIMENSION(1:NNB), INTENT(IN) :: abv
   INTEGER,                   INTENT(IN) :: ib
   REAL                                  :: hi
   REAL, PARAMETER :: EPS=1.E-6, XB2MAX=10.D0
   REAL    :: fr, fi, fabs, farg, h, ebc
   REAL    :: xb, xb2, xc, xc2, xbc
   INTEGER :: ic
   CALL fri(abv, fr, fi, fabs, farg)
   h=-2*(fi*LOG(fabs/K**2) + fr*farg)
   xb =abv(ib)
   xb2=xb*xb
!
! Express Fi in terms of its sum and write 
!
!   hi = kb^2*xb*Exp[-xb^2]/Sum_c kc^2*xc*Exp[-xc^2]
!
! then calculate hi^-1 first, excluding kb^2*xb since
! some components of kb^2*xb might be zero.
   hi=0                                       
   DO ic=1,NNB                               
      xc =abv(ic)                             
      xc2=xc*xc                              
      xbc=xb2-xc2                            
      IF (xbc .LT. XB2MAX) THEN                
         ebc=EXP(xb2-xc2)                    
      ELSE                                   
         hi=0    ! ebc is large and dominates the sum, 
         RETURN  ! and its inverse is almost zero
      ENDIF
      hi=hi + kb2(ic)*xc*ebc                 
   ENDDO
   hi=kb2(ib)*xb/hi  ! invert and include kb^2*xb
   hi=h*hi
 END FUNCTION hi

!
!         /1
!         |              ln[1-u]
! j1(x) = | du e^{-x u} ---------
!         |              Sqrt[u]
!         /0
!
 FUNCTION j1(x)
!
! see globalvars for parameters
!
   USE globalvars   
   IMPLICIT NONE
   REAL, INTENT(IN) :: x
   REAL             :: j1

   REAL,    PARAMETER   :: XMIN=0.1D0, XMAX=20.D0
   REAL    :: x2, x4
   REAL    :: xx, ra, rc
   REAL    :: y, y3, y5
   INTEGER :: n
!
! analytic asymptotic forms
!
  IF (x .LE. XMIN) THEN
     j1=J1MM*x+J1BB
  ELSEIF (x .GT. XMAX) THEN
     y=SQRT(x)           ! x^1/2
     y3=x*y              ! x^3/2
     y5=y3*x             ! x^5/2
     j1=J1AA/y3 +J1GG/y5
  ELSE
!
! numerical asymptotic form
!
     x2=x*x
     x4=x**3.5
     j1=(J1MM*x+J1BB)/(J1CC*x4+1) + J1EE*J1AA*x2/(J1EE*x4+1)
!
! spline correction
! 
     ra=0.E0
     rc=0.E0
     xx=1.E0 
     DO n=0,NMJ1
        ra=ra+J1A(n)*xx
        rc=rc+J1B(n)*xx
        xx=x*xx
     ENDDO
     ra=ra+xx
     rc=rc+xx
     j1=j1*rc/ra
  ENDIF
 END FUNCTION j1
!
!         /1
!         |              
! j2(x) = | du e^{-x u} ln[1-u]*Sqrt[u]
!         |              
!         /0
!
 FUNCTION j2(x)
!
! see globalvars for parameters
!
   USE globalvars
   IMPLICIT NONE
   REAL, INTENT(IN) :: x
   REAL             :: j2

   REAL,    PARAMETER   :: XMIN=0.1, XMAX=30.D0
   REAL    :: x2, x4
   REAL    :: y, y5, y7
   REAL    :: xx, ra, rc
   INTEGER :: n

!
! analytic asymptotic forms
!
  IF (x .LE. XMIN) THEN
     j2=J2MM*x+J2BB
  ELSEIF (x .GT. XMAX) THEN
     y=SQRT(x)          ! x^1/2
     y5=x*x*y           ! x^5/2
     y7=y5*x            ! x^7/2
     j2=J2AA/y5 +J2GG/y7
  ELSE
!
! numerical asymptotic form
!
     x2=x*x
     x4=x**4.5
     j2=(J2MM*x+J2BB)/(J2CC*x4+1) + &
        J2EE*J2AA*x2/(J2EE*x4+1)
!
! spline correction
! 
     ra=0.E0
     rc=0.E0
     xx=1.E0
     DO n=0,NMJ2
        ra=ra+J2A(n)*xx
        rc=rc+J2B(n)*xx
        xx=x*xx
     ENDDO
     ra=ra+xx
     rc=rc+xx
     j2=j2*rc/ra
  ENDIF
 END FUNCTION j2
!
!         /1
!         |               ln[u]
! j3(x) = | du e^{-x u} --------
!         |              Sqrt[u]
!         /0
!
 FUNCTION j3(x)
!
! see globalvars for parameters
!
   USE globalvars
   IMPLICIT NONE
   REAL, INTENT(IN) :: x
   REAL             :: j3

   REAL,    PARAMETER   :: XMIN=0.4, XMAX=2.3D0
   REAL    :: x2, x3, x4
   REAL    :: xx, ra, rc
   REAL    :: y
   INTEGER :: n
!
! analytic asymptotic forms
!
  IF (x .LE. XMIN) THEN
     j3=J3MM*x+J3BB
  ELSEIF (x .GE. XMAX) THEN
     y=SQRT(x)
     j3=(J3AA1*LOG(x) + J3AA2)/y
  ELSE
!
! numerical asymptotic form 
!
     x2 =x*x
     x3 =x**2.5
     x4 =x**3.5
     j3=(J3MM*x+J3BB)/(J3CC*x4+1) + &
        J3EE*(J3AA1*LOG(1+x)+J3AA2)*x2/(J3EE*x3+1)
!
! spline correction
! 
     ra=0.E0
     rc=0.E0
     xx=1.E0
     DO n=0,NMJ3
        ra=ra+J3A(n)*xx
        rc=rc+J3B(n)*xx
        xx=x*xx
     ENDDO
     ra=ra+xx
     rc=rc+xx
     j3=j3*rc/ra
  ENDIF
END FUNCTION j3
!
!         /1
!         |               
! j4(x) = | du e^{-x u} ln[u]*Sqrt[u]
!         |              
!         /0
!
 FUNCTION j4(x)
!
! see globalvars for parameters
!
   USE globalvars
   IMPLICIT NONE
   REAL, INTENT(IN) :: x
   REAL             :: j4

   REAL,    PARAMETER   :: XMIN=0.18D0, XMAX=4.7D0
   REAL    :: x2, x4
   REAL    :: xx, ra, rc
   REAL    :: y, y3
   INTEGER :: n
!
! analytic asymptotic forms
!
  IF (x .LE. XMIN) THEN
     j4=J4MM*x+J4BB
  ELSEIF (x .GE. XMAX) THEN
     y=SQRT(x)                   ! x^1/2
     y3=x*y                      ! x^3/2
     j4=(J4AA1*LOG(x) + J4AA2)/y3
  ELSE 
!
! numerical asymptotic form 
!
     x2 =x*x
     x4 =x**3.5
     j4=(J4MM*x+J4BB)/(J4CC*x4+1) + &
         J4EE*(J4AA1*LOG(1+x)+J4AA2)*x2/(J4EE*x4+1)
!
! spline correction
! 
     ra=0.E0
     rc=0.E0
     xx=1.E0
     DO n=0,NMJ4
        ra=ra+J4A(n)*xx
        rc=rc+J4B(n)*xx
        xx=x*xx
     ENDDO
     ra=ra+xx
     rc=rc+xx
     j4=j4*rc/ra
  ENDIF
 END FUNCTION j4

!
! Returns the dielectric function F in terms of the 
! real part, the imaginary part, the absolute value, 
! and the argument: fr, fi, fabs, farg
!
 SUBROUTINE fri(xb, fr, fi, fabs, farg)
   USE globalvars
   IMPLICIT NONE
   REAL, DIMENSION(1:NNB), INTENT(IN)  :: xb
   REAL,                   INTENT(OUT) :: fr,  fi, fabs, farg
   REAL    :: x, daw, d
   INTEGER :: ib
   fr=0
   fi=0
   DO ib=1,NNB
      x=xb(ib)
      d=daw(x)
      fr=fr+(kb2(ib)*(1-2*x*d))
      fi=fi+kb2(ib)*x*EXP(-x*x)
   ENDDO
   fi=fi*SQPI
   fabs=SQRT(fr*fr + fi*fi)
   farg=ATAN2(fi,fr)
 END SUBROUTINE fri 

!
! The Dawson function takes the form
!
!
!          /x
!          |               
! daw(x) = | dy e^{y^2 -x^2}
!          |              
!          /0
!      
!        =(sqrt(pi)/2)*exp(-x^2)*erfi(x)
!
!
! For small x < XMIN we use the asymptotic form
!
!                 2x^3     4x^5
! daw(x) =  x  +  ----  + ----- + O(x^7) 
!                  3        15
!
! and for large x > XMAX we use 
!
!            1     1       3
! daw(x) =  --- + ---- + ----- + O(x^-7) 
!           2x    4x^3    8x^5
!
! The error is 0.03% for XMIN=0.4 and 0.01% XMAX=5.0. For 
! intermediate values, we approximate daw(x) as a rational 
! function of the form
!
!              x      x^6+b5*x^5+b4*x^4+b3*x^3+b2*x^2+b1*x+b0
! daw(x) = --------- ----------------------------------------
!          2 x^2 + 1  x^6+a5*x^5+a4*x^4+a3*x^3+a2*x^2+a1*x+b0
!
!
! With the values of bn and an chosen below, the error is 0.03%.
!
 FUNCTION daw(x)
!
! As x->0 and x->infty, the Dawson function takes the asymptotic
! forms daw(x)~x and daw(x)~1/(2x), respectively. The first 
! rational function "R(x)=x/(2x**2+1)" reproduces this behavior; 
! the 6-th order polynomial-ratio Q6(x) asymptotes to one at both 
! ends (to preserve the asymptotic form of the previous function), 
! with the coefficients a5, b5, ... b0 being chosen to provide 
! agreement with the exact Dawson integral at the values:
!
! x0=0.92413 daw(x0)=0.541044
! x1=0.2     daw(x1)=0.194751
! x2=0.5     daw(x2)=0.424436
! x3=0.7     daw(x3)=0.510504
! x4=1.2     daw(x4)=0.507273
! x5=1.4     daw(x5)=0.456507
! x6=1.6     daw(x6)=0.399940
! x7=2.0     daw(x7)=0.301340
! x8=3.0     daw(x8)=0.178271
! x9=4.0     daw(x9)=0.129348
! x10=8.0    daw(x10)=0.0630002
!
! See daw.nb for details.
!
   USE globalvars
   IMPLICIT NONE
   REAL, INTENT(IN) :: x
   REAL  daw

   REAL,    PARAMETER  :: XMIN=0.4D0, XMAX=5.D0
   REAL    :: x3, x5, xx, ra, rc
   INTEGER :: n
   IF (x .LE. XMIN) THEN
      x3=x*x*x
      x5=x3*x*x
      daw=x - 2.D0*x3/3.D0 + 4.D0*x5/15.D0
   ELSEIF (x .GE. XMAX) THEN
      x3=x*x*x
      x5=x3*x*x
      daw=1.D0/(2.D0*x)+1.D0/(4.D0*x3)+3.D0/(8.D0*x5)
   ELSE
      ra=0.E0
      rc=0.E0
      xx=1.E0
      DO n=0,NMDAW
         ra=ra+DWA(n)*xx
         rc=rc+DWB(n)*xx
         xx=x*xx
      ENDDO
      ra=ra+xx
      rc=rc+xx
      daw=x/(1.E0+2.E0*x*x)
      daw=daw*rc/ra
   ENDIF
 END FUNCTION daw

\end{verbatim}
}


\pagebreak
\begin{thebibliography}{99}

\bibitem{bps} 
  ``Charged Particle Motion in a Highly Ionized Plasma'', 
  L.~Brown, D.~Preston, and R.~Singleton~Jr., LA-UR-042713, 
  {\it Physics Reports}, {\bf 410/4}, 237 (2005), 
  [arXiv:physics/0501084].


\bibitem{Lifs}
  Section 29 of ``Physical Kinetics'',  E.~M.~Lifshitz and 
  L.~P.~Pitaevskii, Pergamon Press, Oxford, 1981.

\bibitem{abst}
``Handbook of Mathematical Functions'', M.~Abramowitz and 
I.~A.~Stegun, Dover Publications, $9^{\rm th}$ Edition,
1972. 



\end{thebibliography}


\end{document}

