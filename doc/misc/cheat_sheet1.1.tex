\documentclass[preprint,12pt,eqsecnum,nofootinbib,amsmath,amssymb]{revtex4}

% Date file was last changed:
\newcommand{\datechange}{3/4/2020}
\newcommand{\datestart}{3/4/2020}

% version
\newcommand\draftverson{v1.1}
\newcommand{\fname}{cheat\_sheet1.1.tex}
\newcommand{\laurnumber}{\draftverson  ~\today ~\currenttime}
\newcommand{\mydate}{\datechange}

% Person who last changed file:>
\newcommand{\whochange}{Robert Singleton}
%
% Project Name, path, informal author names, title
\newcommand{\projname}{Clog Doc}
\newcommand{\dirname}{Clog/doc/dedx}
\newcommand{\myauthors}{Robert Singleton}
\newcommand{\myrunningtitle}{\fname}
\newcommand{\mytitle}{Cheat Sheet}



% printing margins
%
\textwidth=6.5in
\textheight=9.5in

% packages
%
\usepackage{graphicx}  % Include figure files
\usepackage{dcolumn}  % Align table columns on decimal point
\usepackage{bm}             % Bold math: $\bm{\alpha}$
\usepackage{latexsym}  % Several additional symbols
\usepackage{fancyhdr}  % Fancy header package
\usepackage{wrapfig}
\usepackage{comment}
\usepackage{dsfont}
\usepackage{mathtools}
\usepackage{datetime}
%\usepackage{showkeys}% Displays equation and fig names
%\usepackage{hyperref}% Hyperlinked references

% local commands
\newcommand{\overoverline}[1]{ {\overline{\overline{#1}}} }
\newcommand{\EMPTYSET}{\varnothing}
\newcommand{\PROOF}{{\tiny PROOF}}
\newcommand{\ALTPROOF}{{\tiny ALTERNATE PROOF}}
\newcommand{\PAR}{$\blacktriangleright$}
\newcommand{\ENDPF}{$\blacksquare$}
\newcommand{\ENDPROOF}{$\blacksquare$}
%\newcommand{\ENDPF}{\square}
%\newcommand{\ENDPROOF}{$\square$}
\newcommand{\AND}{\wedge}
\newcommand{\OR}{\vee}
\newcommand{\NOT}{\neg}
\newcommand{\EQ}{\equiv}
\newcommand{\IFF}{\leftrightarrow}
\newcommand{\IMP}{\rightarrow}
\newcommand{\T}{{\rm T}}
\newcommand{\F}{{\rm F}}
\newcommand{\LOGEQ}{\sim}
\newcommand{\smDash}{{\rule[1mm]{0.1cm}{0.1mm}}}
\newcommand{\dbar}{{d\hskip-0.12cm \rule[2.2mm]{0.15cm}{0.1mm}}}
\newcommand{\smA}{{\scriptscriptstyle \rm A}}
\newcommand{\smB}{{\rm\scriptscriptstyle B}}
\newcommand{\smN}{{\rm\scriptscriptstyle N}}
\newcommand{\smX}{{\rm\scriptscriptstyle X}}
\newcommand{\bvec}[1]{\mathbf{#1}}
\newcommand{\smP}{{\rm\scriptscriptstyle P}}
\newcommand{\smL}{{\rm\scriptscriptstyle L}}
\newcommand{\smT}{{\rm\scriptscriptstyle T}}
\newcommand{\smC}{{\rm\scriptscriptstyle C}}
\newcommand{\smI}{{\rm\scriptscriptstyle I}}
\newcommand{\smR}{{\rm\scriptscriptstyle R}}
\newcommand{\smS}{{\rm\scriptscriptstyle S}}
\newcommand{\smQ}{{\rm\scriptscriptstyle Q}}
\newcommand{\smD}{{\rm\scriptscriptstyle D}}
\newcommand{\smO}{{\rm\scriptscriptstyle 0}}
\newcommand{\smW}{{\rm\scriptscriptstyle W}}
\newcommand{\smCT}{{\rm\scriptscriptstyle CT}}
\newcommand{\smQM}{{\rm\scriptscriptstyle QM}}
\newcommand{\smRe}{{\rm\scriptscriptstyle Re}}
\newcommand{\smIm}{{\rm\scriptscriptstyle Im}}
\newcommand{\smCL}{{\rm\scriptscriptstyle CL}}
\newcommand{\smBPS}{{\rm\scriptscriptstyle BPS}}
\newcommand{\smAMU}{{\rm\scriptscriptstyle AMU}}
\newcommand{\smE}{{\rm\scriptscriptstyle E}}
\newcommand{\smae}{{\rm\scriptscriptstyle ae}}
\newcommand{\extend}[2]{ {#1}^\smallfrown{\! #2} }
\newcommand{\smTC}{{\rm\scriptstyle TC}}
\newcommand{\calT}{ {\cal T}}
\newcommand{\calA}{{\cal A}}
\newcommand{\mathfrakA}{\mathfrak{A}}
\newcommand{\mathfrakB}{\mathfrak{B}}
\newcommand{\mathfrakS}{\mathfrak{S}}
\newcommand{\smGr}{{\rm\scriptscriptstyle gr}}
\newcommand{\smLT}{{\rm\scriptscriptstyle <}}
\newcommand{\smGT}{{\rm\scriptscriptstyle >}}
\newcommand{\smY}{{\rm\scriptscriptstyle Y}}

% % baselineskip modes
\newcommand{\bodyskip}{\baselineskip 18pt plus 1pt minus 1pt}
\newcommand{\bibskip}{\baselineskip16pt plus 1pt minus 1pt}
\newcommand{\tableofcontentsskip}{\baselineskip 14pt plus 1pt minus 1pt}
\newcommand{\footnoteskip}{\baselineskip 12pt plus 1pt minus 1pt}
\newcommand{\abstractskip}{\baselineskip 13pt plus 1pt minus 1pt}
\newcommand{\titleskip}{\baselineskip 18pt plus 1pt minus 1pt}
\newcommand{\affiliationskip}{\baselineskip 15pt plus 1pt minus 1pt}
\newcommand{\captionskip}{\footnotesize \baselineskip 12pt plus 1pt minus 1pt}
\newcommand{\enumerateskip}{\baselineskip 14pt plus 1pt minus 1pt}
\newcommand{\theoremskip}{\baselineskip 13pt plus 1pt minus 1pt}

% theorem
%
\newtheorem{theorem}{Theorem}
\newtheorem{corollary}[theorem]{Corollary}
\newtheorem{definition}[theorem]{Definition}
\newtheorem{lemma}[theorem]{Lemma}
\newtheorem{proposition}[theorem]{Proposition}
\newtheorem{example}[theorem]{Example}
%\newtheorem{theorem}{Theorem}
%\newtheorem{corollary}{Corollary}
%\newtheorem{definition}{Definition}

\pagestyle{fancy}
\lhead{\laurnumber}
%\lhead{}
\chead{}
\rhead{}
\lfoot{}
\cfoot{\thepage}
\rfoot{}

%%
%% begin: draw box
%%
%%%%%%%%%%%%%%%%%%%%%%%%%%%%%%
%%
%%  This macro draws a box around around text, taken 
%%  from ``TeX by Example'', by Arvind Borde p76.
%%
%%   To use: 
%%
%%   \vskip0.3cm
%%   \frame{.1}{2}{16.2cm}{\noindent
%%   \begin{eqnarray}
%%     a = b
%%   \end{eqnarray}
%%   }
%%   \vskip0.2cm
%%
%%%%%%%%%%%%%%%%%%%%%%%%%%%%%%%
%%
\def\frame#1#2#3#4{\vbox{\hrule height #1pt    % TOP RULE
  \hbox{\vrule width #1pt\kern #2pt                     % RULE/SPACE ON LEFT
  \vbox{\kern #2pt                                               % TOP SPACE
  \vbox{\hsize #3\noindent #4}                            % BOXED MATERIAL
  \kern #2pt}                                                        % BOTTOM SPACE
  \kern #2pt\vrule width #1pt}                              % RULE/SPACE ON RIGHT
  \hrule height0pt depth #1pt}                            % BOTTOM RULE
}
%%
\def\myframe#1{\vbox{\hrule height 0.1pt    % TOP RULE
  \hbox{\vrule width 0.1pt\kern 2pt                     % RULE/SPACE ON LEFT
  \vbox{\kern 2pt                                               % TOP SPACE
  \vbox{\hsize 16.5cm\noindent #1}                            % BOXED MATERIAL
  \kern 2pt}                                                        % BOTTOM SPACE
  \kern 2pt\vrule width 0.1pt}                              % RULE/SPACE ON RIGHT
  \hrule height0pt depth 0.1pt}                            % BOTTOM RULE
}
%%
%% draws two boxes around text (use sparingly)
%%
\def\fitframe #1#2#3{\vbox{\hrule height#1pt  % TOP RULE
  \hbox{\vrule width#1pt\kern #2pt             % RULE/SPACE ON LEFT
  \vbox{\kern #2pt\hbox{#3}\kern #2pt}         % TOP,MATERIAL,BOT
  \kern #2pt\vrule width#1pt}                  % RULE/SPACE ON RIGHT
  \hrule height0pt depth#1pt}                  % BOTTOM RULE
}
%%
%% draws a box with shadow around text
%%
\def\shframe #1#2#3#4{\vbox{\hrule height 0pt % NO TOP SHADOW
 \hbox{\vrule width #1pt\kern 0pt             % LEFT SHADOW
 \vbox{\kern-#1pt\frame{.3}{#2}{#3}{#4}       % START SHADOW
 \kern-.3pt}                                  % MOVE UP RULE
 \kern-#2pt\vrule width 0pt}                  % STOP SHADOW
 \hrule height #1pt}                          % BOTTOM SHADOW
}
%%
%%
%% end: draw box
%%
%%  To install as a package on a local host.
%%   a. Append the header ``\usepackage{myboxes}'' to the above macro. Name 
%%   the macreo file myboxes.sty.  Move myboxes.sty into $HOME/texmf/tex/mypackages/. 
%%   You might need to type texhash.
%%   b. T use the package write \usepackage{myboxes} in the preamble.

%
\begin{document}

%% notes info page
%\hfill{\laurnumber}
%\vskip0.3cm
\centerline{{ \Large\bf \projname: \fname}}
\vskip0.25cm 
\centerline{\bf \mytitle}
\vskip0.25cm
\centerline{\myauthors}
\vskip0.75cm 
\baselineskip 14pt plus 1pt minus 1pt
\begin{flushright}
Research Notes   \\[3pt]
{\it Project}:          \\
\projname                      \\
  {\it Path of TeX Source}:          \\
\dirname/\fname                      \\[3pt]
{\it Last Modified By}:            \\
\whochange                         \\
\datechange                        \\[3pt]
{\it Date Started:}                \\
\datestart                         \\[3pt]
{\it Date:}                \\
\draftverson~ \today ~\currenttime \\
\end{flushright}

\baselineskip 20pt plus 1pt minus 1pt

%% mini abstract
%\abstractskip
%\noindent
%These are notes on Logic from Ref.~\cite{ref_chang}.  
%\bodyskip

%% title page
\vskip2.0cm
%\pagebreak
\preprint{\laurnumber}

% publication title page
\title{\titleskip
  \mytitle
}

\author{Robert L Singleton Jr}

\affiliation{\affiliationskip
   School of Mathematics\\
   University of Leeds\\
   LS2 9JT
}

%\vskip 0.2cm 
%\affiliation{\affiliationskip
%     %$^1$
%     Los Alamos National Laboratory\\
%     Los Alamos, New Mexico 87545, USA
%}

\date{\datechange}

\begin{abstract}
\abstractskip
\vskip0.3cm 
\noindent
  Physics documentation for the BPS temperature equilibration in the code Clog.
\end{abstract}

%%
\maketitle
%%

% to change page settings
%\thispagestyle{empty}
%\pagestyle{empty}
%\setcounter{page}{0}

\pagebreak
\tableofcontentsskip
\tableofcontents
%\thispagestyle{empty}

%\pagebreak
\newpage
\bodyskip
%\setcounter{page}{1}

\pagebreak
\clearpage

\newpage


\section{Formulae}

%%
\begin{eqnarray}
  F(v) 
  &=& 
  -\int_{-\infty}^\infty \! du \, 
  \frac{\rho_\text{tot}(u)}{v - u + i\eta} 
  ~~~\text{with}~~
  \rho_\text{tot}(u)=\sum_b\rho_b(u)
\label{Fdef}
\\[5pt]
  \rho_b(v) 
  &=& 
  \kappa_b^2\,\sqrt{\frac{\beta_b m_b}{2\pi}}\, v\,
  \exp\!\left\{-\frac{1}{2}\,\beta_b m_b\, v^2\right\} \ ,
\label{rhototdef}
\end{eqnarray}
%%
and its relation to the dielectric function is $k^2 \, \epsilon( {\bf
  k} , {\bf k}\cdot {\bf v}_p ) = k^2 + F(\hat{\bf k} \cdot {\bf
  v}_p)$.  
%%
  The Dawson integral is defined by 
  \begin{eqnarray}
  {\rm daw}(x) = \int_0^x dy\, 
  e^{y^2 - x^2} = \frac{\sqrt{\pi}}{2}\, e^{-x^2}
  {\rm erfi}(x) \ ,
  \end{eqnarray}
%% 
%%
\begin{eqnarray}
  F_\smRe(v)
  &=& 
  \sum_b \kappa_b^2 
  \left[1 - 2 \sqrt{\frac{\beta_b m_b}{2}} ~v~
  {\rm daw}\left(\sqrt{\frac{\beta_b m_b}{2}}\,v 
  \right) \right]
\label{Fru}
\\[5pt]
  F_\smIm(v)
  &=&
  \sqrt{\pi} \sum_b \kappa_b^2 
  \sqrt{\frac{\beta_b m_b}{2}}~~
  v\, \exp\left[-\frac{\beta_b m_b}
  {2}\, v^2\right] = \pi  \rho_{\rm tot}(v)  \ .
\label{Fiu}
\end{eqnarray}
%% 
$F^*(v)=F(-v)$
%%
\begin{eqnarray}
  H(v) 
  &\equiv&
  -i\left[
  F(v) \ln\!\left\{ \frac{F(v)}{\kappa_e^2}\right\} -
  F^*(v)\ln\!\left\{\frac{F^*(v)}{\kappa_e^2} \right\}\right]
\label{Ha}
\\[5pt]
  &=& 
  \phantom{-}
  2\left[
  F_\smRe\, {\rm arg}\{F\}
  +
  F_{\smIm} \ln\!\left\{\frac{\vert F \vert}{\kappa_e^2}\right\}
   \right] \ .
\label{Hb}
\end{eqnarray}
%%
We can write the dielectric function as a sum over plasma components,
%%
\begin{eqnarray}
  F(v)={\sum}_bF_b(v) \ ,
\end{eqnarray}
%%
where we express the contribution from plasma species $b$ as
%%
\begin{eqnarray}
  F_b(v) 
  &=& 
\label{Fbdef}
  -\int_{-\infty}^\infty du\, \frac{\rho_b(v)}{v - u + i\eta}
\\[5pt]
  \rho_b(v) 
  &=& 
  \kappa_b^2\,\sqrt{\frac{\beta_b m_b}{2\pi}}\, v\,
  \exp\!\left\{-\frac{1}{2}\,\beta_b m_b\, v^2\right\} \ .
\label{barrhob}
\end{eqnarray}
%%
The functions $F_b(v)$ have units of wave-number-squared 
[$1/L^2$] and their argument $v$ has units of velocity.
We can express the functions $F_b(v)$ in terms of a single
dimensionless function $\mathbb F(x)$ as follows:
%%
\begin{eqnarray}
  F_b(v) 
  &=& 
  \kappa_b^2 \,\mathbb{F}\left(
  \sqrt{\frac{\beta_b m_b}{2}}\, v ~ \right)
\label{FbFbb}
\end{eqnarray}
%%
with
%%
\begin{eqnarray}
  \mathbb{F}(x) 
  &=& 
  -\int_{-\infty}^\infty dy\, \frac{\bar\rho(y)}{x - y + i \eta}
\label{Fbbdef}
\\[5pt]
  \bar\rho(y) 
  &=& 
  \frac{y}{\sqrt{\pi}}\, e^{-y^2} \ .
\label{rhobardef}
\end{eqnarray}
%%
Relation (\ref{FbFbb}) holds because $\rho_b(u) = \kappa_b^2\,\bar
\rho(y)$ for $u=(2/\beta_b m_b)^{1/2}\, y$. Note the reflection
property
%%
\begin{eqnarray}
  \mathbb{F}(-x) = \mathbb{F}^*(x) \  ,
\label{Freflect}
\end{eqnarray}
%%
which means that the real part is even in $x$ and the imaginary
part is odd,
%%
\begin{eqnarray}
  \mathbb{F}_\smRe(-x) &=& \phantom{-}\mathbb{F}_\smRe(x)
\\[5pt]
  \mathbb{F}_\smIm(-x) &=& -\mathbb{F}_\smIm(x) \ .
\end{eqnarray}
%%
As with expressions (\ref{Fru}) and (\ref{Fiu}), the real and
imaginary parts can be written
%%
\begin{eqnarray}
  \mathbb{F}_\smRe(x)
  &=& 
  1 - 2 x\,  {\rm daw}\left(x  \right) 
\label{FbarRe}
\\[5pt]
  \mathbb{F}_\smIm(x)
  &=&
  \pi\,  \bar\rho(x)  
  =
  \sqrt{\pi}\, x\, e^{-x^2}\ .
\label{FbarIm}
\end{eqnarray}
%% 
Since the Dawson function ${\rm daw}(x)$ is odd we see that
$\mathbb{F}_\smRe(x)$ is even. The real (blue) and the imaginary (red)
parts of $\mathbb{F}(x)$ are graphed in Fig.~** 
For a proof of (\ref{FbarRe}) and (\ref{FbarIm}), see
physics/research/Tei/BPS-classical/notes/cei\_reg.tex




%%
\begin{eqnarray}
  \mathbb{F}(x) 
  &=& 
  \int_{-\infty}^\infty dy\, \frac{\bar\rho(y)}{y - x - i \eta} \ ,
\end{eqnarray}
%%


\subsection{Asymptotic forms of $F$}


From expressions (\ref{FbFbb}), (\ref{FbarRe}) and (\ref{FbarIm}), 
the real and imaginary parts of $F_b(v)$ can be expressed as
%%
\begin{eqnarray}
  F^\smRe_b(v)
  &=& 
  \kappa_b^2 
  \left[1 - 2 \sqrt{\frac{\beta_b m_b}{2}} ~v~
  {\rm daw}\left\{\sqrt{\frac{\beta_b m_b}{2}}\,v 
  \right\} \right]
\label{FbRe}
\\[5pt]
  F^\smIm_b(v)
  &=&
  \kappa_b^2 \,
  \sqrt{\frac{\beta_b m_b\,\pi}{2}}~~
  v\, \exp\left\{-\frac{\beta_b m_b}
  {2}\, v^2\right\} \ .
\label{FbIm}
\end{eqnarray}
%% 

The limits of small and large arguments of the Dawson function
are
%%
\begin{eqnarray}
  {\rm daw}(x) 
  &=& 
  x +\frac{2 x^3}{3} + \frac{4 x^5}{15} + {\cal O}(x^7) 
\label{dawA}
\\[5pt]
  {\rm daw}(x)
  &=&
  \frac{1}{2x} +\frac{1}{4 x^3} + \frac{3}{8x^5} + {\cal O}(x^{-7}) \ .
\label{dawB}
\end{eqnarray}
%%


\vfill
\pagebreak
\section{Numerical Checks}


For a single component plasma note that (\ref{Fiu}) reduces to 
%%
\begin{eqnarray}
  F_\smRe(v)
  &=& 
  \kappa^2 
  \left[1 - 2 \sqrt{\frac{\beta m}{2}} ~v~
  {\rm daw}\left(\sqrt{\frac{\beta m}{2}}\,v 
  \right) \right]
\\[5pt]
  F_\smIm(v)
  &=&
  \sqrt{\pi} \kappa^2 
  \sqrt{\frac{\beta m}{2}}~~
  v\, \exp\left[-\frac{\beta m}
  {2}\, v^2\right] \ ;
\end{eqnarray}
%% 
and therefore $F(0)=\kappa^2$, with $\kappa^2= e^2 n/T$. For large
values of the argument, the Dawson function is\,\footnote{
\footnoteskip 
%%
  For completeness, small values of the argument give ${\rm daw}(x)= x
  + 2 x^3/3 + 4 x^5/15$.
%%
}


%%
\pagebreak
\section{Extra}

\subsection{Numerical Forms}

The following numerical forms have been used in the code:
%%
\begin{eqnarray}
  \kappa_b \cdot {\rm cm}
  &=&
  4.25390 \times 10^{-5} \,\vert Z_b \vert 
  \left(\frac{n_b \cdot {\rm cm^3}}{T_b/{\rm keV}} \right)^{1/2}
\\[5pt]
  \omega_b \cdot {\rm s}
  &=&
  1.32155 \times 10^3 \, \big\vert Z_b \big\vert 
  \Big(n_b \cdot{\rm cm}^3 \Big)^{1/2}
  \Big(\frac{m_\smAMU}{m_b} \Big)^{1/2}
\\[5pt]
  \omega_e \cdot {\rm s} 
  &=&   
  5.62016 \times 10^4 \left(n_e/{\rm cm}^3 \right)^{1/2}
\end{eqnarray}
where 
%%
\begin{eqnarray}
  m_\smAMU c^2 &=& 931.19 \times 10^3\,{\rm keV} \ .
\end{eqnarray}

%\newpage
%\begin{thebibliography}{99}
%\bibskip
%
%
%\bibitem{ref1}
%  Ref. 1 here.
%\end{thebibliography}

\end{document}
  


