\documentclass[preprint,12pt,eqsecnum,nofootinbib,amsmath,amssymb]{revtex4}

% Date file was last changed:
\newcommand{\datechange}{3/4/2020}
\newcommand{\datestart}{3/4/2020}

% version
\newcommand\draftverson{v1.0}
\newcommand{\fname}{acoeff1.0.tex}
\newcommand{\laurnumber}{\draftverson  ~\today ~\currenttime}
\newcommand{\mydate}{\datechange}

% Person who last changed file:>
\newcommand{\whochange}{Robert Singleton}
%
% Project Name, path, informal author names, title
\newcommand{\projname}{Clog Doc}
\newcommand{\dirname}{Clog/doc/acoeff}
\newcommand{\myauthors}{Robert Singleton}
\newcommand{\myrunningtitle}{\fname}
\newcommand{\mytitle}{A-Coefficient in Clog}
%
% printing margins
%

\textwidth=6.5in
\textheight=9.5in

% packages
%
\usepackage{graphicx}  % Include figure files
\usepackage{dcolumn}  % Align table columns on decimal point
\usepackage{bm}             % Bold math: $\bm{\alpha}$
\usepackage{latexsym}  % Several additional symbols
\usepackage{fancyhdr}  % Fancy header package
\usepackage{wrapfig}
\usepackage{comment}
\usepackage{dsfont}
\usepackage{mathtools}
\usepackage{datetime}
%\usepackage{showkeys}% Displays equation and fig names
%\usepackage{hyperref}% Hyperlinked references

% local commands
\newcommand{\overoverline}[1]{ {\overline{\overline{#1}}} }
\newcommand{\EMPTYSET}{\varnothing}
\newcommand{\PROOF}{{\tiny PROOF}}
\newcommand{\ALTPROOF}{{\tiny ALTERNATE PROOF}}
\newcommand{\PAR}{$\blacktriangleright$}
\newcommand{\ENDPF}{$\blacksquare$}
\newcommand{\ENDPROOF}{$\blacksquare$}
%\newcommand{\ENDPF}{\square}
%\newcommand{\ENDPROOF}{$\square$}
\newcommand{\AND}{\wedge}
\newcommand{\OR}{\vee}
\newcommand{\NOT}{\neg}
\newcommand{\EQ}{\equiv}
\newcommand{\IFF}{\leftrightarrow}
\newcommand{\IMP}{\rightarrow}
\newcommand{\T}{{\rm T}}
\newcommand{\F}{{\rm F}}
\newcommand{\LOGEQ}{\sim}
\newcommand{\smDash}{{\rule[1mm]{0.1cm}{0.1mm}}}
\newcommand{\dbar}{{d\hskip-0.12cm \rule[2.2mm]{0.15cm}{0.1mm}}}
\newcommand{\smA}{{\scriptscriptstyle \rm A}}
\newcommand{\smB}{{\rm\scriptscriptstyle B}}
\newcommand{\smN}{{\rm\scriptscriptstyle N}}
\newcommand{\smX}{{\rm\scriptscriptstyle X}}
\newcommand{\bvec}[1]{\mathbf{#1}}
\newcommand{\smP}{{\rm\scriptscriptstyle P}}
\newcommand{\smL}{{\rm\scriptscriptstyle L}}
\newcommand{\smT}{{\rm\scriptscriptstyle T}}
\newcommand{\smC}{{\rm\scriptscriptstyle C}}
\newcommand{\smI}{{\rm\scriptscriptstyle I}}
\newcommand{\smR}{{\rm\scriptscriptstyle R}}
\newcommand{\smS}{{\rm\scriptscriptstyle S}}
\newcommand{\smQ}{{\rm\scriptscriptstyle Q}}
\newcommand{\smD}{{\rm\scriptscriptstyle D}}
\newcommand{\smO}{{\rm\scriptscriptstyle 0}}
\newcommand{\smW}{{\rm\scriptscriptstyle W}}
\newcommand{\smCT}{{\rm\scriptscriptstyle CT}}
\newcommand{\smQM}{{\rm\scriptscriptstyle QM}}
\newcommand{\smRe}{{\rm\scriptscriptstyle Re}}
\newcommand{\smIm}{{\rm\scriptscriptstyle Im}}
\newcommand{\smE}{{\rm\scriptscriptstyle E}}
\newcommand{\smae}{{\rm\scriptscriptstyle ae}}
\newcommand{\extend}[2]{ {#1}^\smallfrown{\! #2} }
\newcommand{\smTC}{{\rm\scriptstyle TC}}
\newcommand{\calT}{ {\cal T}}
\newcommand{\calA}{{\cal A}}
\newcommand{\mathfrakA}{\mathfrak{A}}
\newcommand{\mathfrakB}{\mathfrak{B}}
\newcommand{\mathfrakS}{\mathfrak{S}}
\newcommand{\smGr}{{\rm\scriptscriptstyle gr}}
\newcommand{\smLT}{{\rm\scriptscriptstyle <}}
\newcommand{\smGT}{{\rm\scriptscriptstyle >}}
\newcommand{\smY}{{\rm\scriptscriptstyle Y}}

% % baselineskip modes
\newcommand{\bodyskip}{\baselineskip 18pt plus 1pt minus 1pt}
\newcommand{\bibskip}{\baselineskip16pt plus 1pt minus 1pt}
\newcommand{\tableofcontentsskip}{\baselineskip 14pt plus 1pt minus 1pt}
\newcommand{\footnoteskip}{\baselineskip 12pt plus 1pt minus 1pt}
\newcommand{\abstractskip}{\baselineskip 13pt plus 1pt minus 1pt}
\newcommand{\titleskip}{\baselineskip 18pt plus 1pt minus 1pt}
\newcommand{\affiliationskip}{\baselineskip 15pt plus 1pt minus 1pt}
\newcommand{\captionskip}{\footnotesize \baselineskip 12pt plus 1pt minus 1pt}
\newcommand{\enumerateskip}{\baselineskip 14pt plus 1pt minus 1pt}
\newcommand{\theoremskip}{\baselineskip 13pt plus 1pt minus 1pt}

% theorem
%
\newtheorem{theorem}{Theorem}
\newtheorem{corollary}[theorem]{Corollary}
\newtheorem{definition}[theorem]{Definition}
\newtheorem{lemma}[theorem]{Lemma}
\newtheorem{proposition}[theorem]{Proposition}
\newtheorem{example}[theorem]{Example}
%\newtheorem{theorem}{Theorem}
%\newtheorem{corollary}{Corollary}
%\newtheorem{definition}{Definition}

\pagestyle{fancy}
\lhead{\laurnumber}
%\lhead{}
\chead{}
\rhead{}
\lfoot{}
\cfoot{\thepage}
\rfoot{}

%%
%% begin: draw box
%%
%%%%%%%%%%%%%%%%%%%%%%%%%%%%%%
%%
%%  This macro draws a box around around text, taken 
%%  from ``TeX by Example'', by Arvind Borde p76.
%%
%%   To use: 
%%
%%   \vskip0.3cm
%%   \frame{.1}{2}{16.2cm}{\noindent
%%   \begin{eqnarray}
%%     a = b
%%   \end{eqnarray}
%%   }
%%   \vskip0.2cm
%%
%%%%%%%%%%%%%%%%%%%%%%%%%%%%%%%
%%
\def\frame#1#2#3#4{\vbox{\hrule height #1pt    % TOP RULE
  \hbox{\vrule width #1pt\kern #2pt                     % RULE/SPACE ON LEFT
  \vbox{\kern #2pt                                               % TOP SPACE
  \vbox{\hsize #3\noindent #4}                            % BOXED MATERIAL
  \kern #2pt}                                                        % BOTTOM SPACE
  \kern #2pt\vrule width #1pt}                              % RULE/SPACE ON RIGHT
  \hrule height0pt depth #1pt}                            % BOTTOM RULE
}
%%
\def\myframe#1{\vbox{\hrule height 0.1pt    % TOP RULE
  \hbox{\vrule width 0.1pt\kern 2pt                     % RULE/SPACE ON LEFT
  \vbox{\kern 2pt                                               % TOP SPACE
  \vbox{\hsize 16.5cm\noindent #1}                            % BOXED MATERIAL
  \kern 2pt}                                                        % BOTTOM SPACE
  \kern 2pt\vrule width 0.1pt}                              % RULE/SPACE ON RIGHT
  \hrule height0pt depth 0.1pt}                            % BOTTOM RULE
}
%%
%% draws two boxes around text (use sparingly)
%%
\def\fitframe #1#2#3{\vbox{\hrule height#1pt  % TOP RULE
  \hbox{\vrule width#1pt\kern #2pt             % RULE/SPACE ON LEFT
  \vbox{\kern #2pt\hbox{#3}\kern #2pt}         % TOP,MATERIAL,BOT
  \kern #2pt\vrule width#1pt}                  % RULE/SPACE ON RIGHT
  \hrule height0pt depth#1pt}                  % BOTTOM RULE
}
%%
%% draws a box with shadow around text
%%
\def\shframe #1#2#3#4{\vbox{\hrule height 0pt % NO TOP SHADOW
 \hbox{\vrule width #1pt\kern 0pt             % LEFT SHADOW
 \vbox{\kern-#1pt\frame{.3}{#2}{#3}{#4}       % START SHADOW
 \kern-.3pt}                                  % MOVE UP RULE
 \kern-#2pt\vrule width 0pt}                  % STOP SHADOW
 \hrule height #1pt}                          % BOTTOM SHADOW
}
%%
%%
%% end: draw box
%%
%%  To install as a package on a local host.
%%   a. Append the header ``\usepackage{myboxes}'' to the above macro. Name 
%%   the macreo file myboxes.sty.  Move myboxes.sty into $HOME/texmf/tex/mypackages/. 
%%   You might need to type texhash.
%%   b. T use the package write \usepackage{myboxes} in the preamble.

%
\begin{document}

%% notes info page
%\hfill{\laurnumber}
%\vskip0.3cm
\centerline{{ \Large\bf \projname: \fname}}
\vskip0.25cm 
\centerline{\bf \mytitle}
\vskip0.25cm
\centerline{\myauthors}
\vskip0.75cm 
\baselineskip 14pt plus 1pt minus 1pt
\begin{flushright}
Research Notes   \\[3pt]
{\it Project}:          \\
\projname                      \\
  {\it Path of TeX Source}:          \\
\dirname/\fname                      \\[3pt]
{\it Last Modified By}:            \\
\whochange                         \\
\datechange                        \\[3pt]
{\it Date Started:}                \\
\datestart                         \\[3pt]
{\it Date:}                \\
\draftverson~ \today ~\currenttime \\
\end{flushright}

\baselineskip 20pt plus 1pt minus 1pt

%% mini abstract
%\abstractskip
%\noindent
%These are notes on Logic from Ref.~\cite{ref_chang}.  
%\bodyskip

%% title page
\vskip2.0cm
%\pagebreak
\preprint{\laurnumber}

% publication title page
\title{\titleskip
  \mytitle
}

\author{Robert L Singleton Jr}

\affiliation{\affiliationskip
   School of Mathematics\\
   University of Leeds\\
   LS2 9JT
}

%\vskip 0.2cm 
%\affiliation{\affiliationskip
%     %$^1$
%     Los Alamos National Laboratory\\
%     Los Alamos, New Mexico 87545, USA
%}

\date{\datechange}

\begin{abstract}
\abstractskip
\vskip0.3cm 
\noindent
  Physics documentation for the BPS stopping power in the code Clog.
\end{abstract}

%%
\maketitle
%%

% to change page settings
%\thispagestyle{empty}
%\pagestyle{empty}
%\setcounter{page}{0}

\pagebreak
\tableofcontentsskip
\tableofcontents
%\thispagestyle{empty}

%\pagebreak
\newpage
\bodyskip
%\setcounter{page}{1}

\pagebreak
\clearpage

\section{General Analytic Expressions for the BPS A-Coefficients}

Suppose we have a plasma with various species labeled by an index $b$
at distinct temperatures $T_b$, number densities $n_b$, and species
masses $m_b$.\footnote{By convention $b=1$ will be the electron
component.}  Temperature will be measured in energy units, and we
denote the inverse temperature by $\beta_b =1/T_b$. Electrostatic
units will be rationalized cgs. The projectile will have mass $m_a$,
charge $e_a$ and energy $E_a = \frac{1}{2}\, m_a v_a^2$.  The BPS
{\cal A}-coefficients take the form
%%
\begin{eqnarray}
  {\cal A}_{ab} 
  &=&  
  \Big({\cal A}^\smLT_{ab,\smR}  + {\cal A}^\smC_{ab,\smS} \Big) +
  {\cal A}^{\Delta Q}_{ab} 
\label{all}
\end{eqnarray}
%%
with
%%
\begin{eqnarray}
  {\cal A}^\smLT_{ab,\smR}
  &=&
  \frac{e_a^2}{4 \pi}\, \frac{i}{2 \pi}
  \int_{-1}^1 \! d\cos\theta\, \cos\theta\,
  \frac{\rho_b(v_a\cos\theta)}
  {\rho_{\rm total}(v_a\cos\theta)}\,F(v_a \cos\theta) 
  \ln\!\left\{\frac{F(v_a\cos\theta)}{K^2}\right\} \,, 
\label{nun}
\\[10pt]
  {\cal A}^\smC_{ab,\smS}
  &=& 
  \frac{e_a^2\,\kappa^2_b}{4\pi} \,   
  \left( \frac{\beta_b m_b}{2\pi} \right)^{1/2}\!\!
  v_a\int_0^1 du \, u^{1/2} \,\exp\left\{ - \frac{1}{2} \,
  \beta_b m_b v^2_a \, u \right\}
\nonumber\\
  && \qquad
  \left[ -\ln \left(\beta_b  \frac{e_a e_b}{4\pi} \,
  K \, \frac{m_b}{m_{ab}} \, \frac{u}{1-u} \right) 
  - 2 \gamma + 2 \right] 
\label{wonderclassic}
\\[10pt]
  {\cal A}^{\Delta Q}_{ab}
  &=&
  -\frac{e_a^2\, \kappa_b^2}{4 \pi}\,
  \left( \frac{\beta_b m_b}{2\pi} \right)^{1/2}\,
  \int_0^\infty dv_{ab}
  \bigg\{ {\rm Re} \, \psi \left( 1 + i \eta_{ab}
  \right) - \ln \eta_{ab}  \bigg\}
\nonumber\\
  &&
  \frac{1}{\beta_b m_b v_a v_{ab}} \, 
  \Bigg[ \exp\left\{ - \frac{1}{2}\, \beta_b
  m_b \left( v_a - v_{ab} \right)^2\right\} 
   \left( 1 - \frac{1}{\beta_b m_b v_a v_{ab} } \right) 
\nonumber\\
  && \qquad\qquad\quad
  + \exp\left\{ - \frac{1}{2} \beta_b m_b \left( v_a + v_{ab} 
  \right)^2\right\} 
   \left( 1 + \frac{1}{\beta_b m_b v_a v_{ab} } \right) 
\Bigg]	\,,
\label{regb}
\end{eqnarray}
%% 
and
%%
\begin{eqnarray}
\eta_{ab}= \frac{e_a e_b}{4\pi\hbar v_{ab}} \ .
\end{eqnarray}
%%
The Debye wavenumber $K$ is arbitrary and will typically be chosen as
$K=\kappa_e$. The function $F(v)$ takes the form
%%
\begin{eqnarray}
  F(v) 
  &=& 
  -\int_{-\infty}^\infty \! du \, 
  \frac{\rho_\text{tot}(u)}{v - u + i\eta} 
  ~~~\text{with}~~
  \rho_\text{tot}(u)=\sum_b\rho_b(u)
\label{Fdef}
\\[5pt]
  \rho_b(v) 
  &=& 
  \kappa_b^2\,\sqrt{\frac{\beta_b m_b}{2\pi}}\, v\,
  \exp\!\left\{-\frac{1}{2}\,\beta_b m_b\, v^2\right\} \ ,
\label{rhototdef}
\end{eqnarray}
%%
and its relation to the dielectric function is 
%%
\begin{eqnarray}
  k^2 \, \epsilon({\bf k} , {\bf k}\cdot {\bf v} ) = k^2 + 
  F(\hat{\bf k} \cdot {\bf v}) \ .
\end{eqnarray}
%%
The first term ${\cal A}_{b ,\smR}^\smLT$ arises from long-distance
collective effects from the dielectric function, and it involves {\em
all} plasma species (even species $c$ different from $a$ and $b$).
This is the term I call non-separable, meaning that it cannot be
written as a sum of individual plasma components involving only a
single species. The second term ${\cal A}_{b ,\smS}^\smC$ arises from
short-distance two-body classical scattering, and the third term
${\cal A}_{b}^\smQM$ is the two-body quantum scattering correction to
all orders in the quantum parameters $\bar\eta_{ab}$. Three body and
higher effects are contained in our systematic error term, the
next-to-next-to-leading order term proportional to $g^3$. In a
strongly coupled plasma these higher order effects dominate, but in a
weakly coupled plasma they are negligible.

\pagebreak
\section{The Main Driver}

I will return the ${\cal A}$-coefficients in three forms:
%%
\begin{enumerate}
  \baselineskip 10pt plus 1pt minus 1pt
  \setlength{\itemsep}{3pt} % single spacing
  \setlength{\parskip}{1pt} %
  \setlength{\parsep}{0pt}  %

\item[i.] \verb+bps_acoeff_ab_mass+: For a given pair of indices $p$
and $b$ (the projectile $p$ will often be denoted by species index
$a$), this routine returns the individual component ${\cal A}_{ab}(E)$
for a given energy $E$. The quantum parameter $\eta$ can be arbitrary. 
This routine is used to construct the entries in the next two subroutines.

\item[ii.] \verb+bps_acoeff_ab_matrix+: Returns the complete
matrix of coefficients ${\cal A}_{ab}(E)$. 


\item[iii.] \verb+bps_acoeff_ei_mass+: This routine returns the sum
over the ions ${\cal A}_{p\smI}={\sum}_i {\cal A}_{pi}$ for a given
projectile $p$. It also returns the coulomb logarithm.

\end{enumerate}
%%

\subsection{The Driver Routine:
\lowercase{bps}\_\lowercase{acoeff}\_\lowercase{ab}\_\lowercase{mass}}



This subroutine returns the matrix of values ${\cal A}_{ab}(E)$ for a
given energy $E$. The driving routine that calls and assembles the
singular, regular, and quantum pieces.  
\vskip0.4cm
\noindent
acoeff.f90:bps\_acoeff\_ab\_mass
{
\baselineskip 10pt
\begin{verbatim}
      SUBROUTINE bps_acoeff_ab_mass(nni, ep, mp, zp, ia, ib, betab, zb, mb, nb, &
            a_ab, a_ab_sing, a_ab_reg, a_ab_qm)
      USE physvars
      USE mathvars    
        IMPLICIT NONE                                             ! Plasma:
        INTEGER,                            INTENT(IN)  :: nni    !  number of ions
        REAL,                               INTENT(IN)  :: ep     !  energy input [keV]
        REAL,                               INTENT(IN)  :: mp     !  mass [keV]
        REAL,                               INTENT(IN)  :: zp     !  charge
        INTEGER,                            INTENT(IN)  :: ia     !  
        INTEGER,                            INTENT(IN)  :: ib     !  
        REAL,    DIMENSION(1:nni+1),        INTENT(IN)  :: betab  !  temp array [1/keV]
        REAL,    DIMENSION(1:nni+1),        INTENT(IN)  :: mb     !  mass array [keV]
        REAL,    DIMENSION(1:nni+1),        INTENT(IN)  :: nb     !  density [1/cc]
        REAL,    DIMENSION(1:nni+1),        INTENT(IN)  :: zb     !  charge array
                                                                  !
                                                                  ! A-coeffs [MeV/micron]
        REAL,                               INTENT(OUT) :: a_ab
        REAL,                               INTENT(OUT) :: a_ab_sing
        REAL,                               INTENT(OUT) :: a_ab_reg
        REAL,                               INTENT(OUT) :: a_ab_qm

        REAL,    DIMENSION(1:nni+1)  :: mpb, mbpb, kb2, ab
        REAL                         :: vp, zp2, k, k2, kd, kd2, a, b, eta
        REAL                         :: ac_r, ac_s, aq, c1, c2

        REAL, PARAMETER              :: EPS_SMALL_E=2.E-4
        REAL, PARAMETER              :: EPS_SMALL_E_SING=2.E-4
        REAL, PARAMETER              :: EPS_SMALL_E_REG=2.E-4
!
! initialize components of A-coefficients
!
        kb2=8*PI*A0CM*BEKEV*zb*zb*nb*betab
        kd2 = SUM(kb2)                ! [1/cm^2]
        kd  = SQRT(kd2)               ! [1/cm]
        k2  = kb2(1)                  ! [1/cm^2]
        k   = SQRT(k2)                ! [1/cm]   k = k_e
!
! Loop over charged plasma species
!
        mpb = mp*mb/(mp+mb)            ! [keV]
        mbpb= mb/mpb                   ! [dimensionless]
        vp =CC*SQRT(2*ep/mp)           ! [cm/s]
        zp2=zp**2                      ! [dimensionless]
                                       ! ab=(1/2) betab(ib)*mbc2(ib)*vp2/CC2
        ab  =0.5*betab*mb*vp*vp/CC2    ! [dimensionless] 
        IF (zb(ib) .NE. 0.) THEN
        a  =ab(ib)
        b  =-Log(2*betab(ib)*BEKEV*ABS(zp*zb(ib))*k*A0CM*mbpb(ib) )-2*GAMMA+2
        eta=ABS(zp*zb(ib))*2.1870E8/vp ! defined with projectile velocity vp
        c1=2*zp2*BEKEV*kb2(ib)*A0CM    ! [keV/cm] c1 = e_p^2 kappa_b^2/(4 Pi)
        c1=c1*1.E-7                    ! [MeV/micron]  
        c2=SQRT(a/PI)                  ! [dimensionless] 
                                       ! c2=SQRT(betab(ib)*mb(ib)/TWOPI)*vp/CC 
!
! A_{ab}-classical-singular 
!
        CALL a_sing_mass(a,b,ac_s) 
        a_ab_sing=c1*c2*ac_s
!
! A_{ab}-classical-regular 
!
        CALL a_reg_mass(ib,nni,vp,k2,kb2,betab,mb,ac_r)
        a_ab_reg=c1*ac_r
!
! A_{ab}-quantum
!
        CALL a_quantum_mass(ia,ib,a,eta,aq) ! eta = dimensionless quantum param.
        a_ab_qm=c1*c2*aq
!
! A_{ab}-total
!
        a_ab=a_ab_sing + a_ab_reg + a_ab_qm
        ENDIF
      END SUBROUTINE bps_acoeff_ab_mass
\end{verbatim}
}


\vskip0.4cm 
\noindent
acoeff.f90:bps\_acoeff\_ab\_matrix
{
\baselineskip 10pt
\begin{verbatim}
        SUBROUTINE bps_acoeff_ab_matrix(nni, ep, betab, zb, mb, nb, &
            a_ab, a_ab_sing, a_ab_reg, a_ab_qm)
      USE physvars
      USE mathvars    
        IMPLICIT NONE                                             ! Plasma:
        INTEGER,                            INTENT(IN)  :: nni    !  number of ions
        REAL,                               INTENT(IN)  :: ep     !  energy input [keV]
        REAL,    DIMENSION(1:nni+1),        INTENT(IN)  :: betab  !  temp array [1/keV]
        REAL,    DIMENSION(1:nni+1),        INTENT(IN)  :: zb     !  charge array
        REAL,    DIMENSION(1:nni+1),        INTENT(IN)  :: mb     !  mass array [keV]
        REAL,    DIMENSION(1:nni+1),        INTENT(IN)  :: nb     !  density [1/cc]
                                                                  !
                                                                  ! A-coeffs [MeV/micron]
        REAL,    DIMENSION(1:nni+1,1:nni+1),INTENT(OUT) :: a_ab
        REAL,    DIMENSION(1:nni+1,1:nni+1),INTENT(OUT) :: a_ab_sing
        REAL,    DIMENSION(1:nni+1,1:nni+1),INTENT(OUT) :: a_ab_reg
        REAL,    DIMENSION(1:nni+1,1:nni+1),INTENT(OUT) :: a_ab_qm
        REAL    :: aab, aab_sing, aab_reg, aab_qm, mp, zp
        INTEGER :: ia, ib

! Loop over charged plasma species
!
       DO ia=1,nni+1
          mp=mb(ia)
          zp=zb(ia)
          DO ib=1,nni+1
            CALL bps_acoeff_ab_mass(nni, ep, mp, zp, ia, ib, betab, zb, mb, nb, &
            aab, aab_sing, aab_reg, aab_qm)
            a_ab(ia,ib)=aab
            a_ab_sing(ia,ib)=aab_sing
            a_ab_reg(ia,ib) =aab_reg
            a_ab_qm(ia,ib)  =aab_qm   
          ENDDO
      ENDDO
      END SUBROUTINE bps_acoeff_ab_matrix
\end{verbatim}
%%

\vskip0.4cm 
\noindent
acoeff.f90:bps\_acoeff\_ei\_mass
{
\baselineskip 10pt
\begin{verbatim}
        SUBROUTINE bps_acoeff_ei_mass(nni, ep, zp, mp, betab, zb, mb, nb, &
            a_tot, a_i, a_e, ac_tot, ac_i, ac_e, aq_tot, aq_i, aq_e,&
            ac_s_i, ac_s_e, ac_r_i, ac_r_e)
      USE physvars
      USE mathvars    
      USE controlvars  
        IMPLICIT NONE                                      ! Plasma:
        INTEGER,                     INTENT(IN)  :: nni    !  number of ions
        REAL,    DIMENSION(1:nni+1), INTENT(IN)  :: betab  !  temp array [1/keV]
        REAL,    DIMENSION(1:nni+1), INTENT(IN)  :: mb     !  mass array [keV]
        REAL,    DIMENSION(1:nni+1), INTENT(IN)  :: nb     !  density [1/cc]
        REAL,    DIMENSION(1:nni+1), INTENT(IN)  :: zb     !  charge array
                                                           !
                                                           ! Projectile  
        REAL,                        INTENT(IN)  :: ep     !  projectile energy [keV]
        REAL,                        INTENT(IN)  :: mp     !  projectile mass   [keV]
        REAL,                        INTENT(IN)  :: zp     !  projectile charge
                                                           !
                                                           ! A-coeffs [MeV/micron]
        REAL,                        INTENT(OUT) :: a_tot  !  electron + ion
        REAL,                        INTENT(OUT) :: a_i    !  ion contribution
        REAL,                        INTENT(OUT) :: a_e    !  electron contribution
        REAL,                        INTENT(OUT) :: ac_tot !  classical
        REAL,                        INTENT(OUT) :: ac_i   !  classical
        REAL,                        INTENT(OUT) :: ac_e   !  classical
        REAL,                        INTENT(OUT) :: aq_tot !  quantum
        REAL,                        INTENT(OUT) :: aq_i   !  quantum
        REAL,                        INTENT(OUT) :: aq_e   !  quantum
        REAL,                        INTENT(OUT) :: ac_s_i
        REAL,                        INTENT(OUT) :: ac_s_e 
        REAL,                        INTENT(OUT) :: ac_r_i
        REAL,                        INTENT(OUT) :: ac_r_e

        REAL     :: adum, ac_s, ac_r, aq
        INTEGER  :: ia, ib, nnb

!
! initialize components of A-coefficients
!
        a_tot =0  ! electron + ion
        a_i   =0  ! ion contribution
        a_e   =0  ! electron contribution
        ac_tot=0  ! classical total
        ac_e  =0  ! classical electron
        ac_i  =0  ! classical ion
        aq_tot=0  ! quantum total
        aq_e  =0  ! quantum electron
        aq_i  =0  ! quantum ion
        ac_s_i=0 
        ac_s_e=0 
        ac_r_i=0
        ac_r_e=0

        NNB = nni+1                 ! number of ions + electrons
        ia=1
        DO ib=1,nni+1
        IF (zb(ib) .NE. 0.) THEN
            CALL bps_acoeff_ab_mass(nni, ep, mp, zp, ia, ib, betab, zb, mb, nb, &
            adum, ac_s, ac_r, aq)
            CALL x_collect(ib, NNB, ac_s, ac_r, aq,       &
            a_tot, a_i, a_e, ac_tot, ac_i, ac_e, aq_tot,  &
            aq_i, aq_e, ac_s_i, ac_s_e, ac_r_i, ac_r_e)
        ENDIF
        ENDDO
      END SUBROUTINE bps_acoeff_ei_mass
\end{verbatim}
%%


\subsection{The Regular Contribution: a\_reg\_mass}


The long-distance regular contribution can be expressed as
%%
\begin{eqnarray}
  {\cal A}^\smLT_{ab,\smR} 
  &=&
  \frac{e_a^2}{4\pi}
  \frac{i}{2\pi} \int_{-1}^1 du\,u\,
  \frac{\rho_b(v_a u)}{\rho_\text{total}(v_a u)}\,
  F(v_a u) \ln \left\{ \frac{F(v_a u)}{K^2} \right\}
\\[5pt]
  &=&
  \frac{e_a^2}{4\pi}
  \frac{i}{2\pi} \int_0^1 du\,u\,
  \frac{\rho_b(v_a u)}{\rho_\text{total}(v_a u)}\!
  \left[
  F(v_a u) \ln \left\{ \frac{F(v_a u)}{K^2} \right\}
  -
  F^*(v_a u) \ln \left\{ \frac{F^*(v_a u)}{K^2} \right\}
  \right]
\\[5pt]
  &=&
  -\frac{e_a^2}{4\pi}
  \frac{1}{2\pi} \int_0^1 du\,u\,
  \frac{\rho_b(v_a u)}{\rho_\text{total}(v_a u)}\, H(v_a u) \ ,
\end{eqnarray}
%%
where we have defined
%%
\begin{eqnarray}
  H(v) 
  &\equiv&
  -i\left[
  F(v) \ln\!\left\{ \frac{F(v)}{K^2}\right\} -
  F^*(v)\ln\!\left\{\frac{F^*(v)}{K^2} \right\}\right]
  =
  2\left[
  F_\smRe\, {\rm arg}\{F\}
  +
  F_{\smIm} \ln\!\left\{\frac{\vert F \vert}{K^2}\right\}
   \right] \ .
\nonumber\\
\label{Hbagain}
\end{eqnarray}
%%
We shall factor out a dimensionfull wavenumber $K$ and define
dimensionless quantities $\mathbb{F}(v)$ and $\mathbb{H}(v)$
through
%%
\begin{eqnarray}
  F(v) &=& K^2\,  \mathbb{F}(v) 
  ~~~\text{and}~~~
  H(v) = K^2\,  \mathbb{H}(v) \ .
\end{eqnarray}
%%
Defining the parameters
%%
\begin{eqnarray}
  a_c
  &\equiv&
  \left(\frac{\beta_c m_c}{2} \right)^{1/2}
\\[5pt]
  \bar\kappa_c^2 
  &\equiv &
  \frac{\kappa_c^2}{K^2} 
\end{eqnarray}
%%
gives the real and imaginary parts of $\mathbb{F}$, 
%%
\begin{eqnarray}
  \mathbb{F}_\smRe(\{a_c v\},\{\bar\kappa_c\}) &=& 
  {\sum}_c \bar\kappa_c^2 \Big(1 - 2 a_c\, v\,{\rm daw}\{a_c\, v\}\Big)
\label{defFreA}
\\[5pt]
  \mathbb{F}_\smIm(\{a_c v\},\{\bar\kappa_c\}) &=& 
  \sqrt{\pi}\,{\sum}_c \bar\kappa_c^2\,a_c v\, e^{-a_c^2\, v^2} \ .
\label{defFimA}
\end{eqnarray}
%%
The ratio of weighting factors can be written in terms of a function
$\mathbb{R}_{ab}$ defined by

%%
\begin{eqnarray}
  \frac{\rho_b(v_a u)}{\rho_\text{total}(v_a u)} \cdot H(v_a u)
  &=&
  K^2\,\frac{\rho_b(v_a u)}{\rho_\text{total}(v_a u)} \cdot
  \mathbb{H}(v_a u)
\\[5pt]
  &=&
  K^2 \frac{
  \kappa_b^2 \left(\beta_b m_b/2\pi\right)^{1/2} \,v_a u\,
  e^{-\frac{1}{2}\,\beta_b m_b v_a^2\, u^2}}
  {{\sum}_c\kappa_c^2 \left(\beta_c m_c/2\pi\right)^{1/2} 
  \, v_a u\,  e^{-\frac{1}{2}\,\beta_c m_c v_a^2\, u^2}}
  \cdot \mathbb{H}(v_a u)
\\[5pt]
  &=&
  \kappa_b^2 \cdot 
  \underbrace{~
  \left[{\sum}_c \, \frac{\kappa_c^2}{K^2} 
  \left(\frac{\beta_c m_c}{\beta_b m_b}\right)^{1/2} 
  e^{\frac{1}{2}\,(\beta_b m_b-\beta_c m_c) v_a^2\, u^2}
  \right]^{-1}}_{\mathbb{R}_{ab}(v_a u)} \cdot \,
  \mathbb{H}(v_a u) 
\\[5pt]
  &=&
  \kappa_b^2 \, \mathbb{R}_{ab}(v_a u) \, \mathbb{H}(v_a u) \ .
\end{eqnarray}
%%
We can now express the regular piece as
%%
\begin{eqnarray}
  {\cal A}^\smC_{ab,\smR}
  &=& 
  \underbrace{
  \left[
  \frac{e_a^2\, \kappa_b^2}{4\pi}\,
  \right] 
  }_{c_{ab,1}}
  \cdot \,
  {\sf A}_{ab\,\smR}(v_a,\{a_c\},\{\bar\kappa_c\})
\\
  {\sf A}_{ab\,\smR}(v_a,\{a_c\},\{\bar\kappa_c\})
  &=&
  -\int_0^1 du\, u\,
  \underbrace{
  \mathbb{R}_{ab}(\{a_c v_a u\})\,
  \mathbb{H}(\{a_c\,v_a u\},\{\bar\kappa_c\})  
  }_{\rm dab\_reg}
  \ .
\end{eqnarray}
%%

\vskip0.4cm
\noindent
acoeff.f90: dab\_reg
{
\baselineskip 10pt
\begin{verbatim}
      FUNCTION dab_reg(u, vp, ib, nni, k2, kb2, betab, mb)
      USE mathvars
      USE physvars
        IMPLICIT NONE
        REAL,                        INTENT(IN)  :: u      ! [dimensionless]
        REAL,                        INTENT(IN)  :: vp     ! Projectile velocity [cm/s]
        INTEGER,                     INTENT(IN)  :: ib     ! Species number
        INTEGER,                     INTENT(IN)  :: nni    ! Number of ion species
        REAL,                        INTENT(IN)  :: k2     ! Wave-number squared [1/cm^2]
        REAL,    DIMENSION(1:nni+1), INTENT(IN)  :: kb2    ! Debye wavenumber squared [1/cm^2]
        REAL,    DIMENSION(1:nni+1), INTENT(IN)  :: betab  ! Temperature array [1/keV]
        REAL,    DIMENSION(1:nni+1), INTENT(IN)  :: mb     ! Mass array [keV]
        REAL                                     :: dab_reg! [dimensionless]
        REAL,    DIMENSION(1:nni+1) :: alfb, ab
        REAL                        :: fr, fi, fabs, farg, h, uvp
        REAL                        :: kcb, r_ib, bm_ic, bm_ib, a_ic, a_ib, ex, au
        INTEGER                     :: ic
        ab=SQRT(0.5*betab*mb)*vp/CC
        alfb=kb2/k2
        uvp=u*vp
        CALL frfi(u,nni,alfb,ab,fr,fi,fabs,farg)
        h=2*(fr*farg + fi*LOG(fabs))*u
!
! construct spectral weight ratio Rb=rho_b/rho_tot
!
        r_ib=0
        bm_ib=betab(ib)*mb(ib)
        a_ib =ab(ib)*ab(ib)
        DO ic=1,nni+1
           kcb=kb2(ic)/k2
           bm_ic=betab(ic)*mb(ic)
           a_ic =ab(ic)*ab(ic)
           IF (ic == ib) THEN
              ex=1.
           ELSE
              au=(a_ic-a_ib)*u
              ex=EXP(-au)
           ENDIF
           r_ib=r_ib + kcb*SQRT(bm_ic/bm_ib)*ex
        ENDDO      
        r_ib=1./r_ib
        dab_reg=-r_ib*h/TWOPI
      END FUNCTION dab_reg
\end{verbatim}
}

\noindent
The numerical integration is performed by Gaussian quadrature:
\vskip0.4cm
\noindent
acoeff.f90:
{
\baselineskip 10pt
\begin{verbatim}
      SUBROUTINE a_reg_mass(ib, nni, vp, k2, kb2, betab, mb, ac_r)
        IMPLICIT NONE
        INTEGER,                     INTENT(IN)  :: ib
        INTEGER,                     INTENT(IN)  :: nni 
        REAL,                        INTENT(IN)  :: vp
        REAL,                        INTENT(IN)  :: k2
        REAL,                        INTENT(IN)  :: kb2
        REAL,    DIMENSION(1:nni+1), INTENT(IN)  :: betab
        REAL,    DIMENSION(1:nni+1), INTENT(IN)  :: mb
        REAL,                        INTENT(OUT) :: ac_r
        INTEGER, PARAMETER :: NR=10 ! integration regions: must be even
        REAL,    PARAMETER :: UPM=0.7745966692E0 ! parameters for Gaussian Quad
        REAL,    PARAMETER :: W13=0.5555555556E0, W2=0.8888888889E0
        REAL               :: u0, u1, du, u, um, dab_reg
        INTEGER            :: iu
        ac_r=0
        u0=0.
        u1=1.
        du=(u1-u0)/NR
        u=u0-du
        DO iu=1,NR,2 ! Gaussian quadrature
           u=u+2.E0*du
           ac_r=ac_r+W2*dab_reg(u,vp,ib,nni,k2,kb2,betab,mb)
           um=u-du*UPM
           ac_r=ac_r+W13*dab_reg(um,vp,ib,nni,k2,kb2,betab,mb)
           um=u+du*UPM
           ac_r=ac_r+W13*dab_reg(um,vp,ib,nni,k2,kb2,betab,mb)
        ENDDO
        ac_r=ac_r*du
      END SUBROUTINE a_reg_mass
\end{verbatim}
}



\subsection{The Singular Contribution: a\_sing}

The singular contribution,
%%
\begin{eqnarray}
  {\cal A}^\smC_{b,\smS} 
  &=& 
  \left[
  \frac{e_p^2\, \kappa_b^2}{4\pi}
  \left( \frac{\beta_b m_b}{2\pi} \right)^{1/2} \!\! v_p
  \right]
  \int_0^1 du\, u^{1/2} e^{-\frac{1}{2}\, \beta_b m_b v_p^2 \,u}
  \left[-\ln\!\left\{\frac{\beta_b e_b e_p}{4\pi}\,K\,
  \frac{m_b}{m_{pb}}\,\frac{u}{1-u} \right\} 
  -2\gamma + 2
  \right] ,
\nonumber\\
\end{eqnarray}
%%
is quite easy to code. The integral can be broke into the pieces
%%
\begin{eqnarray}
  \int_0^1 du\, u^{1/2} e^{-\frac{1}{2}\, \beta_b m_b v_p^2 \,u}
  \Bigg[\ln\!\left\{\frac{u}{1-u} \right\} 
  -\ln\left\{\frac{\beta_b e_b e_p}{4\pi}\,K\,
  \frac{m_b}{m_{pb}}\right\} 
  -2\gamma + 2 \Bigg] \ ,
\end{eqnarray}
%%
which motivates the definition
%%
\begin{eqnarray}
  {\cal A}^\smC_{b,\smS} 
  &=& 
  c_{b, 1}\, c_{b, 2} 
  \cdot 
  {\sf A}_\smS(a_{pb},b_{pb})
\\
  {\sf A}_\smS(a,b)
  &=&
  \int_0^1 du\, u^{1/2} e^{-a \,u}
  \left[-\ln\!\left\{\frac{u}{1-u} \right\} + b \right]
\\
  a_{pb} &=& \frac{1}{2}\, \beta_b m_b v_p^2
  \hskip1cm 
  \text{and}
  \hskip1cm 
  b_{pb} =
  -\ln\left\{\frac{\beta_b e_b e_p}{4\pi}\,K\,
  \frac{m_b}{m_{pb}}\right\} -2\gamma + 2 
\\
  c_{b, 1} &=& \frac{e_p^2\, \kappa_b^2}{4\pi}
  ~~~~
  c_{b, 2} =
  \left( \frac{\beta_b m_b}{2\pi} \right)^{1/2} \! v_p  \ .
\end{eqnarray}
%%
The term involving $b$ can be integrated exactly, but we will use
Gaussian quadrature for both pieces. 

\vskip0.5cm 
\noindent
acoeff.f90:
{
\baselineskip 10pt
\begin{verbatim}
    FUNCTION dab_sing(u, a, b)
      IMPLICIT NONE  ! a=(1/2)*beta*mpc2*vp^2/C^2
      REAL,        INTENT(IN)  :: u ! [dimensionless]
      REAL,        INTENT(IN)  :: a ! [dimensionless] 
      REAL,        INTENT(IN)  :: b ! [dimensionless]
      REAL                     :: dab_sing ! [dimensionless]
      dab_sing=SQRT(u)*EXP(-a*u)*(-LOG(u/(1-u)) + b)
    END FUNCTION dab_sing
\end{verbatim}
}

\noindent
The numerical integration is performed by Gaussian quadrature:
\vskip0.4cm 
\noindent
acoeff.f90:
{
\baselineskip 10pt
\begin{verbatim}
      SUBROUTINE a_sing(a, b, ac_s)
        REAL,    INTENT(IN)  :: a
        REAL,    INTENT(IN)  :: b
        REAL,    INTENT(OUT) :: ac_s
        REAL                 :: u0, u1, du, u, um
        INTEGER, PARAMETER :: NS=1000            ! integration regions singular: must be even
        REAL,    PARAMETER :: UPM=0.7745966692E0 ! parameters for Gaussian Quad
        REAL,    PARAMETER :: W13=0.5555555556E0, W2=0.8888888889E0
           ac_s=0
           u0=0
           u1=1
           du=(u1-u0)/NS
           u=u0-du
           DO iu=1,NS,2 ! Gaussian quadrature
              u=u+2.E0*du
              ac_s=ac_s+W2*dab_sing(u,a,b)
              um=u-du*UPM
              ac_s=ac_s+W13*dab_sing(um,a,b)
              um=u+du*UPM
              ac_s=ac_s+W13*dab_sing(um,a,b)
           ENDDO
           ac_s=ac_s*du
      END SUBROUTINE a_sing
\end{verbatim}
}


\subsection{The Quantum Correction: a\_quantum}

For the quantum term we make the change of variables
$v_{pb}= v_p\,u$ so that
%%
\begin{eqnarray}
  {\cal A}^\smQM_b
  &\!=\!&  
  -\frac{e_p^2\, \kappa_b^2}{4\pi}\,
  \left( \frac{\beta_b m_b}{2\pi} \right)^{1/2} \!\! v_p\,
  \int_0^\infty du\, 
  \Big[{\rm Re}\,\psi\!\left\{ 1 + i\, \frac{\bar\eta_{pb}}{u}
  \right\}
  - \ln\!\left\{\frac{\bar\eta_{pb}}{u} \right\} \Big] 
  \frac{1}{\beta_b m_b v_p^2\,u}
\nonumber
\\ &&
  \Bigg[ e^{-\frac{1}{2}\,\beta_b m_b v_p^2 (u-1)^2} 
  \left(1 - \frac{1}{\beta_b m_b v_p^2\, u}\right) +
  e^{-\frac{1}{2}\,\beta_b m_b v_p^2 (u+1)^2}  
  \left(1 + \frac{1}{\beta_b m_b v_p^2\, u}\right)
  \Bigg] \ .
\end{eqnarray}
%%
The quantum function we need to code is therefore
%%
\begin{eqnarray}
  {\cal A}^\smQM_b
  &\!=\!&  
  \underbrace{
  \left[\frac{e_p^2\, \kappa_b^2}{4\pi}\,
  \left( \frac{\beta_b m_b}{2\pi} \right)^{1/2} \!\! v_p\,
  \right] 
  }_{c_{b, 1} \cdot c_{b, 2}}
  \cdot \,
  {\sf A}_1^\smQM(a_{pb}, \tilde\eta_{pb}) \ ,
\label{qmA}
\end{eqnarray}
%%
where the arguments of the function are defined by 
%%
\begin{eqnarray}
  a_{pb} 
  &=& 
  \frac{1}{2}\,\beta_b m_b\, v_p^2
\\
  \tilde\eta_{pb}
  &=& 
  \frac{e_p e_b}{4 \pi \hbar v_p}
  =
  \vert Z_p Z_b\vert\,\frac{e^2}{8 \pi a_0}\,\frac{2 a_0}{\hbar}\,
  \frac{1}{v_p}
  =
  \vert Z_p Z_b\vert\ \cdot 13.606\,{\rm eV}\,\cdot
  \frac{2 \cdot 5.29 \times 10^{-9}\,{\rm cm}}{6.5821 \times
  10^{-16}\,{\rm eV\,s}}\,
  \frac{1}{v_p}
\nonumber
\\
  &=&
  2.1870 \times 10^8 \, \frac{\vert Z_p Z_b\vert}{v_p
  \cdot ({\rm cm/s})^{-1}} \ ,
\end{eqnarray}
%%
and the function itself takes the form
%%
\begin{eqnarray}
  {\sf A}_1^\smQM(a,\eta)
  &\!=\!&  
  -\int_0^\infty \! du\, 
  \left[{\rm Re}\,\psi\! \left\{1 + i\, \frac{\eta}{u} \right\} - 
  \ln\left\{ \frac{\eta}{u}  \right\} \right]
\nonumber
\\ && 
  \frac{1}{2 a\,u}\left[\left(e^{-a(u-1)^2} + e^{-a(u+1)^2} \right) 
  - \frac{e^{-a(u-1)^2} - e^{-a(u+1)^2} } {2 a\, u}\right] \ .
\label{qmAB}
\end{eqnarray}


\vskip1cm 
\noindent
acoeff.f90:
{
\baselineskip 10pt
\begin{verbatim}
      FUNCTION daq(u, a, eta)
      USE physvars
        IMPLICIT NONE
        REAL,                        INTENT(IN)  :: u          ! [dimensionless]
        REAL,                        INTENT(IN)  :: a          ! [dimensionless]
        REAL,                        INTENT(IN)  :: eta        ! [dimensionless]
        REAL                                     :: daq  ! [dimensionless]
        REAL, PARAMETER :: AMAX=25.
        REAL            :: repsi, au, eu, au2, ap, am, psilog, ch, sh
        eu=eta/u 
        psilog=repsi(eu) - LOG(eu)
        au =2*a*u
        au2=a*u*u
        IF (a <= AMAX) THEN      
          ch =EXP(-au2)*COSH(au)
          sh =EXP(-au2)*SINH(au)
        ELSE
          ap = au-au2-a
          am =-au-au2-a
          ch =0.5*(EXP(ap)+EXP(am))
          sh =0.5*(EXP(ap)-EXP(am))
        ENDIF
        daq=-psilog*2*(ch - sh/au)/au
      END FUNCTION daq


      SUBROUTINE a_quantum(ib, a, eta, aq)
        IMPLICIT NONE
        INTEGER, INTENT(IN)  :: ib    ! species index
        REAL,    INTENT(IN)  :: a     ! [dimensionless] (1/2) betab mb vp^2
        REAL,    INTENT(IN)  :: eta   ! [dimensionless] ep eb/4pi hbar vp
        REAL,    INTENT(OUT) :: aq 
        REAL               :: u0, u1, du, u, um
        INTEGER, PARAMETER :: NQ=1000            ! integration regions quantum : must be even
        REAL,    PARAMETER :: UPM=0.7745966692E0 ! parameters for Gaussian Quad
        REAL,    PARAMETER :: W13=0.5555555556E0, W2=0.8888888889E0
        REAL    :: daq
        INTEGER :: iu
        aq=0
        u0=0.
        aq=0
        IF (ib == 1) THEN
           u0=0
           u1=4./SQRT(a)
        ELSE
           u0=1-10./SQRT(a)
           u0=MAX(0.,u0)  
           u1=1+10./SQRT(a)
        ENDIF
        du=(u1-u0)/NQ
        u=u0-du
        DO iu=1,NQ,2 ! Gaussian quadrature
           u=u+2.E0*du
           aq=aq+W2*daq(u,a,eta)
           um=u-du*UPM
           aq=aq+W13*daq(um,a,eta)
           um=u+du*UPM
           aq=aq+W13*daq(um,a,eta)
        ENDDO
        aq=aq*du
      END SUBROUTINE a_quantum
\end{verbatim}
}






\pagebreak
\appendix



\section{Calculating the Real and Imaginary Parts of F}
\label{app:FrFi}

We can write the dielectric function (\ref{Fdef}) as a sum over plasma
components,
%%
\begin{eqnarray}
  F(v)={\sum}_b F_b(v) \ ,
\end{eqnarray}
%%
where we express the contribution from plasma species $b$ as
%%
\begin{eqnarray}
  F_b(v) 
  &=& 
\label{Fbdef}
  -\int_{-\infty}^\infty du\, \frac{\rho_b(v)}{v - u + i\eta}
\\[5pt]
  \rho_b(v) 
  &=& 
  \kappa_b^2\,\sqrt{\frac{\beta_b m_b}{2\pi}}\, v\,
  \exp\!\left\{-\frac{1}{2}\,\beta_b m_b\, v^2\right\} \ .
\label{barrhob}
\end{eqnarray} 
%%
We will often decompose $F$ into its contribution from
electrons and ions and write
%%
\begin{eqnarray}
 F(v)=F_e(v) + F_\smI(v) \ .
\end{eqnarray}
%%
Note the reflection property
%%
\begin{eqnarray}
  F_b(-v) 
  &=& 
  F_b^*(v) \ ,
\label{Fbreflect}
\end{eqnarray}
%%
which means that the real part of $F_b(v)$ is even in $v$ and the
imaginary part is odd. For numerical work it is best to use the
explicit real and imaginary parts of $F$, which can be written
%%
\begin{eqnarray}
  F_\smRe(v)
  &=& 
  \sum_b \kappa_b^2 
  \left[1 - 2 \sqrt{\frac{\beta_b m_b}{2}} ~v~
  {\rm daw}\left\{\sqrt{\frac{\beta_b m_b}{2}}\,v 
  \right\} \right]
\label{Fru}
\\[5pt]
  F_\smIm(v)
  &=&
  \sqrt{\pi} \sum_b \kappa_b^2 
  \sqrt{\frac{\beta_b m_b}{2}}~~
  v\, \exp\left\{-\frac{\beta_b m_b}
  {2}\, v^2\right\} = \pi  \rho_{\rm tot}(v)  \ ,
\label{Fiu}
\end{eqnarray}
%% 
where the Dawson integral is defined by 
%%
  \begin{eqnarray}
  {\rm daw}(x) = \int_0^x dy\, 
  e^{y^2 - x^2} = \frac{\sqrt{\pi}}{2}\, e^{-x^2}
  {\rm erfi}(x) \ .
  \end{eqnarray}
%% 
The limits of small and large arguments of the Dawson function
are
%%
\begin{eqnarray}
  {\rm daw}(x) 
  &=& 
  x +\frac{2 x^3}{3} + \frac{4 x^5}{15} + {\cal O}(x^7) 
\label{dawA}
\\[5pt]
  {\rm daw}(x)
  &=&
  \frac{1}{2x} +\frac{1}{4 x^3} + \frac{3}{8x^5} + {\cal O}(x^{-7}) \ .
\label{dawB}
\end{eqnarray}
%%

The functions $F_b(v)$ have units of wave-number-squared 
[$1/L^2$] and their argument $v$ has units of velocity.
We can express the functions $F_b(v)$ in terms of a single
dimensionless function $\mathbb F(x)$ as follows:
%%
\begin{eqnarray}
  F_b(v) 
  &=& 
  \kappa_b^2 \,\mathbb{F}\left(
  \sqrt{\frac{\beta_b m_b}{2}}\, v ~ \right)
\label{FbFbb}
\end{eqnarray}
%%
with
%%
\begin{eqnarray}
  \mathbb{F}(x) 
  &=& 
  \int_{-\infty}^\infty dy\, \frac{\bar\rho(y)}{y - x - i \eta} 
\\[5pt]
  \bar\rho(y)&=& \frac{1}{\sqrt{\pi}}\,y\,e^{-y^2}  \ .
\end{eqnarray}
%%
Relation (\ref{FbFbb}) holds because $\rho_b(u) = \kappa_b^2\,\bar
\rho(y)$ for $u=(2/\beta_b m_b)^{1/2}\, y$.  The reflection property
(\ref{Fbreflect}) becomes
%%
\begin{eqnarray}
  \mathbb{F}(-x) = \mathbb{F}^*(x) \  ,
\label{Freflect}
\end{eqnarray}
%%
which means that the real part is even in $x$ and the imaginary
part is odd,
%%
\begin{eqnarray}
  \mathbb{F}_\smRe(-x) &=& \phantom{-}\mathbb{F}_\smRe(x)
\\[5pt]
  \mathbb{F}_\smIm(-x) &=& -\mathbb{F}_\smIm(x) \ .
\end{eqnarray}
%%
As with expressions (\ref{Fru}) and (\ref{Fiu}), the real and
imaginary parts can be written
%%
\begin{eqnarray}
  \mathbb{F}_\smRe(x)
  &=& 
  1 - 2 x\,  {\rm daw}\left(x  \right) 
\label{FbarRe}
\\[5pt]
  \mathbb{F}_\smIm(x)
  &=&
  \pi\,  \bar\rho(x)  
  =
  \sqrt{\pi}\, x\, e^{-x^2}\ .
\label{FbarIm}
\end{eqnarray}
%% 



Let us now establish the forms (\ref{Fru}) and (\ref{Fiu}) for the real
and imaginary parts of $F(v)$.  
Staring with
%%
\begin{eqnarray}
  \frac{1}{y - x - i \eta} 
  =
  {\sf P}\frac{1}{y-x} + i \pi\,\delta(y-x) \ ,
\end{eqnarray}
%%
the imaginary part becomes 
%%
\begin{eqnarray}
  \mathbb{F}_\smIm(x) 
  &=& 
  \int_{-\infty}^\infty dy\, 
  {\rm Im}\,\frac{\bar\rho(y)}{y - x - i \eta} 
  =
  \int_{-\infty}^\infty dy\, \bar\rho(y) \,
  \pi \delta(y-x)
  =
  \pi \bar\rho(x) \ .
\end{eqnarray}
%%
The real part of the function must be evaluated by a principal part
integral,
%%
\begin{eqnarray}
  \mathbb{F}_\smRe(x) 
  &=& 
  {\sf P}\int_{-\infty}^\infty dy\, \frac{\bar\rho(y)}{y - x}  \ .
\end{eqnarray}
%%
Let us add and subtract unity in the form
%%
\begin{eqnarray}
  \int_{-\infty}^\infty \frac{dy}{y}\, \bar\rho(y) = 1 \ ,
\end{eqnarray}
%%
so that
%%
\begin{eqnarray}
  \mathbb{F}_\smRe(x) 
  &=& 
  1 +
  {\sf P}\int_{-\infty}^\infty dy\, 
  \left[\frac{\bar\rho(y)}{y - x} - \frac{\bar\rho(y)}{y}\right] 
\\[5pt]
  &=& 
  1 +
  {\sf P}\int_{-\infty}^\infty dy\, \frac{x}{y(y - x)}\, \bar\rho(y)
\\[5pt]
  &=& 
  1 +
  \frac{x}{\sqrt{\pi}}\,
  {\sf P}\int_{-\infty}^\infty dy\, \frac{e^{-y^2}}{y - x} \ .
\end{eqnarray}
%%
Making the change of variables $y^\prime=y-x$ (and dropping the prime)
we can write 
%%
\begin{eqnarray}
  \mathbb{F}_\smRe(x) 
  &=& 
  1 +
  \frac{x}{\sqrt{\pi}}\,
  {\sf P}\int_{-\infty}^\infty \frac{dy}{y}\, e^{-(y+x)^2} 
\\[5pt]
  &=&
  1 +
  \frac{x\,e^{-x^2}}{\sqrt{\pi}}\,
  {\sf P}\int_{-\infty}^\infty \frac{dy}{y}\, e^{-y^2 - 2 x y} 
\\[5pt]
  &=&
  1 +
  \frac{x\,e^{-x^2}}{\sqrt{\pi}}\lim_{\epsilon\to0^+}\left[
  \int_{-\infty}^{-\epsilon} \frac{dy}{y}\, e^{-y^2 - 2 x y} 
  +
  \int_\epsilon^\infty \frac{dy}{y}\, e^{-y^2 - 2 x y} 
  \right] \ .
\end{eqnarray}
%%
In the last expression we have used the definition of the principal
part integration.  Making a change of variables $y^\prime=-y$ in the
first integral in square brackets gives (and again dropping the prime)
%%
\begin{eqnarray}
  \int_{-\infty}^{-\epsilon} \frac{dy}{y}\, e^{-y^2 - 2 x y} 
  =
  -\int_\epsilon^\infty \frac{dy}{y}\, e^{-y^2 + 2 x y} \ ,
\end{eqnarray}
%%
and this allows us to write 
%%
\begin{eqnarray}
  \mathbb{F}_\smRe(x) 
  &=&
  1 -
  \frac{x\,e^{-x^2}}{\sqrt{\pi}}\lim_{\epsilon\to0^+}
  \int_\epsilon^\infty \frac{dy}{y}\, e^{-y^2}\Big[
  e^{2 x y}  - e^{- 2 x y} 
  \Big] \ .
\end{eqnarray}
%%
The term in square braces is just $2\sinh(2xy)$, which renders
the factor $1/y$ harmless when the limit $\epsilon \to 0^+$ is
taken, 
%%
\begin{eqnarray}
  \mathbb{F}_\smRe(x) 
  &=&
  1 -
  \frac{2x\,e^{-x^2}}{\sqrt{\pi}}  \int_0^\infty dy\, 
  e^{-y^2}\,   \frac{\sinh 2 x y }{y} 
  =
  1 - 2 x\, {\rm daw}(x) \ .
\end{eqnarray}
%%
The latter form hold because this is just another integral
representation of the Dawson function,
%%
\begin{eqnarray}
  {\rm daw}(x)
  &=&
  \frac{e^{-x^2}}{\sqrt{\pi}}  \int_0^\infty dy\, 
  e^{-y^2}\,   \frac{\sinh 2 x y }{y} \ .
\end{eqnarray}
%%
Compare this with
%%
\begin{eqnarray}
  {\rm daw}(x)
  &=&
  e^{-x^2} \! \int_0^x dy\,   e^{y^2} \ .
\end{eqnarray}
%%

%\pagebreak
%\appendix
%\section{Coding the A-coefficients}



%\bibitem{ref1}
%  R.J. Goldston and P.H. Rutherford, 
%  {\em Introduction to Plasma Physics},
%  IOP Publishing Ltd., Bristol UK, 2000.
%\end{thebibliography}

\end{document}
  


