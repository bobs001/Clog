\documentclass[preprint,12pt,eqsecnum,nofootinbib,amsmath,amssymb]{revtex4}

% Date file was last changed:
\newcommand{\datechange}{3/4/2020}
\newcommand{\datestart}{3/4/2020}

% version
\newcommand\draftverson{v1.4}
\newcommand{\fname}{rate1.4.tex}
\newcommand{\laurnumber}{\draftverson  ~\today ~\currenttime}
\newcommand{\mydate}{\datechange}

% Person who last changed file:>
\newcommand{\whochange}{Robert Singleton}
%
% Project Name, path, informal author names, title
\newcommand{\projname}{Clog Doc}
\newcommand{\dirname}{Clog/doc/dedx}
\newcommand{\myauthors}{Robert Singleton}
\newcommand{\myrunningtitle}{\fname}
\newcommand{\mytitle}{Temperature Equilibration Rate in Clog}

% log
% * cei_reg1.1.tex: 2008-11-21
% Collected from notes.
%
% * cei_reg1.2.tex: 2008-11-24
% First draft. Contains cei_sing as well. Will now change name
% of file to rate1.0.tex and continue. 
%
% * 2008-11-24 v1.0
%   rate1.0.tex
%
% * 2009-12-21: v1.1
%   rate1.1.tex: Coding the quantum piece.
%
% * 2010-01-11: v1.1
%   The dimensionless quantum integral is working, i.e. it agrees 
%   with mathematica.
%
% * 2010-01-12: v1.1
%   The full rate and the extreme quantum rate in rate.f90 do not 
%   agree (off by about a factor of two). 
%
%   For clarity, I adding an introduction with the code equations.   
%
% * 2010-03-05: v1.2
%   Refactoring code to look more like that in acoeff.f90. 
%   cei_sing and cei_qm are working. cei_reg is not working.
%   
% * 2010.03.24 v1.3
%   The singular and quantum pieces are working. The regular
%   pice of the rate only workds for a single ion species!?!
%
% * 2010.05.10
%   The regular piece is now working and a matrix of values C_{ab}
%   is now returned. I need to use this matrix to compute C_{eI}
%   [as I did with A_{ab} and A_{pI}]
%
% * 2010.03.24 v1.4
%   Using C_{ab} routine to compute C_{eI}.


% printing margins
%
\textwidth=6.5in
\textheight=9.5in

% packages
%
\usepackage{graphicx}  % Include figure files
\usepackage{dcolumn}  % Align table columns on decimal point
\usepackage{bm}             % Bold math: $\bm{\alpha}$
\usepackage{latexsym}  % Several additional symbols
\usepackage{fancyhdr}  % Fancy header package
\usepackage{wrapfig}
\usepackage{comment}
\usepackage{dsfont}
\usepackage{mathtools}
\usepackage{datetime}
%\usepackage{showkeys}% Displays equation and fig names
%\usepackage{hyperref}% Hyperlinked references

% local commands
\newcommand{\overoverline}[1]{ {\overline{\overline{#1}}} }
\newcommand{\EMPTYSET}{\varnothing}
\newcommand{\PROOF}{{\tiny PROOF}}
\newcommand{\ALTPROOF}{{\tiny ALTERNATE PROOF}}
\newcommand{\PAR}{$\blacktriangleright$}
\newcommand{\ENDPF}{$\blacksquare$}
\newcommand{\ENDPROOF}{$\blacksquare$}
%\newcommand{\ENDPF}{\square}
%\newcommand{\ENDPROOF}{$\square$}
\newcommand{\AND}{\wedge}
\newcommand{\OR}{\vee}
\newcommand{\NOT}{\neg}
\newcommand{\EQ}{\equiv}
\newcommand{\IFF}{\leftrightarrow}
\newcommand{\IMP}{\rightarrow}
\newcommand{\T}{{\rm T}}
\newcommand{\F}{{\rm F}}
\newcommand{\LOGEQ}{\sim}
\newcommand{\smDash}{{\rule[1mm]{0.1cm}{0.1mm}}}
\newcommand{\dbar}{{d\hskip-0.12cm \rule[2.2mm]{0.15cm}{0.1mm}}}
\newcommand{\smA}{{\scriptscriptstyle \rm A}}
\newcommand{\smB}{{\rm\scriptscriptstyle B}}
\newcommand{\smN}{{\rm\scriptscriptstyle N}}
\newcommand{\smX}{{\rm\scriptscriptstyle X}}
\newcommand{\bvec}[1]{\mathbf{#1}}
\newcommand{\smP}{{\rm\scriptscriptstyle P}}
\newcommand{\smL}{{\rm\scriptscriptstyle L}}
\newcommand{\smT}{{\rm\scriptscriptstyle T}}
\newcommand{\smC}{{\rm\scriptscriptstyle C}}
\newcommand{\smI}{{\rm\scriptscriptstyle I}}
\newcommand{\smR}{{\rm\scriptscriptstyle R}}
\newcommand{\smS}{{\rm\scriptscriptstyle S}}
\newcommand{\smQ}{{\rm\scriptscriptstyle Q}}
\newcommand{\smD}{{\rm\scriptscriptstyle D}}
\newcommand{\smO}{{\rm\scriptscriptstyle 0}}
\newcommand{\smW}{{\rm\scriptscriptstyle W}}
\newcommand{\smCT}{{\rm\scriptscriptstyle CT}}
\newcommand{\smQM}{{\rm\scriptscriptstyle QM}}
\newcommand{\smRe}{{\rm\scriptscriptstyle Re}}
\newcommand{\smIm}{{\rm\scriptscriptstyle Im}}
\newcommand{\smCL}{{\rm\scriptscriptstyle CL}}
\newcommand{\smBPS}{{\rm\scriptscriptstyle BPS}}
\newcommand{\smE}{{\rm\scriptscriptstyle E}}
\newcommand{\smae}{{\rm\scriptscriptstyle ae}}
\newcommand{\extend}[2]{ {#1}^\smallfrown{\! #2} }
\newcommand{\smTC}{{\rm\scriptstyle TC}}
\newcommand{\calT}{ {\cal T}}
\newcommand{\calA}{{\cal A}}
\newcommand{\mathfrakA}{\mathfrak{A}}
\newcommand{\mathfrakB}{\mathfrak{B}}
\newcommand{\mathfrakS}{\mathfrak{S}}
\newcommand{\smGr}{{\rm\scriptscriptstyle gr}}
\newcommand{\smLT}{{\rm\scriptscriptstyle <}}
\newcommand{\smGT}{{\rm\scriptscriptstyle >}}
\newcommand{\smY}{{\rm\scriptscriptstyle Y}}

% % baselineskip modes
\newcommand{\bodyskip}{\baselineskip 18pt plus 1pt minus 1pt}
\newcommand{\bibskip}{\baselineskip16pt plus 1pt minus 1pt}
\newcommand{\tableofcontentsskip}{\baselineskip 14pt plus 1pt minus 1pt}
\newcommand{\footnoteskip}{\baselineskip 12pt plus 1pt minus 1pt}
\newcommand{\abstractskip}{\baselineskip 13pt plus 1pt minus 1pt}
\newcommand{\titleskip}{\baselineskip 18pt plus 1pt minus 1pt}
\newcommand{\affiliationskip}{\baselineskip 15pt plus 1pt minus 1pt}
\newcommand{\captionskip}{\footnotesize \baselineskip 12pt plus 1pt minus 1pt}
\newcommand{\enumerateskip}{\baselineskip 14pt plus 1pt minus 1pt}
\newcommand{\theoremskip}{\baselineskip 13pt plus 1pt minus 1pt}

% theorem
%
\newtheorem{theorem}{Theorem}
\newtheorem{corollary}[theorem]{Corollary}
\newtheorem{definition}[theorem]{Definition}
\newtheorem{lemma}[theorem]{Lemma}
\newtheorem{proposition}[theorem]{Proposition}
\newtheorem{example}[theorem]{Example}
%\newtheorem{theorem}{Theorem}
%\newtheorem{corollary}{Corollary}
%\newtheorem{definition}{Definition}

\pagestyle{fancy}
\lhead{\laurnumber}
%\lhead{}
\chead{}
\rhead{}
\lfoot{}
\cfoot{\thepage}
\rfoot{}

%%
%% begin: draw box
%%
%%%%%%%%%%%%%%%%%%%%%%%%%%%%%%
%%
%%  This macro draws a box around around text, taken 
%%  from ``TeX by Example'', by Arvind Borde p76.
%%
%%   To use: 
%%
%%   \vskip0.3cm
%%   \frame{.1}{2}{16.2cm}{\noindent
%%   \begin{eqnarray}
%%     a = b
%%   \end{eqnarray}
%%   }
%%   \vskip0.2cm
%%
%%%%%%%%%%%%%%%%%%%%%%%%%%%%%%%
%%
\def\frame#1#2#3#4{\vbox{\hrule height #1pt    % TOP RULE
  \hbox{\vrule width #1pt\kern #2pt                     % RULE/SPACE ON LEFT
  \vbox{\kern #2pt                                               % TOP SPACE
  \vbox{\hsize #3\noindent #4}                            % BOXED MATERIAL
  \kern #2pt}                                                        % BOTTOM SPACE
  \kern #2pt\vrule width #1pt}                              % RULE/SPACE ON RIGHT
  \hrule height0pt depth #1pt}                            % BOTTOM RULE
}
%%
\def\myframe#1{\vbox{\hrule height 0.1pt    % TOP RULE
  \hbox{\vrule width 0.1pt\kern 2pt                     % RULE/SPACE ON LEFT
  \vbox{\kern 2pt                                               % TOP SPACE
  \vbox{\hsize 16.5cm\noindent #1}                            % BOXED MATERIAL
  \kern 2pt}                                                        % BOTTOM SPACE
  \kern 2pt\vrule width 0.1pt}                              % RULE/SPACE ON RIGHT
  \hrule height0pt depth 0.1pt}                            % BOTTOM RULE
}
%%
%% draws two boxes around text (use sparingly)
%%
\def\fitframe #1#2#3{\vbox{\hrule height#1pt  % TOP RULE
  \hbox{\vrule width#1pt\kern #2pt             % RULE/SPACE ON LEFT
  \vbox{\kern #2pt\hbox{#3}\kern #2pt}         % TOP,MATERIAL,BOT
  \kern #2pt\vrule width#1pt}                  % RULE/SPACE ON RIGHT
  \hrule height0pt depth#1pt}                  % BOTTOM RULE
}
%%
%% draws a box with shadow around text
%%
\def\shframe #1#2#3#4{\vbox{\hrule height 0pt % NO TOP SHADOW
 \hbox{\vrule width #1pt\kern 0pt             % LEFT SHADOW
 \vbox{\kern-#1pt\frame{.3}{#2}{#3}{#4}       % START SHADOW
 \kern-.3pt}                                  % MOVE UP RULE
 \kern-#2pt\vrule width 0pt}                  % STOP SHADOW
 \hrule height #1pt}                          % BOTTOM SHADOW
}
%%
%%
%% end: draw box
%%
%%  To install as a package on a local host.
%%   a. Append the header ``\usepackage{myboxes}'' to the above macro. Name 
%%   the macreo file myboxes.sty.  Move myboxes.sty into $HOME/texmf/tex/mypackages/. 
%%   You might need to type texhash.
%%   b. T use the package write \usepackage{myboxes} in the preamble.

%
\begin{document}

%% notes info page
%\hfill{\laurnumber}
%\vskip0.3cm
\centerline{{ \Large\bf \projname: \fname}}
\vskip0.25cm 
\centerline{\bf \mytitle}
\vskip0.25cm
\centerline{\myauthors}
\vskip0.75cm 
\baselineskip 14pt plus 1pt minus 1pt
\begin{flushright}
Research Notes   \\[3pt]
{\it Project}:          \\
\projname                      \\
  {\it Path of TeX Source}:          \\
\dirname/\fname                      \\[3pt]
{\it Last Modified By}:            \\
\whochange                         \\
\datechange                        \\[3pt]
{\it Date Started:}                \\
\datestart                         \\[3pt]
{\it Date:}                \\
\draftverson~ \today ~\currenttime \\
\end{flushright}

\baselineskip 20pt plus 1pt minus 1pt

%% mini abstract
%\abstractskip
%\noindent
%These are notes on Logic from Ref.~\cite{ref_chang}.  
%\bodyskip

%% title page
\vskip2.0cm
%\pagebreak
\preprint{\laurnumber}

% publication title page
\title{\titleskip
  \mytitle
}

\author{Robert L Singleton Jr}

\affiliation{\affiliationskip
   School of Mathematics\\
   University of Leeds\\
   LS2 9JT
}

%\vskip 0.2cm 
%\affiliation{\affiliationskip
%     %$^1$
%     Los Alamos National Laboratory\\
%     Los Alamos, New Mexico 87545, USA
%}

\date{\datechange}

\begin{abstract}
\abstractskip
\vskip0.3cm 
\noindent
  Physics documentation for the BPS temperature equilibration in the code Clog.
\end{abstract}

%%
\maketitle
%%

% to change page settings
%\thispagestyle{empty}
%\pagestyle{empty}
%\setcounter{page}{0}

\pagebreak
\tableofcontentsskip
\tableofcontents
%\thispagestyle{empty}

%\pagebreak
\newpage
\bodyskip
%\setcounter{page}{1}

\pagebreak
\clearpage

\newpage
%%
\section{General Analytic Expressions for the BPS Rate Coefficients}
\label{sec:rate}

Suppose we have a plasma with various species labeled by an index $b$
at distinct temperatures $T_b$, number densities $n_b$, and species
masses $m_b$.\footnote{By convention, $b=1$ will be the electron
component.}  Temperature will be measured in energy units, and we
denote the inverse temperature by $\beta_b =1/T_b$. Electrostatic
units will be rationalized cgs.  Species $a$ and $b$ will exchange
energy through coulomb interactions, and the rate of change in the
energy density between species $a$ and $b$ at temperatures $T_a$ and
$T_b$ is given by the usual rate equation
%%
\begin{eqnarray}
  \frac{d{\cal E}_{ab}}{dt}
  =
  -\, {\cal C}_{ab} \,\Big(T_a - T_b \Big)  \ .
\label{eq:Cab}
\end{eqnarray}
%% 
The rate coefficients ${\cal C}_{ab}$ as calculated by BPS take the
form 
%%
\begin{eqnarray}
  {\cal C}_{ab}
  =
  \Big({\cal C}^\smC_{ab,\smR} 
  +
  {\cal C}^\smC_{ab,\smS} 
  \Big)
  +
  {\cal C}^\smQM_{ab} \ ,
\end{eqnarray}
%%
where the first two term are purely classical and are given by a
long- and a short-distance contribution, 
%%
\begin{eqnarray}
  {\cal C}^\smC_{ab,\smR} 
  &\!=\!&
  \frac{\kappa_a^2\, \kappa_b^2}{2\pi}
  \left(\frac{\beta_a m_a}{2\pi} \right)^{\!\! 1/2} \!\!
  \left(\frac{\beta_b m_b}{2\pi} \right)^{\!\! 1/2} \!\!\!
  \int_{-\infty}^{\infty} \!\!\!\! dv \, v^2 
  e^{- \frac{1}{2}( \beta_a m_a + \beta_b m_b) v^2 }  
  \frac{i}{2 \pi} \,\frac{F(v)}{\rho_\text{tot}(v)}
  \ln\! \left\{ \frac{F(v)}{\kappa_e^2}\right\} 
\label{CabA}
\\[10pt]
  {\cal C}^\smC_{ab,\smS} 
  &=& 
  \!-{\kappa_a^2\, \kappa_b^2 } \,
  \frac{ (\beta_a m_a \beta_b m_b)^{1/2}}{\left( \beta_a m_a + 
  \beta_b m_b \right)^{3/2} } \,\left( \frac{1}{2\pi} \right)^{\!\!3/2}\, 
  \left[\,\ln\!\left\{ \frac{e_a\,e_b}{4 \pi}\,
  \frac{\kappa_e}{4 \, m_{ab} \, V^2_{ab}}\right\} 
  + 2 \gamma\,  \right]  
\label{CabB}
\end{eqnarray}
%%
respectively, and the third term is the short-distance quantum
scattering piece (this term vanishes as $\hbar \to 0$)
%%
\begin{eqnarray}
  {\cal C}^\smQM_{ab} 
  &\!=\!&  
  \!-\frac{1}{2} \, \kappa_a^2\, \kappa_b^2 \, 
  \frac{(\beta_a m_a \, \beta_b m_b)^{1/2}}{(\beta_a m_a \!+\! 
  \beta_b m_b)^{3/2}} 
  \left(\frac{1}{2\pi}\right)^{\!\!3/2} \!\!\!\!
  \int_0^\infty \!\!\! d \zeta\, e^{-\zeta/2} 
  \left[\,{\rm Re}\,
  \psi\!\left(\!1 + i\frac{\bar\eta_{ab}}{\zeta^{1/2}}\right) \!-\! 
  \ln\!\left\{ \frac{\bar\eta_{ab}}{\zeta^{1/2}}\right\}
  \right] \ .
\nonumber \\
\label{CabC}
\end{eqnarray}
%%
These expressions are accurate to leading and next-to-leading order in
the plasma coupling $g=e^2 \kappa/4\pi T$, and are therefore
essentially exact in a weakly coupled plasma, {\em i.e.} a plasma for
which $g \ll 1$.  For convenience we have set the arbitrary wave
number $K$ that appears in the regular and singular terms to the value
$K=\kappa_e$.\footnote{
  %%
  \footnoteskip Recall that the sum ${\cal C}_{ab}^\smCL={\cal C}_{ab,
  \smR}^\smCL + {\cal C}_{ab,\smS}^\smCL$ is independent $K$. This is
  because the BPS calculation introduced an arbitrary wave number $K$
  which hold no physical significance (it is akin to a renormalization
  scale in quantum field theory). 
  %%
}
The reduced mass of species $a$ and $b$ is
%%
\begin{eqnarray}
  \frac{1}{m_{ab}} = \frac{1}{m_a} + \frac{1}{m_b} \ ,
\end{eqnarray}
%%
while the thermal velocity and the quantum parameter are determined by
%%
\begin{eqnarray}
  V_{ab}^2 &=& \frac{1}{\beta_a m_a} + \frac{1}{\beta_b m_b}
\label{defVab}
\\[5pt]
 \bar\eta_{ab}&=& \frac{e_a e_b}{4\pi\, \hbar V_{ab}} \ .
\label{defbareta}
\end{eqnarray}
%%
The function $F(v)$ takes the form
%%
\begin{eqnarray}
  F(v) 
  &=& 
  -\int_{-\infty}^\infty \! du \, 
  \frac{\rho_\text{tot}(u)}{v - u + i\eta} 
  ~~~\text{with}~~
  \rho_\text{tot}(u)=\sum_b\rho_b(u)
\label{Fdef}
\\[5pt]
  \rho_b(v) 
  &=& 
  \kappa_b^2\,\sqrt{\frac{\beta_b m_b}{2\pi}}\, v\,
  \exp\!\left\{-\frac{1}{2}\,\beta_b m_b\, v^2\right\} \ ,
\label{rhototdef}
\end{eqnarray}
%%
and its relation to the dielectric function is $k^2 \, \epsilon(
{\bf k} , {\bf k}\cdot {\bf v} ) = k^2 + F(\hat{\bf k} \cdot {\bf
v})$. The first term ${\cal C}_{ab ,\smR}^\smC$ arises from long-distance
collective effects from the dielectric function, and it involves {\em
  all} plasma species (even species $c$ different from $a$ and $b$).
This is the term I call non-separable, meaning that it cannot be
written as a sum of individual plasma components involving only a
single species. The second term ${\cal C}_{ab ,\smS}^\smC$ arises from
short-distance two-body classical scattering, and the third term
${\cal C}_{ab}^\smQM$ is the two-body quantum scattering correction to
all orders in the quantum parameters $\bar\eta_{ab}$. Three body and
higher effects are contained in our systematic error term, the
next-to-next-to-leading order term proportional to $g^3$. In a 
strongly coupled plasma these higher order effects dominate, but
in a weakly coupled plasma they are negligible.

A dramatic simplification occurs under the following conditions: (i)
the extreme quantum limit is realized, {\em i.e.} $\bar\eta_{ab} \ll
1$, (ii) there is a large mass hierarchy so that $m_e/m_\smI \ll 1$ (in
which case the ions have the same temperature $T_\smI$), and (iii) sum
over the ions to construct the effective rate coefficient. The rate
equation (\ref{eq:Cab}) becomes
%%
\begin{eqnarray}
  \frac{d{\cal E}_{e\smI}}{dt}
  =
  -\, {\cal C}_{e\smI} \,\Big(T_e - T_\smI \Big)  \ ,
\label{eq:rate_eI}
\end{eqnarray}
%% 
\noindent
where $d{\cal E}_{e\smI}/dt={\sum}_i d{\cal E}_{ei}/dt$ and ${\cal
C}_{e\smI}={\sum}_i {\cal C}_{ei}$.  Because of the aforementioned
sum-rule and the extreme quantum limit, the result simplifies to
%%
\begin{eqnarray}
  {\cal C}_{e\smI}^\smBPS
  = 
  \frac{\kappa_e^2\,\omega_\smI^2}{2\pi}\, 
  \sqrt{\frac{m_e}{2\pi\, T_e}}\, \ln\Lambda_\smBPS \ ,
  ~~~\text{with}~~~
  \ln\Lambda_\smBPS
  =
  \frac{1}{2}\left[\ln\!\left\{\frac{8 T_e^2}{\hbar^2 \omega_e^2}
  \right\} - \gamma - 1 \right] \ ,
\label{bpsrate}
\end{eqnarray}
%%
\noindent
where $\omega_\smI^2={\sum}_i\omega_i^2$.  As opposed to the model of
Lee-More, there is no ion temperature dependence inside the
logarithm. The lack of ion dependence has also been observed by
Diamante and Daligault in their classical MD simulations. 

\pagebreak
\section{The Born Approximation}

We will code the rate in the Born approximation first:
%%
\begin{eqnarray}
  {\cal C}_{e\smI}^\smBPS
  = 
  \underbrace{
  \frac{\kappa_e^2\,\omega_\smI^2}{2\pi}\, 
  \sqrt{\frac{m_e}{2\pi\, T_e}}
  }_{\rm cn} \,\cdot\,  \ln\Lambda_\smBPS \ ,
  ~~~\text{with}~~~
  \ln\Lambda_\smBPS
  =
  \frac{1}{2}\left[\ln\!\left\{\frac{8 T_e^2}{\hbar^2 \omega_e^2}
  \right\} - \gamma - 1 \right] \ .
\label{bpsrate_again}
\end{eqnarray}
%%
\vskip0.1cm 
\noindent
rate.f90:bps\_rate\_cei\_born
{
\baselineskip 10pt
\begin{verbatim}
      SUBROUTINE bps_rate_cei_born(nni, betab, zb, mb, nb, ln_bps_born, cei_born)
      USE bpsvars
      USE mathvars
      USE physvars
      IMPLICIT NONE
      INTEGER,                     INTENT(IN)  :: nni     !  Number of ions
      REAL,    DIMENSION(1:nni+1), INTENT(IN)  :: betab   !  Temperature array    [1/keV]
      REAL,    DIMENSION(1:nni+1), INTENT(IN)  :: zb      !  Charge array 
      REAL,    DIMENSION(1:nni+1), INTENT(IN)  :: mb      !  Mass array [keV]
      REAL,    DIMENSION(1:nni+1), INTENT(IN)  :: nb      !  Number density array [cm^-3]

      REAL                       , INTENT(OUT) :: ln_bps_born! BPS Coulomb log 
      REAL                       , INTENT(OUT) :: cei_born   ! equilibration rate

      REAL,    PARAMETER :: UPM=0.7745966692E0
      REAL,    PARAMETER :: W13=0.5555555556E0, W2=0.8888888889E0

      REAL    :: mi, ni, zi, omi2, ome2, cn
      REAL    :: me, ne, betae, te2, ke2
      INTEGER :: ib
!
! construct plasma quantities: k_e^2, omega_I^2
!
      ne=nb(1)
      me=mb(1)
      betae=betab(1)
      te2 =1/betae**2
      ke2 =8*PI*A0CM*BEKEV*ne*betae
      ome2=8*PI*A0CM*BEKEV*ne*CC2/me
      omi2=0
      DO ib=2,nni+1
         ni=nb(ib)
         mi=mb(ib)
         zi=zb(ib)
         omi2=omi2 + 8*PI*A0CM*BEKEV*ni*zi*zi*CC2/mi
      ENDDO
!
!                           om_I^2*k_e^2    [ beta_e m_e*c^2 ]^1/2    1
! construct prefactor: cn = ------------  * |--------------- |     * ---
!                               2 pi        [   2 pi         ]        c
!
      cn=omi2*ke2/TWOPI
      cn=cn*SQRT(betae*me/TWOPI)/CC
!
!                                          1   [   [    8 T_e^2    ]             ]
! construct BPS Coulomb log: ln_bps_born = --- [ ln[ ------------- ] - GAMMA -1  ]
!                                          2   [   [  hbar^2 ome^2 ]             ]

      ln_bps_born=0.5*(LOG(8*te2*CC2/(HBARC2*ome2)) -GAMMA - 1)
!
! construct rate: cei_born
!
      cei_born=cn*ln_bps_born

      END SUBROUTINE bps_rate_cei_born
\end{verbatim}
%%

\pagebreak
\section{General Case: the Main Driver}

I will return the general rate coefficients in three forms:
%%
\begin{enumerate}
  \baselineskip 10pt plus 1pt minus 1pt
  \setlength{\itemsep}{3pt} % single spacing
  \setlength{\parskip}{1pt} %
  \setlength{\parsep}{0pt}  %

\item[i.] \verb+bps_rate_cab_mass+: For a given pair of indices $a$
  and $b$, this subroutines returns the coefficient ${\cal C}_{ab}$.
  The quantum parameter $\eta$ and the electron mass $m_e$ can be
  arbitrary. This routine is used to construct the entries in the next
  two subroutines.

\item[ii.] \verb+bps_rate_cab_matrix+: Returns the complete matrix of
coefficients ${\cal C}_{ab}(E)$.

\item[iii.] \verb+bps_rate_cei_mass+: This routine returns the sum
over the ions ${\cal C}_{e\smI}={\sum}_i {\cal C}_{ei}$. It also
returns the coulomb logarithm.
\end{enumerate}

\subsection{The Driver Routine: bps\_rate\_cab\_mass}

%%
\vskip0.2cm 
\noindent
rate.f90:bps\_rate\_cab\_mass
{
\baselineskip 10pt
\begin{verbatim}
      SUBROUTINE bps_rate_cab_mass(nni, ia, ib, betab, zb, mb, nb, &
        c_ab, c_ab_sing, c_ab_reg, c_ab_qm)
      USE mathvars
      USE physvars
        IMPLICIT NONE
        INTEGER,                             INTENT(IN)  :: nni     !  Number of ions
        INTEGER,                             INTENT(IN)  :: ia      !  Species number
        INTEGER,                             INTENT(IN)  :: ib      !  Species number
        REAL,    DIMENSION(1:nni+1),         INTENT(IN)  :: betab   !  Temperature array    [1/keV]
        REAL,    DIMENSION(1:nni+1),         INTENT(IN)  :: zb      !  Charge array 
        REAL,    DIMENSION(1:nni+1),         INTENT(IN)  :: mb      !  Mass array [keV]
        REAL,    DIMENSION(1:nni+1),         INTENT(IN)  :: nb      !  Number density array [cm^-3]
        REAL,                                INTENT(OUT) :: c_ab
        REAL,                                INTENT(OUT) :: c_ab_sing
        REAL,                                INTENT(OUT) :: c_ab_reg
        REAL,                                INTENT(OUT) :: c_ab_qm
        REAL,    DIMENSION(1:nni+1)  :: kb2, ob2
        REAL    ::  c_s, c_r, c_q
        REAL    :: kia2, kib2, bmia, bmib, nab, omi2, nab_reg
        INTEGER :: ibmax

        ibmax=nni+1
!
! initialize components of rate-coefficients
!
!
! construct plasma quantities: kb^2
!
        ob2=8*PI*A0CM*BEKEV*zb*zb*nb*CC2/mb
        kb2=8*PI*A0CM*BEKEV*zb*zb*nb*betab
        omi2=SUM(ob2(2:ibmax))

        kia2  =kb2(ia)
        bmia  =betab(ia)*mb(ia)
        kib2=kb2(ib)
        bmib=betab(ib)*mb(ib)
        nab=kia2*kib2*CC*SQRT(bmia*bmib)/(bmia + bmib)**1.5   !
        nab=nab/TWOPI**1.5                                    ! normalization
!
! C_{ab}-classical-singular 
!
        CALL cab_sing_mass(nni,ia,ib,betab,zb,mb,nb,c_s)
        c_ab_sing=nab*c_s 
!
! C_{ab}-classical-regular 
!
        CALL cab_reg_mass(nni,ia,ib,betab,zb,mb,nb,c_r)
        nab_reg=kb2(ia)*omi2/TWOPI
        nab_reg=nab_reg*SQRT(betab(ia)*mb(ia)/TWOPI)/CC
        c_ab_reg=nab_reg*c_r  
!
! C_{ab}-quantum
!
        CALL cab_qm_mass(nni,ia,ib,betab,zb,mb,c_q)
        c_ab_qm=nab*c_q
!
! C_{ab}-total
!
        c_ab=c_ab_sing + c_ab_reg + c_ab_qm
      END SUBROUTINE bps_rate_cab_mass
\end{verbatim}
%%

\vskip0.2cm 
\noindent
rate.f90:bps\_rate\_cab\_matrix
{
\baselineskip 10pt
\begin{verbatim}
      SUBROUTINE bps_rate_cab_matrix(nni, betab, zb, mb, nb, &
            c_ab, c_ab_sing, c_ab_reg, c_ab_qm)
      USE physvars
      USE mathvars    
        IMPLICIT NONE                                             ! Plasma:
        INTEGER,                            INTENT(IN)  :: nni    !  number of ions
        REAL,    DIMENSION(1:nni+1),        INTENT(IN)  :: betab  !  temp array [1/keV]
        REAL,    DIMENSION(1:nni+1),        INTENT(IN)  :: zb     !  charge array
        REAL,    DIMENSION(1:nni+1),        INTENT(IN)  :: mb     !  mass array [keV]
        REAL,    DIMENSION(1:nni+1),        INTENT(IN)  :: nb     !  density [1/cc]
                                                                  !
                                                                  ! A-coeffs [MeV/micron]
        REAL,    DIMENSION(1:nni+1,1:nni+1),INTENT(OUT) :: c_ab
        REAL,    DIMENSION(1:nni+1,1:nni+1),INTENT(OUT) :: c_ab_sing
        REAL,    DIMENSION(1:nni+1,1:nni+1),INTENT(OUT) :: c_ab_reg
        REAL,    DIMENSION(1:nni+1,1:nni+1),INTENT(OUT) :: c_ab_qm
        REAL    :: cab, cab_sing, cab_reg, cab_qm
        INTEGER :: ia, ib
        DO ia=1,nni+1
          DO ib=1,nni+1
          CALL bps_rate_cab_mass(nni, ia, ib, betab, zb, mb, nb, & 
            cab, cab_sing, cab_reg,cab_qm)  
            c_ab(ia,ib)=cab
            c_ab_sing(ia,ib)=cab_sing
            c_ab_reg(ia,ib) =cab_reg
            c_ab_qm(ia,ib)  =cab_qm   
          ENDDO
        ENDDO
      END SUBROUTINE bps_rate_cab_matrix
\end{verbatim}
%%


\vskip0.2cm 
\noindent
rate.f90:bps\_rate\_cei\_mass
{
\baselineskip 10pt
\begin{verbatim}
      SUBROUTINE bps_rate_cei_mass(nni, betab, zb, mb, nb, ln_bps,      & 
      delta, cei_tot, cei_i, cei_e, ceic_tot, ceic_i, ceic_e, ceiq_tot, &
      ceiq_i, ceiq_e, ceic_s_i, ceic_s_e, ceic_r_i , ceic_r_e, ceib)
      USE physvars
      USE mathvars    
        IMPLICIT NONE  
        INTEGER,                     INTENT(IN)  :: nni     !  Number of ions
        REAL,    DIMENSION(1:nni+1), INTENT(IN)  :: betab   !  Temperature array    [1/keV]
        REAL,    DIMENSION(1:nni+1), INTENT(IN)  :: zb      !  Charge array 
        REAL,    DIMENSION(1:nni+1), INTENT(IN)  :: mb      !  Mass array [keV]
        REAL,    DIMENSION(1:nni+1), INTENT(IN)  :: nb      !  Number density array [cm^-3]
        REAL,                        INTENT(OUT) :: ln_bps  !  Coulomb logarithm
        REAL,                        INTENT(OUT) :: delta   !  C_reg =-1/2+delta
        REAL,                        INTENT(OUT) :: cei_tot ! [cm^-3 s^-1]
        REAL,                        INTENT(OUT) :: cei_i   !
        REAL,                        INTENT(OUT) :: cei_e   !
        REAL,                        INTENT(OUT) :: ceic_tot!
        REAL,                        INTENT(OUT) :: ceic_i  !
        REAL,                        INTENT(OUT) :: ceic_e  !
        REAL,                        INTENT(OUT) :: ceiq_tot!
        REAL,                        INTENT(OUT) :: ceiq_i  !
        REAL,                        INTENT(OUT) :: ceiq_e  !
        REAL,                        INTENT(OUT) :: ceic_s_i!
        REAL,                        INTENT(OUT) :: ceic_s_e!
        REAL,                        INTENT(OUT) :: ceic_r_i!
        REAL,                        INTENT(OUT) :: ceic_r_e!
        REAL,    DIMENSION(1:nni+1), INTENT(OUT) :: ceib    !
        REAL,    DIMENSION(1:nni+1)  :: ob2
        REAL     :: omi2, cn, kb2e
        REAL     :: cab, c_ab_sing, c_ab_reg, c_ab_qm
        INTEGER  :: ia, ib, nnb
!
! initialize components of A-coefficients
!
        delta=0
        cei_tot =0  ! electron + ion
        cei_i   =0  ! ion contribution
        cei_e   =0  ! electron contribution
        ceic_tot=0  ! classical total
        ceic_e  =0  ! classical electron
        ceic_i  =0  ! classical ion
        ceiq_tot=0  ! quantum total
        ceiq_e  =0  ! quantum electron
        ceiq_i  =0  ! quantum ion
        ceic_s_i=0 
        ceic_s_e=0 
        ceic_r_i=0
        ceic_r_e=0
        ceib=0

        kb2e=8*PI*A0CM*BEKEV*zb(1)*zb(1)*nb(1)*betab(1)
        ob2=8*PI*A0CM*BEKEV*zb*zb*nb*CC2/mb
        omi2=SUM(ob2(2:nni+1))
        cn=kb2e*omi2/TWOPI
        cn=cn*SQRT(betab(1)*mb(1)/TWOPI)/CC

        ia=1
        NNB = nni+1    ! number of ions + electrons
        DO ib=1,nni+1  ! loop over electrons to calculate cei_e
           CALL bps_rate_cab_mass(nni, ia, ib, betab, zb, mb, nb, &
           cab, c_ab_sing, c_ab_reg, c_ab_qm)
           IF (ib .NE. 1) delta=delta+c_ab_reg/cn
           ceib(ib)=c_ab_sing + c_ab_reg + c_ab_qm
           CALL x_collect(ib, NNB, c_ab_sing, c_ab_reg, c_ab_qm,      &
           cei_tot, cei_i, cei_e, ceic_tot, ceic_i, ceic_e, ceiq_tot, &
           ceiq_i, ceiq_e, ceic_s_i, ceic_s_e, ceic_r_i, ceic_r_e)
        ENDDO
        delta=0.5+delta
        ln_bps=cei_i/cn
      END SUBROUTINE bps_rate_cei_mass
\end{verbatim}
%%

The quantity $\Delta$ in the last subroutine is now described. To
obtain the leading order term in the electron mass, we can employ the
sum rule
%%
\begin{eqnarray}
  \frac{i}{2 \pi}\int_{-\infty}^{\infty}  dv \, v\, 
  \Big[\kappa_e^2 + F_\smI(v)\Big]
  \ln\! \left\{ 1 + \frac{F_\smI(v)}{\kappa_e^2}\right\} 
  =
  -\frac{1}{2}\,{\sum}_i \omega_i^2 \ ,
\end{eqnarray}
%%
which gives (the superscript denotes leading order in $m_e$)
%%
\begin{eqnarray}
  {\cal C}^0_{e\smI,\smR} 
  &=&
  - \frac{\kappa_e^2}{2\pi} \,
  \left( \frac{\beta_e m_e}{2\pi} \right)^{1/2} \, 
  \frac{\omega_\smI^2}{2} 
  ~~~~\text{with}~~ \omega_\smI^2={\sum}_i \omega_i^2 \ .
\label{CeIsmallme}
\end{eqnarray}
%%
We will express the rate coefficient (\ref{CeIF}) in terms
of a correction $\Delta$ that vanishes in the $m_e \to 0$
limit: 
%%
\begin{eqnarray}
  {\cal C}_{e\smI,\smR} 
  &=&
  \frac{\kappa_e^2}{2\pi} \,
  \left( \frac{\beta_e m_e}{2\pi} \right)^{1/2} \, 
  \omega_\smI^2\,\Big[ -\frac{1}{2} + \Delta \Big] \ ,
\label{CeIDelta}
\end{eqnarray}
%%
with
%%
\begin{eqnarray}
  \Delta
  &\equiv&
  \frac{1}{2} + \frac{1}{\omega_\smI^2}\,
  \frac{i}{2 \pi} \int_{-\infty}^{\infty} \!\! dv \, v\, 
  e^{- \frac{1}{2}\,\beta_e m_e  v^2 }  
  \frac{\rho_\smI(v)}{\rho_\text{tot}(v)}\,
  F(v) \ln\! \left\{ \frac{F(v)}{\kappa_e^2}\right\} 
\label{defDeltaA}
\\[10pt]
  &=&
  \frac{1}{\omega_\smI^2}\,
  \frac{i}{2 \pi} \int_{-\infty}^{\infty} \!\! dv \, v\, 
  \Bigg[
  e^{- \frac{1}{2}\,\beta_e m_e  v^2 }  
  \frac{\rho_\smI(v)}{\rho_\text{tot}(v)}\,
  F(v) \ln\! \left\{ \frac{F(v)}{\kappa_e^2}\right\} 
  -
  \Big[\kappa_e^2 + F_\smI(v)\Big]
  \ln\! \left\{ 1 + \frac{F_\smI(v)}{\kappa_e^2}\right\} 
  \Bigg] \ .
\label{defDeltaB}
\nonumber
\\
\end{eqnarray}
%%
The form (\ref{defDeltaB}) can serve as a starting point for
an analytic calculation (see notes: this doesn't seem to lead to
particularly simple results). For numerical work it is more
useful to compute the integral in (\ref{defDeltaA}) directly
(using some quadrature method), and if desired, we may construct
$\Delta$ from (\ref{CeIDelta}). 

\subsection{The Singular Contribution: cab\_sing\_mass}

From Eq.~(\ref{CabB}) of  Section~\ref{sec:rate} the short distance
contribution to the total electron-ion rate from the singular piece
%%
\begin{eqnarray}
  {\cal C}^\smC_{ab,\smS} 
  &=& 
  -{\kappa_a^2\, \kappa_b^2 } \,
  \frac{ (\beta_a m_a \beta_b m_b)^{1/2}}{\left( \beta_a m_a + 
  \beta_b m_b \right)^{3/2} } \,
  \left( \frac{1}{2\pi} \right)^{\!\!3/2}\, 
  \left[\,\ln\!\left\{ \frac{e_a\,e_b}{4 \pi}\,
  \frac{\kappa_e}{4 \, m_{ab} \, V^2_{ab}}\right\} 
  + 2 \gamma\,  \right]  ,
\label{CeISA}
\end{eqnarray}
%%
where
%%
\begin{eqnarray}
  \frac{1}{m_{ab}} &=& \frac{1}{m_a} + \frac{1}{m_b} 
\\[5pt]
  V_{ab}^2 &=& \frac{1}{\beta_a m_a} + \frac{1}{\beta_b m_b} \ .
\end{eqnarray}
%%
It will prove more convenient to use the form
%%
\begin{eqnarray}
  {\cal C}^\smC_{e\smI,\smS} 
  &=& 
  \underbrace{
  {\kappa_a^2\, \kappa_b^2 } \,
  \frac{ (\beta_a m_a \beta_b m_b)^{1/2}}{\left( \beta_a m_a + 
  \beta_b m_b \right)^{3/2} }\,   \left( \frac{1}{2\pi}
  \right)^{\!\!3/2}}_\text{nab} \cdot \, 
  \sf{C}_{ab, \smS}
\\[5pt]
  \sf{C}_{ab, \smS}
  &=&
  -\,\ln\!\left\{ \frac{g_e\,Z_a Z_b}{4 \beta_e\, m_{ab} V_{ab}^2}\right\} 
  - 2 \gamma \ .
\label{CeISB}
\end{eqnarray}
%%
The subroutine below gives $\sf{C}_{ab}$ and the coefficient resides
in the driver or calling routine. 

\vskip0.4cm 
\noindent
rate.f90:cab\_sing\_mass
{
\baselineskip 10pt
\begin{verbatim}
      SUBROUTINE cab_sing_mass(nni, ia, ib, betab, zb, mb, nb, cab_sing)
      USE mathvars
      USE physvars
        IMPLICIT NONE
        INTEGER,                     INTENT(IN)  :: nni  !Number of ion species
        INTEGER,                     INTENT(IN)  :: ia
        INTEGER,                     INTENT(IN)  :: ib
        REAL,    DIMENSION(1:nni+1), INTENT(IN)  :: zb   !Charge array
        REAL,    DIMENSION(1:nni+1), INTENT(IN)  :: betab!Temperature array [1/keV]
        REAL,    DIMENSION(1:nni+1), INTENT(IN)  :: mb   !Mass array [keV]
        REAL,    DIMENSION(1:nni+1), INTENT(IN)  :: nb   !density array [1/cm^3]
        REAL,                        INTENT(OUT) :: cab_sing

        REAL    :: betae, ne, ge, mabc2, zia, zib, vab2
!
! note: om_b=(1.32155E+3)*SQRT(zb*zb*nb*AMUKEV/mb) ! Plasma frequency [1/s]
!       ge=(6.1260E-15)*SQRT(ne)/te**1.5

        betae=betab(1)
        ne=nb(1)
        ge=GECOEFF*SQRT(ne)*betae**1.5
        zia=zb(ia)
        zib=zb(ib)
        mabc2=mb(ia)*mb(ib)/(mb(ia) + mb(ib)) ! [keV]
        vab2=1./(betab(ia)*mb(ia)) + 1./(betab(ib)*mb(ib)) ! [dimensionless]
        cab_sing=-LOG(0.25*ge*ABS(zia*zib)/(mabc2*betae*vab2)) - 2*GAMMA 
      END SUBROUTINE cab_sing_mass
\end{verbatim}
%%


\subsection{The Regular Contribution: cab\_reg\_mass}

From Section~\ref{sec:rate}, the long distance regular piece of the
electron-ion equilibration rate is given by 
%%
\begin{eqnarray}
  {\cal C}^\smC_{ab,\smR} 
  &=&
  \phantom{-}
  \frac{\kappa_a^2}{2\pi}
  \left(\frac{\beta_a m_a}{2\pi} \right)^{ 1/2} 
  \frac{i}{2 \pi}\int_{-\infty}^{\infty}  dv \, v\, 
  e^{- \frac{1}{2}\,\beta_a m_a  v^2 }  
  \frac{\rho_b(v)}{\rho_\text{tot}(v)} \,  F(v)
  \ln\! \left\{ \frac{F(v)}{\kappa_e^2}\right\} 
\label{CeIF}
\\[10pt]
  &=&
  -\frac{\kappa_a^2}{2\pi}
  \left(\frac{\beta_a m_a}{2\pi} \right)^{ 1/2} 
  \frac{1}{2\pi}\int_0^{\infty}  dv \, v\, 
  e^{- \frac{1}{2}\,\beta_a m_a  v^2 }  
  \frac{\rho_b(v)}{\rho_\text{tot}(v)}\, H(v) \ ,
\label{CeIH}
\end{eqnarray}
%%
where we have defined
%%
\begin{eqnarray}
  H(v) 
  &\equiv&
  -i\left[
  F(v) \ln\!\left\{ \frac{F(v)}{\kappa_e^2}\right\} -
  F^*(v)\ln\!\left\{\frac{F^*(v)}{\kappa_e^2} \right\}\right]
  =
  2\left[
  F_\smRe\, {\rm arg}\{F\}
  +
  F_{\smIm} \ln\!\left\{\frac{\vert F \vert}{\kappa_e^2}\right\}
   \right] \ .
\nonumber\\
\label{Hb}
\end{eqnarray}
%%
Equation~(\ref{CeIH}) follows from expression (\ref{CeIF}) upon using
the relation $F(-v)=F^*(v)$ in the negative $v$-regime.  
To calculate $\Delta$ it is convenient to express the regular piece
in the form
%%
\begin{eqnarray}
  {\cal C}^\smC_{ab,\smR} 
  &=&
  \underbrace{
  \frac{\kappa_a^2}{2\pi}
  \left(\frac{\beta_a m_a}{2\pi} \right)^{ 1/2} 
  \omega_\smI^2 }_\text{nab\_reg}\, \cdot \,  \sf{C}_{ab,\smR}
\\[5pt]
  \sf{C}_{ab,\smR}
  &=&
  -\frac{1}{2\pi\,\omega_\smI^2}\int_0^{\infty}  dv \, v\, 
  e^{- \frac{1}{2}\,\beta_a m_a  v^2 }  
  \frac{\rho_b(v)}{\rho_\text{tot}(v)}\, H(v) \ .
\label{CeIHnew}
\end{eqnarray}
%%
We will rescale the velocity integration to form the dimensionless variable
%%
\begin{eqnarray}
  x = \frac{v \,\mu}{\sqrt{2\,}} ~~~~~ \mu ={\sum}_i \beta_i m_i 
%= {\sum}_i \beta_i (m_i c^2)/c^2 \ ,
\end{eqnarray}
%%
and defining $\mathbb{H}(x)=H(v)/\kappa_e^2$ gives
%%
\begin{eqnarray}
  \sf{C}_{ab,\smR}
  &=&
  -\frac{\kappa_e^2}{\pi\,\mu^2\,\omega_\smI^2}\int_0^{\infty}  dx \, x\, 
  e^{-\beta_a m_a  x^2/\mu^2 }  
  \frac{\kappa_b^2 (\beta_b m_b)^{1/2}e^{-\beta_b m_b x^2/\mu^2}}
  {\sum_c\kappa_c^2 (\beta_c m_c)^{1/2}e^{-\beta_c m_c x^2/\mu^2}}\,
  \mathbb{H}(x)
\\[5pt]
  &=&
  -\frac{\kappa_e^2}{\pi\,\mu^2\,\omega_\smI^2}\int_0^{\infty}  dx \,
  \left[
  \sum_c \frac{\kappa_c^2\, (\beta_c m_c)^{1/2}}{\kappa_b^2\, (\beta_b
  m_b)^{1/2}} \, 
  e^{(\beta_a m_a + \beta_b m_b-\beta_c m_c )x^2/\mu^2}
  \right]^{-1}
  \mathbb{H}(x)\, x \ .
\end{eqnarray}
%%
The quantity passed to the subroutine \verb+frfi+ is
%%
\begin{eqnarray}
  a_b = \left( \frac{\beta_b m_b}{2}\right)^{1/2} \! v
      = \left(\beta_b m_b\right)^{1/2} \frac{x}{\mu}
\end{eqnarray}
%%



\vskip0.4cm 
\noindent
rate.f90:
{
\baselineskip 10pt
\begin{verbatim}
      FUNCTION dcab_reg(x, nni, ia, ib, betab, zb, mb, nb)
      USE physvars
      USE mathvars
      IMPLICIT NONE
      REAL,                        INTENT(IN)  :: x
      INTEGER,                     INTENT(IN)  :: nni    !  Number of ion species
      INTEGER,                     INTENT(IN)  :: ia     !  Species type
      INTEGER,                     INTENT(IN)  :: ib     !  Species type
      REAL,    DIMENSION(1:nni+1), INTENT(IN)  :: zb     !  Charge array
      REAL,    DIMENSION(1:nni+1), INTENT(IN)  :: betab  !  Temperature array [1/keV]
      REAL,    DIMENSION(1:nni+1), INTENT(IN)  :: mb     !  Mass array [keV]
      REAL,    DIMENSION(1:nni+1), INTENT(IN)  :: nb     !  density array [1/cm^3]
      REAL                                     :: dcab_reg
      REAL,    DIMENSION(1:nni+1) :: ab, kbar2b
      REAL                        :: fr, fi, fabs, farg
      REAL                        :: mu, mu2, rx, hx, ne, betae, cn, kaic, kaia, abca2, rxic
      INTEGER                     :: ic
!
! construct parameters
!
      mu=SUM(SQRT(betab(2:nni+1)*mb(2:nni+1)))  ! inverse thermal velocity
      mu2=mu*mu                                 ! mu^2
      ab=SQRT(betab*mb)/mu                      !

      ne=nb(1)                                  !
      betae=betab(1)                            !
      kbar2b=zb*zb*betab*nb/(betae*ne)          ! kbar=k_b/k_e
!
! construct H(x)*x from  F_re, F_im, |F|, arg(F)
!
      CALL frfi(x,nni,kbar2b,ab,fr,fi,fabs,farg)! F(x) 
      hx=2*(fi*LOG(fabs) + fr*farg)*x           ! H(x)*x
!
! construct spectral weight ratio R_ab(x)

      rxic=0
      DO ic=1,nni+1
         kaic=kbar2b(ic)*ab(ic)
         IF (ic == ib) kaia=kaic*EXP(-ab(ib)*ab(ib)*x*x)
         abca2 =ab(ic)**2 - ab(ia)**2
         rxic = rxic + kaic*EXP(-abca2*x*x)
      ENDDO
      rx=kaia/rxic
!
! construct un-normalized rate integrand
!
      dcab_reg=rx*hx
!
! construct normalization coefficient
!
      cn=0
      DO ic=2,nni+1                                            !
         cn = cn + zb(ic)*zb(ic)*nb(ic)/(mb(ic)*ne*betae)      ! omega_I^2/kappa_e^2
      ENDDO                                                    ! 
      cn=-1./(PI*mu2*cn)                                       ! ke^2/(PI*mu^2 omI^2)
      dcab_reg=dcab_reg*cn 

      END FUNCTION dcab_reg 
      SUBROUTINE cab_reg_mass(nni, ia, ib, betab, zb, mb, nb, c_r)
        IMPLICIT NONE
        INTEGER,                     INTENT(IN)  :: nni    !  Number of ion species
        INTEGER,                     INTENT(IN)  :: ia     !
        INTEGER,                     INTENT(IN)  :: ib     !
        REAL,    DIMENSION(1:nni+1), INTENT(IN)  :: zb     !  Charge array
        REAL,    DIMENSION(1:nni+1), INTENT(IN)  :: betab  !  Temperature array [1/keV]
        REAL,    DIMENSION(1:nni+1), INTENT(IN)  :: mb     !  Mass array [keV]
        REAL,    DIMENSION(1:nni+1), INTENT(IN)  :: nb     !  density array [1/cm^3]
        REAL,                        INTENT(OUT) :: c_r

        REAL,    PARAMETER :: UPM=0.7745966692E0
        REAL,    PARAMETER :: W13=0.5555555556E0, W2=0.8888888889E0

        REAL    :: y, dcab_reg
        REAL    :: xmin, xmax, x, dx, xm, xc
        INTEGER :: nmax, ic
!
! integration cutoff determined by thermal velocity of ions
!
        xc=2. ! ti=1, te=0.01, 0.1, 1, 10, 100
        xmin=0.
        xmax=5*xc ! automate this choice later.
        nmax=1000
        dx=(xmax-xmin)/nmax 
        x=xmin-dx
        c_r=0
        DO ic=1,nmax,2
!     
           x=x+2.E0*dx
           y=dcab_reg(x,nni,ia,ib,betab,zb,mb,nb)
           c_r=c_r + W2*y
!
           xm=x-dx*UPM
           y=dcab_reg(xm,nni,ia,ib,betab,zb,mb,nb)
           c_r=c_r + W13*y
!
           xm=x+dx*UPM
           y=dcab_reg(xm,nni,ia,ib,betab,zb,mb,nb)
           c_r=c_r + W13*y
        ENDDO
        c_r=c_r*dx
      END SUBROUTINE cab_reg_mass
\end{verbatim}
%%


***===

\subsection{The Quantum Correction: cab\_qm\_mass}

From Eq.~(\ref{CabC}) of Section~\ref{sec:rate}, the quantum
correction is
%%
\begin{eqnarray}
  {\cal C}^\smQM_{ab} 
  &\!=\!&  
  \underbrace{
  \kappa_a^2\, \kappa_b^2 \, 
  \frac{(\beta_a m_a \, \beta_b m_b)^{1/2}}{(\beta_a m_a \!+\! 
  \beta_b m_b)^{3/2}} 
  \left(\frac{1}{2\pi}\right)^{\!\!3/2} 
  }_{\rm nab}  {\scriptstyle\times}
  \underbrace{
    \!-\frac{1}{2} \, 
  \int_0^\infty \!\!\! d \zeta\, e^{-\zeta/2} 
  \left[\,{\rm Re}\,
  \psi\!\left(\!1 + i\frac{\bar\eta_{ab}}{\zeta^{1/2}}\right) \!-\! 
  \ln\!\left\{ \frac{\bar\eta_{ab}}{\zeta^{1/2}}\right\}
  \right] }_{\rm c\_q}
  \ .
\nonumber \\
\label{CabC_two}
\end{eqnarray}
%%




\vskip0.1cm 
\noindent
rate.f90:
{
\baselineskip 10pt
\begin{verbatim}

      FUNCTION dcei_qm(x, eta)
      IMPLICIT NONE
      REAL  :: x       ! xi
      REAL  :: eta     ! quantum parameter
      REAL  :: dcei_qm ! quantum integrand
      REAL  :: xh, repsilog
      xh=eta/SQRT(x)
      dcei_qm=EXP(-x/2)*repsilog(xh) 
      END FUNCTION dcei_qm


      SUBROUTINE cab_qm_mass(nni, ia, ib, betab, zb, mb, qm)
      USE physvars
      USE mathvars
        IMPLICIT NONE
        INTEGER,                     INTENT(IN)  :: nni    !  Number of ion species
        INTEGER,                     INTENT(IN)  :: ia     !  
        INTEGER,                     INTENT(IN)  :: ib     !  
        REAL,    DIMENSION(1:nni+1), INTENT(IN)  :: zb     !  Charge array
        REAL,    DIMENSION(1:nni+1), INTENT(IN)  :: betab  !  Temperature array [1/keV]
        REAL,    DIMENSION(1:nni+1), INTENT(IN)  :: mb     !  Mass array [keV]
        REAL,                        INTENT(OUT) :: qm

        REAL,    PARAMETER :: UPM=0.7745966692E0
        REAL,    PARAMETER :: W13=0.5555555556E0, W2=0.8888888889E0
        INTEGER, PARAMETER :: NMAX=1000
        REAL    :: xmin, xmax, dx, x, xm, y, dcei_qm, eta_ab, vab
        INTEGER :: ix

        vab=SQRT(1./(betab(ia)*mb(ia)) + 1./(betab(ib)*mb(ib)))
        eta_ab=ABS(zb(ia)*zb(ib))*2*BEKEV*A0CM/HBARC/vab

        xmin=0.
        xmax=15
        dx=(xmax-xmin)/NMAX
        x=xmin-dx
        qm=0
        DO ix=1,NMAX,2
!     
           x=x+2*dx
           y=dcei_qm(x, eta_ab)
           qm=qm + W2*y
!
           xm=x-dx*UPM
           y=dcei_qm(xm,eta_ab)
           qm=qm + W13*y
!
           xm=x+dx*UPM
           y=dcei_qm(xm,eta_ab)
           qm=qm + W13*y
        ENDDO
        qm=-0.5*qm*dx
      END SUBROUTINE cab_qm_mass
\end{verbatim}
%%



\pagebreak
\appendix



\section{Calculating the Real and Imaginary Parts of F}
\label{app:FrFi}

We can write the dielectric function (\ref{Fdef}) as a sum over plasma
components,
%%
\begin{eqnarray}
  F(v)={\sum}_b F_b(v) \ ,
\end{eqnarray}
%%
where we express the contribution from plasma species $b$ as
%%
\begin{eqnarray}
  F_b(v) 
  &=& 
\label{Fbdef}
  -\int_{-\infty}^\infty du\, \frac{\rho_b(v)}{v - u + i\eta}
\\[5pt]
  \rho_b(v) 
  &=& 
  \kappa_b^2\,\sqrt{\frac{\beta_b m_b}{2\pi}}\, v\,
  \exp\!\left\{-\frac{1}{2}\,\beta_b m_b\, v^2\right\} \ .
\label{barrhob}
\end{eqnarray} 
%%
We will often decompose $F$ into its contribution from
electrons and ions and write
%%
\begin{eqnarray}
 F(v)=F_e(v) + F_\smI(v) \ .
\end{eqnarray}
%%
Note the reflection property
%%
\begin{eqnarray}
  F_b(-v) 
  &=& 
  F_b^*(v) \ ,
\label{Fbreflect}
\end{eqnarray}
%%
which means that the real part of $F_b(v)$ is even in $v$ and the
imaginary part is odd. For numerical work it is best to use the
explicit real and imaginary parts of $F$, which can be written
%%
\begin{eqnarray}
  F_\smRe(v)
  &=& 
  \sum_b \kappa_b^2 
  \left[1 - 2 \sqrt{\frac{\beta_b m_b}{2}} ~v~
  {\rm daw}\left\{\sqrt{\frac{\beta_b m_b}{2}}\,v 
  \right\} \right]
\label{Fru}
\\[5pt]
  F_\smIm(v)
  &=&
  \sqrt{\pi} \sum_b \kappa_b^2 
  \sqrt{\frac{\beta_b m_b}{2}}~~
  v\, \exp\left\{-\frac{\beta_b m_b}
  {2}\, v^2\right\} = \pi  \rho_{\rm tot}(v)  \ ,
\label{Fiu}
\end{eqnarray}
%% 
where the Dawson integral is defined by 
%%
  \begin{eqnarray}
  {\rm daw}(x) = \int_0^x dy\, 
  e^{y^2 - x^2} = \frac{\sqrt{\pi}}{2}\, e^{-x^2}
  {\rm erfi}(x) \ .
  \end{eqnarray}
%% 
The limits of small and large arguments of the Dawson function
are
%%
\begin{eqnarray}
  {\rm daw}(x) 
  &=& 
  x +\frac{2 x^3}{3} + \frac{4 x^5}{15} + {\cal O}(x^7) 
\label{dawA}
\\[5pt]
  {\rm daw}(x)
  &=&
  \frac{1}{2x} +\frac{1}{4 x^3} + \frac{3}{8x^5} + {\cal O}(x^{-7}) \ .
\label{dawB}
\end{eqnarray}
%%

The functions $F_b(v)$ have units of wave-number-squared 
[$1/L^2$] and their argument $v$ has units of velocity.
We can express the functions $F_b(v)$ in terms of a single
dimensionless function $\mathbb F(x)$ as follows:
%%
\begin{eqnarray}
  F_b(v) 
  &=& 
  \kappa_b^2 \,\mathbb{F}\left(
  \sqrt{\frac{\beta_b m_b}{2}}\, v ~ \right)
\label{FbFbb}
\end{eqnarray}
%%
with
%%
\begin{eqnarray}
  \mathbb{F}(x) 
  &=& 
  \int_{-\infty}^\infty dy\, \frac{\bar\rho(y)}{y - x - i \eta} 
\\[5pt]
  \bar\rho(y)&=& \frac{1}{\sqrt{\pi}}\,y\,e^{-y^2}  \ .
\end{eqnarray}
%%
Relation (\ref{FbFbb}) holds because $\rho_b(u) = \kappa_b^2\,\bar
\rho(y)$ for $u=(2/\beta_b m_b)^{1/2}\, y$.  The reflection property
(\ref{Fbreflect}) becomes
%%
\begin{eqnarray}
  \mathbb{F}(-x) = \mathbb{F}^*(x) \  ,
\label{Freflect}
\end{eqnarray}
%%
which means that the real part is even in $x$ and the imaginary
part is odd,
%%
\begin{eqnarray}
  \mathbb{F}_\smRe(-x) &=& \phantom{-}\mathbb{F}_\smRe(x)
\\[5pt]
  \mathbb{F}_\smIm(-x) &=& -\mathbb{F}_\smIm(x) \ .
\end{eqnarray}
%%
As with expressions (\ref{Fru}) and (\ref{Fiu}), the real and
imaginary parts can be written
%%
\begin{eqnarray}
  \mathbb{F}_\smRe(x)
  &=& 
  1 - 2 x\,  {\rm daw}\left(x  \right) 
\label{FbarRe}
\\[5pt]
  \mathbb{F}_\smIm(x)
  &=&
  \pi\,  \bar\rho(x)  
  =
  \sqrt{\pi}\, x\, e^{-x^2}\ .
\label{FbarIm}
\end{eqnarray}
%% 



Let us now establish the forms (\ref{Fru}) and (\ref{Fiu}) for the real
and imaginary parts of $F(v)$.  
Staring with
%%
\begin{eqnarray}
  \frac{1}{y - x - i \eta} 
  =
  {\sf P}\frac{1}{y-x} + i \pi\,\delta(y-x) \ ,
\end{eqnarray}
%%
the imaginary part becomes 
%%
\begin{eqnarray}
  \mathbb{F}_\smIm(x) 
  &=& 
  \int_{-\infty}^\infty dy\, 
  {\rm Im}\,\frac{\bar\rho(y)}{y - x - i \eta} 
  =
  \int_{-\infty}^\infty dy\, \bar\rho(y) \,
  \pi \delta(y-x)
  =
  \pi \bar\rho(x) \ .
\end{eqnarray}
%%
The real part of the function must be evaluated by a principal part
integral,
%%
\begin{eqnarray}
  \mathbb{F}_\smRe(x) 
  &=& 
  {\sf P}\int_{-\infty}^\infty dy\, \frac{\bar\rho(y)}{y - x}  \ .
\end{eqnarray}
%%
Let us add and subtract unity in the form
%%
\begin{eqnarray}
  \int_{-\infty}^\infty \frac{dy}{y}\, \bar\rho(y) = 1 \ ,
\end{eqnarray}
%%
so that
%%
\begin{eqnarray}
  \mathbb{F}_\smRe(x) 
  &=& 
  1 +
  {\sf P}\int_{-\infty}^\infty dy\, 
  \left[\frac{\bar\rho(y)}{y - x} - \frac{\bar\rho(y)}{y}\right] 
\\[5pt]
  &=& 
  1 +
  {\sf P}\int_{-\infty}^\infty dy\, \frac{x}{y(y - x)}\, \bar\rho(y)
\\[5pt]
  &=& 
  1 +
  \frac{x}{\sqrt{\pi}}\,
  {\sf P}\int_{-\infty}^\infty dy\, \frac{e^{-y^2}}{y - x} \ .
\end{eqnarray}
%%
Making the change of variables $y^\prime=y-x$ (and dropping the prime)
we can write 
%%
\begin{eqnarray}
  \mathbb{F}_\smRe(x) 
  &=& 
  1 +
  \frac{x}{\sqrt{\pi}}\,
  {\sf P}\int_{-\infty}^\infty \frac{dy}{y}\, e^{-(y+x)^2} 
\\[5pt]
  &=&
  1 +
  \frac{x\,e^{-x^2}}{\sqrt{\pi}}\,
  {\sf P}\int_{-\infty}^\infty \frac{dy}{y}\, e^{-y^2 - 2 x y} 
\\[5pt]
  &=&
  1 +
  \frac{x\,e^{-x^2}}{\sqrt{\pi}}\lim_{\epsilon\to0^+}\left[
  \int_{-\infty}^{-\epsilon} \frac{dy}{y}\, e^{-y^2 - 2 x y} 
  +
  \int_\epsilon^\infty \frac{dy}{y}\, e^{-y^2 - 2 x y} 
  \right] \ .
\end{eqnarray}
%%
In the last expression we have used the definition of the principal
part integration.  Making a change of variables $y^\prime=-y$ in the
first integral in square brackets gives (and again dropping the prime)
%%
\begin{eqnarray}
  \int_{-\infty}^{-\epsilon} \frac{dy}{y}\, e^{-y^2 - 2 x y} 
  =
  -\int_\epsilon^\infty \frac{dy}{y}\, e^{-y^2 + 2 x y} \ ,
\end{eqnarray}
%%
and this allows us to write 
%%
\begin{eqnarray}
  \mathbb{F}_\smRe(x) 
  &=&
  1 -
  \frac{x\,e^{-x^2}}{\sqrt{\pi}}\lim_{\epsilon\to0^+}
  \int_\epsilon^\infty \frac{dy}{y}\, e^{-y^2}\Big[
  e^{2 x y}  - e^{- 2 x y} 
  \Big] \ .
\end{eqnarray}
%%
The term in square braces is just $2\sinh(2xy)$, which renders
the factor $1/y$ harmless when the limit $\epsilon \to 0^+$ is
taken, 
%%
\begin{eqnarray}
  \mathbb{F}_\smRe(x) 
  &=&
  1 -
  \frac{2x\,e^{-x^2}}{\sqrt{\pi}}  \int_0^\infty dy\, 
  e^{-y^2}\,   \frac{\sinh 2 x y }{y} 
  =
  1 - 2 x\, {\rm daw}(x) \ .
\end{eqnarray}
%%
The latter form hold because this is just another integral
representation of the Dawson function,
%%
\begin{eqnarray}
  {\rm daw}(x)
  &=&
  \frac{e^{-x^2}}{\sqrt{\pi}}  \int_0^\infty dy\, 
  e^{-y^2}\,   \frac{\sinh 2 x y }{y} \ .
\end{eqnarray}
%%
Compare this with
%%
\begin{eqnarray}
  {\rm daw}(x)
  &=&
  e^{-x^2} \! \int_0^x dy\,   e^{y^2} \ .
\end{eqnarray}
%%

\pagebreak
\section{Proving the Sum Rule}
\label{app:sumrule}

Let's review the calculation of Eq.~(\ref{CeIsmallme}).  In taking the
zero mass limit $m_e \to 0$ of Eq.~(\ref{CeIF}), we can replace
%%
\begin{eqnarray}
  e^{-\frac{1}{2}\,\beta_e m_e v^2} \to 1 
\end{eqnarray}
%%
This is because the term $\rho_\smI(v)$ and the
logarithmic term provide the necessary convergence.\footnote{
\footnoteskip
%%
  The support of the integrand lies near $\bar v_\smI$ determined by
  $\beta_\smI m_\smI \bar v_\smI^2 \sim 1$, and for velocities in this
  vicinity, we see that \hbox{$0 < \frac{1}{2}\, m_e \beta_e v^2 \ll
    1$} when $\beta_e m_e \ll \beta_\smI m_\smI$.  We can therefore
  replace the exponential by 1 for $v \sim \bar v_\smI$.
%%
} In this limit we can also can replace the electron spectral weight
by
%%
\begin{eqnarray}
  \rho_e(v) \to \kappa_e^2 \,\left(\frac{\beta_e m_e}{2\pi}\right)^{1/2} v \ ,
\end{eqnarray}
%%
and we can therefore substitute
%%
\begin{eqnarray}
  \frac{\rho_\smI(v)}{\rho_\text{tot}(v)}
  =
  \frac{\rho_\smI(v)}{\rho_\smI(v) + \rho_e(v)}
  \to
  \frac{\rho_\smI(v)}{\rho_\smI(v) + \kappa_e^2\, 
  (\beta_e m_e/2\pi)^{1/2}\,v }   \   .
\end{eqnarray}
%%
In fact, since the convergence is supplied by the logarithmic term, we
can take the $m_e \to 0$ limit in the second term of the integrand,
%%
\begin{eqnarray}
  \frac{\rho_\smI(v)}{\rho_\smI(v) + \kappa_e^2\, 
  (\beta_e m_e/2\pi)^{1/2}\,v } \to 1 \ ,
\end{eqnarray}
%%
which allows us to write
%%
\begin{eqnarray}
  \bar{\cal C}^0_{e\smI,\smR} 
  &=&
  \int_{-\infty}^{\infty} \!\! dv \, v\, 
  \frac{\rho_\smI(v)}{\rho_\smI(v) + \kappa_e^2\, 
  (\beta_e m_e/2\pi)^{1/2}\,v }
  \frac{i}{2 \pi} \,F(v) 
  \ln\! \left\{\frac{F(v)}{\kappa_e^2}\right\} 
\\[5pt]
  &=&
  \int_{-\infty}^{\infty} \!\! dv \, v\, 
  \frac{i}{2 \pi} \,F(v) 
  \ln\! \left\{\frac{F(v)}{\kappa_e^2}\right\} \ .
\label{CzeroONE}  
\end{eqnarray}
%%
We can expand the dielectric function as $F(v)= F_e(v) + F_\smI(v)$,
where the ion contribution is
%%
\begin{eqnarray}
  F_\smI(v) 
  =
  \int_{-\infty}^\infty du\, \frac{\rho_\smI(u)}{u -v - i \eta}
\end{eqnarray}
%%
and the electron contribution can be written
%%
\begin{eqnarray}
  F_e(v) = \int_{-\infty}^\infty du\, \frac{\rho_e(u)}{u -v - i \eta}
  = \kappa_e^2 \,\left(\frac{\beta_e m_e}{2\pi}\right)^{1/2}
  \int_{-\infty}^\infty du\, \frac{u\,\exp\{-\frac{1}{2}\, \beta_e m_e
    u^2\}}{u -v - i \eta} \ .
\end{eqnarray}
%%
The support in $u$ comes from $\beta_e m_e \bar u_e^2 \sim 1$, while
we take the argument $v$ to lie in the region about $\bar v_\smI$
determined by $\beta_\smI m_\smI \bar v_\smI^2 \sim 1$. In other
words, the typical velocity $u$ is the electron thermal velocity,
which is much faster than the typical ion velocity velocity $v$, that
is to say $0 < v \ll u $.  This means to leading order we can replace
$u/(u-v-i\eta) \to 1$; therefore,
%%
\begin{eqnarray}
  F_e(v) 
  = 
  \kappa_e^2 \,\left(\frac{\beta_e m_e}{2\pi}\right)^{1/2}
  \int_{-\infty}^\infty du\, \exp\{-\frac{1}{2}\, \beta_e m_e  u^2\}
  =
  \kappa_e^2 \ .
\end{eqnarray}
%%
To leading order we can express the total dielectric function as
%%
\begin{eqnarray}
  F(v) = \kappa_e^2 + F_\smI(v) \ ,
\end{eqnarray}
%%
and therefore
%%
\begin{eqnarray}
  \bar{\cal C}^0_{e\smI,\smR} 
  &=&
  \int_{-\infty}^{\infty} \!\! dv \, v\, 
  \frac{i}{2 \pi} \,\Big[\kappa_e^2 + F_\smI(v) \Big]
  \ln\! \left\{1 + \frac{F_\smI(v)}{\kappa_e^2}\right\} \ .
\label{CzeroTWO}  
\\[5pt]
  &=&
  \lim_{V \to \infty}
  \int_{-V}^V dv \, v\, \frac{i}{2 \pi} \,\Big[\kappa_e^2 + F_\smI(v) 
  \Big] \ln\! \left\{1 + \frac{F_\smI(v)}{\kappa_e^2}\right\} \ .
\label{CzeroTHREE}  
\end{eqnarray}
%%
Note that the integrand is analytic in the upper half plane, which
means that its integral along any closed path in the upper half plane
will vanish,
%%
\begin{eqnarray}
  \oint dz \, z\, 
  \frac{i}{2 \pi} \,\Big[\kappa_e^2 + F_\smI(z) \Big]
  \ln\! \left\{1 + \frac{F_\smI(z)}{\kappa_e^2}\right\} 
  = 0 \ .
\label{CzeroTHREEFOUR}  
\end{eqnarray}
%%
Let $C_V$ denote the counter-clockwise circle at the origin with
radius $V$, and we shall consider the closed contour defined by
traversing the real axis along $[-V,V]$ and then along $C_V$ in the
the upper half plane. Expression (\ref{CzeroTHREEFOUR}) can then be
written
%%
\begin{eqnarray}
  \int_{-V}^V dv \, v\, \frac{i}{2 \pi} \,\Big[\kappa_e^2 + F_\smI(v) 
  \Big] \ln\! \left\{1 + \frac{F_\smI(v)}{\kappa_e^2}\right\} 
  +
  \int_{C_V} dz \, z\, 
  \frac{i}{2 \pi} \,\Big[\kappa_e^2 + F_\smI(z) \Big]
  \ln\! \left\{1 + \frac{F_\smI(z)}{\kappa_e^2}\right\} 
  = 0 \ .
\nonumber \\ 
\end{eqnarray}
%%
Taking the $V \to \infty$ limit gives
%%
\begin{eqnarray}
  \bar{\cal C}^0_{e\smI,\smR} 
  &=&
  -\lim_{V \to \infty}\int_{C_V} dz \, z\, 
  \frac{i}{2 \pi} \,\Big[\kappa_e^2 + F_\smI(z) \Big]
  \ln\! \left\{1 + \frac{F_\smI(z)}{\kappa_e^2}\right\} \ .
\label{CzeroFOUR}  
\end{eqnarray}
%%
Complex numbers on $C_V$ take the form $z=V e^{i\theta}$. This
means that along $C_V$ we find $dz = i z\, d\theta$ (or $dz/z= i
d\theta$) and $\vert z \vert = V$. To complete the calculation
we need $F_\smI(z)$ for large values of $\vert z \vert$, which takes
the form
%%
\begin{eqnarray}
  F_\smI(z) 
  &=&
  -\frac{1}{z^2}{\sum}_i \omega_i^2 
  =
  - \frac{\omega_\smI^2}{z^2} \ .
\label{FIlargez}
\end{eqnarray}
%%
To leading order we find
%%
\begin{eqnarray}
  \Big[\kappa_e^2 + F_\smI(z) \Big]
  \ln\! \left\{1 + \frac{F_\smI(z)}{\kappa_e^2} 
  \right\}
  = 
  -\frac{\omega_\smI^2}{z^2} \ ,
\end{eqnarray}
%%
and therefore 
%%
\begin{eqnarray}
  \bar{\cal C}^0_{e\smI,\smR} 
  &=&
  \omega_\smI^2
  \lim_{V \to \infty}\frac{i}{2\pi}\int_{C_V} \frac{dz}{z}\,
  \Bigg\vert_{z = V e^{i\theta}}
  =
  \omega_\smI^2\frac{i}{2\pi}\int_0^\pi i d\theta
  = 
  -\frac{\omega_\smI^2}{2} \ .
\label{CzeroFIVE}  
\end{eqnarray}
%%
To verify (\ref{FIlargez}) note the following:
%%
\begin{eqnarray}
  F_\smI(z) 
  &=&
  -\sum\kappa_i^2 \left(\frac{\beta_i m_i}{2\pi}\right)^{1/2}
  \int_{-\infty}^\infty du\, \frac{u}{z}\left(1 - \frac{u}{z} -
  i \eta \right)^{-1}
  e^{-\frac{1}{2}\,\beta_i m_i u^2}
\\[5pt]
  &=&
  -\sum\kappa_i^2 \left(\frac{\beta_i m_i}{2\pi}\right)^{1/2}
  \int_{-\infty}^\infty du\, \frac{u}{z}\left(1 + \frac{u}{z} + \cdots\right)
  e^{-\frac{1}{2}\,\beta_i m_i u^2}
\\[5pt]
  &=&
  -\sum\kappa_i^2 \left(\frac{\beta_i m_i}{2\pi}\right)^{1/2}
  \frac{1}{z^2}\int_{-\infty}^\infty du\, u^2 
  e^{-\frac{1}{2}\,\beta_i m_i u^2}
\\[5pt]
  &=&
  -\sum\kappa_i^2 \left(\frac{\beta_i m_i}{2\pi}\right)^{1/2}
  \frac{1}{z^2}\,\frac{\sqrt{\pi}}{2}\, 
  \left(\frac{2}{\beta_i m_i}\right)^{3/2}
  =
  -\frac{1}{z^2}{\sum}_i \frac{\kappa_i^2}{\beta_i m_i} 
\\[5pt]
  &=&
  -\frac{1}{z^2}{\sum}_i \omega_i^2 \ .
\end{eqnarray}
%%

% ***===
\pagebreak
\section{F-Function: \lowercase{frfi}}

We can write the dielectric function (\ref{Fdef}) as a sum over plasma
\begin{eqnarray}
  F(v) 
  &=& 
\label{Fbdef2}
  -{\sum}_b\int_{-\infty}^\infty du\, \frac{\rho_b(v)}{v - u + i\eta}
\\[5pt]
  \rho_b(v) 
  &=& 
  \kappa_b^2\,\sqrt{\frac{\beta_b m_b}{2\pi}}\, v\,
  \exp\!\left\{-\frac{1}{2}\,\beta_b m_b\, v^2\right\} \ .
\label{barrhob2}
\end{eqnarray} 
%%
For numerical work it is best to use the
explicit real and imaginary parts of $F$, which can be written
%%
\begin{eqnarray}
  F_\smRe(v)
  &=& 
  \sum_b \kappa_b^2 
  \left[1 - 2 \sqrt{\frac{\beta_b m_b}{2}} ~v~
  {\rm daw}\left\{\sqrt{\frac{\beta_b m_b}{2}}\,v 
  \right\} \right]
\label{Fru2}
\\[5pt]
  F_\smIm(v)
  &=&
  \sqrt{\pi} \sum_b \kappa_b^2 
  \sqrt{\frac{\beta_b m_b}{2}}~~
  v\, \exp\left\{-\frac{\beta_b m_b}
  {2}\, v^2\right\} = \pi  \rho_{\rm tot}(v)  \ ,
\label{Fiu2}
\end{eqnarray}
%% 
where the Dawson integral is defined by 
%%
  \begin{eqnarray}
  {\rm daw}(x) = \int_0^x dy\, 
  e^{y^2 - x^2} = \frac{\sqrt{\pi}}{2}\, e^{-x^2}
  {\rm erfi}(x) \ .
  \end{eqnarray}
%% 
The function $F$ takes the form
%%
\begin{eqnarray}
  F(v) = {\sum}_b\mathbb{F}_b\left(a_b \right) 
  ~~~\text{with}~~~ 
  a_b = \left( \frac{\beta_b m_b}{2}\right)^{1/2} \!v
  =
  \left( \frac{\beta_b\, \bar m_b}{2}\right)^{1/2}\frac{v}{c}
\end{eqnarray}
%%
\noindent
where $\bar m_b = m_b c^2$.  


\vskip0.4cm 
\noindent
rate.f90:frfi
{
\baselineskip 10pt
\begin{verbatim}
      SUBROUTINE frfi(x,nni,alfb,ab,fr,fi,fabs,farg) 
      USE mathvars
      IMPLICIT NONE
      REAL,                        INTENT(IN)  :: x
      INTEGER,                     INTENT(IN)  :: nni    ! Number of ion species
      REAL,    DIMENSION(1:nni+1), INTENT(IN)  :: alfb   ! alpha(b)
      REAL,    DIMENSION(1:nni+1), INTENT(IN)  :: ab     ! a(b)
      REAL,                        INTENT(OUT) :: fr     ! Dimensionless: multiply fr and
      REAL,                        INTENT(OUT) :: fi     ! fi by ke^2 for physical units
      REAL,                        INTENT(OUT) :: fabs   !
      REAL,                        INTENT(OUT) :: farg   !

      REAL                :: xib, daw
      INTEGER             :: ib
      fr=0
      fi=0
      DO ib=1,nni+1
        xib=ab(ib)*x
        fr=fr + alfb(ib)*(1-2*xib*daw(xib))
        fi=fi + alfb(ib)*xib*EXP(-xib*xib)
      ENDDO
      fi=fi*SQRT(PI)
      fabs=SQRT(fr*fr + fi*fi)
      farg=ATAN2(fi,fr)
      END SUBROUTINE frfi
\end{verbatim}
%%


%\vfill
%
%\begin{thebibliography}{99}
%\bibskip
%
%
%\bibitem{ref1}
%  R.J. Goldston and P.H. Rutherford, 
%  {\em Introduction to Plasma Physics},
%  IOP Publishing Ltd., Bristol UK, 2000.
%\end{thebibliography}

\end{document}
  


